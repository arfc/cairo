\section{Introduction}
Reducing carbon emissions has become a priority for many countries in response
to the rising threat of climate change. The goal set by the 2015 Paris
Agreement is to prevent the global temperature from rising more than 1.5
$^\circ$C above pre-industrial levels \cite{noauthor_paris_nodate}. Virtually
all current plans to reduce carbon emissions depend on increasing the share of
energy production by
renewable and clean energy sources, especially solar and wind energy \cite{cany_nuclear_2018, chilvers_realising_2017}. While solar and wind are low-carbon sources, these
forms of electricity generation are variable and unpredictable. This variability
is found to be major cause of blackouts and power system failures
\cite{haes_alhelou_survey_2019}. Further, even modest penetrations of renewable
energy negatively affect the economics of other types of clean energy, such as
nuclear power \cite{cany_nuclear_2018, keppler_carbon_2011}. There has been
some work done to quantify the economic benefit of improving forecasts of
renewable energy \cite{wang_quantifying_2016, mc_garrigle_quantifying_2015, brancucci_martinez-anido_value_2016}. Some of the benefits of improving
forecasts are: 1) It is often cheaper than building storage devices
\cite{wang_quantifying_2016}. 2) Would reduce curtailment and allow for
efficient
use of non-renewable sources \cite{mc_garrigle_quantifying_2015}. 3) Enable a
slight, but important, amount of load-following from nuclear and bio-mass generators which are not designed for rapid load following \cite{brancucci_martinez-anido_value_2016}.
