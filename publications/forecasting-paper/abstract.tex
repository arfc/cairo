\begin{abstract}
Calls for decarbonization have led to rapid growth of solar photovoltaic cells
and wind turbines. These energy sources vary based on weather
features such as solar irradiance and wind speed. This variability challenges
grid operators to reliably meet demand through scheduling dispatchable
resources. This paper uses weather and energy data
from the diverse microgrid at the University of Illinois at Urbana-Champaign
to develop accurate forecasts for total electricity demand, wind power, and
solar power with echo state networks. We found that solar elevation angle was
the only parameter that improved the forecast in every case. Other parameters
must be chosen carefully for each application.
\end{abstract}
