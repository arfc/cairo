\section{Discussion}

The forecast accuracy of our \gls{esn} for the Lorenz model does not persist
for quite as long as in other works \cite{pathak_using_2017}. However, our
model successfully replicates the environment that produces the Lorenz
Attractor. Further, optimal parameters may be unique for each randomly
instantiated reservoir. It is impossible to replicate the exact conditions of
other works without information about a seed for the random state. We have
included this information for future work to compare with our results.

For each target variable, demand, wind, and solar, we found that air pressure was
the only meteorological factor that improved the forecast error in every case.
Solar elevation angle also decreased the error in most cases with one exception,
48-hour ahead solar production. One possible reason for the improved performance
from adding air pressure is that the data may contain implicit information about
weather dynamics. For example, air pressure typically changes throughout the
day due to solar heating and has close relationship to air temperature
\cite{aguilar_seasonal_2001}, thus
it contains implicit information about both the amount of solar energy reaching
the ground and the ambient temperature which influences electricity demand
and solar energy generation.
Similarly, the height of the sun in the sky has a strong influence on
ambient weather and thus on  demand and the production of renewable energy.
Elevation angle lacks information about how much solar energy reaches the
ground, which is perhaps why air pressure performed better in some cases.
Using the solar angle to improve forecasting has a couple of important
advantages over measured weather data. First, it can be calculated accurately
within a minute time-resolution \cite{us_department_of_commerce_esrl_nodate}.
Second, because solar angle can be calculated deterministically, it reduces the
amount of data processing required. Based on this, we recommend using solar
elevation as a simple first attempt at improving forecasts.

The forecast lengths were decided based on the requirements for improved
economics and planning mentioned in the literature
 \cite{wang_quantifying_2016,mc_garrigle_quantifying_2015,brancucci_martinez-anido_value_2016}. The \gls{esn} model performed reasonably well at predicting
 four hours ahead but is not an improvement over the state-of-the-art
 \cite{wang_quantifying_2016,powers_weather_2017}. The model did not perform
 well at the 48-hour ahead forecasts. This could be due to the lack of higher
 resolution data. \glspl{esn} are known for their ability to predict highly non-
 linear systems \cite{jaeger_harnessing_2004,lukosevicius_reservoir_2009} yet
 using hourly data could add superfluous complexity \cite{garland_model-free_2014} that confounds the model.
