\subsection{Benchmark: Lorenz 1963}

We first verified that our choice of implementation for a conventional \gls{esn}
produced similar results to those found in the literature
\cite{pathak_using_2017}. Table \ref{tab:lorenzparam} contains the hyper-parameters
that minimized the RMSE of the model. Our optimized values differ
somewhat from the literature, but our \gls{esn} implementation
succesfully replicated the climate of the Lorenz Attractor similar to Pathak
et. al 2017. Figure \ref{fig:lorenz63} shows the \gls{esn} forecast ten seconds
into the future for Lorenz 1963 model.


\begin{table}[h]
  \centering
  \caption{Hyper-parameters for the Lorenz 1963 model. The random seed was generated by the open source package \texttt{numpy}.}
  \label{tab:lorenzparam}
  \begin{tabular}{l r r}
    \hline
    Parameter & This paper & Literature \cite{pathak_using_2017}\\
    \hline
    $N$ & 2000& 300\\
    $\rho$& 0.9&1.2\\
    \texttt{sparsity}& 0.1& 0.1\\
    \texttt{noise}& 0.001& 0\\
    Training Length & 3200& Not Specified\\
    Random Seed & 85 & Not Specified\\
    \hline
  \end{tabular}
\end{table}
\begin{figure}[h]
  \centering
  % \includegraphics[width=\columnwidth]{lorenz63_prediction.png}
  %% Creator: Matplotlib, PGF backend
%%
%% To include the figure in your LaTeX document, write
%%   \input{<filename>.pgf}
%%
%% Make sure the required packages are loaded in your preamble
%%   \usepackage{pgf}
%%
%% Figures using additional raster images can only be included by \input if
%% they are in the same directory as the main LaTeX file. For loading figures
%% from other directories you can use the `import` package
%%   \usepackage{import}
%% and then include the figures with
%%   \import{<path to file>}{<filename>.pgf}
%%
%% Matplotlib used the following preamble
%%
\begingroup%
\makeatletter%
\begin{pgfpicture}%
\pgfpathrectangle{\pgfpointorigin}{\pgfqpoint{6.967413in}{4.173027in}}%
\pgfusepath{use as bounding box, clip}%
\begin{pgfscope}%
\pgfsetbuttcap%
\pgfsetmiterjoin%
\definecolor{currentfill}{rgb}{1.000000,1.000000,1.000000}%
\pgfsetfillcolor{currentfill}%
\pgfsetlinewidth{0.000000pt}%
\definecolor{currentstroke}{rgb}{1.000000,1.000000,1.000000}%
\pgfsetstrokecolor{currentstroke}%
\pgfsetdash{}{0pt}%
\pgfpathmoveto{\pgfqpoint{0.000000in}{0.000000in}}%
\pgfpathlineto{\pgfqpoint{6.967413in}{0.000000in}}%
\pgfpathlineto{\pgfqpoint{6.967413in}{4.173027in}}%
\pgfpathlineto{\pgfqpoint{0.000000in}{4.173027in}}%
\pgfpathclose%
\pgfusepath{fill}%
\end{pgfscope}%
\begin{pgfscope}%
\pgfsetbuttcap%
\pgfsetmiterjoin%
\definecolor{currentfill}{rgb}{0.898039,0.898039,0.898039}%
\pgfsetfillcolor{currentfill}%
\pgfsetlinewidth{0.000000pt}%
\definecolor{currentstroke}{rgb}{0.000000,0.000000,0.000000}%
\pgfsetstrokecolor{currentstroke}%
\pgfsetstrokeopacity{0.000000}%
\pgfsetdash{}{0pt}%
\pgfpathmoveto{\pgfqpoint{0.647840in}{3.185866in}}%
\pgfpathlineto{\pgfqpoint{5.410091in}{3.185866in}}%
\pgfpathlineto{\pgfqpoint{5.410091in}{4.073027in}}%
\pgfpathlineto{\pgfqpoint{0.647840in}{4.073027in}}%
\pgfpathclose%
\pgfusepath{fill}%
\end{pgfscope}%
\begin{pgfscope}%
\pgfpathrectangle{\pgfqpoint{0.647840in}{3.185866in}}{\pgfqpoint{4.762251in}{0.887161in}}%
\pgfusepath{clip}%
\pgfsetrectcap%
\pgfsetroundjoin%
\pgfsetlinewidth{0.803000pt}%
\definecolor{currentstroke}{rgb}{1.000000,1.000000,1.000000}%
\pgfsetstrokecolor{currentstroke}%
\pgfsetdash{}{0pt}%
\pgfpathmoveto{\pgfqpoint{0.864306in}{3.185866in}}%
\pgfpathlineto{\pgfqpoint{0.864306in}{4.073027in}}%
\pgfusepath{stroke}%
\end{pgfscope}%
\begin{pgfscope}%
\pgfsetbuttcap%
\pgfsetroundjoin%
\definecolor{currentfill}{rgb}{0.333333,0.333333,0.333333}%
\pgfsetfillcolor{currentfill}%
\pgfsetlinewidth{0.803000pt}%
\definecolor{currentstroke}{rgb}{0.333333,0.333333,0.333333}%
\pgfsetstrokecolor{currentstroke}%
\pgfsetdash{}{0pt}%
\pgfsys@defobject{currentmarker}{\pgfqpoint{0.000000in}{-0.048611in}}{\pgfqpoint{0.000000in}{0.000000in}}{%
\pgfpathmoveto{\pgfqpoint{0.000000in}{0.000000in}}%
\pgfpathlineto{\pgfqpoint{0.000000in}{-0.048611in}}%
\pgfusepath{stroke,fill}%
}%
\begin{pgfscope}%
\pgfsys@transformshift{0.864306in}{3.185866in}%
\pgfsys@useobject{currentmarker}{}%
\end{pgfscope}%
\end{pgfscope}%
\begin{pgfscope}%
\definecolor{textcolor}{rgb}{0.333333,0.333333,0.333333}%
\pgfsetstrokecolor{textcolor}%
\pgfsetfillcolor{textcolor}%
\pgftext[x=0.864306in,y=3.088644in,,top]{\color{textcolor}\rmfamily\fontsize{10.000000}{12.000000}\selectfont \(\displaystyle 80.0\)}%
\end{pgfscope}%
\begin{pgfscope}%
\pgfpathrectangle{\pgfqpoint{0.647840in}{3.185866in}}{\pgfqpoint{4.762251in}{0.887161in}}%
\pgfusepath{clip}%
\pgfsetrectcap%
\pgfsetroundjoin%
\pgfsetlinewidth{0.803000pt}%
\definecolor{currentstroke}{rgb}{1.000000,1.000000,1.000000}%
\pgfsetstrokecolor{currentstroke}%
\pgfsetdash{}{0pt}%
\pgfpathmoveto{\pgfqpoint{1.406012in}{3.185866in}}%
\pgfpathlineto{\pgfqpoint{1.406012in}{4.073027in}}%
\pgfusepath{stroke}%
\end{pgfscope}%
\begin{pgfscope}%
\pgfsetbuttcap%
\pgfsetroundjoin%
\definecolor{currentfill}{rgb}{0.333333,0.333333,0.333333}%
\pgfsetfillcolor{currentfill}%
\pgfsetlinewidth{0.803000pt}%
\definecolor{currentstroke}{rgb}{0.333333,0.333333,0.333333}%
\pgfsetstrokecolor{currentstroke}%
\pgfsetdash{}{0pt}%
\pgfsys@defobject{currentmarker}{\pgfqpoint{0.000000in}{-0.048611in}}{\pgfqpoint{0.000000in}{0.000000in}}{%
\pgfpathmoveto{\pgfqpoint{0.000000in}{0.000000in}}%
\pgfpathlineto{\pgfqpoint{0.000000in}{-0.048611in}}%
\pgfusepath{stroke,fill}%
}%
\begin{pgfscope}%
\pgfsys@transformshift{1.406012in}{3.185866in}%
\pgfsys@useobject{currentmarker}{}%
\end{pgfscope}%
\end{pgfscope}%
\begin{pgfscope}%
\definecolor{textcolor}{rgb}{0.333333,0.333333,0.333333}%
\pgfsetstrokecolor{textcolor}%
\pgfsetfillcolor{textcolor}%
\pgftext[x=1.406012in,y=3.088644in,,top]{\color{textcolor}\rmfamily\fontsize{10.000000}{12.000000}\selectfont \(\displaystyle 82.5\)}%
\end{pgfscope}%
\begin{pgfscope}%
\pgfpathrectangle{\pgfqpoint{0.647840in}{3.185866in}}{\pgfqpoint{4.762251in}{0.887161in}}%
\pgfusepath{clip}%
\pgfsetrectcap%
\pgfsetroundjoin%
\pgfsetlinewidth{0.803000pt}%
\definecolor{currentstroke}{rgb}{1.000000,1.000000,1.000000}%
\pgfsetstrokecolor{currentstroke}%
\pgfsetdash{}{0pt}%
\pgfpathmoveto{\pgfqpoint{1.947719in}{3.185866in}}%
\pgfpathlineto{\pgfqpoint{1.947719in}{4.073027in}}%
\pgfusepath{stroke}%
\end{pgfscope}%
\begin{pgfscope}%
\pgfsetbuttcap%
\pgfsetroundjoin%
\definecolor{currentfill}{rgb}{0.333333,0.333333,0.333333}%
\pgfsetfillcolor{currentfill}%
\pgfsetlinewidth{0.803000pt}%
\definecolor{currentstroke}{rgb}{0.333333,0.333333,0.333333}%
\pgfsetstrokecolor{currentstroke}%
\pgfsetdash{}{0pt}%
\pgfsys@defobject{currentmarker}{\pgfqpoint{0.000000in}{-0.048611in}}{\pgfqpoint{0.000000in}{0.000000in}}{%
\pgfpathmoveto{\pgfqpoint{0.000000in}{0.000000in}}%
\pgfpathlineto{\pgfqpoint{0.000000in}{-0.048611in}}%
\pgfusepath{stroke,fill}%
}%
\begin{pgfscope}%
\pgfsys@transformshift{1.947719in}{3.185866in}%
\pgfsys@useobject{currentmarker}{}%
\end{pgfscope}%
\end{pgfscope}%
\begin{pgfscope}%
\definecolor{textcolor}{rgb}{0.333333,0.333333,0.333333}%
\pgfsetstrokecolor{textcolor}%
\pgfsetfillcolor{textcolor}%
\pgftext[x=1.947719in,y=3.088644in,,top]{\color{textcolor}\rmfamily\fontsize{10.000000}{12.000000}\selectfont \(\displaystyle 85.0\)}%
\end{pgfscope}%
\begin{pgfscope}%
\pgfpathrectangle{\pgfqpoint{0.647840in}{3.185866in}}{\pgfqpoint{4.762251in}{0.887161in}}%
\pgfusepath{clip}%
\pgfsetrectcap%
\pgfsetroundjoin%
\pgfsetlinewidth{0.803000pt}%
\definecolor{currentstroke}{rgb}{1.000000,1.000000,1.000000}%
\pgfsetstrokecolor{currentstroke}%
\pgfsetdash{}{0pt}%
\pgfpathmoveto{\pgfqpoint{2.489425in}{3.185866in}}%
\pgfpathlineto{\pgfqpoint{2.489425in}{4.073027in}}%
\pgfusepath{stroke}%
\end{pgfscope}%
\begin{pgfscope}%
\pgfsetbuttcap%
\pgfsetroundjoin%
\definecolor{currentfill}{rgb}{0.333333,0.333333,0.333333}%
\pgfsetfillcolor{currentfill}%
\pgfsetlinewidth{0.803000pt}%
\definecolor{currentstroke}{rgb}{0.333333,0.333333,0.333333}%
\pgfsetstrokecolor{currentstroke}%
\pgfsetdash{}{0pt}%
\pgfsys@defobject{currentmarker}{\pgfqpoint{0.000000in}{-0.048611in}}{\pgfqpoint{0.000000in}{0.000000in}}{%
\pgfpathmoveto{\pgfqpoint{0.000000in}{0.000000in}}%
\pgfpathlineto{\pgfqpoint{0.000000in}{-0.048611in}}%
\pgfusepath{stroke,fill}%
}%
\begin{pgfscope}%
\pgfsys@transformshift{2.489425in}{3.185866in}%
\pgfsys@useobject{currentmarker}{}%
\end{pgfscope}%
\end{pgfscope}%
\begin{pgfscope}%
\definecolor{textcolor}{rgb}{0.333333,0.333333,0.333333}%
\pgfsetstrokecolor{textcolor}%
\pgfsetfillcolor{textcolor}%
\pgftext[x=2.489425in,y=3.088644in,,top]{\color{textcolor}\rmfamily\fontsize{10.000000}{12.000000}\selectfont \(\displaystyle 87.5\)}%
\end{pgfscope}%
\begin{pgfscope}%
\pgfpathrectangle{\pgfqpoint{0.647840in}{3.185866in}}{\pgfqpoint{4.762251in}{0.887161in}}%
\pgfusepath{clip}%
\pgfsetrectcap%
\pgfsetroundjoin%
\pgfsetlinewidth{0.803000pt}%
\definecolor{currentstroke}{rgb}{1.000000,1.000000,1.000000}%
\pgfsetstrokecolor{currentstroke}%
\pgfsetdash{}{0pt}%
\pgfpathmoveto{\pgfqpoint{3.031132in}{3.185866in}}%
\pgfpathlineto{\pgfqpoint{3.031132in}{4.073027in}}%
\pgfusepath{stroke}%
\end{pgfscope}%
\begin{pgfscope}%
\pgfsetbuttcap%
\pgfsetroundjoin%
\definecolor{currentfill}{rgb}{0.333333,0.333333,0.333333}%
\pgfsetfillcolor{currentfill}%
\pgfsetlinewidth{0.803000pt}%
\definecolor{currentstroke}{rgb}{0.333333,0.333333,0.333333}%
\pgfsetstrokecolor{currentstroke}%
\pgfsetdash{}{0pt}%
\pgfsys@defobject{currentmarker}{\pgfqpoint{0.000000in}{-0.048611in}}{\pgfqpoint{0.000000in}{0.000000in}}{%
\pgfpathmoveto{\pgfqpoint{0.000000in}{0.000000in}}%
\pgfpathlineto{\pgfqpoint{0.000000in}{-0.048611in}}%
\pgfusepath{stroke,fill}%
}%
\begin{pgfscope}%
\pgfsys@transformshift{3.031132in}{3.185866in}%
\pgfsys@useobject{currentmarker}{}%
\end{pgfscope}%
\end{pgfscope}%
\begin{pgfscope}%
\definecolor{textcolor}{rgb}{0.333333,0.333333,0.333333}%
\pgfsetstrokecolor{textcolor}%
\pgfsetfillcolor{textcolor}%
\pgftext[x=3.031132in,y=3.088644in,,top]{\color{textcolor}\rmfamily\fontsize{10.000000}{12.000000}\selectfont \(\displaystyle 90.0\)}%
\end{pgfscope}%
\begin{pgfscope}%
\pgfpathrectangle{\pgfqpoint{0.647840in}{3.185866in}}{\pgfqpoint{4.762251in}{0.887161in}}%
\pgfusepath{clip}%
\pgfsetrectcap%
\pgfsetroundjoin%
\pgfsetlinewidth{0.803000pt}%
\definecolor{currentstroke}{rgb}{1.000000,1.000000,1.000000}%
\pgfsetstrokecolor{currentstroke}%
\pgfsetdash{}{0pt}%
\pgfpathmoveto{\pgfqpoint{3.572839in}{3.185866in}}%
\pgfpathlineto{\pgfqpoint{3.572839in}{4.073027in}}%
\pgfusepath{stroke}%
\end{pgfscope}%
\begin{pgfscope}%
\pgfsetbuttcap%
\pgfsetroundjoin%
\definecolor{currentfill}{rgb}{0.333333,0.333333,0.333333}%
\pgfsetfillcolor{currentfill}%
\pgfsetlinewidth{0.803000pt}%
\definecolor{currentstroke}{rgb}{0.333333,0.333333,0.333333}%
\pgfsetstrokecolor{currentstroke}%
\pgfsetdash{}{0pt}%
\pgfsys@defobject{currentmarker}{\pgfqpoint{0.000000in}{-0.048611in}}{\pgfqpoint{0.000000in}{0.000000in}}{%
\pgfpathmoveto{\pgfqpoint{0.000000in}{0.000000in}}%
\pgfpathlineto{\pgfqpoint{0.000000in}{-0.048611in}}%
\pgfusepath{stroke,fill}%
}%
\begin{pgfscope}%
\pgfsys@transformshift{3.572839in}{3.185866in}%
\pgfsys@useobject{currentmarker}{}%
\end{pgfscope}%
\end{pgfscope}%
\begin{pgfscope}%
\definecolor{textcolor}{rgb}{0.333333,0.333333,0.333333}%
\pgfsetstrokecolor{textcolor}%
\pgfsetfillcolor{textcolor}%
\pgftext[x=3.572839in,y=3.088644in,,top]{\color{textcolor}\rmfamily\fontsize{10.000000}{12.000000}\selectfont \(\displaystyle 92.5\)}%
\end{pgfscope}%
\begin{pgfscope}%
\pgfpathrectangle{\pgfqpoint{0.647840in}{3.185866in}}{\pgfqpoint{4.762251in}{0.887161in}}%
\pgfusepath{clip}%
\pgfsetrectcap%
\pgfsetroundjoin%
\pgfsetlinewidth{0.803000pt}%
\definecolor{currentstroke}{rgb}{1.000000,1.000000,1.000000}%
\pgfsetstrokecolor{currentstroke}%
\pgfsetdash{}{0pt}%
\pgfpathmoveto{\pgfqpoint{4.114545in}{3.185866in}}%
\pgfpathlineto{\pgfqpoint{4.114545in}{4.073027in}}%
\pgfusepath{stroke}%
\end{pgfscope}%
\begin{pgfscope}%
\pgfsetbuttcap%
\pgfsetroundjoin%
\definecolor{currentfill}{rgb}{0.333333,0.333333,0.333333}%
\pgfsetfillcolor{currentfill}%
\pgfsetlinewidth{0.803000pt}%
\definecolor{currentstroke}{rgb}{0.333333,0.333333,0.333333}%
\pgfsetstrokecolor{currentstroke}%
\pgfsetdash{}{0pt}%
\pgfsys@defobject{currentmarker}{\pgfqpoint{0.000000in}{-0.048611in}}{\pgfqpoint{0.000000in}{0.000000in}}{%
\pgfpathmoveto{\pgfqpoint{0.000000in}{0.000000in}}%
\pgfpathlineto{\pgfqpoint{0.000000in}{-0.048611in}}%
\pgfusepath{stroke,fill}%
}%
\begin{pgfscope}%
\pgfsys@transformshift{4.114545in}{3.185866in}%
\pgfsys@useobject{currentmarker}{}%
\end{pgfscope}%
\end{pgfscope}%
\begin{pgfscope}%
\definecolor{textcolor}{rgb}{0.333333,0.333333,0.333333}%
\pgfsetstrokecolor{textcolor}%
\pgfsetfillcolor{textcolor}%
\pgftext[x=4.114545in,y=3.088644in,,top]{\color{textcolor}\rmfamily\fontsize{10.000000}{12.000000}\selectfont \(\displaystyle 95.0\)}%
\end{pgfscope}%
\begin{pgfscope}%
\pgfpathrectangle{\pgfqpoint{0.647840in}{3.185866in}}{\pgfqpoint{4.762251in}{0.887161in}}%
\pgfusepath{clip}%
\pgfsetrectcap%
\pgfsetroundjoin%
\pgfsetlinewidth{0.803000pt}%
\definecolor{currentstroke}{rgb}{1.000000,1.000000,1.000000}%
\pgfsetstrokecolor{currentstroke}%
\pgfsetdash{}{0pt}%
\pgfpathmoveto{\pgfqpoint{4.656252in}{3.185866in}}%
\pgfpathlineto{\pgfqpoint{4.656252in}{4.073027in}}%
\pgfusepath{stroke}%
\end{pgfscope}%
\begin{pgfscope}%
\pgfsetbuttcap%
\pgfsetroundjoin%
\definecolor{currentfill}{rgb}{0.333333,0.333333,0.333333}%
\pgfsetfillcolor{currentfill}%
\pgfsetlinewidth{0.803000pt}%
\definecolor{currentstroke}{rgb}{0.333333,0.333333,0.333333}%
\pgfsetstrokecolor{currentstroke}%
\pgfsetdash{}{0pt}%
\pgfsys@defobject{currentmarker}{\pgfqpoint{0.000000in}{-0.048611in}}{\pgfqpoint{0.000000in}{0.000000in}}{%
\pgfpathmoveto{\pgfqpoint{0.000000in}{0.000000in}}%
\pgfpathlineto{\pgfqpoint{0.000000in}{-0.048611in}}%
\pgfusepath{stroke,fill}%
}%
\begin{pgfscope}%
\pgfsys@transformshift{4.656252in}{3.185866in}%
\pgfsys@useobject{currentmarker}{}%
\end{pgfscope}%
\end{pgfscope}%
\begin{pgfscope}%
\definecolor{textcolor}{rgb}{0.333333,0.333333,0.333333}%
\pgfsetstrokecolor{textcolor}%
\pgfsetfillcolor{textcolor}%
\pgftext[x=4.656252in,y=3.088644in,,top]{\color{textcolor}\rmfamily\fontsize{10.000000}{12.000000}\selectfont \(\displaystyle 97.5\)}%
\end{pgfscope}%
\begin{pgfscope}%
\pgfpathrectangle{\pgfqpoint{0.647840in}{3.185866in}}{\pgfqpoint{4.762251in}{0.887161in}}%
\pgfusepath{clip}%
\pgfsetrectcap%
\pgfsetroundjoin%
\pgfsetlinewidth{0.803000pt}%
\definecolor{currentstroke}{rgb}{1.000000,1.000000,1.000000}%
\pgfsetstrokecolor{currentstroke}%
\pgfsetdash{}{0pt}%
\pgfpathmoveto{\pgfqpoint{5.197958in}{3.185866in}}%
\pgfpathlineto{\pgfqpoint{5.197958in}{4.073027in}}%
\pgfusepath{stroke}%
\end{pgfscope}%
\begin{pgfscope}%
\pgfsetbuttcap%
\pgfsetroundjoin%
\definecolor{currentfill}{rgb}{0.333333,0.333333,0.333333}%
\pgfsetfillcolor{currentfill}%
\pgfsetlinewidth{0.803000pt}%
\definecolor{currentstroke}{rgb}{0.333333,0.333333,0.333333}%
\pgfsetstrokecolor{currentstroke}%
\pgfsetdash{}{0pt}%
\pgfsys@defobject{currentmarker}{\pgfqpoint{0.000000in}{-0.048611in}}{\pgfqpoint{0.000000in}{0.000000in}}{%
\pgfpathmoveto{\pgfqpoint{0.000000in}{0.000000in}}%
\pgfpathlineto{\pgfqpoint{0.000000in}{-0.048611in}}%
\pgfusepath{stroke,fill}%
}%
\begin{pgfscope}%
\pgfsys@transformshift{5.197958in}{3.185866in}%
\pgfsys@useobject{currentmarker}{}%
\end{pgfscope}%
\end{pgfscope}%
\begin{pgfscope}%
\definecolor{textcolor}{rgb}{0.333333,0.333333,0.333333}%
\pgfsetstrokecolor{textcolor}%
\pgfsetfillcolor{textcolor}%
\pgftext[x=5.197958in,y=3.088644in,,top]{\color{textcolor}\rmfamily\fontsize{10.000000}{12.000000}\selectfont \(\displaystyle 100.0\)}%
\end{pgfscope}%
\begin{pgfscope}%
\pgfpathrectangle{\pgfqpoint{0.647840in}{3.185866in}}{\pgfqpoint{4.762251in}{0.887161in}}%
\pgfusepath{clip}%
\pgfsetrectcap%
\pgfsetroundjoin%
\pgfsetlinewidth{0.803000pt}%
\definecolor{currentstroke}{rgb}{1.000000,1.000000,1.000000}%
\pgfsetstrokecolor{currentstroke}%
\pgfsetdash{}{0pt}%
\pgfpathmoveto{\pgfqpoint{0.647840in}{3.384585in}}%
\pgfpathlineto{\pgfqpoint{5.410091in}{3.384585in}}%
\pgfusepath{stroke}%
\end{pgfscope}%
\begin{pgfscope}%
\pgfsetbuttcap%
\pgfsetroundjoin%
\definecolor{currentfill}{rgb}{0.333333,0.333333,0.333333}%
\pgfsetfillcolor{currentfill}%
\pgfsetlinewidth{0.803000pt}%
\definecolor{currentstroke}{rgb}{0.333333,0.333333,0.333333}%
\pgfsetstrokecolor{currentstroke}%
\pgfsetdash{}{0pt}%
\pgfsys@defobject{currentmarker}{\pgfqpoint{-0.048611in}{0.000000in}}{\pgfqpoint{0.000000in}{0.000000in}}{%
\pgfpathmoveto{\pgfqpoint{0.000000in}{0.000000in}}%
\pgfpathlineto{\pgfqpoint{-0.048611in}{0.000000in}}%
\pgfusepath{stroke,fill}%
}%
\begin{pgfscope}%
\pgfsys@transformshift{0.647840in}{3.384585in}%
\pgfsys@useobject{currentmarker}{}%
\end{pgfscope}%
\end{pgfscope}%
\begin{pgfscope}%
\definecolor{textcolor}{rgb}{0.333333,0.333333,0.333333}%
\pgfsetstrokecolor{textcolor}%
\pgfsetfillcolor{textcolor}%
\pgftext[x=0.303703in,y=3.336360in,left,base]{\color{textcolor}\rmfamily\fontsize{10.000000}{12.000000}\selectfont \(\displaystyle -10\)}%
\end{pgfscope}%
\begin{pgfscope}%
\pgfpathrectangle{\pgfqpoint{0.647840in}{3.185866in}}{\pgfqpoint{4.762251in}{0.887161in}}%
\pgfusepath{clip}%
\pgfsetrectcap%
\pgfsetroundjoin%
\pgfsetlinewidth{0.803000pt}%
\definecolor{currentstroke}{rgb}{1.000000,1.000000,1.000000}%
\pgfsetstrokecolor{currentstroke}%
\pgfsetdash{}{0pt}%
\pgfpathmoveto{\pgfqpoint{0.647840in}{3.626846in}}%
\pgfpathlineto{\pgfqpoint{5.410091in}{3.626846in}}%
\pgfusepath{stroke}%
\end{pgfscope}%
\begin{pgfscope}%
\pgfsetbuttcap%
\pgfsetroundjoin%
\definecolor{currentfill}{rgb}{0.333333,0.333333,0.333333}%
\pgfsetfillcolor{currentfill}%
\pgfsetlinewidth{0.803000pt}%
\definecolor{currentstroke}{rgb}{0.333333,0.333333,0.333333}%
\pgfsetstrokecolor{currentstroke}%
\pgfsetdash{}{0pt}%
\pgfsys@defobject{currentmarker}{\pgfqpoint{-0.048611in}{0.000000in}}{\pgfqpoint{0.000000in}{0.000000in}}{%
\pgfpathmoveto{\pgfqpoint{0.000000in}{0.000000in}}%
\pgfpathlineto{\pgfqpoint{-0.048611in}{0.000000in}}%
\pgfusepath{stroke,fill}%
}%
\begin{pgfscope}%
\pgfsys@transformshift{0.647840in}{3.626846in}%
\pgfsys@useobject{currentmarker}{}%
\end{pgfscope}%
\end{pgfscope}%
\begin{pgfscope}%
\definecolor{textcolor}{rgb}{0.333333,0.333333,0.333333}%
\pgfsetstrokecolor{textcolor}%
\pgfsetfillcolor{textcolor}%
\pgftext[x=0.481173in,y=3.578621in,left,base]{\color{textcolor}\rmfamily\fontsize{10.000000}{12.000000}\selectfont \(\displaystyle 0\)}%
\end{pgfscope}%
\begin{pgfscope}%
\pgfpathrectangle{\pgfqpoint{0.647840in}{3.185866in}}{\pgfqpoint{4.762251in}{0.887161in}}%
\pgfusepath{clip}%
\pgfsetrectcap%
\pgfsetroundjoin%
\pgfsetlinewidth{0.803000pt}%
\definecolor{currentstroke}{rgb}{1.000000,1.000000,1.000000}%
\pgfsetstrokecolor{currentstroke}%
\pgfsetdash{}{0pt}%
\pgfpathmoveto{\pgfqpoint{0.647840in}{3.869107in}}%
\pgfpathlineto{\pgfqpoint{5.410091in}{3.869107in}}%
\pgfusepath{stroke}%
\end{pgfscope}%
\begin{pgfscope}%
\pgfsetbuttcap%
\pgfsetroundjoin%
\definecolor{currentfill}{rgb}{0.333333,0.333333,0.333333}%
\pgfsetfillcolor{currentfill}%
\pgfsetlinewidth{0.803000pt}%
\definecolor{currentstroke}{rgb}{0.333333,0.333333,0.333333}%
\pgfsetstrokecolor{currentstroke}%
\pgfsetdash{}{0pt}%
\pgfsys@defobject{currentmarker}{\pgfqpoint{-0.048611in}{0.000000in}}{\pgfqpoint{0.000000in}{0.000000in}}{%
\pgfpathmoveto{\pgfqpoint{0.000000in}{0.000000in}}%
\pgfpathlineto{\pgfqpoint{-0.048611in}{0.000000in}}%
\pgfusepath{stroke,fill}%
}%
\begin{pgfscope}%
\pgfsys@transformshift{0.647840in}{3.869107in}%
\pgfsys@useobject{currentmarker}{}%
\end{pgfscope}%
\end{pgfscope}%
\begin{pgfscope}%
\definecolor{textcolor}{rgb}{0.333333,0.333333,0.333333}%
\pgfsetstrokecolor{textcolor}%
\pgfsetfillcolor{textcolor}%
\pgftext[x=0.411728in,y=3.820882in,left,base]{\color{textcolor}\rmfamily\fontsize{10.000000}{12.000000}\selectfont \(\displaystyle 10\)}%
\end{pgfscope}%
\begin{pgfscope}%
\definecolor{textcolor}{rgb}{0.333333,0.333333,0.333333}%
\pgfsetstrokecolor{textcolor}%
\pgfsetfillcolor{textcolor}%
\pgftext[x=0.248148in,y=3.629446in,,bottom,rotate=90.000000]{\color{textcolor}\rmfamily\fontsize{12.000000}{14.400000}\selectfont x}%
\end{pgfscope}%
\begin{pgfscope}%
\pgfpathrectangle{\pgfqpoint{0.647840in}{3.185866in}}{\pgfqpoint{4.762251in}{0.887161in}}%
\pgfusepath{clip}%
\pgfsetrectcap%
\pgfsetroundjoin%
\pgfsetlinewidth{1.505625pt}%
\definecolor{currentstroke}{rgb}{0.886275,0.290196,0.200000}%
\pgfsetstrokecolor{currentstroke}%
\pgfsetdash{}{0pt}%
\pgfpathmoveto{\pgfqpoint{0.864306in}{3.528079in}}%
\pgfpathlineto{\pgfqpoint{0.868639in}{3.512089in}}%
\pgfpathlineto{\pgfqpoint{0.872973in}{3.493262in}}%
\pgfpathlineto{\pgfqpoint{0.877306in}{3.471302in}}%
\pgfpathlineto{\pgfqpoint{0.881640in}{3.446031in}}%
\pgfpathlineto{\pgfqpoint{0.885974in}{3.417501in}}%
\pgfpathlineto{\pgfqpoint{0.894641in}{3.353140in}}%
\pgfpathlineto{\pgfqpoint{0.898975in}{3.320323in}}%
\pgfpathlineto{\pgfqpoint{0.903308in}{3.290589in}}%
\pgfpathlineto{\pgfqpoint{0.907642in}{3.267546in}}%
\pgfpathlineto{\pgfqpoint{0.911976in}{3.254910in}}%
\pgfpathlineto{\pgfqpoint{0.916309in}{3.255508in}}%
\pgfpathlineto{\pgfqpoint{0.920643in}{3.270266in}}%
\pgfpathlineto{\pgfqpoint{0.924977in}{3.297770in}}%
\pgfpathlineto{\pgfqpoint{0.929310in}{3.334667in}}%
\pgfpathlineto{\pgfqpoint{0.942311in}{3.460783in}}%
\pgfpathlineto{\pgfqpoint{0.946645in}{3.497674in}}%
\pgfpathlineto{\pgfqpoint{0.950979in}{3.529499in}}%
\pgfpathlineto{\pgfqpoint{0.955312in}{3.556081in}}%
\pgfpathlineto{\pgfqpoint{0.959646in}{3.577743in}}%
\pgfpathlineto{\pgfqpoint{0.963980in}{3.595082in}}%
\pgfpathlineto{\pgfqpoint{0.968313in}{3.608805in}}%
\pgfpathlineto{\pgfqpoint{0.972647in}{3.619623in}}%
\pgfpathlineto{\pgfqpoint{0.976981in}{3.628197in}}%
\pgfpathlineto{\pgfqpoint{0.981314in}{3.635110in}}%
\pgfpathlineto{\pgfqpoint{0.985648in}{3.640861in}}%
\pgfpathlineto{\pgfqpoint{0.994315in}{3.650487in}}%
\pgfpathlineto{\pgfqpoint{1.002982in}{3.659715in}}%
\pgfpathlineto{\pgfqpoint{1.007316in}{3.664825in}}%
\pgfpathlineto{\pgfqpoint{1.011650in}{3.670576in}}%
\pgfpathlineto{\pgfqpoint{1.015983in}{3.677197in}}%
\pgfpathlineto{\pgfqpoint{1.020317in}{3.684933in}}%
\pgfpathlineto{\pgfqpoint{1.024651in}{3.694050in}}%
\pgfpathlineto{\pgfqpoint{1.028984in}{3.704840in}}%
\pgfpathlineto{\pgfqpoint{1.033318in}{3.717630in}}%
\pgfpathlineto{\pgfqpoint{1.037652in}{3.732773in}}%
\pgfpathlineto{\pgfqpoint{1.041985in}{3.750636in}}%
\pgfpathlineto{\pgfqpoint{1.046319in}{3.771569in}}%
\pgfpathlineto{\pgfqpoint{1.050653in}{3.795844in}}%
\pgfpathlineto{\pgfqpoint{1.054986in}{3.823554in}}%
\pgfpathlineto{\pgfqpoint{1.059320in}{3.854454in}}%
\pgfpathlineto{\pgfqpoint{1.072321in}{3.954196in}}%
\pgfpathlineto{\pgfqpoint{1.076655in}{3.981288in}}%
\pgfpathlineto{\pgfqpoint{1.080988in}{3.999162in}}%
\pgfpathlineto{\pgfqpoint{1.085322in}{4.004311in}}%
\pgfpathlineto{\pgfqpoint{1.089655in}{3.994845in}}%
\pgfpathlineto{\pgfqpoint{1.093989in}{3.971268in}}%
\pgfpathlineto{\pgfqpoint{1.098323in}{3.936415in}}%
\pgfpathlineto{\pgfqpoint{1.106990in}{3.850101in}}%
\pgfpathlineto{\pgfqpoint{1.111324in}{3.806800in}}%
\pgfpathlineto{\pgfqpoint{1.115657in}{3.767159in}}%
\pgfpathlineto{\pgfqpoint{1.119991in}{3.732524in}}%
\pgfpathlineto{\pgfqpoint{1.124325in}{3.703316in}}%
\pgfpathlineto{\pgfqpoint{1.128658in}{3.679334in}}%
\pgfpathlineto{\pgfqpoint{1.132992in}{3.660019in}}%
\pgfpathlineto{\pgfqpoint{1.137326in}{3.644650in}}%
\pgfpathlineto{\pgfqpoint{1.141659in}{3.632472in}}%
\pgfpathlineto{\pgfqpoint{1.145993in}{3.622771in}}%
\pgfpathlineto{\pgfqpoint{1.150327in}{3.614904in}}%
\pgfpathlineto{\pgfqpoint{1.154660in}{3.608319in}}%
\pgfpathlineto{\pgfqpoint{1.163328in}{3.597185in}}%
\pgfpathlineto{\pgfqpoint{1.171995in}{3.586388in}}%
\pgfpathlineto{\pgfqpoint{1.176329in}{3.580381in}}%
\pgfpathlineto{\pgfqpoint{1.180662in}{3.573619in}}%
\pgfpathlineto{\pgfqpoint{1.184996in}{3.565840in}}%
\pgfpathlineto{\pgfqpoint{1.189329in}{3.556774in}}%
\pgfpathlineto{\pgfqpoint{1.193663in}{3.546130in}}%
\pgfpathlineto{\pgfqpoint{1.197997in}{3.533597in}}%
\pgfpathlineto{\pgfqpoint{1.202330in}{3.518844in}}%
\pgfpathlineto{\pgfqpoint{1.206664in}{3.501534in}}%
\pgfpathlineto{\pgfqpoint{1.210998in}{3.481355in}}%
\pgfpathlineto{\pgfqpoint{1.215331in}{3.458072in}}%
\pgfpathlineto{\pgfqpoint{1.219665in}{3.431625in}}%
\pgfpathlineto{\pgfqpoint{1.223999in}{3.402264in}}%
\pgfpathlineto{\pgfqpoint{1.237000in}{3.307846in}}%
\pgfpathlineto{\pgfqpoint{1.241333in}{3.281926in}}%
\pgfpathlineto{\pgfqpoint{1.245667in}{3.264296in}}%
\pgfpathlineto{\pgfqpoint{1.250001in}{3.258167in}}%
\pgfpathlineto{\pgfqpoint{1.254334in}{3.265406in}}%
\pgfpathlineto{\pgfqpoint{1.258668in}{3.285813in}}%
\pgfpathlineto{\pgfqpoint{1.263002in}{3.317083in}}%
\pgfpathlineto{\pgfqpoint{1.267335in}{3.355505in}}%
\pgfpathlineto{\pgfqpoint{1.276003in}{3.438070in}}%
\pgfpathlineto{\pgfqpoint{1.280336in}{3.476148in}}%
\pgfpathlineto{\pgfqpoint{1.284670in}{3.509780in}}%
\pgfpathlineto{\pgfqpoint{1.289004in}{3.538387in}}%
\pgfpathlineto{\pgfqpoint{1.293337in}{3.562018in}}%
\pgfpathlineto{\pgfqpoint{1.297671in}{3.581101in}}%
\pgfpathlineto{\pgfqpoint{1.302004in}{3.596254in}}%
\pgfpathlineto{\pgfqpoint{1.306338in}{3.608154in}}%
\pgfpathlineto{\pgfqpoint{1.310672in}{3.617459in}}%
\pgfpathlineto{\pgfqpoint{1.315005in}{3.624763in}}%
\pgfpathlineto{\pgfqpoint{1.319339in}{3.630583in}}%
\pgfpathlineto{\pgfqpoint{1.323673in}{3.635355in}}%
\pgfpathlineto{\pgfqpoint{1.328006in}{3.639439in}}%
\pgfpathlineto{\pgfqpoint{1.345341in}{3.654270in}}%
\pgfpathlineto{\pgfqpoint{1.349675in}{3.658660in}}%
\pgfpathlineto{\pgfqpoint{1.354008in}{3.663702in}}%
\pgfpathlineto{\pgfqpoint{1.358342in}{3.669595in}}%
\pgfpathlineto{\pgfqpoint{1.362676in}{3.676557in}}%
\pgfpathlineto{\pgfqpoint{1.367009in}{3.684834in}}%
\pgfpathlineto{\pgfqpoint{1.371343in}{3.694710in}}%
\pgfpathlineto{\pgfqpoint{1.375677in}{3.706509in}}%
\pgfpathlineto{\pgfqpoint{1.380010in}{3.720594in}}%
\pgfpathlineto{\pgfqpoint{1.384344in}{3.737368in}}%
\pgfpathlineto{\pgfqpoint{1.388678in}{3.757244in}}%
\pgfpathlineto{\pgfqpoint{1.393011in}{3.780606in}}%
\pgfpathlineto{\pgfqpoint{1.397345in}{3.807723in}}%
\pgfpathlineto{\pgfqpoint{1.401678in}{3.838608in}}%
\pgfpathlineto{\pgfqpoint{1.410346in}{3.909125in}}%
\pgfpathlineto{\pgfqpoint{1.414679in}{3.945330in}}%
\pgfpathlineto{\pgfqpoint{1.419013in}{3.978018in}}%
\pgfpathlineto{\pgfqpoint{1.423347in}{4.002852in}}%
\pgfpathlineto{\pgfqpoint{1.427680in}{4.015387in}}%
\pgfpathlineto{\pgfqpoint{1.432014in}{4.012400in}}%
\pgfpathlineto{\pgfqpoint{1.436348in}{3.993144in}}%
\pgfpathlineto{\pgfqpoint{1.440681in}{3.959786in}}%
\pgfpathlineto{\pgfqpoint{1.445015in}{3.916686in}}%
\pgfpathlineto{\pgfqpoint{1.453682in}{3.821262in}}%
\pgfpathlineto{\pgfqpoint{1.458016in}{3.776826in}}%
\pgfpathlineto{\pgfqpoint{1.462350in}{3.737564in}}%
\pgfpathlineto{\pgfqpoint{1.466683in}{3.704195in}}%
\pgfpathlineto{\pgfqpoint{1.471017in}{3.676631in}}%
\pgfpathlineto{\pgfqpoint{1.475351in}{3.654304in}}%
\pgfpathlineto{\pgfqpoint{1.479684in}{3.636421in}}%
\pgfpathlineto{\pgfqpoint{1.484018in}{3.622124in}}%
\pgfpathlineto{\pgfqpoint{1.488352in}{3.610585in}}%
\pgfpathlineto{\pgfqpoint{1.492685in}{3.601056in}}%
\pgfpathlineto{\pgfqpoint{1.497019in}{3.592886in}}%
\pgfpathlineto{\pgfqpoint{1.505686in}{3.578465in}}%
\pgfpathlineto{\pgfqpoint{1.514353in}{3.563738in}}%
\pgfpathlineto{\pgfqpoint{1.518687in}{3.555368in}}%
\pgfpathlineto{\pgfqpoint{1.523021in}{3.545905in}}%
\pgfpathlineto{\pgfqpoint{1.527354in}{3.535045in}}%
\pgfpathlineto{\pgfqpoint{1.531688in}{3.522490in}}%
\pgfpathlineto{\pgfqpoint{1.536022in}{3.507948in}}%
\pgfpathlineto{\pgfqpoint{1.540355in}{3.491150in}}%
\pgfpathlineto{\pgfqpoint{1.544689in}{3.471880in}}%
\pgfpathlineto{\pgfqpoint{1.549023in}{3.450025in}}%
\pgfpathlineto{\pgfqpoint{1.553356in}{3.425663in}}%
\pgfpathlineto{\pgfqpoint{1.562024in}{3.371362in}}%
\pgfpathlineto{\pgfqpoint{1.566357in}{3.343622in}}%
\pgfpathlineto{\pgfqpoint{1.570691in}{3.317965in}}%
\pgfpathlineto{\pgfqpoint{1.575025in}{3.296937in}}%
\pgfpathlineto{\pgfqpoint{1.579358in}{3.283260in}}%
\pgfpathlineto{\pgfqpoint{1.583692in}{3.279250in}}%
\pgfpathlineto{\pgfqpoint{1.588026in}{3.286141in}}%
\pgfpathlineto{\pgfqpoint{1.592359in}{3.303641in}}%
\pgfpathlineto{\pgfqpoint{1.596693in}{3.329942in}}%
\pgfpathlineto{\pgfqpoint{1.601027in}{3.362206in}}%
\pgfpathlineto{\pgfqpoint{1.614027in}{3.465247in}}%
\pgfpathlineto{\pgfqpoint{1.618361in}{3.494656in}}%
\pgfpathlineto{\pgfqpoint{1.622695in}{3.519926in}}%
\pgfpathlineto{\pgfqpoint{1.627028in}{3.540938in}}%
\pgfpathlineto{\pgfqpoint{1.631362in}{3.557918in}}%
\pgfpathlineto{\pgfqpoint{1.635696in}{3.571283in}}%
\pgfpathlineto{\pgfqpoint{1.640029in}{3.581525in}}%
\pgfpathlineto{\pgfqpoint{1.644363in}{3.589141in}}%
\pgfpathlineto{\pgfqpoint{1.648697in}{3.594587in}}%
\pgfpathlineto{\pgfqpoint{1.653030in}{3.598254in}}%
\pgfpathlineto{\pgfqpoint{1.657364in}{3.600463in}}%
\pgfpathlineto{\pgfqpoint{1.661698in}{3.601460in}}%
\pgfpathlineto{\pgfqpoint{1.666031in}{3.601425in}}%
\pgfpathlineto{\pgfqpoint{1.670365in}{3.600477in}}%
\pgfpathlineto{\pgfqpoint{1.674699in}{3.598681in}}%
\pgfpathlineto{\pgfqpoint{1.679032in}{3.596051in}}%
\pgfpathlineto{\pgfqpoint{1.683366in}{3.592557in}}%
\pgfpathlineto{\pgfqpoint{1.687700in}{3.588125in}}%
\pgfpathlineto{\pgfqpoint{1.692033in}{3.582639in}}%
\pgfpathlineto{\pgfqpoint{1.696367in}{3.575940in}}%
\pgfpathlineto{\pgfqpoint{1.700701in}{3.567822in}}%
\pgfpathlineto{\pgfqpoint{1.705034in}{3.558033in}}%
\pgfpathlineto{\pgfqpoint{1.709368in}{3.546269in}}%
\pgfpathlineto{\pgfqpoint{1.713701in}{3.532180in}}%
\pgfpathlineto{\pgfqpoint{1.718035in}{3.515378in}}%
\pgfpathlineto{\pgfqpoint{1.722369in}{3.495459in}}%
\pgfpathlineto{\pgfqpoint{1.726702in}{3.472051in}}%
\pgfpathlineto{\pgfqpoint{1.731036in}{3.444899in}}%
\pgfpathlineto{\pgfqpoint{1.735370in}{3.414007in}}%
\pgfpathlineto{\pgfqpoint{1.744037in}{3.343664in}}%
\pgfpathlineto{\pgfqpoint{1.748371in}{3.307694in}}%
\pgfpathlineto{\pgfqpoint{1.752704in}{3.275360in}}%
\pgfpathlineto{\pgfqpoint{1.757038in}{3.250982in}}%
\pgfpathlineto{\pgfqpoint{1.761372in}{3.238950in}}%
\pgfpathlineto{\pgfqpoint{1.765705in}{3.242397in}}%
\pgfpathlineto{\pgfqpoint{1.770039in}{3.261978in}}%
\pgfpathlineto{\pgfqpoint{1.774373in}{3.295471in}}%
\pgfpathlineto{\pgfqpoint{1.778706in}{3.338514in}}%
\pgfpathlineto{\pgfqpoint{1.787374in}{3.433439in}}%
\pgfpathlineto{\pgfqpoint{1.791707in}{3.477550in}}%
\pgfpathlineto{\pgfqpoint{1.796041in}{3.516495in}}%
\pgfpathlineto{\pgfqpoint{1.800375in}{3.549578in}}%
\pgfpathlineto{\pgfqpoint{1.804708in}{3.576897in}}%
\pgfpathlineto{\pgfqpoint{1.809042in}{3.599018in}}%
\pgfpathlineto{\pgfqpoint{1.813375in}{3.616729in}}%
\pgfpathlineto{\pgfqpoint{1.817709in}{3.630884in}}%
\pgfpathlineto{\pgfqpoint{1.822043in}{3.642300in}}%
\pgfpathlineto{\pgfqpoint{1.826376in}{3.651720in}}%
\pgfpathlineto{\pgfqpoint{1.830710in}{3.659789in}}%
\pgfpathlineto{\pgfqpoint{1.839377in}{3.674005in}}%
\pgfpathlineto{\pgfqpoint{1.848045in}{3.688496in}}%
\pgfpathlineto{\pgfqpoint{1.852378in}{3.696727in}}%
\pgfpathlineto{\pgfqpoint{1.856712in}{3.706035in}}%
\pgfpathlineto{\pgfqpoint{1.861046in}{3.716722in}}%
\pgfpathlineto{\pgfqpoint{1.865379in}{3.729088in}}%
\pgfpathlineto{\pgfqpoint{1.869713in}{3.743426in}}%
\pgfpathlineto{\pgfqpoint{1.874047in}{3.760013in}}%
\pgfpathlineto{\pgfqpoint{1.878380in}{3.779080in}}%
\pgfpathlineto{\pgfqpoint{1.882714in}{3.800759in}}%
\pgfpathlineto{\pgfqpoint{1.887048in}{3.825005in}}%
\pgfpathlineto{\pgfqpoint{1.895715in}{3.879444in}}%
\pgfpathlineto{\pgfqpoint{1.904382in}{3.933843in}}%
\pgfpathlineto{\pgfqpoint{1.908716in}{3.955781in}}%
\pgfpathlineto{\pgfqpoint{1.913050in}{3.970588in}}%
\pgfpathlineto{\pgfqpoint{1.917383in}{3.975825in}}%
\pgfpathlineto{\pgfqpoint{1.921717in}{3.970078in}}%
\pgfpathlineto{\pgfqpoint{1.926050in}{3.953465in}}%
\pgfpathlineto{\pgfqpoint{1.930384in}{3.927676in}}%
\pgfpathlineto{\pgfqpoint{1.934718in}{3.895522in}}%
\pgfpathlineto{\pgfqpoint{1.947719in}{3.791132in}}%
\pgfpathlineto{\pgfqpoint{1.952052in}{3.761046in}}%
\pgfpathlineto{\pgfqpoint{1.956386in}{3.735115in}}%
\pgfpathlineto{\pgfqpoint{1.960720in}{3.713494in}}%
\pgfpathlineto{\pgfqpoint{1.965053in}{3.695973in}}%
\pgfpathlineto{\pgfqpoint{1.969387in}{3.682139in}}%
\pgfpathlineto{\pgfqpoint{1.973721in}{3.671490in}}%
\pgfpathlineto{\pgfqpoint{1.978054in}{3.663518in}}%
\pgfpathlineto{\pgfqpoint{1.982388in}{3.657754in}}%
\pgfpathlineto{\pgfqpoint{1.986722in}{3.653789in}}%
\pgfpathlineto{\pgfqpoint{1.991055in}{3.651290in}}%
\pgfpathlineto{\pgfqpoint{1.995389in}{3.649995in}}%
\pgfpathlineto{\pgfqpoint{1.999723in}{3.649711in}}%
\pgfpathlineto{\pgfqpoint{2.004056in}{3.650303in}}%
\pgfpathlineto{\pgfqpoint{2.008390in}{3.651696in}}%
\pgfpathlineto{\pgfqpoint{2.012724in}{3.653857in}}%
\pgfpathlineto{\pgfqpoint{2.017057in}{3.656802in}}%
\pgfpathlineto{\pgfqpoint{2.021391in}{3.660585in}}%
\pgfpathlineto{\pgfqpoint{2.025724in}{3.665302in}}%
\pgfpathlineto{\pgfqpoint{2.030058in}{3.671089in}}%
\pgfpathlineto{\pgfqpoint{2.034392in}{3.678124in}}%
\pgfpathlineto{\pgfqpoint{2.038725in}{3.686632in}}%
\pgfpathlineto{\pgfqpoint{2.043059in}{3.696889in}}%
\pgfpathlineto{\pgfqpoint{2.047393in}{3.709217in}}%
\pgfpathlineto{\pgfqpoint{2.051726in}{3.723990in}}%
\pgfpathlineto{\pgfqpoint{2.056060in}{3.741618in}}%
\pgfpathlineto{\pgfqpoint{2.060394in}{3.762521in}}%
\pgfpathlineto{\pgfqpoint{2.064727in}{3.787076in}}%
\pgfpathlineto{\pgfqpoint{2.069061in}{3.815512in}}%
\pgfpathlineto{\pgfqpoint{2.073395in}{3.847757in}}%
\pgfpathlineto{\pgfqpoint{2.082062in}{3.920337in}}%
\pgfpathlineto{\pgfqpoint{2.086396in}{3.956605in}}%
\pgfpathlineto{\pgfqpoint{2.090729in}{3.988187in}}%
\pgfpathlineto{\pgfqpoint{2.095063in}{4.010467in}}%
\pgfpathlineto{\pgfqpoint{2.099397in}{4.019053in}}%
\pgfpathlineto{\pgfqpoint{2.103730in}{4.011219in}}%
\pgfpathlineto{\pgfqpoint{2.108064in}{3.987055in}}%
\pgfpathlineto{\pgfqpoint{2.112398in}{3.949578in}}%
\pgfpathlineto{\pgfqpoint{2.116731in}{3.903685in}}%
\pgfpathlineto{\pgfqpoint{2.125398in}{3.806749in}}%
\pgfpathlineto{\pgfqpoint{2.129732in}{3.762996in}}%
\pgfpathlineto{\pgfqpoint{2.134066in}{3.724880in}}%
\pgfpathlineto{\pgfqpoint{2.138399in}{3.692826in}}%
\pgfpathlineto{\pgfqpoint{2.142733in}{3.666544in}}%
\pgfpathlineto{\pgfqpoint{2.147067in}{3.645347in}}%
\pgfpathlineto{\pgfqpoint{2.151400in}{3.628382in}}%
\pgfpathlineto{\pgfqpoint{2.155734in}{3.614769in}}%
\pgfpathlineto{\pgfqpoint{2.160068in}{3.603679in}}%
\pgfpathlineto{\pgfqpoint{2.164401in}{3.594377in}}%
\pgfpathlineto{\pgfqpoint{2.168735in}{3.586224in}}%
\pgfpathlineto{\pgfqpoint{2.181736in}{3.563599in}}%
\pgfpathlineto{\pgfqpoint{2.186070in}{3.555305in}}%
\pgfpathlineto{\pgfqpoint{2.190403in}{3.546059in}}%
\pgfpathlineto{\pgfqpoint{2.194737in}{3.535547in}}%
\pgfpathlineto{\pgfqpoint{2.199071in}{3.523470in}}%
\pgfpathlineto{\pgfqpoint{2.203404in}{3.509538in}}%
\pgfpathlineto{\pgfqpoint{2.207738in}{3.493488in}}%
\pgfpathlineto{\pgfqpoint{2.212072in}{3.475101in}}%
\pgfpathlineto{\pgfqpoint{2.216405in}{3.454257in}}%
\pgfpathlineto{\pgfqpoint{2.220739in}{3.430997in}}%
\pgfpathlineto{\pgfqpoint{2.229406in}{3.378853in}}%
\pgfpathlineto{\pgfqpoint{2.238073in}{3.326489in}}%
\pgfpathlineto{\pgfqpoint{2.242407in}{3.305010in}}%
\pgfpathlineto{\pgfqpoint{2.246741in}{3.290005in}}%
\pgfpathlineto{\pgfqpoint{2.251074in}{3.283785in}}%
\pgfpathlineto{\pgfqpoint{2.255408in}{3.287795in}}%
\pgfpathlineto{\pgfqpoint{2.259742in}{3.302122in}}%
\pgfpathlineto{\pgfqpoint{2.264075in}{3.325401in}}%
\pgfpathlineto{\pgfqpoint{2.268409in}{3.355157in}}%
\pgfpathlineto{\pgfqpoint{2.281410in}{3.454906in}}%
\pgfpathlineto{\pgfqpoint{2.285744in}{3.484355in}}%
\pgfpathlineto{\pgfqpoint{2.290077in}{3.509966in}}%
\pgfpathlineto{\pgfqpoint{2.294411in}{3.531474in}}%
\pgfpathlineto{\pgfqpoint{2.298745in}{3.548992in}}%
\pgfpathlineto{\pgfqpoint{2.303078in}{3.562857in}}%
\pgfpathlineto{\pgfqpoint{2.307412in}{3.573511in}}%
\pgfpathlineto{\pgfqpoint{2.311746in}{3.581419in}}%
\pgfpathlineto{\pgfqpoint{2.316079in}{3.587019in}}%
\pgfpathlineto{\pgfqpoint{2.320413in}{3.590695in}}%
\pgfpathlineto{\pgfqpoint{2.324747in}{3.592759in}}%
\pgfpathlineto{\pgfqpoint{2.329080in}{3.593455in}}%
\pgfpathlineto{\pgfqpoint{2.333414in}{3.592955in}}%
\pgfpathlineto{\pgfqpoint{2.337747in}{3.591369in}}%
\pgfpathlineto{\pgfqpoint{2.342081in}{3.588746in}}%
\pgfpathlineto{\pgfqpoint{2.346415in}{3.585082in}}%
\pgfpathlineto{\pgfqpoint{2.350748in}{3.580321in}}%
\pgfpathlineto{\pgfqpoint{2.355082in}{3.574358in}}%
\pgfpathlineto{\pgfqpoint{2.359416in}{3.567041in}}%
\pgfpathlineto{\pgfqpoint{2.363749in}{3.558169in}}%
\pgfpathlineto{\pgfqpoint{2.368083in}{3.547493in}}%
\pgfpathlineto{\pgfqpoint{2.372417in}{3.534720in}}%
\pgfpathlineto{\pgfqpoint{2.376750in}{3.519518in}}%
\pgfpathlineto{\pgfqpoint{2.381084in}{3.501537in}}%
\pgfpathlineto{\pgfqpoint{2.385418in}{3.480441in}}%
\pgfpathlineto{\pgfqpoint{2.389751in}{3.455977in}}%
\pgfpathlineto{\pgfqpoint{2.394085in}{3.428074in}}%
\pgfpathlineto{\pgfqpoint{2.398419in}{3.397012in}}%
\pgfpathlineto{\pgfqpoint{2.411420in}{3.297517in}}%
\pgfpathlineto{\pgfqpoint{2.415753in}{3.270951in}}%
\pgfpathlineto{\pgfqpoint{2.420087in}{3.253860in}}%
\pgfpathlineto{\pgfqpoint{2.424421in}{3.249670in}}%
\pgfpathlineto{\pgfqpoint{2.428754in}{3.260125in}}%
\pgfpathlineto{\pgfqpoint{2.433088in}{3.284550in}}%
\pgfpathlineto{\pgfqpoint{2.437421in}{3.319985in}}%
\pgfpathlineto{\pgfqpoint{2.446089in}{3.406558in}}%
\pgfpathlineto{\pgfqpoint{2.450422in}{3.449643in}}%
\pgfpathlineto{\pgfqpoint{2.454756in}{3.488950in}}%
\pgfpathlineto{\pgfqpoint{2.459090in}{3.523202in}}%
\pgfpathlineto{\pgfqpoint{2.463423in}{3.552030in}}%
\pgfpathlineto{\pgfqpoint{2.467757in}{3.575664in}}%
\pgfpathlineto{\pgfqpoint{2.472091in}{3.594680in}}%
\pgfpathlineto{\pgfqpoint{2.476424in}{3.609801in}}%
\pgfpathlineto{\pgfqpoint{2.480758in}{3.621781in}}%
\pgfpathlineto{\pgfqpoint{2.485092in}{3.631331in}}%
\pgfpathlineto{\pgfqpoint{2.489425in}{3.639084in}}%
\pgfpathlineto{\pgfqpoint{2.493759in}{3.645588in}}%
\pgfpathlineto{\pgfqpoint{2.502426in}{3.656634in}}%
\pgfpathlineto{\pgfqpoint{2.511094in}{3.667413in}}%
\pgfpathlineto{\pgfqpoint{2.515427in}{3.673431in}}%
\pgfpathlineto{\pgfqpoint{2.519761in}{3.680216in}}%
\pgfpathlineto{\pgfqpoint{2.524095in}{3.688029in}}%
\pgfpathlineto{\pgfqpoint{2.528428in}{3.697141in}}%
\pgfpathlineto{\pgfqpoint{2.532762in}{3.707842in}}%
\pgfpathlineto{\pgfqpoint{2.537095in}{3.720446in}}%
\pgfpathlineto{\pgfqpoint{2.541429in}{3.735283in}}%
\pgfpathlineto{\pgfqpoint{2.545763in}{3.752693in}}%
\pgfpathlineto{\pgfqpoint{2.550096in}{3.772986in}}%
\pgfpathlineto{\pgfqpoint{2.554430in}{3.796394in}}%
\pgfpathlineto{\pgfqpoint{2.558764in}{3.822972in}}%
\pgfpathlineto{\pgfqpoint{2.563097in}{3.852457in}}%
\pgfpathlineto{\pgfqpoint{2.576098in}{3.946963in}}%
\pgfpathlineto{\pgfqpoint{2.580432in}{3.972714in}}%
\pgfpathlineto{\pgfqpoint{2.584766in}{3.990041in}}%
\pgfpathlineto{\pgfqpoint{2.589099in}{3.995751in}}%
\pgfpathlineto{\pgfqpoint{2.593433in}{3.988032in}}%
\pgfpathlineto{\pgfqpoint{2.597767in}{3.967161in}}%
\pgfpathlineto{\pgfqpoint{2.602100in}{3.935514in}}%
\pgfpathlineto{\pgfqpoint{2.606434in}{3.896842in}}%
\pgfpathlineto{\pgfqpoint{2.615101in}{3.814157in}}%
\pgfpathlineto{\pgfqpoint{2.619435in}{3.776155in}}%
\pgfpathlineto{\pgfqpoint{2.623769in}{3.742645in}}%
\pgfpathlineto{\pgfqpoint{2.628102in}{3.714177in}}%
\pgfpathlineto{\pgfqpoint{2.632436in}{3.690683in}}%
\pgfpathlineto{\pgfqpoint{2.636770in}{3.671724in}}%
\pgfpathlineto{\pgfqpoint{2.641103in}{3.656676in}}%
\pgfpathlineto{\pgfqpoint{2.645437in}{3.644858in}}%
\pgfpathlineto{\pgfqpoint{2.649770in}{3.635613in}}%
\pgfpathlineto{\pgfqpoint{2.654104in}{3.628346in}}%
\pgfpathlineto{\pgfqpoint{2.658438in}{3.622543in}}%
\pgfpathlineto{\pgfqpoint{2.662771in}{3.617771in}}%
\pgfpathlineto{\pgfqpoint{2.671439in}{3.609935in}}%
\pgfpathlineto{\pgfqpoint{2.680106in}{3.602602in}}%
\pgfpathlineto{\pgfqpoint{2.684440in}{3.598587in}}%
\pgfpathlineto{\pgfqpoint{2.688773in}{3.594084in}}%
\pgfpathlineto{\pgfqpoint{2.693107in}{3.588904in}}%
\pgfpathlineto{\pgfqpoint{2.697441in}{3.582846in}}%
\pgfpathlineto{\pgfqpoint{2.701774in}{3.575685in}}%
\pgfpathlineto{\pgfqpoint{2.706108in}{3.567170in}}%
\pgfpathlineto{\pgfqpoint{2.710442in}{3.557012in}}%
\pgfpathlineto{\pgfqpoint{2.714775in}{3.544883in}}%
\pgfpathlineto{\pgfqpoint{2.719109in}{3.530412in}}%
\pgfpathlineto{\pgfqpoint{2.723443in}{3.513196in}}%
\pgfpathlineto{\pgfqpoint{2.727776in}{3.492827in}}%
\pgfpathlineto{\pgfqpoint{2.732110in}{3.468934in}}%
\pgfpathlineto{\pgfqpoint{2.736444in}{3.441284in}}%
\pgfpathlineto{\pgfqpoint{2.740777in}{3.409923in}}%
\pgfpathlineto{\pgfqpoint{2.749444in}{3.339067in}}%
\pgfpathlineto{\pgfqpoint{2.753778in}{3.303312in}}%
\pgfpathlineto{\pgfqpoint{2.758112in}{3.271697in}}%
\pgfpathlineto{\pgfqpoint{2.762445in}{3.248614in}}%
\pgfpathlineto{\pgfqpoint{2.766779in}{3.238378in}}%
\pgfpathlineto{\pgfqpoint{2.771113in}{3.243875in}}%
\pgfpathlineto{\pgfqpoint{2.775446in}{3.265396in}}%
\pgfpathlineto{\pgfqpoint{2.779780in}{3.300387in}}%
\pgfpathlineto{\pgfqpoint{2.784114in}{3.344305in}}%
\pgfpathlineto{\pgfqpoint{2.792781in}{3.439252in}}%
\pgfpathlineto{\pgfqpoint{2.797115in}{3.482818in}}%
\pgfpathlineto{\pgfqpoint{2.801448in}{3.521068in}}%
\pgfpathlineto{\pgfqpoint{2.805782in}{3.553427in}}%
\pgfpathlineto{\pgfqpoint{2.810116in}{3.580069in}}%
\pgfpathlineto{\pgfqpoint{2.814449in}{3.601601in}}%
\pgfpathlineto{\pgfqpoint{2.818783in}{3.618829in}}%
\pgfpathlineto{\pgfqpoint{2.823117in}{3.632604in}}%
\pgfpathlineto{\pgfqpoint{2.827450in}{3.643740in}}%
\pgfpathlineto{\pgfqpoint{2.831784in}{3.652966in}}%
\pgfpathlineto{\pgfqpoint{2.836118in}{3.660917in}}%
\pgfpathlineto{\pgfqpoint{2.853452in}{3.689730in}}%
\pgfpathlineto{\pgfqpoint{2.857786in}{3.698105in}}%
\pgfpathlineto{\pgfqpoint{2.862119in}{3.707598in}}%
\pgfpathlineto{\pgfqpoint{2.866453in}{3.718512in}}%
\pgfpathlineto{\pgfqpoint{2.870787in}{3.731145in}}%
\pgfpathlineto{\pgfqpoint{2.875120in}{3.745790in}}%
\pgfpathlineto{\pgfqpoint{2.879454in}{3.762719in}}%
\pgfpathlineto{\pgfqpoint{2.883788in}{3.782150in}}%
\pgfpathlineto{\pgfqpoint{2.888121in}{3.804195in}}%
\pgfpathlineto{\pgfqpoint{2.892455in}{3.828774in}}%
\pgfpathlineto{\pgfqpoint{2.901122in}{3.883542in}}%
\pgfpathlineto{\pgfqpoint{2.905456in}{3.911482in}}%
\pgfpathlineto{\pgfqpoint{2.909790in}{3.937263in}}%
\pgfpathlineto{\pgfqpoint{2.914123in}{3.958295in}}%
\pgfpathlineto{\pgfqpoint{2.918457in}{3.971813in}}%
\pgfpathlineto{\pgfqpoint{2.922791in}{3.975494in}}%
\pgfpathlineto{\pgfqpoint{2.927124in}{3.968128in}}%
\pgfpathlineto{\pgfqpoint{2.931458in}{3.950071in}}%
\pgfpathlineto{\pgfqpoint{2.935792in}{3.923212in}}%
\pgfpathlineto{\pgfqpoint{2.940125in}{3.890456in}}%
\pgfpathlineto{\pgfqpoint{2.948793in}{3.819679in}}%
\pgfpathlineto{\pgfqpoint{2.953126in}{3.786623in}}%
\pgfpathlineto{\pgfqpoint{2.957460in}{3.757151in}}%
\pgfpathlineto{\pgfqpoint{2.961793in}{3.731877in}}%
\pgfpathlineto{\pgfqpoint{2.966127in}{3.710895in}}%
\pgfpathlineto{\pgfqpoint{2.970461in}{3.693957in}}%
\pgfpathlineto{\pgfqpoint{2.974794in}{3.680634in}}%
\pgfpathlineto{\pgfqpoint{2.979128in}{3.670420in}}%
\pgfpathlineto{\pgfqpoint{2.983462in}{3.662812in}}%
\pgfpathlineto{\pgfqpoint{2.987795in}{3.657349in}}%
\pgfpathlineto{\pgfqpoint{2.992129in}{3.653636in}}%
\pgfpathlineto{\pgfqpoint{2.996463in}{3.651349in}}%
\pgfpathlineto{\pgfqpoint{3.000796in}{3.650239in}}%
\pgfpathlineto{\pgfqpoint{3.005130in}{3.650122in}}%
\pgfpathlineto{\pgfqpoint{3.009464in}{3.650874in}}%
\pgfpathlineto{\pgfqpoint{3.013797in}{3.652427in}}%
\pgfpathlineto{\pgfqpoint{3.018131in}{3.654759in}}%
\pgfpathlineto{\pgfqpoint{3.022465in}{3.657891in}}%
\pgfpathlineto{\pgfqpoint{3.026798in}{3.661886in}}%
\pgfpathlineto{\pgfqpoint{3.031132in}{3.666848in}}%
\pgfpathlineto{\pgfqpoint{3.035466in}{3.672923in}}%
\pgfpathlineto{\pgfqpoint{3.039799in}{3.680298in}}%
\pgfpathlineto{\pgfqpoint{3.044133in}{3.689210in}}%
\pgfpathlineto{\pgfqpoint{3.048467in}{3.699944in}}%
\pgfpathlineto{\pgfqpoint{3.052800in}{3.712834in}}%
\pgfpathlineto{\pgfqpoint{3.057134in}{3.728263in}}%
\pgfpathlineto{\pgfqpoint{3.061467in}{3.746644in}}%
\pgfpathlineto{\pgfqpoint{3.065801in}{3.768389in}}%
\pgfpathlineto{\pgfqpoint{3.070135in}{3.793848in}}%
\pgfpathlineto{\pgfqpoint{3.074468in}{3.823189in}}%
\pgfpathlineto{\pgfqpoint{3.078802in}{3.856226in}}%
\pgfpathlineto{\pgfqpoint{3.091803in}{3.964637in}}%
\pgfpathlineto{\pgfqpoint{3.096137in}{3.994254in}}%
\pgfpathlineto{\pgfqpoint{3.100470in}{4.013454in}}%
\pgfpathlineto{\pgfqpoint{3.104804in}{4.018145in}}%
\pgfpathlineto{\pgfqpoint{3.109138in}{4.006231in}}%
\pgfpathlineto{\pgfqpoint{3.113471in}{3.978554in}}%
\pgfpathlineto{\pgfqpoint{3.117805in}{3.938695in}}%
\pgfpathlineto{\pgfqpoint{3.135140in}{3.753535in}}%
\pgfpathlineto{\pgfqpoint{3.139473in}{3.716990in}}%
\pgfpathlineto{\pgfqpoint{3.143807in}{3.686465in}}%
\pgfpathlineto{\pgfqpoint{3.148141in}{3.661553in}}%
\pgfpathlineto{\pgfqpoint{3.152474in}{3.641517in}}%
\pgfpathlineto{\pgfqpoint{3.156808in}{3.625490in}}%
\pgfpathlineto{\pgfqpoint{3.161141in}{3.612604in}}%
\pgfpathlineto{\pgfqpoint{3.165475in}{3.602056in}}%
\pgfpathlineto{\pgfqpoint{3.169809in}{3.593135in}}%
\pgfpathlineto{\pgfqpoint{3.178476in}{3.577822in}}%
\pgfpathlineto{\pgfqpoint{3.187143in}{3.562757in}}%
\pgfpathlineto{\pgfqpoint{3.191477in}{3.554362in}}%
\pgfpathlineto{\pgfqpoint{3.195811in}{3.544951in}}%
\pgfpathlineto{\pgfqpoint{3.200144in}{3.534217in}}%
\pgfpathlineto{\pgfqpoint{3.204478in}{3.521859in}}%
\pgfpathlineto{\pgfqpoint{3.208812in}{3.507592in}}%
\pgfpathlineto{\pgfqpoint{3.213145in}{3.491152in}}%
\pgfpathlineto{\pgfqpoint{3.217479in}{3.472332in}}%
\pgfpathlineto{\pgfqpoint{3.221813in}{3.451024in}}%
\pgfpathlineto{\pgfqpoint{3.226146in}{3.427302in}}%
\pgfpathlineto{\pgfqpoint{3.234814in}{3.374460in}}%
\pgfpathlineto{\pgfqpoint{3.239147in}{3.347414in}}%
\pgfpathlineto{\pgfqpoint{3.243481in}{3.322283in}}%
\pgfpathlineto{\pgfqpoint{3.247815in}{3.301471in}}%
\pgfpathlineto{\pgfqpoint{3.252148in}{3.287575in}}%
\pgfpathlineto{\pgfqpoint{3.256482in}{3.282841in}}%
\pgfpathlineto{\pgfqpoint{3.260816in}{3.288545in}}%
\pgfpathlineto{\pgfqpoint{3.265149in}{3.304541in}}%
\pgfpathlineto{\pgfqpoint{3.269483in}{3.329236in}}%
\pgfpathlineto{\pgfqpoint{3.273816in}{3.360001in}}%
\pgfpathlineto{\pgfqpoint{3.286817in}{3.460255in}}%
\pgfpathlineto{\pgfqpoint{3.291151in}{3.489310in}}%
\pgfpathlineto{\pgfqpoint{3.295485in}{3.514416in}}%
\pgfpathlineto{\pgfqpoint{3.299818in}{3.535386in}}%
\pgfpathlineto{\pgfqpoint{3.304152in}{3.552385in}}%
\pgfpathlineto{\pgfqpoint{3.308486in}{3.565782in}}%
\pgfpathlineto{\pgfqpoint{3.312819in}{3.576035in}}%
\pgfpathlineto{\pgfqpoint{3.317153in}{3.583614in}}%
\pgfpathlineto{\pgfqpoint{3.321487in}{3.588955in}}%
\pgfpathlineto{\pgfqpoint{3.325820in}{3.592437in}}%
\pgfpathlineto{\pgfqpoint{3.330154in}{3.594367in}}%
\pgfpathlineto{\pgfqpoint{3.334488in}{3.594979in}}%
\pgfpathlineto{\pgfqpoint{3.338821in}{3.594441in}}%
\pgfpathlineto{\pgfqpoint{3.343155in}{3.592858in}}%
\pgfpathlineto{\pgfqpoint{3.347489in}{3.590274in}}%
\pgfpathlineto{\pgfqpoint{3.351822in}{3.586684in}}%
\pgfpathlineto{\pgfqpoint{3.356156in}{3.582029in}}%
\pgfpathlineto{\pgfqpoint{3.360490in}{3.576207in}}%
\pgfpathlineto{\pgfqpoint{3.364823in}{3.569064in}}%
\pgfpathlineto{\pgfqpoint{3.369157in}{3.560400in}}%
\pgfpathlineto{\pgfqpoint{3.373490in}{3.549968in}}%
\pgfpathlineto{\pgfqpoint{3.377824in}{3.537474in}}%
\pgfpathlineto{\pgfqpoint{3.382158in}{3.522584in}}%
\pgfpathlineto{\pgfqpoint{3.386491in}{3.504939in}}%
\pgfpathlineto{\pgfqpoint{3.390825in}{3.484188in}}%
\pgfpathlineto{\pgfqpoint{3.395159in}{3.460049in}}%
\pgfpathlineto{\pgfqpoint{3.399492in}{3.432405in}}%
\pgfpathlineto{\pgfqpoint{3.403826in}{3.401464in}}%
\pgfpathlineto{\pgfqpoint{3.416827in}{3.300481in}}%
\pgfpathlineto{\pgfqpoint{3.421161in}{3.272510in}}%
\pgfpathlineto{\pgfqpoint{3.425494in}{3.253602in}}%
\pgfpathlineto{\pgfqpoint{3.429828in}{3.247415in}}%
\pgfpathlineto{\pgfqpoint{3.434162in}{3.256034in}}%
\pgfpathlineto{\pgfqpoint{3.438495in}{3.279113in}}%
\pgfpathlineto{\pgfqpoint{3.442829in}{3.313883in}}%
\pgfpathlineto{\pgfqpoint{3.447163in}{3.356048in}}%
\pgfpathlineto{\pgfqpoint{3.455830in}{3.445030in}}%
\pgfpathlineto{\pgfqpoint{3.460164in}{3.485390in}}%
\pgfpathlineto{\pgfqpoint{3.464497in}{3.520704in}}%
\pgfpathlineto{\pgfqpoint{3.468831in}{3.550514in}}%
\pgfpathlineto{\pgfqpoint{3.473164in}{3.575012in}}%
\pgfpathlineto{\pgfqpoint{3.477498in}{3.594762in}}%
\pgfpathlineto{\pgfqpoint{3.481832in}{3.610496in}}%
\pgfpathlineto{\pgfqpoint{3.486165in}{3.622985in}}%
\pgfpathlineto{\pgfqpoint{3.490499in}{3.632964in}}%
\pgfpathlineto{\pgfqpoint{3.494833in}{3.641089in}}%
\pgfpathlineto{\pgfqpoint{3.499166in}{3.647928in}}%
\pgfpathlineto{\pgfqpoint{3.507834in}{3.659613in}}%
\pgfpathlineto{\pgfqpoint{3.516501in}{3.671099in}}%
\pgfpathlineto{\pgfqpoint{3.520835in}{3.677530in}}%
\pgfpathlineto{\pgfqpoint{3.525168in}{3.684786in}}%
\pgfpathlineto{\pgfqpoint{3.529502in}{3.693139in}}%
\pgfpathlineto{\pgfqpoint{3.533836in}{3.702868in}}%
\pgfpathlineto{\pgfqpoint{3.538169in}{3.714273in}}%
\pgfpathlineto{\pgfqpoint{3.542503in}{3.727669in}}%
\pgfpathlineto{\pgfqpoint{3.546837in}{3.743382in}}%
\pgfpathlineto{\pgfqpoint{3.551170in}{3.761731in}}%
\pgfpathlineto{\pgfqpoint{3.555504in}{3.782984in}}%
\pgfpathlineto{\pgfqpoint{3.559838in}{3.807296in}}%
\pgfpathlineto{\pgfqpoint{3.564171in}{3.834594in}}%
\pgfpathlineto{\pgfqpoint{3.572839in}{3.895780in}}%
\pgfpathlineto{\pgfqpoint{3.577172in}{3.926880in}}%
\pgfpathlineto{\pgfqpoint{3.581506in}{3.955132in}}%
\pgfpathlineto{\pgfqpoint{3.585839in}{3.977282in}}%
\pgfpathlineto{\pgfqpoint{3.590173in}{3.989961in}}%
\pgfpathlineto{\pgfqpoint{3.594507in}{3.990545in}}%
\pgfpathlineto{\pgfqpoint{3.598840in}{3.978038in}}%
\pgfpathlineto{\pgfqpoint{3.603174in}{3.953512in}}%
\pgfpathlineto{\pgfqpoint{3.607508in}{3.919831in}}%
\pgfpathlineto{\pgfqpoint{3.624842in}{3.765564in}}%
\pgfpathlineto{\pgfqpoint{3.629176in}{3.734608in}}%
\pgfpathlineto{\pgfqpoint{3.633510in}{3.708573in}}%
\pgfpathlineto{\pgfqpoint{3.637843in}{3.687263in}}%
\pgfpathlineto{\pgfqpoint{3.642177in}{3.670191in}}%
\pgfpathlineto{\pgfqpoint{3.646511in}{3.656737in}}%
\pgfpathlineto{\pgfqpoint{3.650844in}{3.646255in}}%
\pgfpathlineto{\pgfqpoint{3.655178in}{3.638135in}}%
\pgfpathlineto{\pgfqpoint{3.659512in}{3.631836in}}%
\pgfpathlineto{\pgfqpoint{3.663845in}{3.626897in}}%
\pgfpathlineto{\pgfqpoint{3.668179in}{3.622931in}}%
\pgfpathlineto{\pgfqpoint{3.672513in}{3.619624in}}%
\pgfpathlineto{\pgfqpoint{3.681180in}{3.614001in}}%
\pgfpathlineto{\pgfqpoint{3.689847in}{3.608438in}}%
\pgfpathlineto{\pgfqpoint{3.694181in}{3.605288in}}%
\pgfpathlineto{\pgfqpoint{3.698514in}{3.601700in}}%
\pgfpathlineto{\pgfqpoint{3.702848in}{3.597523in}}%
\pgfpathlineto{\pgfqpoint{3.707182in}{3.592594in}}%
\pgfpathlineto{\pgfqpoint{3.711515in}{3.586724in}}%
\pgfpathlineto{\pgfqpoint{3.715849in}{3.579697in}}%
\pgfpathlineto{\pgfqpoint{3.720183in}{3.571259in}}%
\pgfpathlineto{\pgfqpoint{3.724516in}{3.561108in}}%
\pgfpathlineto{\pgfqpoint{3.728850in}{3.548894in}}%
\pgfpathlineto{\pgfqpoint{3.733184in}{3.534218in}}%
\pgfpathlineto{\pgfqpoint{3.737517in}{3.516633in}}%
\pgfpathlineto{\pgfqpoint{3.741851in}{3.495672in}}%
\pgfpathlineto{\pgfqpoint{3.746185in}{3.470900in}}%
\pgfpathlineto{\pgfqpoint{3.750518in}{3.442006in}}%
\pgfpathlineto{\pgfqpoint{3.754852in}{3.408972in}}%
\pgfpathlineto{\pgfqpoint{3.763519in}{3.333480in}}%
\pgfpathlineto{\pgfqpoint{3.767853in}{3.295057in}}%
\pgfpathlineto{\pgfqpoint{3.772187in}{3.261050in}}%
\pgfpathlineto{\pgfqpoint{3.776520in}{3.236457in}}%
\pgfpathlineto{\pgfqpoint{3.780854in}{3.226191in}}%
\pgfpathlineto{\pgfqpoint{3.785187in}{3.233451in}}%
\pgfpathlineto{\pgfqpoint{3.789521in}{3.258344in}}%
\pgfpathlineto{\pgfqpoint{3.793855in}{3.297686in}}%
\pgfpathlineto{\pgfqpoint{3.798188in}{3.346146in}}%
\pgfpathlineto{\pgfqpoint{3.806856in}{3.448499in}}%
\pgfpathlineto{\pgfqpoint{3.811189in}{3.494527in}}%
\pgfpathlineto{\pgfqpoint{3.815523in}{3.534501in}}%
\pgfpathlineto{\pgfqpoint{3.819857in}{3.568026in}}%
\pgfpathlineto{\pgfqpoint{3.824190in}{3.595467in}}%
\pgfpathlineto{\pgfqpoint{3.828524in}{3.617593in}}%
\pgfpathlineto{\pgfqpoint{3.832858in}{3.635341in}}%
\pgfpathlineto{\pgfqpoint{3.837191in}{3.649661in}}%
\pgfpathlineto{\pgfqpoint{3.841525in}{3.661440in}}%
\pgfpathlineto{\pgfqpoint{3.845859in}{3.671463in}}%
\pgfpathlineto{\pgfqpoint{3.854526in}{3.688848in}}%
\pgfpathlineto{\pgfqpoint{3.863193in}{3.706121in}}%
\pgfpathlineto{\pgfqpoint{3.867527in}{3.715750in}}%
\pgfpathlineto{\pgfqpoint{3.871861in}{3.726499in}}%
\pgfpathlineto{\pgfqpoint{3.876194in}{3.738672in}}%
\pgfpathlineto{\pgfqpoint{3.880528in}{3.752541in}}%
\pgfpathlineto{\pgfqpoint{3.884862in}{3.768334in}}%
\pgfpathlineto{\pgfqpoint{3.889195in}{3.786214in}}%
\pgfpathlineto{\pgfqpoint{3.893529in}{3.806227in}}%
\pgfpathlineto{\pgfqpoint{3.897862in}{3.828248in}}%
\pgfpathlineto{\pgfqpoint{3.915197in}{3.922774in}}%
\pgfpathlineto{\pgfqpoint{3.919531in}{3.940967in}}%
\pgfpathlineto{\pgfqpoint{3.923864in}{3.953023in}}%
\pgfpathlineto{\pgfqpoint{3.928198in}{3.957137in}}%
\pgfpathlineto{\pgfqpoint{3.932532in}{3.952296in}}%
\pgfpathlineto{\pgfqpoint{3.936865in}{3.938610in}}%
\pgfpathlineto{\pgfqpoint{3.941199in}{3.917345in}}%
\pgfpathlineto{\pgfqpoint{3.945533in}{3.890619in}}%
\pgfpathlineto{\pgfqpoint{3.958534in}{3.801555in}}%
\pgfpathlineto{\pgfqpoint{3.962867in}{3.775130in}}%
\pgfpathlineto{\pgfqpoint{3.967201in}{3.752066in}}%
\pgfpathlineto{\pgfqpoint{3.971535in}{3.732655in}}%
\pgfpathlineto{\pgfqpoint{3.975868in}{3.716861in}}%
\pgfpathlineto{\pgfqpoint{3.980202in}{3.704445in}}%
\pgfpathlineto{\pgfqpoint{3.984536in}{3.695069in}}%
\pgfpathlineto{\pgfqpoint{3.988869in}{3.688363in}}%
\pgfpathlineto{\pgfqpoint{3.993203in}{3.683980in}}%
\pgfpathlineto{\pgfqpoint{3.997536in}{3.681614in}}%
\pgfpathlineto{\pgfqpoint{4.001870in}{3.681026in}}%
\pgfpathlineto{\pgfqpoint{4.006204in}{3.682038in}}%
\pgfpathlineto{\pgfqpoint{4.010537in}{3.684542in}}%
\pgfpathlineto{\pgfqpoint{4.014871in}{3.688495in}}%
\pgfpathlineto{\pgfqpoint{4.019205in}{3.693912in}}%
\pgfpathlineto{\pgfqpoint{4.023538in}{3.700866in}}%
\pgfpathlineto{\pgfqpoint{4.027872in}{3.709484in}}%
\pgfpathlineto{\pgfqpoint{4.032206in}{3.719938in}}%
\pgfpathlineto{\pgfqpoint{4.036539in}{3.732436in}}%
\pgfpathlineto{\pgfqpoint{4.040873in}{3.747214in}}%
\pgfpathlineto{\pgfqpoint{4.045207in}{3.764503in}}%
\pgfpathlineto{\pgfqpoint{4.049540in}{3.784495in}}%
\pgfpathlineto{\pgfqpoint{4.053874in}{3.807274in}}%
\pgfpathlineto{\pgfqpoint{4.058208in}{3.832726in}}%
\pgfpathlineto{\pgfqpoint{4.066875in}{3.889359in}}%
\pgfpathlineto{\pgfqpoint{4.071209in}{3.918036in}}%
\pgfpathlineto{\pgfqpoint{4.075542in}{3.944174in}}%
\pgfpathlineto{\pgfqpoint{4.079876in}{3.964974in}}%
\pgfpathlineto{\pgfqpoint{4.084210in}{3.977525in}}%
\pgfpathlineto{\pgfqpoint{4.088543in}{3.979488in}}%
\pgfpathlineto{\pgfqpoint{4.092877in}{3.969827in}}%
\pgfpathlineto{\pgfqpoint{4.097210in}{3.949214in}}%
\pgfpathlineto{\pgfqpoint{4.101544in}{3.919900in}}%
\pgfpathlineto{\pgfqpoint{4.110211in}{3.848119in}}%
\pgfpathlineto{\pgfqpoint{4.114545in}{3.811858in}}%
\pgfpathlineto{\pgfqpoint{4.118879in}{3.778345in}}%
\pgfpathlineto{\pgfqpoint{4.123212in}{3.748775in}}%
\pgfpathlineto{\pgfqpoint{4.127546in}{3.723633in}}%
\pgfpathlineto{\pgfqpoint{4.131880in}{3.702900in}}%
\pgfpathlineto{\pgfqpoint{4.136213in}{3.686244in}}%
\pgfpathlineto{\pgfqpoint{4.140547in}{3.673170in}}%
\pgfpathlineto{\pgfqpoint{4.144881in}{3.663132in}}%
\pgfpathlineto{\pgfqpoint{4.149214in}{3.655598in}}%
\pgfpathlineto{\pgfqpoint{4.153548in}{3.650091in}}%
\pgfpathlineto{\pgfqpoint{4.157882in}{3.646202in}}%
\pgfpathlineto{\pgfqpoint{4.162215in}{3.643598in}}%
\pgfpathlineto{\pgfqpoint{4.166549in}{3.642017in}}%
\pgfpathlineto{\pgfqpoint{4.170883in}{3.641263in}}%
\pgfpathlineto{\pgfqpoint{4.175216in}{3.641194in}}%
\pgfpathlineto{\pgfqpoint{4.179550in}{3.641717in}}%
\pgfpathlineto{\pgfqpoint{4.183884in}{3.642781in}}%
\pgfpathlineto{\pgfqpoint{4.188217in}{3.644369in}}%
\pgfpathlineto{\pgfqpoint{4.192551in}{3.646497in}}%
\pgfpathlineto{\pgfqpoint{4.196884in}{3.649211in}}%
\pgfpathlineto{\pgfqpoint{4.201218in}{3.652586in}}%
\pgfpathlineto{\pgfqpoint{4.205552in}{3.656728in}}%
\pgfpathlineto{\pgfqpoint{4.209885in}{3.661777in}}%
\pgfpathlineto{\pgfqpoint{4.214219in}{3.667909in}}%
\pgfpathlineto{\pgfqpoint{4.218553in}{3.675344in}}%
\pgfpathlineto{\pgfqpoint{4.222886in}{3.684350in}}%
\pgfpathlineto{\pgfqpoint{4.227220in}{3.695249in}}%
\pgfpathlineto{\pgfqpoint{4.231554in}{3.708419in}}%
\pgfpathlineto{\pgfqpoint{4.235887in}{3.724299in}}%
\pgfpathlineto{\pgfqpoint{4.240221in}{3.743368in}}%
\pgfpathlineto{\pgfqpoint{4.244555in}{3.766121in}}%
\pgfpathlineto{\pgfqpoint{4.248888in}{3.792995in}}%
\pgfpathlineto{\pgfqpoint{4.253222in}{3.824250in}}%
\pgfpathlineto{\pgfqpoint{4.257556in}{3.859752in}}%
\pgfpathlineto{\pgfqpoint{4.270557in}{3.977651in}}%
\pgfpathlineto{\pgfqpoint{4.274890in}{4.009637in}}%
\pgfpathlineto{\pgfqpoint{4.279224in}{4.029559in}}%
\pgfpathlineto{\pgfqpoint{4.283558in}{4.032702in}}%
\pgfpathlineto{\pgfqpoint{4.287891in}{4.016923in}}%
\pgfpathlineto{\pgfqpoint{4.292225in}{3.983705in}}%
\pgfpathlineto{\pgfqpoint{4.296559in}{3.937683in}}%
\pgfpathlineto{\pgfqpoint{4.309559in}{3.780770in}}%
\pgfpathlineto{\pgfqpoint{4.313893in}{3.735989in}}%
\pgfpathlineto{\pgfqpoint{4.318227in}{3.697872in}}%
\pgfpathlineto{\pgfqpoint{4.322560in}{3.666362in}}%
\pgfpathlineto{\pgfqpoint{4.326894in}{3.640807in}}%
\pgfpathlineto{\pgfqpoint{4.331228in}{3.620272in}}%
\pgfpathlineto{\pgfqpoint{4.335561in}{3.603747in}}%
\pgfpathlineto{\pgfqpoint{4.339895in}{3.590259in}}%
\pgfpathlineto{\pgfqpoint{4.344229in}{3.578932in}}%
\pgfpathlineto{\pgfqpoint{4.348562in}{3.569003in}}%
\pgfpathlineto{\pgfqpoint{4.365897in}{3.531676in}}%
\pgfpathlineto{\pgfqpoint{4.370231in}{3.520738in}}%
\pgfpathlineto{\pgfqpoint{4.374564in}{3.508477in}}%
\pgfpathlineto{\pgfqpoint{4.378898in}{3.494634in}}%
\pgfpathlineto{\pgfqpoint{4.383232in}{3.479008in}}%
\pgfpathlineto{\pgfqpoint{4.387565in}{3.461480in}}%
\pgfpathlineto{\pgfqpoint{4.391899in}{3.442054in}}%
\pgfpathlineto{\pgfqpoint{4.400566in}{3.398524in}}%
\pgfpathlineto{\pgfqpoint{4.409233in}{3.353481in}}%
\pgfpathlineto{\pgfqpoint{4.413567in}{3.333574in}}%
\pgfpathlineto{\pgfqpoint{4.417901in}{3.317761in}}%
\pgfpathlineto{\pgfqpoint{4.422234in}{3.307877in}}%
\pgfpathlineto{\pgfqpoint{4.426568in}{3.305393in}}%
\pgfpathlineto{\pgfqpoint{4.430902in}{3.311035in}}%
\pgfpathlineto{\pgfqpoint{4.435235in}{3.324543in}}%
\pgfpathlineto{\pgfqpoint{4.439569in}{3.344694in}}%
\pgfpathlineto{\pgfqpoint{4.443903in}{3.369572in}}%
\pgfpathlineto{\pgfqpoint{4.456904in}{3.451526in}}%
\pgfpathlineto{\pgfqpoint{4.461237in}{3.475799in}}%
\pgfpathlineto{\pgfqpoint{4.465571in}{3.496993in}}%
\pgfpathlineto{\pgfqpoint{4.469905in}{3.514822in}}%
\pgfpathlineto{\pgfqpoint{4.474238in}{3.529288in}}%
\pgfpathlineto{\pgfqpoint{4.478572in}{3.540575in}}%
\pgfpathlineto{\pgfqpoint{4.482906in}{3.548961in}}%
\pgfpathlineto{\pgfqpoint{4.487239in}{3.554750in}}%
\pgfpathlineto{\pgfqpoint{4.491573in}{3.558237in}}%
\pgfpathlineto{\pgfqpoint{4.495907in}{3.559671in}}%
\pgfpathlineto{\pgfqpoint{4.500240in}{3.559249in}}%
\pgfpathlineto{\pgfqpoint{4.504574in}{3.557104in}}%
\pgfpathlineto{\pgfqpoint{4.508907in}{3.553309in}}%
\pgfpathlineto{\pgfqpoint{4.513241in}{3.547875in}}%
\pgfpathlineto{\pgfqpoint{4.517575in}{3.540758in}}%
\pgfpathlineto{\pgfqpoint{4.521908in}{3.531863in}}%
\pgfpathlineto{\pgfqpoint{4.526242in}{3.521056in}}%
\pgfpathlineto{\pgfqpoint{4.530576in}{3.508171in}}%
\pgfpathlineto{\pgfqpoint{4.534909in}{3.493037in}}%
\pgfpathlineto{\pgfqpoint{4.539243in}{3.475506in}}%
\pgfpathlineto{\pgfqpoint{4.543577in}{3.455501in}}%
\pgfpathlineto{\pgfqpoint{4.547910in}{3.433086in}}%
\pgfpathlineto{\pgfqpoint{4.556578in}{3.382595in}}%
\pgfpathlineto{\pgfqpoint{4.565245in}{3.331376in}}%
\pgfpathlineto{\pgfqpoint{4.569579in}{3.309983in}}%
\pgfpathlineto{\pgfqpoint{4.573912in}{3.294572in}}%
\pgfpathlineto{\pgfqpoint{4.578246in}{3.287403in}}%
\pgfpathlineto{\pgfqpoint{4.582580in}{3.289979in}}%
\pgfpathlineto{\pgfqpoint{4.586913in}{3.302561in}}%
\pgfpathlineto{\pgfqpoint{4.591247in}{3.324025in}}%
\pgfpathlineto{\pgfqpoint{4.595581in}{3.352130in}}%
\pgfpathlineto{\pgfqpoint{4.612915in}{3.478253in}}%
\pgfpathlineto{\pgfqpoint{4.617249in}{3.503833in}}%
\pgfpathlineto{\pgfqpoint{4.621582in}{3.525448in}}%
\pgfpathlineto{\pgfqpoint{4.625916in}{3.543140in}}%
\pgfpathlineto{\pgfqpoint{4.630250in}{3.557189in}}%
\pgfpathlineto{\pgfqpoint{4.634583in}{3.568000in}}%
\pgfpathlineto{\pgfqpoint{4.638917in}{3.576011in}}%
\pgfpathlineto{\pgfqpoint{4.643251in}{3.581642in}}%
\pgfpathlineto{\pgfqpoint{4.647584in}{3.585264in}}%
\pgfpathlineto{\pgfqpoint{4.651918in}{3.587184in}}%
\pgfpathlineto{\pgfqpoint{4.656252in}{3.587636in}}%
\pgfpathlineto{\pgfqpoint{4.660585in}{3.586787in}}%
\pgfpathlineto{\pgfqpoint{4.664919in}{3.584736in}}%
\pgfpathlineto{\pgfqpoint{4.669253in}{3.581523in}}%
\pgfpathlineto{\pgfqpoint{4.673586in}{3.577129in}}%
\pgfpathlineto{\pgfqpoint{4.677920in}{3.571483in}}%
\pgfpathlineto{\pgfqpoint{4.682254in}{3.564462in}}%
\pgfpathlineto{\pgfqpoint{4.686587in}{3.555893in}}%
\pgfpathlineto{\pgfqpoint{4.690921in}{3.545556in}}%
\pgfpathlineto{\pgfqpoint{4.695255in}{3.533186in}}%
\pgfpathlineto{\pgfqpoint{4.699588in}{3.518485in}}%
\pgfpathlineto{\pgfqpoint{4.703922in}{3.501134in}}%
\pgfpathlineto{\pgfqpoint{4.708256in}{3.480834in}}%
\pgfpathlineto{\pgfqpoint{4.712589in}{3.457359in}}%
\pgfpathlineto{\pgfqpoint{4.716923in}{3.430656in}}%
\pgfpathlineto{\pgfqpoint{4.721256in}{3.400991in}}%
\pgfpathlineto{\pgfqpoint{4.734257in}{3.305721in}}%
\pgfpathlineto{\pgfqpoint{4.738591in}{3.279747in}}%
\pgfpathlineto{\pgfqpoint{4.742925in}{3.262301in}}%
\pgfpathlineto{\pgfqpoint{4.747258in}{3.256625in}}%
\pgfpathlineto{\pgfqpoint{4.751592in}{3.264551in}}%
\pgfpathlineto{\pgfqpoint{4.755926in}{3.285780in}}%
\pgfpathlineto{\pgfqpoint{4.760259in}{3.317875in}}%
\pgfpathlineto{\pgfqpoint{4.764593in}{3.357013in}}%
\pgfpathlineto{\pgfqpoint{4.773260in}{3.440476in}}%
\pgfpathlineto{\pgfqpoint{4.777594in}{3.478748in}}%
\pgfpathlineto{\pgfqpoint{4.781928in}{3.512452in}}%
\pgfpathlineto{\pgfqpoint{4.786261in}{3.541056in}}%
\pgfpathlineto{\pgfqpoint{4.790595in}{3.564644in}}%
\pgfpathlineto{\pgfqpoint{4.794929in}{3.583674in}}%
\pgfpathlineto{\pgfqpoint{4.799262in}{3.598784in}}%
\pgfpathlineto{\pgfqpoint{4.803596in}{3.610666in}}%
\pgfpathlineto{\pgfqpoint{4.807930in}{3.619988in}}%
\pgfpathlineto{\pgfqpoint{4.812263in}{3.627350in}}%
\pgfpathlineto{\pgfqpoint{4.816597in}{3.633273in}}%
\pgfpathlineto{\pgfqpoint{4.820930in}{3.638199in}}%
\pgfpathlineto{\pgfqpoint{4.829598in}{3.646461in}}%
\pgfpathlineto{\pgfqpoint{4.838265in}{3.654428in}}%
\pgfpathlineto{\pgfqpoint{4.842599in}{3.658862in}}%
\pgfpathlineto{\pgfqpoint{4.846932in}{3.663867in}}%
\pgfpathlineto{\pgfqpoint{4.851266in}{3.669648in}}%
\pgfpathlineto{\pgfqpoint{4.855600in}{3.676423in}}%
\pgfpathlineto{\pgfqpoint{4.859933in}{3.684434in}}%
\pgfpathlineto{\pgfqpoint{4.864267in}{3.693954in}}%
\pgfpathlineto{\pgfqpoint{4.868601in}{3.705294in}}%
\pgfpathlineto{\pgfqpoint{4.872934in}{3.718801in}}%
\pgfpathlineto{\pgfqpoint{4.877268in}{3.734857in}}%
\pgfpathlineto{\pgfqpoint{4.881602in}{3.753859in}}%
\pgfpathlineto{\pgfqpoint{4.885935in}{3.776181in}}%
\pgfpathlineto{\pgfqpoint{4.890269in}{3.802101in}}%
\pgfpathlineto{\pgfqpoint{4.894603in}{3.831679in}}%
\pgfpathlineto{\pgfqpoint{4.898936in}{3.864568in}}%
\pgfpathlineto{\pgfqpoint{4.911937in}{3.968227in}}%
\pgfpathlineto{\pgfqpoint{4.916271in}{3.994449in}}%
\pgfpathlineto{\pgfqpoint{4.920605in}{4.009681in}}%
\pgfpathlineto{\pgfqpoint{4.924938in}{4.010472in}}%
\pgfpathlineto{\pgfqpoint{4.929272in}{3.995491in}}%
\pgfpathlineto{\pgfqpoint{4.933605in}{3.966161in}}%
\pgfpathlineto{\pgfqpoint{4.937939in}{3.926236in}}%
\pgfpathlineto{\pgfqpoint{4.950940in}{3.789445in}}%
\pgfpathlineto{\pgfqpoint{4.955274in}{3.749708in}}%
\pgfpathlineto{\pgfqpoint{4.959607in}{3.715571in}}%
\pgfpathlineto{\pgfqpoint{4.963941in}{3.687151in}}%
\pgfpathlineto{\pgfqpoint{4.968275in}{3.664018in}}%
\pgfpathlineto{\pgfqpoint{4.972608in}{3.645461in}}%
\pgfpathlineto{\pgfqpoint{4.976942in}{3.630664in}}%
\pgfpathlineto{\pgfqpoint{4.981276in}{3.618821in}}%
\pgfpathlineto{\pgfqpoint{4.985609in}{3.609187in}}%
\pgfpathlineto{\pgfqpoint{4.989943in}{3.601107in}}%
\pgfpathlineto{\pgfqpoint{4.998610in}{3.587446in}}%
\pgfpathlineto{\pgfqpoint{5.007278in}{3.574250in}}%
\pgfpathlineto{\pgfqpoint{5.011611in}{3.566948in}}%
\pgfpathlineto{\pgfqpoint{5.015945in}{3.558765in}}%
\pgfpathlineto{\pgfqpoint{5.020279in}{3.549406in}}%
\pgfpathlineto{\pgfqpoint{5.024612in}{3.538574in}}%
\pgfpathlineto{\pgfqpoint{5.028946in}{3.525968in}}%
\pgfpathlineto{\pgfqpoint{5.033279in}{3.511284in}}%
\pgfpathlineto{\pgfqpoint{5.037613in}{3.494232in}}%
\pgfpathlineto{\pgfqpoint{5.041947in}{3.474565in}}%
\pgfpathlineto{\pgfqpoint{5.046280in}{3.452136in}}%
\pgfpathlineto{\pgfqpoint{5.050614in}{3.426984in}}%
\pgfpathlineto{\pgfqpoint{5.059281in}{3.370367in}}%
\pgfpathlineto{\pgfqpoint{5.063615in}{3.341142in}}%
\pgfpathlineto{\pgfqpoint{5.067949in}{3.313917in}}%
\pgfpathlineto{\pgfqpoint{5.072282in}{3.291441in}}%
\pgfpathlineto{\pgfqpoint{5.076616in}{3.276705in}}%
\pgfpathlineto{\pgfqpoint{5.080950in}{3.272283in}}%
\pgfpathlineto{\pgfqpoint{5.085283in}{3.279570in}}%
\pgfpathlineto{\pgfqpoint{5.089617in}{3.298256in}}%
\pgfpathlineto{\pgfqpoint{5.093951in}{3.326345in}}%
\pgfpathlineto{\pgfqpoint{5.098284in}{3.360709in}}%
\pgfpathlineto{\pgfqpoint{5.106952in}{3.434908in}}%
\pgfpathlineto{\pgfqpoint{5.111285in}{3.469453in}}%
\pgfpathlineto{\pgfqpoint{5.115619in}{3.500177in}}%
\pgfpathlineto{\pgfqpoint{5.119953in}{3.526465in}}%
\pgfpathlineto{\pgfqpoint{5.124286in}{3.548262in}}%
\pgfpathlineto{\pgfqpoint{5.128620in}{3.565863in}}%
\pgfpathlineto{\pgfqpoint{5.132953in}{3.579751in}}%
\pgfpathlineto{\pgfqpoint{5.137287in}{3.590481in}}%
\pgfpathlineto{\pgfqpoint{5.141621in}{3.598602in}}%
\pgfpathlineto{\pgfqpoint{5.145954in}{3.604614in}}%
\pgfpathlineto{\pgfqpoint{5.150288in}{3.608950in}}%
\pgfpathlineto{\pgfqpoint{5.154622in}{3.611966in}}%
\pgfpathlineto{\pgfqpoint{5.158955in}{3.613944in}}%
\pgfpathlineto{\pgfqpoint{5.163289in}{3.615103in}}%
\pgfpathlineto{\pgfqpoint{5.167623in}{3.615604in}}%
\pgfpathlineto{\pgfqpoint{5.171956in}{3.615559in}}%
\pgfpathlineto{\pgfqpoint{5.176290in}{3.615040in}}%
\pgfpathlineto{\pgfqpoint{5.180624in}{3.614085in}}%
\pgfpathlineto{\pgfqpoint{5.184957in}{3.612703in}}%
\pgfpathlineto{\pgfqpoint{5.189291in}{3.610875in}}%
\pgfpathlineto{\pgfqpoint{5.193625in}{3.608558in}}%
\pgfpathlineto{\pgfqpoint{5.193625in}{3.608558in}}%
\pgfusepath{stroke}%
\end{pgfscope}%
\begin{pgfscope}%
\pgfpathrectangle{\pgfqpoint{0.647840in}{3.185866in}}{\pgfqpoint{4.762251in}{0.887161in}}%
\pgfusepath{clip}%
\pgfsetrectcap%
\pgfsetroundjoin%
\pgfsetlinewidth{1.505625pt}%
\definecolor{currentstroke}{rgb}{0.203922,0.541176,0.741176}%
\pgfsetstrokecolor{currentstroke}%
\pgfsetdash{}{0pt}%
\pgfpathmoveto{\pgfqpoint{3.031132in}{3.666859in}}%
\pgfpathlineto{\pgfqpoint{3.035466in}{3.672800in}}%
\pgfpathlineto{\pgfqpoint{3.039799in}{3.680030in}}%
\pgfpathlineto{\pgfqpoint{3.044133in}{3.688901in}}%
\pgfpathlineto{\pgfqpoint{3.048467in}{3.699746in}}%
\pgfpathlineto{\pgfqpoint{3.052800in}{3.712528in}}%
\pgfpathlineto{\pgfqpoint{3.057134in}{3.727878in}}%
\pgfpathlineto{\pgfqpoint{3.061467in}{3.746210in}}%
\pgfpathlineto{\pgfqpoint{3.065801in}{3.767679in}}%
\pgfpathlineto{\pgfqpoint{3.070135in}{3.792599in}}%
\pgfpathlineto{\pgfqpoint{3.074468in}{3.821464in}}%
\pgfpathlineto{\pgfqpoint{3.078802in}{3.853849in}}%
\pgfpathlineto{\pgfqpoint{3.091803in}{3.962108in}}%
\pgfpathlineto{\pgfqpoint{3.096137in}{3.992542in}}%
\pgfpathlineto{\pgfqpoint{3.100470in}{4.013222in}}%
\pgfpathlineto{\pgfqpoint{3.104804in}{4.019891in}}%
\pgfpathlineto{\pgfqpoint{3.109138in}{4.010265in}}%
\pgfpathlineto{\pgfqpoint{3.113471in}{3.984487in}}%
\pgfpathlineto{\pgfqpoint{3.117805in}{3.945186in}}%
\pgfpathlineto{\pgfqpoint{3.122139in}{3.898286in}}%
\pgfpathlineto{\pgfqpoint{3.130806in}{3.800401in}}%
\pgfpathlineto{\pgfqpoint{3.135140in}{3.756782in}}%
\pgfpathlineto{\pgfqpoint{3.139473in}{3.719213in}}%
\pgfpathlineto{\pgfqpoint{3.143807in}{3.687491in}}%
\pgfpathlineto{\pgfqpoint{3.148141in}{3.661474in}}%
\pgfpathlineto{\pgfqpoint{3.152474in}{3.640399in}}%
\pgfpathlineto{\pgfqpoint{3.156808in}{3.623626in}}%
\pgfpathlineto{\pgfqpoint{3.161141in}{3.610073in}}%
\pgfpathlineto{\pgfqpoint{3.165475in}{3.599015in}}%
\pgfpathlineto{\pgfqpoint{3.169809in}{3.589518in}}%
\pgfpathlineto{\pgfqpoint{3.178476in}{3.573191in}}%
\pgfpathlineto{\pgfqpoint{3.187143in}{3.557582in}}%
\pgfpathlineto{\pgfqpoint{3.191477in}{3.548727in}}%
\pgfpathlineto{\pgfqpoint{3.195811in}{3.538925in}}%
\pgfpathlineto{\pgfqpoint{3.200144in}{3.527606in}}%
\pgfpathlineto{\pgfqpoint{3.204478in}{3.514752in}}%
\pgfpathlineto{\pgfqpoint{3.208812in}{3.499953in}}%
\pgfpathlineto{\pgfqpoint{3.213145in}{3.482852in}}%
\pgfpathlineto{\pgfqpoint{3.217479in}{3.463594in}}%
\pgfpathlineto{\pgfqpoint{3.221813in}{3.442009in}}%
\pgfpathlineto{\pgfqpoint{3.230480in}{3.392748in}}%
\pgfpathlineto{\pgfqpoint{3.239147in}{3.341078in}}%
\pgfpathlineto{\pgfqpoint{3.243481in}{3.318436in}}%
\pgfpathlineto{\pgfqpoint{3.247815in}{3.300731in}}%
\pgfpathlineto{\pgfqpoint{3.252148in}{3.290279in}}%
\pgfpathlineto{\pgfqpoint{3.256482in}{3.289012in}}%
\pgfpathlineto{\pgfqpoint{3.260816in}{3.297692in}}%
\pgfpathlineto{\pgfqpoint{3.265149in}{3.315780in}}%
\pgfpathlineto{\pgfqpoint{3.269483in}{3.341159in}}%
\pgfpathlineto{\pgfqpoint{3.273816in}{3.371657in}}%
\pgfpathlineto{\pgfqpoint{3.282484in}{3.436421in}}%
\pgfpathlineto{\pgfqpoint{3.286817in}{3.466455in}}%
\pgfpathlineto{\pgfqpoint{3.291151in}{3.493245in}}%
\pgfpathlineto{\pgfqpoint{3.295485in}{3.516242in}}%
\pgfpathlineto{\pgfqpoint{3.299818in}{3.535368in}}%
\pgfpathlineto{\pgfqpoint{3.304152in}{3.550690in}}%
\pgfpathlineto{\pgfqpoint{3.308486in}{3.562564in}}%
\pgfpathlineto{\pgfqpoint{3.312819in}{3.571303in}}%
\pgfpathlineto{\pgfqpoint{3.317153in}{3.577757in}}%
\pgfpathlineto{\pgfqpoint{3.321487in}{3.581992in}}%
\pgfpathlineto{\pgfqpoint{3.325820in}{3.584488in}}%
\pgfpathlineto{\pgfqpoint{3.330154in}{3.585375in}}%
\pgfpathlineto{\pgfqpoint{3.334488in}{3.584921in}}%
\pgfpathlineto{\pgfqpoint{3.338821in}{3.583125in}}%
\pgfpathlineto{\pgfqpoint{3.343155in}{3.580167in}}%
\pgfpathlineto{\pgfqpoint{3.347489in}{3.576053in}}%
\pgfpathlineto{\pgfqpoint{3.351822in}{3.570503in}}%
\pgfpathlineto{\pgfqpoint{3.356156in}{3.563809in}}%
\pgfpathlineto{\pgfqpoint{3.360490in}{3.555258in}}%
\pgfpathlineto{\pgfqpoint{3.364823in}{3.545062in}}%
\pgfpathlineto{\pgfqpoint{3.369157in}{3.532901in}}%
\pgfpathlineto{\pgfqpoint{3.373490in}{3.518444in}}%
\pgfpathlineto{\pgfqpoint{3.377824in}{3.501589in}}%
\pgfpathlineto{\pgfqpoint{3.382158in}{3.481852in}}%
\pgfpathlineto{\pgfqpoint{3.386491in}{3.459047in}}%
\pgfpathlineto{\pgfqpoint{3.390825in}{3.432827in}}%
\pgfpathlineto{\pgfqpoint{3.395159in}{3.403628in}}%
\pgfpathlineto{\pgfqpoint{3.408160in}{3.309310in}}%
\pgfpathlineto{\pgfqpoint{3.412493in}{3.282985in}}%
\pgfpathlineto{\pgfqpoint{3.416827in}{3.264788in}}%
\pgfpathlineto{\pgfqpoint{3.421161in}{3.258069in}}%
\pgfpathlineto{\pgfqpoint{3.425494in}{3.264855in}}%
\pgfpathlineto{\pgfqpoint{3.429828in}{3.284734in}}%
\pgfpathlineto{\pgfqpoint{3.434162in}{3.315669in}}%
\pgfpathlineto{\pgfqpoint{3.438495in}{3.353841in}}%
\pgfpathlineto{\pgfqpoint{3.447163in}{3.436365in}}%
\pgfpathlineto{\pgfqpoint{3.451496in}{3.474484in}}%
\pgfpathlineto{\pgfqpoint{3.455830in}{3.508108in}}%
\pgfpathlineto{\pgfqpoint{3.460164in}{3.536770in}}%
\pgfpathlineto{\pgfqpoint{3.464497in}{3.560528in}}%
\pgfpathlineto{\pgfqpoint{3.468831in}{3.579820in}}%
\pgfpathlineto{\pgfqpoint{3.473164in}{3.595019in}}%
\pgfpathlineto{\pgfqpoint{3.477498in}{3.606986in}}%
\pgfpathlineto{\pgfqpoint{3.481832in}{3.616424in}}%
\pgfpathlineto{\pgfqpoint{3.486165in}{3.623742in}}%
\pgfpathlineto{\pgfqpoint{3.490499in}{3.629669in}}%
\pgfpathlineto{\pgfqpoint{3.494833in}{3.634587in}}%
\pgfpathlineto{\pgfqpoint{3.499166in}{3.638648in}}%
\pgfpathlineto{\pgfqpoint{3.516501in}{3.652948in}}%
\pgfpathlineto{\pgfqpoint{3.520835in}{3.657173in}}%
\pgfpathlineto{\pgfqpoint{3.525168in}{3.662078in}}%
\pgfpathlineto{\pgfqpoint{3.529502in}{3.667789in}}%
\pgfpathlineto{\pgfqpoint{3.533836in}{3.674399in}}%
\pgfpathlineto{\pgfqpoint{3.538169in}{3.682275in}}%
\pgfpathlineto{\pgfqpoint{3.542503in}{3.691597in}}%
\pgfpathlineto{\pgfqpoint{3.546837in}{3.702707in}}%
\pgfpathlineto{\pgfqpoint{3.551170in}{3.716039in}}%
\pgfpathlineto{\pgfqpoint{3.555504in}{3.731954in}}%
\pgfpathlineto{\pgfqpoint{3.559838in}{3.750644in}}%
\pgfpathlineto{\pgfqpoint{3.564171in}{3.772799in}}%
\pgfpathlineto{\pgfqpoint{3.568505in}{3.798122in}}%
\pgfpathlineto{\pgfqpoint{3.572839in}{3.826795in}}%
\pgfpathlineto{\pgfqpoint{3.577172in}{3.858925in}}%
\pgfpathlineto{\pgfqpoint{3.590173in}{3.965392in}}%
\pgfpathlineto{\pgfqpoint{3.594507in}{3.994402in}}%
\pgfpathlineto{\pgfqpoint{3.598840in}{4.013604in}}%
\pgfpathlineto{\pgfqpoint{3.603174in}{4.018674in}}%
\pgfpathlineto{\pgfqpoint{3.607508in}{4.007299in}}%
\pgfpathlineto{\pgfqpoint{3.611841in}{3.980352in}}%
\pgfpathlineto{\pgfqpoint{3.616175in}{3.940612in}}%
\pgfpathlineto{\pgfqpoint{3.620509in}{3.893977in}}%
\pgfpathlineto{\pgfqpoint{3.629176in}{3.796500in}}%
\pgfpathlineto{\pgfqpoint{3.633510in}{3.753328in}}%
\pgfpathlineto{\pgfqpoint{3.637843in}{3.715977in}}%
\pgfpathlineto{\pgfqpoint{3.642177in}{3.684641in}}%
\pgfpathlineto{\pgfqpoint{3.646511in}{3.659001in}}%
\pgfpathlineto{\pgfqpoint{3.650844in}{3.638473in}}%
\pgfpathlineto{\pgfqpoint{3.655178in}{3.622022in}}%
\pgfpathlineto{\pgfqpoint{3.659512in}{3.608866in}}%
\pgfpathlineto{\pgfqpoint{3.663845in}{3.597919in}}%
\pgfpathlineto{\pgfqpoint{3.668179in}{3.588774in}}%
\pgfpathlineto{\pgfqpoint{3.676846in}{3.572547in}}%
\pgfpathlineto{\pgfqpoint{3.685513in}{3.556941in}}%
\pgfpathlineto{\pgfqpoint{3.689847in}{3.547953in}}%
\pgfpathlineto{\pgfqpoint{3.694181in}{3.537887in}}%
\pgfpathlineto{\pgfqpoint{3.698514in}{3.526506in}}%
\pgfpathlineto{\pgfqpoint{3.702848in}{3.513493in}}%
\pgfpathlineto{\pgfqpoint{3.707182in}{3.498523in}}%
\pgfpathlineto{\pgfqpoint{3.711515in}{3.481277in}}%
\pgfpathlineto{\pgfqpoint{3.715849in}{3.461682in}}%
\pgfpathlineto{\pgfqpoint{3.720183in}{3.439812in}}%
\pgfpathlineto{\pgfqpoint{3.728850in}{3.390342in}}%
\pgfpathlineto{\pgfqpoint{3.737517in}{3.338711in}}%
\pgfpathlineto{\pgfqpoint{3.741851in}{3.316297in}}%
\pgfpathlineto{\pgfqpoint{3.746185in}{3.298978in}}%
\pgfpathlineto{\pgfqpoint{3.750518in}{3.289352in}}%
\pgfpathlineto{\pgfqpoint{3.754852in}{3.289138in}}%
\pgfpathlineto{\pgfqpoint{3.759186in}{3.298577in}}%
\pgfpathlineto{\pgfqpoint{3.763519in}{3.317351in}}%
\pgfpathlineto{\pgfqpoint{3.767853in}{3.343252in}}%
\pgfpathlineto{\pgfqpoint{3.776520in}{3.406708in}}%
\pgfpathlineto{\pgfqpoint{3.780854in}{3.438816in}}%
\pgfpathlineto{\pgfqpoint{3.785187in}{3.468543in}}%
\pgfpathlineto{\pgfqpoint{3.789521in}{3.494851in}}%
\pgfpathlineto{\pgfqpoint{3.793855in}{3.517358in}}%
\pgfpathlineto{\pgfqpoint{3.798188in}{3.536072in}}%
\pgfpathlineto{\pgfqpoint{3.802522in}{3.551122in}}%
\pgfpathlineto{\pgfqpoint{3.806856in}{3.562713in}}%
\pgfpathlineto{\pgfqpoint{3.811189in}{3.571335in}}%
\pgfpathlineto{\pgfqpoint{3.815523in}{3.577295in}}%
\pgfpathlineto{\pgfqpoint{3.819857in}{3.581301in}}%
\pgfpathlineto{\pgfqpoint{3.824190in}{3.583584in}}%
\pgfpathlineto{\pgfqpoint{3.828524in}{3.584330in}}%
\pgfpathlineto{\pgfqpoint{3.832858in}{3.583710in}}%
\pgfpathlineto{\pgfqpoint{3.837191in}{3.581765in}}%
\pgfpathlineto{\pgfqpoint{3.841525in}{3.578542in}}%
\pgfpathlineto{\pgfqpoint{3.845859in}{3.574033in}}%
\pgfpathlineto{\pgfqpoint{3.850192in}{3.568206in}}%
\pgfpathlineto{\pgfqpoint{3.854526in}{3.560885in}}%
\pgfpathlineto{\pgfqpoint{3.858860in}{3.552038in}}%
\pgfpathlineto{\pgfqpoint{3.863193in}{3.541486in}}%
\pgfpathlineto{\pgfqpoint{3.867527in}{3.528741in}}%
\pgfpathlineto{\pgfqpoint{3.871861in}{3.513803in}}%
\pgfpathlineto{\pgfqpoint{3.876194in}{3.496028in}}%
\pgfpathlineto{\pgfqpoint{3.880528in}{3.475375in}}%
\pgfpathlineto{\pgfqpoint{3.884862in}{3.451689in}}%
\pgfpathlineto{\pgfqpoint{3.889195in}{3.424846in}}%
\pgfpathlineto{\pgfqpoint{3.897862in}{3.363756in}}%
\pgfpathlineto{\pgfqpoint{3.902196in}{3.332166in}}%
\pgfpathlineto{\pgfqpoint{3.906530in}{3.302739in}}%
\pgfpathlineto{\pgfqpoint{3.910863in}{3.278906in}}%
\pgfpathlineto{\pgfqpoint{3.915197in}{3.263877in}}%
\pgfpathlineto{\pgfqpoint{3.919531in}{3.260638in}}%
\pgfpathlineto{\pgfqpoint{3.923864in}{3.270495in}}%
\pgfpathlineto{\pgfqpoint{3.928198in}{3.292702in}}%
\pgfpathlineto{\pgfqpoint{3.932532in}{3.324901in}}%
\pgfpathlineto{\pgfqpoint{3.941199in}{3.404645in}}%
\pgfpathlineto{\pgfqpoint{3.945533in}{3.444779in}}%
\pgfpathlineto{\pgfqpoint{3.949866in}{3.482021in}}%
\pgfpathlineto{\pgfqpoint{3.954200in}{3.514529in}}%
\pgfpathlineto{\pgfqpoint{3.958534in}{3.541829in}}%
\pgfpathlineto{\pgfqpoint{3.962867in}{3.564499in}}%
\pgfpathlineto{\pgfqpoint{3.967201in}{3.582731in}}%
\pgfpathlineto{\pgfqpoint{3.971535in}{3.597187in}}%
\pgfpathlineto{\pgfqpoint{3.975868in}{3.608456in}}%
\pgfpathlineto{\pgfqpoint{3.980202in}{3.617186in}}%
\pgfpathlineto{\pgfqpoint{3.984536in}{3.624028in}}%
\pgfpathlineto{\pgfqpoint{3.988869in}{3.629549in}}%
\pgfpathlineto{\pgfqpoint{3.993203in}{3.634042in}}%
\pgfpathlineto{\pgfqpoint{3.997536in}{3.638044in}}%
\pgfpathlineto{\pgfqpoint{4.006204in}{3.644931in}}%
\pgfpathlineto{\pgfqpoint{4.014871in}{3.651814in}}%
\pgfpathlineto{\pgfqpoint{4.019205in}{3.655932in}}%
\pgfpathlineto{\pgfqpoint{4.023538in}{3.660816in}}%
\pgfpathlineto{\pgfqpoint{4.027872in}{3.666365in}}%
\pgfpathlineto{\pgfqpoint{4.032206in}{3.672924in}}%
\pgfpathlineto{\pgfqpoint{4.036539in}{3.680589in}}%
\pgfpathlineto{\pgfqpoint{4.040873in}{3.689949in}}%
\pgfpathlineto{\pgfqpoint{4.045207in}{3.701192in}}%
\pgfpathlineto{\pgfqpoint{4.049540in}{3.714438in}}%
\pgfpathlineto{\pgfqpoint{4.053874in}{3.730082in}}%
\pgfpathlineto{\pgfqpoint{4.058208in}{3.748977in}}%
\pgfpathlineto{\pgfqpoint{4.062541in}{3.771216in}}%
\pgfpathlineto{\pgfqpoint{4.066875in}{3.797460in}}%
\pgfpathlineto{\pgfqpoint{4.071209in}{3.828047in}}%
\pgfpathlineto{\pgfqpoint{4.075542in}{3.862289in}}%
\pgfpathlineto{\pgfqpoint{4.088543in}{3.971453in}}%
\pgfpathlineto{\pgfqpoint{4.092877in}{3.999726in}}%
\pgfpathlineto{\pgfqpoint{4.097210in}{4.016159in}}%
\pgfpathlineto{\pgfqpoint{4.101544in}{4.017232in}}%
\pgfpathlineto{\pgfqpoint{4.105878in}{4.001352in}}%
\pgfpathlineto{\pgfqpoint{4.110211in}{3.970293in}}%
\pgfpathlineto{\pgfqpoint{4.114545in}{3.928405in}}%
\pgfpathlineto{\pgfqpoint{4.127546in}{3.785878in}}%
\pgfpathlineto{\pgfqpoint{4.131880in}{3.744945in}}%
\pgfpathlineto{\pgfqpoint{4.136213in}{3.709781in}}%
\pgfpathlineto{\pgfqpoint{4.140547in}{3.680549in}}%
\pgfpathlineto{\pgfqpoint{4.144881in}{3.656650in}}%
\pgfpathlineto{\pgfqpoint{4.149214in}{3.637632in}}%
\pgfpathlineto{\pgfqpoint{4.153548in}{3.622359in}}%
\pgfpathlineto{\pgfqpoint{4.157882in}{3.610193in}}%
\pgfpathlineto{\pgfqpoint{4.162215in}{3.600001in}}%
\pgfpathlineto{\pgfqpoint{4.166549in}{3.591321in}}%
\pgfpathlineto{\pgfqpoint{4.183884in}{3.561392in}}%
\pgfpathlineto{\pgfqpoint{4.188217in}{3.552716in}}%
\pgfpathlineto{\pgfqpoint{4.192551in}{3.543124in}}%
\pgfpathlineto{\pgfqpoint{4.196884in}{3.532147in}}%
\pgfpathlineto{\pgfqpoint{4.201218in}{3.519365in}}%
\pgfpathlineto{\pgfqpoint{4.205552in}{3.504653in}}%
\pgfpathlineto{\pgfqpoint{4.209885in}{3.487572in}}%
\pgfpathlineto{\pgfqpoint{4.214219in}{3.468367in}}%
\pgfpathlineto{\pgfqpoint{4.218553in}{3.446268in}}%
\pgfpathlineto{\pgfqpoint{4.222886in}{3.422139in}}%
\pgfpathlineto{\pgfqpoint{4.231554in}{3.368897in}}%
\pgfpathlineto{\pgfqpoint{4.235887in}{3.342233in}}%
\pgfpathlineto{\pgfqpoint{4.240221in}{3.317981in}}%
\pgfpathlineto{\pgfqpoint{4.244555in}{3.298549in}}%
\pgfpathlineto{\pgfqpoint{4.248888in}{3.286587in}}%
\pgfpathlineto{\pgfqpoint{4.253222in}{3.284160in}}%
\pgfpathlineto{\pgfqpoint{4.257556in}{3.292337in}}%
\pgfpathlineto{\pgfqpoint{4.261889in}{3.310396in}}%
\pgfpathlineto{\pgfqpoint{4.266223in}{3.336181in}}%
\pgfpathlineto{\pgfqpoint{4.270557in}{3.367401in}}%
\pgfpathlineto{\pgfqpoint{4.279224in}{3.434090in}}%
\pgfpathlineto{\pgfqpoint{4.283558in}{3.465286in}}%
\pgfpathlineto{\pgfqpoint{4.287891in}{3.493223in}}%
\pgfpathlineto{\pgfqpoint{4.292225in}{3.517087in}}%
\pgfpathlineto{\pgfqpoint{4.296559in}{3.537152in}}%
\pgfpathlineto{\pgfqpoint{4.300892in}{3.553225in}}%
\pgfpathlineto{\pgfqpoint{4.305226in}{3.565694in}}%
\pgfpathlineto{\pgfqpoint{4.309559in}{3.575269in}}%
\pgfpathlineto{\pgfqpoint{4.313893in}{3.582399in}}%
\pgfpathlineto{\pgfqpoint{4.318227in}{3.587235in}}%
\pgfpathlineto{\pgfqpoint{4.322560in}{3.590212in}}%
\pgfpathlineto{\pgfqpoint{4.326894in}{3.591747in}}%
\pgfpathlineto{\pgfqpoint{4.331228in}{3.591849in}}%
\pgfpathlineto{\pgfqpoint{4.335561in}{3.590737in}}%
\pgfpathlineto{\pgfqpoint{4.339895in}{3.588494in}}%
\pgfpathlineto{\pgfqpoint{4.344229in}{3.585292in}}%
\pgfpathlineto{\pgfqpoint{4.348562in}{3.580926in}}%
\pgfpathlineto{\pgfqpoint{4.352896in}{3.575384in}}%
\pgfpathlineto{\pgfqpoint{4.357230in}{3.568517in}}%
\pgfpathlineto{\pgfqpoint{4.361563in}{3.560178in}}%
\pgfpathlineto{\pgfqpoint{4.365897in}{3.550177in}}%
\pgfpathlineto{\pgfqpoint{4.370231in}{3.538214in}}%
\pgfpathlineto{\pgfqpoint{4.374564in}{3.523932in}}%
\pgfpathlineto{\pgfqpoint{4.378898in}{3.506833in}}%
\pgfpathlineto{\pgfqpoint{4.383232in}{3.486983in}}%
\pgfpathlineto{\pgfqpoint{4.387565in}{3.463761in}}%
\pgfpathlineto{\pgfqpoint{4.391899in}{3.437226in}}%
\pgfpathlineto{\pgfqpoint{4.396233in}{3.407478in}}%
\pgfpathlineto{\pgfqpoint{4.409233in}{3.309181in}}%
\pgfpathlineto{\pgfqpoint{4.413567in}{3.280830in}}%
\pgfpathlineto{\pgfqpoint{4.417901in}{3.260365in}}%
\pgfpathlineto{\pgfqpoint{4.422234in}{3.251527in}}%
\pgfpathlineto{\pgfqpoint{4.426568in}{3.256671in}}%
\pgfpathlineto{\pgfqpoint{4.430902in}{3.275870in}}%
\pgfpathlineto{\pgfqpoint{4.435235in}{3.306801in}}%
\pgfpathlineto{\pgfqpoint{4.439569in}{3.346197in}}%
\pgfpathlineto{\pgfqpoint{4.452570in}{3.473570in}}%
\pgfpathlineto{\pgfqpoint{4.456904in}{3.509189in}}%
\pgfpathlineto{\pgfqpoint{4.461237in}{3.539582in}}%
\pgfpathlineto{\pgfqpoint{4.465571in}{3.564774in}}%
\pgfpathlineto{\pgfqpoint{4.469905in}{3.585160in}}%
\pgfpathlineto{\pgfqpoint{4.474238in}{3.601429in}}%
\pgfpathlineto{\pgfqpoint{4.478572in}{3.614392in}}%
\pgfpathlineto{\pgfqpoint{4.482906in}{3.624755in}}%
\pgfpathlineto{\pgfqpoint{4.487239in}{3.632856in}}%
\pgfpathlineto{\pgfqpoint{4.491573in}{3.639552in}}%
\pgfpathlineto{\pgfqpoint{4.495907in}{3.645078in}}%
\pgfpathlineto{\pgfqpoint{4.504574in}{3.654890in}}%
\pgfpathlineto{\pgfqpoint{4.508907in}{3.659696in}}%
\pgfpathlineto{\pgfqpoint{4.513241in}{3.664978in}}%
\pgfpathlineto{\pgfqpoint{4.517575in}{3.670912in}}%
\pgfpathlineto{\pgfqpoint{4.521908in}{3.677631in}}%
\pgfpathlineto{\pgfqpoint{4.526242in}{3.685422in}}%
\pgfpathlineto{\pgfqpoint{4.530576in}{3.694423in}}%
\pgfpathlineto{\pgfqpoint{4.534909in}{3.705091in}}%
\pgfpathlineto{\pgfqpoint{4.539243in}{3.717741in}}%
\pgfpathlineto{\pgfqpoint{4.543577in}{3.732715in}}%
\pgfpathlineto{\pgfqpoint{4.547910in}{3.750392in}}%
\pgfpathlineto{\pgfqpoint{4.552244in}{3.771008in}}%
\pgfpathlineto{\pgfqpoint{4.556578in}{3.794378in}}%
\pgfpathlineto{\pgfqpoint{4.560911in}{3.821485in}}%
\pgfpathlineto{\pgfqpoint{4.565245in}{3.852333in}}%
\pgfpathlineto{\pgfqpoint{4.578246in}{3.951422in}}%
\pgfpathlineto{\pgfqpoint{4.582580in}{3.978396in}}%
\pgfpathlineto{\pgfqpoint{4.586913in}{3.996104in}}%
\pgfpathlineto{\pgfqpoint{4.591247in}{4.001737in}}%
\pgfpathlineto{\pgfqpoint{4.595581in}{3.993247in}}%
\pgfpathlineto{\pgfqpoint{4.599914in}{3.970628in}}%
\pgfpathlineto{\pgfqpoint{4.604248in}{3.936171in}}%
\pgfpathlineto{\pgfqpoint{4.612915in}{3.851009in}}%
\pgfpathlineto{\pgfqpoint{4.617249in}{3.808166in}}%
\pgfpathlineto{\pgfqpoint{4.621582in}{3.768868in}}%
\pgfpathlineto{\pgfqpoint{4.625916in}{3.734638in}}%
\pgfpathlineto{\pgfqpoint{4.630250in}{3.705546in}}%
\pgfpathlineto{\pgfqpoint{4.634583in}{3.681670in}}%
\pgfpathlineto{\pgfqpoint{4.638917in}{3.662451in}}%
\pgfpathlineto{\pgfqpoint{4.643251in}{3.647256in}}%
\pgfpathlineto{\pgfqpoint{4.647584in}{3.634908in}}%
\pgfpathlineto{\pgfqpoint{4.651918in}{3.625290in}}%
\pgfpathlineto{\pgfqpoint{4.656252in}{3.617389in}}%
\pgfpathlineto{\pgfqpoint{4.660585in}{3.610762in}}%
\pgfpathlineto{\pgfqpoint{4.669253in}{3.599991in}}%
\pgfpathlineto{\pgfqpoint{4.677920in}{3.589629in}}%
\pgfpathlineto{\pgfqpoint{4.686587in}{3.577581in}}%
\pgfpathlineto{\pgfqpoint{4.690921in}{3.570248in}}%
\pgfpathlineto{\pgfqpoint{4.695255in}{3.561761in}}%
\pgfpathlineto{\pgfqpoint{4.699588in}{3.551927in}}%
\pgfpathlineto{\pgfqpoint{4.703922in}{3.540173in}}%
\pgfpathlineto{\pgfqpoint{4.708256in}{3.526342in}}%
\pgfpathlineto{\pgfqpoint{4.712589in}{3.510077in}}%
\pgfpathlineto{\pgfqpoint{4.716923in}{3.491096in}}%
\pgfpathlineto{\pgfqpoint{4.721256in}{3.468992in}}%
\pgfpathlineto{\pgfqpoint{4.725590in}{3.443560in}}%
\pgfpathlineto{\pgfqpoint{4.729924in}{3.415071in}}%
\pgfpathlineto{\pgfqpoint{4.738591in}{3.351213in}}%
\pgfpathlineto{\pgfqpoint{4.742925in}{3.318993in}}%
\pgfpathlineto{\pgfqpoint{4.747258in}{3.289922in}}%
\pgfpathlineto{\pgfqpoint{4.751592in}{3.267799in}}%
\pgfpathlineto{\pgfqpoint{4.755926in}{3.256103in}}%
\pgfpathlineto{\pgfqpoint{4.760259in}{3.257436in}}%
\pgfpathlineto{\pgfqpoint{4.764593in}{3.272642in}}%
\pgfpathlineto{\pgfqpoint{4.768927in}{3.300466in}}%
\pgfpathlineto{\pgfqpoint{4.773260in}{3.337454in}}%
\pgfpathlineto{\pgfqpoint{4.786261in}{3.462509in}}%
\pgfpathlineto{\pgfqpoint{4.790595in}{3.498819in}}%
\pgfpathlineto{\pgfqpoint{4.794929in}{3.529972in}}%
\pgfpathlineto{\pgfqpoint{4.799262in}{3.555891in}}%
\pgfpathlineto{\pgfqpoint{4.803596in}{3.577249in}}%
\pgfpathlineto{\pgfqpoint{4.807930in}{3.594298in}}%
\pgfpathlineto{\pgfqpoint{4.812263in}{3.607792in}}%
\pgfpathlineto{\pgfqpoint{4.816597in}{3.618514in}}%
\pgfpathlineto{\pgfqpoint{4.820930in}{3.626837in}}%
\pgfpathlineto{\pgfqpoint{4.825264in}{3.633518in}}%
\pgfpathlineto{\pgfqpoint{4.829598in}{3.638991in}}%
\pgfpathlineto{\pgfqpoint{4.833931in}{3.643884in}}%
\pgfpathlineto{\pgfqpoint{4.851266in}{3.661763in}}%
\pgfpathlineto{\pgfqpoint{4.855600in}{3.667287in}}%
\pgfpathlineto{\pgfqpoint{4.859933in}{3.673455in}}%
\pgfpathlineto{\pgfqpoint{4.864267in}{3.680634in}}%
\pgfpathlineto{\pgfqpoint{4.868601in}{3.689176in}}%
\pgfpathlineto{\pgfqpoint{4.872934in}{3.699362in}}%
\pgfpathlineto{\pgfqpoint{4.877268in}{3.711272in}}%
\pgfpathlineto{\pgfqpoint{4.881602in}{3.725186in}}%
\pgfpathlineto{\pgfqpoint{4.885935in}{3.741822in}}%
\pgfpathlineto{\pgfqpoint{4.890269in}{3.761488in}}%
\pgfpathlineto{\pgfqpoint{4.894603in}{3.784219in}}%
\pgfpathlineto{\pgfqpoint{4.898936in}{3.810289in}}%
\pgfpathlineto{\pgfqpoint{4.903270in}{3.839736in}}%
\pgfpathlineto{\pgfqpoint{4.911937in}{3.906277in}}%
\pgfpathlineto{\pgfqpoint{4.916271in}{3.939943in}}%
\pgfpathlineto{\pgfqpoint{4.920605in}{3.970462in}}%
\pgfpathlineto{\pgfqpoint{4.924938in}{3.994139in}}%
\pgfpathlineto{\pgfqpoint{4.929272in}{4.006732in}}%
\pgfpathlineto{\pgfqpoint{4.933605in}{4.005120in}}%
\pgfpathlineto{\pgfqpoint{4.937939in}{3.988498in}}%
\pgfpathlineto{\pgfqpoint{4.942273in}{3.958323in}}%
\pgfpathlineto{\pgfqpoint{4.946606in}{3.918657in}}%
\pgfpathlineto{\pgfqpoint{4.959607in}{3.785807in}}%
\pgfpathlineto{\pgfqpoint{4.963941in}{3.747574in}}%
\pgfpathlineto{\pgfqpoint{4.968275in}{3.714890in}}%
\pgfpathlineto{\pgfqpoint{4.972608in}{3.687719in}}%
\pgfpathlineto{\pgfqpoint{4.976942in}{3.665747in}}%
\pgfpathlineto{\pgfqpoint{4.981276in}{3.648041in}}%
\pgfpathlineto{\pgfqpoint{4.985609in}{3.634028in}}%
\pgfpathlineto{\pgfqpoint{4.989943in}{3.622948in}}%
\pgfpathlineto{\pgfqpoint{4.994277in}{3.613843in}}%
\pgfpathlineto{\pgfqpoint{4.998610in}{3.606216in}}%
\pgfpathlineto{\pgfqpoint{5.002944in}{3.599713in}}%
\pgfpathlineto{\pgfqpoint{5.015945in}{3.581709in}}%
\pgfpathlineto{\pgfqpoint{5.020279in}{3.575107in}}%
\pgfpathlineto{\pgfqpoint{5.024612in}{3.567708in}}%
\pgfpathlineto{\pgfqpoint{5.028946in}{3.559294in}}%
\pgfpathlineto{\pgfqpoint{5.033279in}{3.549433in}}%
\pgfpathlineto{\pgfqpoint{5.037613in}{3.537999in}}%
\pgfpathlineto{\pgfqpoint{5.041947in}{3.524732in}}%
\pgfpathlineto{\pgfqpoint{5.046280in}{3.509230in}}%
\pgfpathlineto{\pgfqpoint{5.050614in}{3.491010in}}%
\pgfpathlineto{\pgfqpoint{5.054948in}{3.469938in}}%
\pgfpathlineto{\pgfqpoint{5.059281in}{3.446092in}}%
\pgfpathlineto{\pgfqpoint{5.063615in}{3.419257in}}%
\pgfpathlineto{\pgfqpoint{5.072282in}{3.359180in}}%
\pgfpathlineto{\pgfqpoint{5.076616in}{3.328677in}}%
\pgfpathlineto{\pgfqpoint{5.080950in}{3.301127in}}%
\pgfpathlineto{\pgfqpoint{5.085283in}{3.279451in}}%
\pgfpathlineto{\pgfqpoint{5.089617in}{3.266840in}}%
\pgfpathlineto{\pgfqpoint{5.093951in}{3.265859in}}%
\pgfpathlineto{\pgfqpoint{5.098284in}{3.277531in}}%
\pgfpathlineto{\pgfqpoint{5.102618in}{3.301071in}}%
\pgfpathlineto{\pgfqpoint{5.106952in}{3.333835in}}%
\pgfpathlineto{\pgfqpoint{5.124286in}{3.484622in}}%
\pgfpathlineto{\pgfqpoint{5.128620in}{3.515124in}}%
\pgfpathlineto{\pgfqpoint{5.132953in}{3.540794in}}%
\pgfpathlineto{\pgfqpoint{5.137287in}{3.561945in}}%
\pgfpathlineto{\pgfqpoint{5.141621in}{3.579041in}}%
\pgfpathlineto{\pgfqpoint{5.145954in}{3.592333in}}%
\pgfpathlineto{\pgfqpoint{5.150288in}{3.602696in}}%
\pgfpathlineto{\pgfqpoint{5.154622in}{3.610701in}}%
\pgfpathlineto{\pgfqpoint{5.158955in}{3.616687in}}%
\pgfpathlineto{\pgfqpoint{5.163289in}{3.621313in}}%
\pgfpathlineto{\pgfqpoint{5.167623in}{3.625002in}}%
\pgfpathlineto{\pgfqpoint{5.171956in}{3.627881in}}%
\pgfpathlineto{\pgfqpoint{5.180624in}{3.632248in}}%
\pgfpathlineto{\pgfqpoint{5.193625in}{3.638096in}}%
\pgfpathlineto{\pgfqpoint{5.193625in}{3.638096in}}%
\pgfusepath{stroke}%
\end{pgfscope}%
\begin{pgfscope}%
\pgfsetrectcap%
\pgfsetmiterjoin%
\pgfsetlinewidth{1.003750pt}%
\definecolor{currentstroke}{rgb}{1.000000,1.000000,1.000000}%
\pgfsetstrokecolor{currentstroke}%
\pgfsetdash{}{0pt}%
\pgfpathmoveto{\pgfqpoint{0.647840in}{3.185866in}}%
\pgfpathlineto{\pgfqpoint{0.647840in}{4.073027in}}%
\pgfusepath{stroke}%
\end{pgfscope}%
\begin{pgfscope}%
\pgfsetrectcap%
\pgfsetmiterjoin%
\pgfsetlinewidth{1.003750pt}%
\definecolor{currentstroke}{rgb}{1.000000,1.000000,1.000000}%
\pgfsetstrokecolor{currentstroke}%
\pgfsetdash{}{0pt}%
\pgfpathmoveto{\pgfqpoint{5.410091in}{3.185866in}}%
\pgfpathlineto{\pgfqpoint{5.410091in}{4.073027in}}%
\pgfusepath{stroke}%
\end{pgfscope}%
\begin{pgfscope}%
\pgfsetrectcap%
\pgfsetmiterjoin%
\pgfsetlinewidth{1.003750pt}%
\definecolor{currentstroke}{rgb}{1.000000,1.000000,1.000000}%
\pgfsetstrokecolor{currentstroke}%
\pgfsetdash{}{0pt}%
\pgfpathmoveto{\pgfqpoint{0.647840in}{3.185866in}}%
\pgfpathlineto{\pgfqpoint{5.410091in}{3.185866in}}%
\pgfusepath{stroke}%
\end{pgfscope}%
\begin{pgfscope}%
\pgfsetrectcap%
\pgfsetmiterjoin%
\pgfsetlinewidth{1.003750pt}%
\definecolor{currentstroke}{rgb}{1.000000,1.000000,1.000000}%
\pgfsetstrokecolor{currentstroke}%
\pgfsetdash{}{0pt}%
\pgfpathmoveto{\pgfqpoint{0.647840in}{4.073027in}}%
\pgfpathlineto{\pgfqpoint{5.410091in}{4.073027in}}%
\pgfusepath{stroke}%
\end{pgfscope}%
\begin{pgfscope}%
\pgfsetbuttcap%
\pgfsetmiterjoin%
\definecolor{currentfill}{rgb}{0.898039,0.898039,0.898039}%
\pgfsetfillcolor{currentfill}%
\pgfsetlinewidth{0.000000pt}%
\definecolor{currentstroke}{rgb}{0.000000,0.000000,0.000000}%
\pgfsetstrokecolor{currentstroke}%
\pgfsetstrokeopacity{0.000000}%
\pgfsetdash{}{0pt}%
\pgfpathmoveto{\pgfqpoint{0.647840in}{1.855124in}}%
\pgfpathlineto{\pgfqpoint{5.410091in}{1.855124in}}%
\pgfpathlineto{\pgfqpoint{5.410091in}{2.742285in}}%
\pgfpathlineto{\pgfqpoint{0.647840in}{2.742285in}}%
\pgfpathclose%
\pgfusepath{fill}%
\end{pgfscope}%
\begin{pgfscope}%
\pgfpathrectangle{\pgfqpoint{0.647840in}{1.855124in}}{\pgfqpoint{4.762251in}{0.887161in}}%
\pgfusepath{clip}%
\pgfsetrectcap%
\pgfsetroundjoin%
\pgfsetlinewidth{0.803000pt}%
\definecolor{currentstroke}{rgb}{1.000000,1.000000,1.000000}%
\pgfsetstrokecolor{currentstroke}%
\pgfsetdash{}{0pt}%
\pgfpathmoveto{\pgfqpoint{0.864306in}{1.855124in}}%
\pgfpathlineto{\pgfqpoint{0.864306in}{2.742285in}}%
\pgfusepath{stroke}%
\end{pgfscope}%
\begin{pgfscope}%
\pgfsetbuttcap%
\pgfsetroundjoin%
\definecolor{currentfill}{rgb}{0.333333,0.333333,0.333333}%
\pgfsetfillcolor{currentfill}%
\pgfsetlinewidth{0.803000pt}%
\definecolor{currentstroke}{rgb}{0.333333,0.333333,0.333333}%
\pgfsetstrokecolor{currentstroke}%
\pgfsetdash{}{0pt}%
\pgfsys@defobject{currentmarker}{\pgfqpoint{0.000000in}{-0.048611in}}{\pgfqpoint{0.000000in}{0.000000in}}{%
\pgfpathmoveto{\pgfqpoint{0.000000in}{0.000000in}}%
\pgfpathlineto{\pgfqpoint{0.000000in}{-0.048611in}}%
\pgfusepath{stroke,fill}%
}%
\begin{pgfscope}%
\pgfsys@transformshift{0.864306in}{1.855124in}%
\pgfsys@useobject{currentmarker}{}%
\end{pgfscope}%
\end{pgfscope}%
\begin{pgfscope}%
\definecolor{textcolor}{rgb}{0.333333,0.333333,0.333333}%
\pgfsetstrokecolor{textcolor}%
\pgfsetfillcolor{textcolor}%
\pgftext[x=0.864306in,y=1.757902in,,top]{\color{textcolor}\rmfamily\fontsize{10.000000}{12.000000}\selectfont \(\displaystyle 80.0\)}%
\end{pgfscope}%
\begin{pgfscope}%
\pgfpathrectangle{\pgfqpoint{0.647840in}{1.855124in}}{\pgfqpoint{4.762251in}{0.887161in}}%
\pgfusepath{clip}%
\pgfsetrectcap%
\pgfsetroundjoin%
\pgfsetlinewidth{0.803000pt}%
\definecolor{currentstroke}{rgb}{1.000000,1.000000,1.000000}%
\pgfsetstrokecolor{currentstroke}%
\pgfsetdash{}{0pt}%
\pgfpathmoveto{\pgfqpoint{1.406012in}{1.855124in}}%
\pgfpathlineto{\pgfqpoint{1.406012in}{2.742285in}}%
\pgfusepath{stroke}%
\end{pgfscope}%
\begin{pgfscope}%
\pgfsetbuttcap%
\pgfsetroundjoin%
\definecolor{currentfill}{rgb}{0.333333,0.333333,0.333333}%
\pgfsetfillcolor{currentfill}%
\pgfsetlinewidth{0.803000pt}%
\definecolor{currentstroke}{rgb}{0.333333,0.333333,0.333333}%
\pgfsetstrokecolor{currentstroke}%
\pgfsetdash{}{0pt}%
\pgfsys@defobject{currentmarker}{\pgfqpoint{0.000000in}{-0.048611in}}{\pgfqpoint{0.000000in}{0.000000in}}{%
\pgfpathmoveto{\pgfqpoint{0.000000in}{0.000000in}}%
\pgfpathlineto{\pgfqpoint{0.000000in}{-0.048611in}}%
\pgfusepath{stroke,fill}%
}%
\begin{pgfscope}%
\pgfsys@transformshift{1.406012in}{1.855124in}%
\pgfsys@useobject{currentmarker}{}%
\end{pgfscope}%
\end{pgfscope}%
\begin{pgfscope}%
\definecolor{textcolor}{rgb}{0.333333,0.333333,0.333333}%
\pgfsetstrokecolor{textcolor}%
\pgfsetfillcolor{textcolor}%
\pgftext[x=1.406012in,y=1.757902in,,top]{\color{textcolor}\rmfamily\fontsize{10.000000}{12.000000}\selectfont \(\displaystyle 82.5\)}%
\end{pgfscope}%
\begin{pgfscope}%
\pgfpathrectangle{\pgfqpoint{0.647840in}{1.855124in}}{\pgfqpoint{4.762251in}{0.887161in}}%
\pgfusepath{clip}%
\pgfsetrectcap%
\pgfsetroundjoin%
\pgfsetlinewidth{0.803000pt}%
\definecolor{currentstroke}{rgb}{1.000000,1.000000,1.000000}%
\pgfsetstrokecolor{currentstroke}%
\pgfsetdash{}{0pt}%
\pgfpathmoveto{\pgfqpoint{1.947719in}{1.855124in}}%
\pgfpathlineto{\pgfqpoint{1.947719in}{2.742285in}}%
\pgfusepath{stroke}%
\end{pgfscope}%
\begin{pgfscope}%
\pgfsetbuttcap%
\pgfsetroundjoin%
\definecolor{currentfill}{rgb}{0.333333,0.333333,0.333333}%
\pgfsetfillcolor{currentfill}%
\pgfsetlinewidth{0.803000pt}%
\definecolor{currentstroke}{rgb}{0.333333,0.333333,0.333333}%
\pgfsetstrokecolor{currentstroke}%
\pgfsetdash{}{0pt}%
\pgfsys@defobject{currentmarker}{\pgfqpoint{0.000000in}{-0.048611in}}{\pgfqpoint{0.000000in}{0.000000in}}{%
\pgfpathmoveto{\pgfqpoint{0.000000in}{0.000000in}}%
\pgfpathlineto{\pgfqpoint{0.000000in}{-0.048611in}}%
\pgfusepath{stroke,fill}%
}%
\begin{pgfscope}%
\pgfsys@transformshift{1.947719in}{1.855124in}%
\pgfsys@useobject{currentmarker}{}%
\end{pgfscope}%
\end{pgfscope}%
\begin{pgfscope}%
\definecolor{textcolor}{rgb}{0.333333,0.333333,0.333333}%
\pgfsetstrokecolor{textcolor}%
\pgfsetfillcolor{textcolor}%
\pgftext[x=1.947719in,y=1.757902in,,top]{\color{textcolor}\rmfamily\fontsize{10.000000}{12.000000}\selectfont \(\displaystyle 85.0\)}%
\end{pgfscope}%
\begin{pgfscope}%
\pgfpathrectangle{\pgfqpoint{0.647840in}{1.855124in}}{\pgfqpoint{4.762251in}{0.887161in}}%
\pgfusepath{clip}%
\pgfsetrectcap%
\pgfsetroundjoin%
\pgfsetlinewidth{0.803000pt}%
\definecolor{currentstroke}{rgb}{1.000000,1.000000,1.000000}%
\pgfsetstrokecolor{currentstroke}%
\pgfsetdash{}{0pt}%
\pgfpathmoveto{\pgfqpoint{2.489425in}{1.855124in}}%
\pgfpathlineto{\pgfqpoint{2.489425in}{2.742285in}}%
\pgfusepath{stroke}%
\end{pgfscope}%
\begin{pgfscope}%
\pgfsetbuttcap%
\pgfsetroundjoin%
\definecolor{currentfill}{rgb}{0.333333,0.333333,0.333333}%
\pgfsetfillcolor{currentfill}%
\pgfsetlinewidth{0.803000pt}%
\definecolor{currentstroke}{rgb}{0.333333,0.333333,0.333333}%
\pgfsetstrokecolor{currentstroke}%
\pgfsetdash{}{0pt}%
\pgfsys@defobject{currentmarker}{\pgfqpoint{0.000000in}{-0.048611in}}{\pgfqpoint{0.000000in}{0.000000in}}{%
\pgfpathmoveto{\pgfqpoint{0.000000in}{0.000000in}}%
\pgfpathlineto{\pgfqpoint{0.000000in}{-0.048611in}}%
\pgfusepath{stroke,fill}%
}%
\begin{pgfscope}%
\pgfsys@transformshift{2.489425in}{1.855124in}%
\pgfsys@useobject{currentmarker}{}%
\end{pgfscope}%
\end{pgfscope}%
\begin{pgfscope}%
\definecolor{textcolor}{rgb}{0.333333,0.333333,0.333333}%
\pgfsetstrokecolor{textcolor}%
\pgfsetfillcolor{textcolor}%
\pgftext[x=2.489425in,y=1.757902in,,top]{\color{textcolor}\rmfamily\fontsize{10.000000}{12.000000}\selectfont \(\displaystyle 87.5\)}%
\end{pgfscope}%
\begin{pgfscope}%
\pgfpathrectangle{\pgfqpoint{0.647840in}{1.855124in}}{\pgfqpoint{4.762251in}{0.887161in}}%
\pgfusepath{clip}%
\pgfsetrectcap%
\pgfsetroundjoin%
\pgfsetlinewidth{0.803000pt}%
\definecolor{currentstroke}{rgb}{1.000000,1.000000,1.000000}%
\pgfsetstrokecolor{currentstroke}%
\pgfsetdash{}{0pt}%
\pgfpathmoveto{\pgfqpoint{3.031132in}{1.855124in}}%
\pgfpathlineto{\pgfqpoint{3.031132in}{2.742285in}}%
\pgfusepath{stroke}%
\end{pgfscope}%
\begin{pgfscope}%
\pgfsetbuttcap%
\pgfsetroundjoin%
\definecolor{currentfill}{rgb}{0.333333,0.333333,0.333333}%
\pgfsetfillcolor{currentfill}%
\pgfsetlinewidth{0.803000pt}%
\definecolor{currentstroke}{rgb}{0.333333,0.333333,0.333333}%
\pgfsetstrokecolor{currentstroke}%
\pgfsetdash{}{0pt}%
\pgfsys@defobject{currentmarker}{\pgfqpoint{0.000000in}{-0.048611in}}{\pgfqpoint{0.000000in}{0.000000in}}{%
\pgfpathmoveto{\pgfqpoint{0.000000in}{0.000000in}}%
\pgfpathlineto{\pgfqpoint{0.000000in}{-0.048611in}}%
\pgfusepath{stroke,fill}%
}%
\begin{pgfscope}%
\pgfsys@transformshift{3.031132in}{1.855124in}%
\pgfsys@useobject{currentmarker}{}%
\end{pgfscope}%
\end{pgfscope}%
\begin{pgfscope}%
\definecolor{textcolor}{rgb}{0.333333,0.333333,0.333333}%
\pgfsetstrokecolor{textcolor}%
\pgfsetfillcolor{textcolor}%
\pgftext[x=3.031132in,y=1.757902in,,top]{\color{textcolor}\rmfamily\fontsize{10.000000}{12.000000}\selectfont \(\displaystyle 90.0\)}%
\end{pgfscope}%
\begin{pgfscope}%
\pgfpathrectangle{\pgfqpoint{0.647840in}{1.855124in}}{\pgfqpoint{4.762251in}{0.887161in}}%
\pgfusepath{clip}%
\pgfsetrectcap%
\pgfsetroundjoin%
\pgfsetlinewidth{0.803000pt}%
\definecolor{currentstroke}{rgb}{1.000000,1.000000,1.000000}%
\pgfsetstrokecolor{currentstroke}%
\pgfsetdash{}{0pt}%
\pgfpathmoveto{\pgfqpoint{3.572839in}{1.855124in}}%
\pgfpathlineto{\pgfqpoint{3.572839in}{2.742285in}}%
\pgfusepath{stroke}%
\end{pgfscope}%
\begin{pgfscope}%
\pgfsetbuttcap%
\pgfsetroundjoin%
\definecolor{currentfill}{rgb}{0.333333,0.333333,0.333333}%
\pgfsetfillcolor{currentfill}%
\pgfsetlinewidth{0.803000pt}%
\definecolor{currentstroke}{rgb}{0.333333,0.333333,0.333333}%
\pgfsetstrokecolor{currentstroke}%
\pgfsetdash{}{0pt}%
\pgfsys@defobject{currentmarker}{\pgfqpoint{0.000000in}{-0.048611in}}{\pgfqpoint{0.000000in}{0.000000in}}{%
\pgfpathmoveto{\pgfqpoint{0.000000in}{0.000000in}}%
\pgfpathlineto{\pgfqpoint{0.000000in}{-0.048611in}}%
\pgfusepath{stroke,fill}%
}%
\begin{pgfscope}%
\pgfsys@transformshift{3.572839in}{1.855124in}%
\pgfsys@useobject{currentmarker}{}%
\end{pgfscope}%
\end{pgfscope}%
\begin{pgfscope}%
\definecolor{textcolor}{rgb}{0.333333,0.333333,0.333333}%
\pgfsetstrokecolor{textcolor}%
\pgfsetfillcolor{textcolor}%
\pgftext[x=3.572839in,y=1.757902in,,top]{\color{textcolor}\rmfamily\fontsize{10.000000}{12.000000}\selectfont \(\displaystyle 92.5\)}%
\end{pgfscope}%
\begin{pgfscope}%
\pgfpathrectangle{\pgfqpoint{0.647840in}{1.855124in}}{\pgfqpoint{4.762251in}{0.887161in}}%
\pgfusepath{clip}%
\pgfsetrectcap%
\pgfsetroundjoin%
\pgfsetlinewidth{0.803000pt}%
\definecolor{currentstroke}{rgb}{1.000000,1.000000,1.000000}%
\pgfsetstrokecolor{currentstroke}%
\pgfsetdash{}{0pt}%
\pgfpathmoveto{\pgfqpoint{4.114545in}{1.855124in}}%
\pgfpathlineto{\pgfqpoint{4.114545in}{2.742285in}}%
\pgfusepath{stroke}%
\end{pgfscope}%
\begin{pgfscope}%
\pgfsetbuttcap%
\pgfsetroundjoin%
\definecolor{currentfill}{rgb}{0.333333,0.333333,0.333333}%
\pgfsetfillcolor{currentfill}%
\pgfsetlinewidth{0.803000pt}%
\definecolor{currentstroke}{rgb}{0.333333,0.333333,0.333333}%
\pgfsetstrokecolor{currentstroke}%
\pgfsetdash{}{0pt}%
\pgfsys@defobject{currentmarker}{\pgfqpoint{0.000000in}{-0.048611in}}{\pgfqpoint{0.000000in}{0.000000in}}{%
\pgfpathmoveto{\pgfqpoint{0.000000in}{0.000000in}}%
\pgfpathlineto{\pgfqpoint{0.000000in}{-0.048611in}}%
\pgfusepath{stroke,fill}%
}%
\begin{pgfscope}%
\pgfsys@transformshift{4.114545in}{1.855124in}%
\pgfsys@useobject{currentmarker}{}%
\end{pgfscope}%
\end{pgfscope}%
\begin{pgfscope}%
\definecolor{textcolor}{rgb}{0.333333,0.333333,0.333333}%
\pgfsetstrokecolor{textcolor}%
\pgfsetfillcolor{textcolor}%
\pgftext[x=4.114545in,y=1.757902in,,top]{\color{textcolor}\rmfamily\fontsize{10.000000}{12.000000}\selectfont \(\displaystyle 95.0\)}%
\end{pgfscope}%
\begin{pgfscope}%
\pgfpathrectangle{\pgfqpoint{0.647840in}{1.855124in}}{\pgfqpoint{4.762251in}{0.887161in}}%
\pgfusepath{clip}%
\pgfsetrectcap%
\pgfsetroundjoin%
\pgfsetlinewidth{0.803000pt}%
\definecolor{currentstroke}{rgb}{1.000000,1.000000,1.000000}%
\pgfsetstrokecolor{currentstroke}%
\pgfsetdash{}{0pt}%
\pgfpathmoveto{\pgfqpoint{4.656252in}{1.855124in}}%
\pgfpathlineto{\pgfqpoint{4.656252in}{2.742285in}}%
\pgfusepath{stroke}%
\end{pgfscope}%
\begin{pgfscope}%
\pgfsetbuttcap%
\pgfsetroundjoin%
\definecolor{currentfill}{rgb}{0.333333,0.333333,0.333333}%
\pgfsetfillcolor{currentfill}%
\pgfsetlinewidth{0.803000pt}%
\definecolor{currentstroke}{rgb}{0.333333,0.333333,0.333333}%
\pgfsetstrokecolor{currentstroke}%
\pgfsetdash{}{0pt}%
\pgfsys@defobject{currentmarker}{\pgfqpoint{0.000000in}{-0.048611in}}{\pgfqpoint{0.000000in}{0.000000in}}{%
\pgfpathmoveto{\pgfqpoint{0.000000in}{0.000000in}}%
\pgfpathlineto{\pgfqpoint{0.000000in}{-0.048611in}}%
\pgfusepath{stroke,fill}%
}%
\begin{pgfscope}%
\pgfsys@transformshift{4.656252in}{1.855124in}%
\pgfsys@useobject{currentmarker}{}%
\end{pgfscope}%
\end{pgfscope}%
\begin{pgfscope}%
\definecolor{textcolor}{rgb}{0.333333,0.333333,0.333333}%
\pgfsetstrokecolor{textcolor}%
\pgfsetfillcolor{textcolor}%
\pgftext[x=4.656252in,y=1.757902in,,top]{\color{textcolor}\rmfamily\fontsize{10.000000}{12.000000}\selectfont \(\displaystyle 97.5\)}%
\end{pgfscope}%
\begin{pgfscope}%
\pgfpathrectangle{\pgfqpoint{0.647840in}{1.855124in}}{\pgfqpoint{4.762251in}{0.887161in}}%
\pgfusepath{clip}%
\pgfsetrectcap%
\pgfsetroundjoin%
\pgfsetlinewidth{0.803000pt}%
\definecolor{currentstroke}{rgb}{1.000000,1.000000,1.000000}%
\pgfsetstrokecolor{currentstroke}%
\pgfsetdash{}{0pt}%
\pgfpathmoveto{\pgfqpoint{5.197958in}{1.855124in}}%
\pgfpathlineto{\pgfqpoint{5.197958in}{2.742285in}}%
\pgfusepath{stroke}%
\end{pgfscope}%
\begin{pgfscope}%
\pgfsetbuttcap%
\pgfsetroundjoin%
\definecolor{currentfill}{rgb}{0.333333,0.333333,0.333333}%
\pgfsetfillcolor{currentfill}%
\pgfsetlinewidth{0.803000pt}%
\definecolor{currentstroke}{rgb}{0.333333,0.333333,0.333333}%
\pgfsetstrokecolor{currentstroke}%
\pgfsetdash{}{0pt}%
\pgfsys@defobject{currentmarker}{\pgfqpoint{0.000000in}{-0.048611in}}{\pgfqpoint{0.000000in}{0.000000in}}{%
\pgfpathmoveto{\pgfqpoint{0.000000in}{0.000000in}}%
\pgfpathlineto{\pgfqpoint{0.000000in}{-0.048611in}}%
\pgfusepath{stroke,fill}%
}%
\begin{pgfscope}%
\pgfsys@transformshift{5.197958in}{1.855124in}%
\pgfsys@useobject{currentmarker}{}%
\end{pgfscope}%
\end{pgfscope}%
\begin{pgfscope}%
\definecolor{textcolor}{rgb}{0.333333,0.333333,0.333333}%
\pgfsetstrokecolor{textcolor}%
\pgfsetfillcolor{textcolor}%
\pgftext[x=5.197958in,y=1.757902in,,top]{\color{textcolor}\rmfamily\fontsize{10.000000}{12.000000}\selectfont \(\displaystyle 100.0\)}%
\end{pgfscope}%
\begin{pgfscope}%
\pgfpathrectangle{\pgfqpoint{0.647840in}{1.855124in}}{\pgfqpoint{4.762251in}{0.887161in}}%
\pgfusepath{clip}%
\pgfsetrectcap%
\pgfsetroundjoin%
\pgfsetlinewidth{0.803000pt}%
\definecolor{currentstroke}{rgb}{1.000000,1.000000,1.000000}%
\pgfsetstrokecolor{currentstroke}%
\pgfsetdash{}{0pt}%
\pgfpathmoveto{\pgfqpoint{0.647840in}{1.920226in}}%
\pgfpathlineto{\pgfqpoint{5.410091in}{1.920226in}}%
\pgfusepath{stroke}%
\end{pgfscope}%
\begin{pgfscope}%
\pgfsetbuttcap%
\pgfsetroundjoin%
\definecolor{currentfill}{rgb}{0.333333,0.333333,0.333333}%
\pgfsetfillcolor{currentfill}%
\pgfsetlinewidth{0.803000pt}%
\definecolor{currentstroke}{rgb}{0.333333,0.333333,0.333333}%
\pgfsetstrokecolor{currentstroke}%
\pgfsetdash{}{0pt}%
\pgfsys@defobject{currentmarker}{\pgfqpoint{-0.048611in}{0.000000in}}{\pgfqpoint{0.000000in}{0.000000in}}{%
\pgfpathmoveto{\pgfqpoint{0.000000in}{0.000000in}}%
\pgfpathlineto{\pgfqpoint{-0.048611in}{0.000000in}}%
\pgfusepath{stroke,fill}%
}%
\begin{pgfscope}%
\pgfsys@transformshift{0.647840in}{1.920226in}%
\pgfsys@useobject{currentmarker}{}%
\end{pgfscope}%
\end{pgfscope}%
\begin{pgfscope}%
\definecolor{textcolor}{rgb}{0.333333,0.333333,0.333333}%
\pgfsetstrokecolor{textcolor}%
\pgfsetfillcolor{textcolor}%
\pgftext[x=0.303703in,y=1.872000in,left,base]{\color{textcolor}\rmfamily\fontsize{10.000000}{12.000000}\selectfont \(\displaystyle -20\)}%
\end{pgfscope}%
\begin{pgfscope}%
\pgfpathrectangle{\pgfqpoint{0.647840in}{1.855124in}}{\pgfqpoint{4.762251in}{0.887161in}}%
\pgfusepath{clip}%
\pgfsetrectcap%
\pgfsetroundjoin%
\pgfsetlinewidth{0.803000pt}%
\definecolor{currentstroke}{rgb}{1.000000,1.000000,1.000000}%
\pgfsetstrokecolor{currentstroke}%
\pgfsetdash{}{0pt}%
\pgfpathmoveto{\pgfqpoint{0.647840in}{2.292965in}}%
\pgfpathlineto{\pgfqpoint{5.410091in}{2.292965in}}%
\pgfusepath{stroke}%
\end{pgfscope}%
\begin{pgfscope}%
\pgfsetbuttcap%
\pgfsetroundjoin%
\definecolor{currentfill}{rgb}{0.333333,0.333333,0.333333}%
\pgfsetfillcolor{currentfill}%
\pgfsetlinewidth{0.803000pt}%
\definecolor{currentstroke}{rgb}{0.333333,0.333333,0.333333}%
\pgfsetstrokecolor{currentstroke}%
\pgfsetdash{}{0pt}%
\pgfsys@defobject{currentmarker}{\pgfqpoint{-0.048611in}{0.000000in}}{\pgfqpoint{0.000000in}{0.000000in}}{%
\pgfpathmoveto{\pgfqpoint{0.000000in}{0.000000in}}%
\pgfpathlineto{\pgfqpoint{-0.048611in}{0.000000in}}%
\pgfusepath{stroke,fill}%
}%
\begin{pgfscope}%
\pgfsys@transformshift{0.647840in}{2.292965in}%
\pgfsys@useobject{currentmarker}{}%
\end{pgfscope}%
\end{pgfscope}%
\begin{pgfscope}%
\definecolor{textcolor}{rgb}{0.333333,0.333333,0.333333}%
\pgfsetstrokecolor{textcolor}%
\pgfsetfillcolor{textcolor}%
\pgftext[x=0.481173in,y=2.244740in,left,base]{\color{textcolor}\rmfamily\fontsize{10.000000}{12.000000}\selectfont \(\displaystyle 0\)}%
\end{pgfscope}%
\begin{pgfscope}%
\pgfpathrectangle{\pgfqpoint{0.647840in}{1.855124in}}{\pgfqpoint{4.762251in}{0.887161in}}%
\pgfusepath{clip}%
\pgfsetrectcap%
\pgfsetroundjoin%
\pgfsetlinewidth{0.803000pt}%
\definecolor{currentstroke}{rgb}{1.000000,1.000000,1.000000}%
\pgfsetstrokecolor{currentstroke}%
\pgfsetdash{}{0pt}%
\pgfpathmoveto{\pgfqpoint{0.647840in}{2.665705in}}%
\pgfpathlineto{\pgfqpoint{5.410091in}{2.665705in}}%
\pgfusepath{stroke}%
\end{pgfscope}%
\begin{pgfscope}%
\pgfsetbuttcap%
\pgfsetroundjoin%
\definecolor{currentfill}{rgb}{0.333333,0.333333,0.333333}%
\pgfsetfillcolor{currentfill}%
\pgfsetlinewidth{0.803000pt}%
\definecolor{currentstroke}{rgb}{0.333333,0.333333,0.333333}%
\pgfsetstrokecolor{currentstroke}%
\pgfsetdash{}{0pt}%
\pgfsys@defobject{currentmarker}{\pgfqpoint{-0.048611in}{0.000000in}}{\pgfqpoint{0.000000in}{0.000000in}}{%
\pgfpathmoveto{\pgfqpoint{0.000000in}{0.000000in}}%
\pgfpathlineto{\pgfqpoint{-0.048611in}{0.000000in}}%
\pgfusepath{stroke,fill}%
}%
\begin{pgfscope}%
\pgfsys@transformshift{0.647840in}{2.665705in}%
\pgfsys@useobject{currentmarker}{}%
\end{pgfscope}%
\end{pgfscope}%
\begin{pgfscope}%
\definecolor{textcolor}{rgb}{0.333333,0.333333,0.333333}%
\pgfsetstrokecolor{textcolor}%
\pgfsetfillcolor{textcolor}%
\pgftext[x=0.411728in,y=2.617479in,left,base]{\color{textcolor}\rmfamily\fontsize{10.000000}{12.000000}\selectfont \(\displaystyle 20\)}%
\end{pgfscope}%
\begin{pgfscope}%
\definecolor{textcolor}{rgb}{0.333333,0.333333,0.333333}%
\pgfsetstrokecolor{textcolor}%
\pgfsetfillcolor{textcolor}%
\pgftext[x=0.248148in,y=2.298704in,,bottom,rotate=90.000000]{\color{textcolor}\rmfamily\fontsize{12.000000}{14.400000}\selectfont y}%
\end{pgfscope}%
\begin{pgfscope}%
\pgfpathrectangle{\pgfqpoint{0.647840in}{1.855124in}}{\pgfqpoint{4.762251in}{0.887161in}}%
\pgfusepath{clip}%
\pgfsetrectcap%
\pgfsetroundjoin%
\pgfsetlinewidth{1.505625pt}%
\definecolor{currentstroke}{rgb}{0.886275,0.290196,0.200000}%
\pgfsetstrokecolor{currentstroke}%
\pgfsetdash{}{0pt}%
\pgfpathmoveto{\pgfqpoint{0.864306in}{2.160519in}}%
\pgfpathlineto{\pgfqpoint{0.868639in}{2.137929in}}%
\pgfpathlineto{\pgfqpoint{0.872973in}{2.111919in}}%
\pgfpathlineto{\pgfqpoint{0.877306in}{2.082529in}}%
\pgfpathlineto{\pgfqpoint{0.885974in}{2.016335in}}%
\pgfpathlineto{\pgfqpoint{0.890307in}{1.983093in}}%
\pgfpathlineto{\pgfqpoint{0.894641in}{1.954218in}}%
\pgfpathlineto{\pgfqpoint{0.898975in}{1.934735in}}%
\pgfpathlineto{\pgfqpoint{0.903308in}{1.930323in}}%
\pgfpathlineto{\pgfqpoint{0.907642in}{1.945698in}}%
\pgfpathlineto{\pgfqpoint{0.911976in}{1.982376in}}%
\pgfpathlineto{\pgfqpoint{0.916309in}{2.036982in}}%
\pgfpathlineto{\pgfqpoint{0.924977in}{2.166260in}}%
\pgfpathlineto{\pgfqpoint{0.929310in}{2.222784in}}%
\pgfpathlineto{\pgfqpoint{0.933644in}{2.266613in}}%
\pgfpathlineto{\pgfqpoint{0.937978in}{2.297053in}}%
\pgfpathlineto{\pgfqpoint{0.942311in}{2.315927in}}%
\pgfpathlineto{\pgfqpoint{0.946645in}{2.326079in}}%
\pgfpathlineto{\pgfqpoint{0.950979in}{2.330328in}}%
\pgfpathlineto{\pgfqpoint{0.955312in}{2.330990in}}%
\pgfpathlineto{\pgfqpoint{0.959646in}{2.329765in}}%
\pgfpathlineto{\pgfqpoint{0.968313in}{2.325837in}}%
\pgfpathlineto{\pgfqpoint{0.972647in}{2.324303in}}%
\pgfpathlineto{\pgfqpoint{0.976981in}{2.323444in}}%
\pgfpathlineto{\pgfqpoint{0.981314in}{2.323385in}}%
\pgfpathlineto{\pgfqpoint{0.985648in}{2.324193in}}%
\pgfpathlineto{\pgfqpoint{0.989981in}{2.325905in}}%
\pgfpathlineto{\pgfqpoint{0.994315in}{2.328563in}}%
\pgfpathlineto{\pgfqpoint{0.998649in}{2.332223in}}%
\pgfpathlineto{\pgfqpoint{1.002982in}{2.336970in}}%
\pgfpathlineto{\pgfqpoint{1.007316in}{2.342925in}}%
\pgfpathlineto{\pgfqpoint{1.011650in}{2.350249in}}%
\pgfpathlineto{\pgfqpoint{1.015983in}{2.359151in}}%
\pgfpathlineto{\pgfqpoint{1.020317in}{2.369884in}}%
\pgfpathlineto{\pgfqpoint{1.024651in}{2.382751in}}%
\pgfpathlineto{\pgfqpoint{1.028984in}{2.398097in}}%
\pgfpathlineto{\pgfqpoint{1.033318in}{2.416291in}}%
\pgfpathlineto{\pgfqpoint{1.037652in}{2.437696in}}%
\pgfpathlineto{\pgfqpoint{1.041985in}{2.462600in}}%
\pgfpathlineto{\pgfqpoint{1.046319in}{2.491113in}}%
\pgfpathlineto{\pgfqpoint{1.050653in}{2.522979in}}%
\pgfpathlineto{\pgfqpoint{1.063654in}{2.624611in}}%
\pgfpathlineto{\pgfqpoint{1.067987in}{2.649672in}}%
\pgfpathlineto{\pgfqpoint{1.072321in}{2.661589in}}%
\pgfpathlineto{\pgfqpoint{1.076655in}{2.654656in}}%
\pgfpathlineto{\pgfqpoint{1.080988in}{2.625535in}}%
\pgfpathlineto{\pgfqpoint{1.085322in}{2.575543in}}%
\pgfpathlineto{\pgfqpoint{1.098323in}{2.380606in}}%
\pgfpathlineto{\pgfqpoint{1.102656in}{2.330193in}}%
\pgfpathlineto{\pgfqpoint{1.106990in}{2.293951in}}%
\pgfpathlineto{\pgfqpoint{1.111324in}{2.270668in}}%
\pgfpathlineto{\pgfqpoint{1.115657in}{2.257527in}}%
\pgfpathlineto{\pgfqpoint{1.119991in}{2.251446in}}%
\pgfpathlineto{\pgfqpoint{1.124325in}{2.249770in}}%
\pgfpathlineto{\pgfqpoint{1.128658in}{2.250503in}}%
\pgfpathlineto{\pgfqpoint{1.141659in}{2.255651in}}%
\pgfpathlineto{\pgfqpoint{1.145993in}{2.256424in}}%
\pgfpathlineto{\pgfqpoint{1.150327in}{2.256310in}}%
\pgfpathlineto{\pgfqpoint{1.154660in}{2.255218in}}%
\pgfpathlineto{\pgfqpoint{1.158994in}{2.253091in}}%
\pgfpathlineto{\pgfqpoint{1.163328in}{2.249874in}}%
\pgfpathlineto{\pgfqpoint{1.167661in}{2.245496in}}%
\pgfpathlineto{\pgfqpoint{1.171995in}{2.239860in}}%
\pgfpathlineto{\pgfqpoint{1.176329in}{2.232827in}}%
\pgfpathlineto{\pgfqpoint{1.180662in}{2.224220in}}%
\pgfpathlineto{\pgfqpoint{1.184996in}{2.213815in}}%
\pgfpathlineto{\pgfqpoint{1.189329in}{2.201344in}}%
\pgfpathlineto{\pgfqpoint{1.193663in}{2.186504in}}%
\pgfpathlineto{\pgfqpoint{1.197997in}{2.168969in}}%
\pgfpathlineto{\pgfqpoint{1.202330in}{2.148429in}}%
\pgfpathlineto{\pgfqpoint{1.206664in}{2.124647in}}%
\pgfpathlineto{\pgfqpoint{1.210998in}{2.097560in}}%
\pgfpathlineto{\pgfqpoint{1.215331in}{2.067447in}}%
\pgfpathlineto{\pgfqpoint{1.228332in}{1.972099in}}%
\pgfpathlineto{\pgfqpoint{1.232666in}{1.948418in}}%
\pgfpathlineto{\pgfqpoint{1.237000in}{1.936547in}}%
\pgfpathlineto{\pgfqpoint{1.241333in}{1.941568in}}%
\pgfpathlineto{\pgfqpoint{1.245667in}{1.966611in}}%
\pgfpathlineto{\pgfqpoint{1.250001in}{2.010886in}}%
\pgfpathlineto{\pgfqpoint{1.254334in}{2.068935in}}%
\pgfpathlineto{\pgfqpoint{1.263002in}{2.191327in}}%
\pgfpathlineto{\pgfqpoint{1.267335in}{2.240473in}}%
\pgfpathlineto{\pgfqpoint{1.271669in}{2.276923in}}%
\pgfpathlineto{\pgfqpoint{1.276003in}{2.301207in}}%
\pgfpathlineto{\pgfqpoint{1.280336in}{2.315571in}}%
\pgfpathlineto{\pgfqpoint{1.284670in}{2.322742in}}%
\pgfpathlineto{\pgfqpoint{1.289004in}{2.325199in}}%
\pgfpathlineto{\pgfqpoint{1.293337in}{2.324886in}}%
\pgfpathlineto{\pgfqpoint{1.297671in}{2.323189in}}%
\pgfpathlineto{\pgfqpoint{1.306338in}{2.318962in}}%
\pgfpathlineto{\pgfqpoint{1.310672in}{2.317329in}}%
\pgfpathlineto{\pgfqpoint{1.315005in}{2.316300in}}%
\pgfpathlineto{\pgfqpoint{1.319339in}{2.315956in}}%
\pgfpathlineto{\pgfqpoint{1.323673in}{2.316331in}}%
\pgfpathlineto{\pgfqpoint{1.328006in}{2.317443in}}%
\pgfpathlineto{\pgfqpoint{1.332340in}{2.319311in}}%
\pgfpathlineto{\pgfqpoint{1.336674in}{2.321970in}}%
\pgfpathlineto{\pgfqpoint{1.341007in}{2.325481in}}%
\pgfpathlineto{\pgfqpoint{1.345341in}{2.329935in}}%
\pgfpathlineto{\pgfqpoint{1.349675in}{2.335457in}}%
\pgfpathlineto{\pgfqpoint{1.354008in}{2.342217in}}%
\pgfpathlineto{\pgfqpoint{1.358342in}{2.350426in}}%
\pgfpathlineto{\pgfqpoint{1.362676in}{2.360347in}}%
\pgfpathlineto{\pgfqpoint{1.367009in}{2.372295in}}%
\pgfpathlineto{\pgfqpoint{1.371343in}{2.386636in}}%
\pgfpathlineto{\pgfqpoint{1.375677in}{2.403782in}}%
\pgfpathlineto{\pgfqpoint{1.380010in}{2.424169in}}%
\pgfpathlineto{\pgfqpoint{1.384344in}{2.448211in}}%
\pgfpathlineto{\pgfqpoint{1.388678in}{2.476214in}}%
\pgfpathlineto{\pgfqpoint{1.393011in}{2.508222in}}%
\pgfpathlineto{\pgfqpoint{1.401678in}{2.581537in}}%
\pgfpathlineto{\pgfqpoint{1.406012in}{2.618869in}}%
\pgfpathlineto{\pgfqpoint{1.410346in}{2.651400in}}%
\pgfpathlineto{\pgfqpoint{1.414679in}{2.673015in}}%
\pgfpathlineto{\pgfqpoint{1.419013in}{2.676710in}}%
\pgfpathlineto{\pgfqpoint{1.423347in}{2.656740in}}%
\pgfpathlineto{\pgfqpoint{1.427680in}{2.611630in}}%
\pgfpathlineto{\pgfqpoint{1.432014in}{2.546313in}}%
\pgfpathlineto{\pgfqpoint{1.440681in}{2.398805in}}%
\pgfpathlineto{\pgfqpoint{1.445015in}{2.338043in}}%
\pgfpathlineto{\pgfqpoint{1.449349in}{2.293137in}}%
\pgfpathlineto{\pgfqpoint{1.453682in}{2.263642in}}%
\pgfpathlineto{\pgfqpoint{1.458016in}{2.246582in}}%
\pgfpathlineto{\pgfqpoint{1.462350in}{2.238299in}}%
\pgfpathlineto{\pgfqpoint{1.466683in}{2.235528in}}%
\pgfpathlineto{\pgfqpoint{1.471017in}{2.235773in}}%
\pgfpathlineto{\pgfqpoint{1.484018in}{2.240152in}}%
\pgfpathlineto{\pgfqpoint{1.488352in}{2.240388in}}%
\pgfpathlineto{\pgfqpoint{1.492685in}{2.239473in}}%
\pgfpathlineto{\pgfqpoint{1.497019in}{2.237278in}}%
\pgfpathlineto{\pgfqpoint{1.501352in}{2.233716in}}%
\pgfpathlineto{\pgfqpoint{1.505686in}{2.228706in}}%
\pgfpathlineto{\pgfqpoint{1.510020in}{2.222152in}}%
\pgfpathlineto{\pgfqpoint{1.514353in}{2.213928in}}%
\pgfpathlineto{\pgfqpoint{1.518687in}{2.203876in}}%
\pgfpathlineto{\pgfqpoint{1.523021in}{2.191804in}}%
\pgfpathlineto{\pgfqpoint{1.527354in}{2.177501in}}%
\pgfpathlineto{\pgfqpoint{1.531688in}{2.160753in}}%
\pgfpathlineto{\pgfqpoint{1.536022in}{2.141385in}}%
\pgfpathlineto{\pgfqpoint{1.540355in}{2.119321in}}%
\pgfpathlineto{\pgfqpoint{1.544689in}{2.094682in}}%
\pgfpathlineto{\pgfqpoint{1.562024in}{1.988500in}}%
\pgfpathlineto{\pgfqpoint{1.566357in}{1.970900in}}%
\pgfpathlineto{\pgfqpoint{1.570691in}{1.963807in}}%
\pgfpathlineto{\pgfqpoint{1.575025in}{1.970719in}}%
\pgfpathlineto{\pgfqpoint{1.579358in}{1.993413in}}%
\pgfpathlineto{\pgfqpoint{1.583692in}{2.030769in}}%
\pgfpathlineto{\pgfqpoint{1.588026in}{2.078483in}}%
\pgfpathlineto{\pgfqpoint{1.596693in}{2.179206in}}%
\pgfpathlineto{\pgfqpoint{1.601027in}{2.220875in}}%
\pgfpathlineto{\pgfqpoint{1.605360in}{2.252868in}}%
\pgfpathlineto{\pgfqpoint{1.609694in}{2.275157in}}%
\pgfpathlineto{\pgfqpoint{1.614027in}{2.289112in}}%
\pgfpathlineto{\pgfqpoint{1.618361in}{2.296653in}}%
\pgfpathlineto{\pgfqpoint{1.622695in}{2.299666in}}%
\pgfpathlineto{\pgfqpoint{1.627028in}{2.299722in}}%
\pgfpathlineto{\pgfqpoint{1.631362in}{2.297996in}}%
\pgfpathlineto{\pgfqpoint{1.635696in}{2.295296in}}%
\pgfpathlineto{\pgfqpoint{1.644363in}{2.288807in}}%
\pgfpathlineto{\pgfqpoint{1.661698in}{2.275191in}}%
\pgfpathlineto{\pgfqpoint{1.670365in}{2.267378in}}%
\pgfpathlineto{\pgfqpoint{1.674699in}{2.262791in}}%
\pgfpathlineto{\pgfqpoint{1.679032in}{2.257530in}}%
\pgfpathlineto{\pgfqpoint{1.683366in}{2.251403in}}%
\pgfpathlineto{\pgfqpoint{1.687700in}{2.244191in}}%
\pgfpathlineto{\pgfqpoint{1.692033in}{2.235639in}}%
\pgfpathlineto{\pgfqpoint{1.696367in}{2.225454in}}%
\pgfpathlineto{\pgfqpoint{1.700701in}{2.213296in}}%
\pgfpathlineto{\pgfqpoint{1.705034in}{2.198784in}}%
\pgfpathlineto{\pgfqpoint{1.709368in}{2.181495in}}%
\pgfpathlineto{\pgfqpoint{1.713701in}{2.160986in}}%
\pgfpathlineto{\pgfqpoint{1.718035in}{2.136847in}}%
\pgfpathlineto{\pgfqpoint{1.722369in}{2.108776in}}%
\pgfpathlineto{\pgfqpoint{1.726702in}{2.076746in}}%
\pgfpathlineto{\pgfqpoint{1.735370in}{2.003626in}}%
\pgfpathlineto{\pgfqpoint{1.739703in}{1.966586in}}%
\pgfpathlineto{\pgfqpoint{1.744037in}{1.934516in}}%
\pgfpathlineto{\pgfqpoint{1.748371in}{1.913530in}}%
\pgfpathlineto{\pgfqpoint{1.752704in}{1.910559in}}%
\pgfpathlineto{\pgfqpoint{1.757038in}{1.931194in}}%
\pgfpathlineto{\pgfqpoint{1.761372in}{1.976711in}}%
\pgfpathlineto{\pgfqpoint{1.765705in}{2.042028in}}%
\pgfpathlineto{\pgfqpoint{1.774373in}{2.188462in}}%
\pgfpathlineto{\pgfqpoint{1.778706in}{2.248504in}}%
\pgfpathlineto{\pgfqpoint{1.783040in}{2.292795in}}%
\pgfpathlineto{\pgfqpoint{1.787374in}{2.321843in}}%
\pgfpathlineto{\pgfqpoint{1.791707in}{2.338621in}}%
\pgfpathlineto{\pgfqpoint{1.796041in}{2.346746in}}%
\pgfpathlineto{\pgfqpoint{1.800375in}{2.349440in}}%
\pgfpathlineto{\pgfqpoint{1.804708in}{2.349163in}}%
\pgfpathlineto{\pgfqpoint{1.817709in}{2.344746in}}%
\pgfpathlineto{\pgfqpoint{1.822043in}{2.344482in}}%
\pgfpathlineto{\pgfqpoint{1.826376in}{2.345352in}}%
\pgfpathlineto{\pgfqpoint{1.830710in}{2.347484in}}%
\pgfpathlineto{\pgfqpoint{1.835044in}{2.350963in}}%
\pgfpathlineto{\pgfqpoint{1.839377in}{2.355870in}}%
\pgfpathlineto{\pgfqpoint{1.843711in}{2.362299in}}%
\pgfpathlineto{\pgfqpoint{1.848045in}{2.370377in}}%
\pgfpathlineto{\pgfqpoint{1.852378in}{2.380262in}}%
\pgfpathlineto{\pgfqpoint{1.856712in}{2.392147in}}%
\pgfpathlineto{\pgfqpoint{1.861046in}{2.406249in}}%
\pgfpathlineto{\pgfqpoint{1.865379in}{2.422790in}}%
\pgfpathlineto{\pgfqpoint{1.869713in}{2.441959in}}%
\pgfpathlineto{\pgfqpoint{1.874047in}{2.463857in}}%
\pgfpathlineto{\pgfqpoint{1.878380in}{2.488400in}}%
\pgfpathlineto{\pgfqpoint{1.887048in}{2.543300in}}%
\pgfpathlineto{\pgfqpoint{1.891381in}{2.571131in}}%
\pgfpathlineto{\pgfqpoint{1.895715in}{2.596167in}}%
\pgfpathlineto{\pgfqpoint{1.900049in}{2.614999in}}%
\pgfpathlineto{\pgfqpoint{1.904382in}{2.623653in}}%
\pgfpathlineto{\pgfqpoint{1.908716in}{2.618443in}}%
\pgfpathlineto{\pgfqpoint{1.913050in}{2.597273in}}%
\pgfpathlineto{\pgfqpoint{1.917383in}{2.560906in}}%
\pgfpathlineto{\pgfqpoint{1.921717in}{2.513398in}}%
\pgfpathlineto{\pgfqpoint{1.930384in}{2.410929in}}%
\pgfpathlineto{\pgfqpoint{1.934718in}{2.367861in}}%
\pgfpathlineto{\pgfqpoint{1.939051in}{2.334521in}}%
\pgfpathlineto{\pgfqpoint{1.943385in}{2.311116in}}%
\pgfpathlineto{\pgfqpoint{1.947719in}{2.296329in}}%
\pgfpathlineto{\pgfqpoint{1.952052in}{2.288219in}}%
\pgfpathlineto{\pgfqpoint{1.956386in}{2.284841in}}%
\pgfpathlineto{\pgfqpoint{1.960720in}{2.284560in}}%
\pgfpathlineto{\pgfqpoint{1.965053in}{2.286146in}}%
\pgfpathlineto{\pgfqpoint{1.969387in}{2.288747in}}%
\pgfpathlineto{\pgfqpoint{1.982388in}{2.298268in}}%
\pgfpathlineto{\pgfqpoint{1.999723in}{2.311214in}}%
\pgfpathlineto{\pgfqpoint{2.004056in}{2.314862in}}%
\pgfpathlineto{\pgfqpoint{2.008390in}{2.318921in}}%
\pgfpathlineto{\pgfqpoint{2.012724in}{2.323543in}}%
\pgfpathlineto{\pgfqpoint{2.017057in}{2.328901in}}%
\pgfpathlineto{\pgfqpoint{2.021391in}{2.335194in}}%
\pgfpathlineto{\pgfqpoint{2.025724in}{2.342649in}}%
\pgfpathlineto{\pgfqpoint{2.030058in}{2.351530in}}%
\pgfpathlineto{\pgfqpoint{2.034392in}{2.362147in}}%
\pgfpathlineto{\pgfqpoint{2.038725in}{2.374856in}}%
\pgfpathlineto{\pgfqpoint{2.043059in}{2.390059in}}%
\pgfpathlineto{\pgfqpoint{2.047393in}{2.408201in}}%
\pgfpathlineto{\pgfqpoint{2.051726in}{2.429741in}}%
\pgfpathlineto{\pgfqpoint{2.056060in}{2.455099in}}%
\pgfpathlineto{\pgfqpoint{2.060394in}{2.484554in}}%
\pgfpathlineto{\pgfqpoint{2.064727in}{2.518067in}}%
\pgfpathlineto{\pgfqpoint{2.077728in}{2.631031in}}%
\pgfpathlineto{\pgfqpoint{2.082062in}{2.662009in}}%
\pgfpathlineto{\pgfqpoint{2.086396in}{2.679926in}}%
\pgfpathlineto{\pgfqpoint{2.090729in}{2.677591in}}%
\pgfpathlineto{\pgfqpoint{2.095063in}{2.649918in}}%
\pgfpathlineto{\pgfqpoint{2.099397in}{2.597073in}}%
\pgfpathlineto{\pgfqpoint{2.103730in}{2.526055in}}%
\pgfpathlineto{\pgfqpoint{2.108064in}{2.448795in}}%
\pgfpathlineto{\pgfqpoint{2.112398in}{2.377454in}}%
\pgfpathlineto{\pgfqpoint{2.116731in}{2.320117in}}%
\pgfpathlineto{\pgfqpoint{2.121065in}{2.279398in}}%
\pgfpathlineto{\pgfqpoint{2.125398in}{2.253751in}}%
\pgfpathlineto{\pgfqpoint{2.129732in}{2.239678in}}%
\pgfpathlineto{\pgfqpoint{2.134066in}{2.233431in}}%
\pgfpathlineto{\pgfqpoint{2.138399in}{2.231876in}}%
\pgfpathlineto{\pgfqpoint{2.142733in}{2.232709in}}%
\pgfpathlineto{\pgfqpoint{2.151400in}{2.235891in}}%
\pgfpathlineto{\pgfqpoint{2.155734in}{2.236667in}}%
\pgfpathlineto{\pgfqpoint{2.160068in}{2.236364in}}%
\pgfpathlineto{\pgfqpoint{2.164401in}{2.234794in}}%
\pgfpathlineto{\pgfqpoint{2.168735in}{2.231843in}}%
\pgfpathlineto{\pgfqpoint{2.173069in}{2.227425in}}%
\pgfpathlineto{\pgfqpoint{2.177402in}{2.221455in}}%
\pgfpathlineto{\pgfqpoint{2.181736in}{2.213825in}}%
\pgfpathlineto{\pgfqpoint{2.186070in}{2.204399in}}%
\pgfpathlineto{\pgfqpoint{2.190403in}{2.193011in}}%
\pgfpathlineto{\pgfqpoint{2.194737in}{2.179475in}}%
\pgfpathlineto{\pgfqpoint{2.199071in}{2.163597in}}%
\pgfpathlineto{\pgfqpoint{2.203404in}{2.145217in}}%
\pgfpathlineto{\pgfqpoint{2.207738in}{2.124259in}}%
\pgfpathlineto{\pgfqpoint{2.212072in}{2.100815in}}%
\pgfpathlineto{\pgfqpoint{2.220739in}{2.048473in}}%
\pgfpathlineto{\pgfqpoint{2.225072in}{2.021903in}}%
\pgfpathlineto{\pgfqpoint{2.229406in}{1.997843in}}%
\pgfpathlineto{\pgfqpoint{2.233740in}{1.979386in}}%
\pgfpathlineto{\pgfqpoint{2.238073in}{1.970154in}}%
\pgfpathlineto{\pgfqpoint{2.242407in}{1.973571in}}%
\pgfpathlineto{\pgfqpoint{2.246741in}{1.991733in}}%
\pgfpathlineto{\pgfqpoint{2.251074in}{2.024266in}}%
\pgfpathlineto{\pgfqpoint{2.255408in}{2.067827in}}%
\pgfpathlineto{\pgfqpoint{2.264075in}{2.164892in}}%
\pgfpathlineto{\pgfqpoint{2.268409in}{2.207106in}}%
\pgfpathlineto{\pgfqpoint{2.272743in}{2.240596in}}%
\pgfpathlineto{\pgfqpoint{2.277076in}{2.264752in}}%
\pgfpathlineto{\pgfqpoint{2.281410in}{2.280515in}}%
\pgfpathlineto{\pgfqpoint{2.285744in}{2.289563in}}%
\pgfpathlineto{\pgfqpoint{2.290077in}{2.293697in}}%
\pgfpathlineto{\pgfqpoint{2.294411in}{2.294491in}}%
\pgfpathlineto{\pgfqpoint{2.298745in}{2.293168in}}%
\pgfpathlineto{\pgfqpoint{2.303078in}{2.290595in}}%
\pgfpathlineto{\pgfqpoint{2.307412in}{2.287338in}}%
\pgfpathlineto{\pgfqpoint{2.316079in}{2.279942in}}%
\pgfpathlineto{\pgfqpoint{2.324747in}{2.271917in}}%
\pgfpathlineto{\pgfqpoint{2.333414in}{2.262831in}}%
\pgfpathlineto{\pgfqpoint{2.337747in}{2.257564in}}%
\pgfpathlineto{\pgfqpoint{2.342081in}{2.251583in}}%
\pgfpathlineto{\pgfqpoint{2.346415in}{2.244684in}}%
\pgfpathlineto{\pgfqpoint{2.350748in}{2.236631in}}%
\pgfpathlineto{\pgfqpoint{2.355082in}{2.227159in}}%
\pgfpathlineto{\pgfqpoint{2.359416in}{2.215966in}}%
\pgfpathlineto{\pgfqpoint{2.363749in}{2.202711in}}%
\pgfpathlineto{\pgfqpoint{2.368083in}{2.187022in}}%
\pgfpathlineto{\pgfqpoint{2.372417in}{2.168511in}}%
\pgfpathlineto{\pgfqpoint{2.376750in}{2.146805in}}%
\pgfpathlineto{\pgfqpoint{2.381084in}{2.121612in}}%
\pgfpathlineto{\pgfqpoint{2.385418in}{2.092835in}}%
\pgfpathlineto{\pgfqpoint{2.389751in}{2.060760in}}%
\pgfpathlineto{\pgfqpoint{2.402752in}{1.959455in}}%
\pgfpathlineto{\pgfqpoint{2.407086in}{1.935119in}}%
\pgfpathlineto{\pgfqpoint{2.411420in}{1.924328in}}%
\pgfpathlineto{\pgfqpoint{2.415753in}{1.932697in}}%
\pgfpathlineto{\pgfqpoint{2.420087in}{1.963305in}}%
\pgfpathlineto{\pgfqpoint{2.424421in}{2.014465in}}%
\pgfpathlineto{\pgfqpoint{2.437421in}{2.209046in}}%
\pgfpathlineto{\pgfqpoint{2.441755in}{2.258486in}}%
\pgfpathlineto{\pgfqpoint{2.446089in}{2.293771in}}%
\pgfpathlineto{\pgfqpoint{2.450422in}{2.316270in}}%
\pgfpathlineto{\pgfqpoint{2.454756in}{2.328842in}}%
\pgfpathlineto{\pgfqpoint{2.459090in}{2.334556in}}%
\pgfpathlineto{\pgfqpoint{2.463423in}{2.336018in}}%
\pgfpathlineto{\pgfqpoint{2.467757in}{2.335181in}}%
\pgfpathlineto{\pgfqpoint{2.480758in}{2.330042in}}%
\pgfpathlineto{\pgfqpoint{2.485092in}{2.329314in}}%
\pgfpathlineto{\pgfqpoint{2.489425in}{2.329477in}}%
\pgfpathlineto{\pgfqpoint{2.493759in}{2.330618in}}%
\pgfpathlineto{\pgfqpoint{2.498093in}{2.332794in}}%
\pgfpathlineto{\pgfqpoint{2.502426in}{2.336060in}}%
\pgfpathlineto{\pgfqpoint{2.506760in}{2.340486in}}%
\pgfpathlineto{\pgfqpoint{2.511094in}{2.346174in}}%
\pgfpathlineto{\pgfqpoint{2.515427in}{2.353261in}}%
\pgfpathlineto{\pgfqpoint{2.519761in}{2.361929in}}%
\pgfpathlineto{\pgfqpoint{2.524095in}{2.372403in}}%
\pgfpathlineto{\pgfqpoint{2.528428in}{2.384954in}}%
\pgfpathlineto{\pgfqpoint{2.532762in}{2.399888in}}%
\pgfpathlineto{\pgfqpoint{2.537095in}{2.417529in}}%
\pgfpathlineto{\pgfqpoint{2.541429in}{2.438190in}}%
\pgfpathlineto{\pgfqpoint{2.545763in}{2.462106in}}%
\pgfpathlineto{\pgfqpoint{2.550096in}{2.489332in}}%
\pgfpathlineto{\pgfqpoint{2.554430in}{2.519577in}}%
\pgfpathlineto{\pgfqpoint{2.567431in}{2.614981in}}%
\pgfpathlineto{\pgfqpoint{2.571765in}{2.638416in}}%
\pgfpathlineto{\pgfqpoint{2.576098in}{2.649840in}}%
\pgfpathlineto{\pgfqpoint{2.580432in}{2.644181in}}%
\pgfpathlineto{\pgfqpoint{2.584766in}{2.618401in}}%
\pgfpathlineto{\pgfqpoint{2.589099in}{2.573445in}}%
\pgfpathlineto{\pgfqpoint{2.593433in}{2.514934in}}%
\pgfpathlineto{\pgfqpoint{2.597767in}{2.451669in}}%
\pgfpathlineto{\pgfqpoint{2.602100in}{2.392538in}}%
\pgfpathlineto{\pgfqpoint{2.606434in}{2.343726in}}%
\pgfpathlineto{\pgfqpoint{2.610768in}{2.307673in}}%
\pgfpathlineto{\pgfqpoint{2.615101in}{2.283758in}}%
\pgfpathlineto{\pgfqpoint{2.619435in}{2.269690in}}%
\pgfpathlineto{\pgfqpoint{2.623769in}{2.262729in}}%
\pgfpathlineto{\pgfqpoint{2.628102in}{2.260405in}}%
\pgfpathlineto{\pgfqpoint{2.632436in}{2.260790in}}%
\pgfpathlineto{\pgfqpoint{2.636770in}{2.262516in}}%
\pgfpathlineto{\pgfqpoint{2.645437in}{2.266716in}}%
\pgfpathlineto{\pgfqpoint{2.649770in}{2.268309in}}%
\pgfpathlineto{\pgfqpoint{2.654104in}{2.269287in}}%
\pgfpathlineto{\pgfqpoint{2.658438in}{2.269570in}}%
\pgfpathlineto{\pgfqpoint{2.662771in}{2.269124in}}%
\pgfpathlineto{\pgfqpoint{2.667105in}{2.267932in}}%
\pgfpathlineto{\pgfqpoint{2.671439in}{2.265970in}}%
\pgfpathlineto{\pgfqpoint{2.675772in}{2.263203in}}%
\pgfpathlineto{\pgfqpoint{2.680106in}{2.259565in}}%
\pgfpathlineto{\pgfqpoint{2.684440in}{2.254963in}}%
\pgfpathlineto{\pgfqpoint{2.688773in}{2.249265in}}%
\pgfpathlineto{\pgfqpoint{2.693107in}{2.242298in}}%
\pgfpathlineto{\pgfqpoint{2.697441in}{2.233843in}}%
\pgfpathlineto{\pgfqpoint{2.701774in}{2.223631in}}%
\pgfpathlineto{\pgfqpoint{2.706108in}{2.211343in}}%
\pgfpathlineto{\pgfqpoint{2.710442in}{2.196604in}}%
\pgfpathlineto{\pgfqpoint{2.714775in}{2.179001in}}%
\pgfpathlineto{\pgfqpoint{2.719109in}{2.158103in}}%
\pgfpathlineto{\pgfqpoint{2.723443in}{2.133508in}}%
\pgfpathlineto{\pgfqpoint{2.727776in}{2.104947in}}%
\pgfpathlineto{\pgfqpoint{2.732110in}{2.072439in}}%
\pgfpathlineto{\pgfqpoint{2.740777in}{1.998817in}}%
\pgfpathlineto{\pgfqpoint{2.745111in}{1.962091in}}%
\pgfpathlineto{\pgfqpoint{2.749444in}{1.931013in}}%
\pgfpathlineto{\pgfqpoint{2.753778in}{1.911877in}}%
\pgfpathlineto{\pgfqpoint{2.758112in}{1.911605in}}%
\pgfpathlineto{\pgfqpoint{2.762445in}{1.935422in}}%
\pgfpathlineto{\pgfqpoint{2.766779in}{1.983879in}}%
\pgfpathlineto{\pgfqpoint{2.771113in}{2.051080in}}%
\pgfpathlineto{\pgfqpoint{2.779780in}{2.196861in}}%
\pgfpathlineto{\pgfqpoint{2.784114in}{2.255079in}}%
\pgfpathlineto{\pgfqpoint{2.788447in}{2.297380in}}%
\pgfpathlineto{\pgfqpoint{2.792781in}{2.324704in}}%
\pgfpathlineto{\pgfqpoint{2.797115in}{2.340198in}}%
\pgfpathlineto{\pgfqpoint{2.801448in}{2.347473in}}%
\pgfpathlineto{\pgfqpoint{2.805782in}{2.349671in}}%
\pgfpathlineto{\pgfqpoint{2.810116in}{2.349152in}}%
\pgfpathlineto{\pgfqpoint{2.818783in}{2.345882in}}%
\pgfpathlineto{\pgfqpoint{2.823117in}{2.344807in}}%
\pgfpathlineto{\pgfqpoint{2.827450in}{2.344684in}}%
\pgfpathlineto{\pgfqpoint{2.831784in}{2.345719in}}%
\pgfpathlineto{\pgfqpoint{2.836118in}{2.348031in}}%
\pgfpathlineto{\pgfqpoint{2.840451in}{2.351703in}}%
\pgfpathlineto{\pgfqpoint{2.844785in}{2.356817in}}%
\pgfpathlineto{\pgfqpoint{2.849118in}{2.363471in}}%
\pgfpathlineto{\pgfqpoint{2.853452in}{2.371795in}}%
\pgfpathlineto{\pgfqpoint{2.857786in}{2.381953in}}%
\pgfpathlineto{\pgfqpoint{2.862119in}{2.394142in}}%
\pgfpathlineto{\pgfqpoint{2.866453in}{2.408578in}}%
\pgfpathlineto{\pgfqpoint{2.870787in}{2.425481in}}%
\pgfpathlineto{\pgfqpoint{2.875120in}{2.445030in}}%
\pgfpathlineto{\pgfqpoint{2.879454in}{2.467306in}}%
\pgfpathlineto{\pgfqpoint{2.883788in}{2.492184in}}%
\pgfpathlineto{\pgfqpoint{2.901122in}{2.599217in}}%
\pgfpathlineto{\pgfqpoint{2.905456in}{2.616770in}}%
\pgfpathlineto{\pgfqpoint{2.909790in}{2.623580in}}%
\pgfpathlineto{\pgfqpoint{2.914123in}{2.616115in}}%
\pgfpathlineto{\pgfqpoint{2.918457in}{2.592638in}}%
\pgfpathlineto{\pgfqpoint{2.922791in}{2.554393in}}%
\pgfpathlineto{\pgfqpoint{2.927124in}{2.505859in}}%
\pgfpathlineto{\pgfqpoint{2.935792in}{2.404251in}}%
\pgfpathlineto{\pgfqpoint{2.940125in}{2.362564in}}%
\pgfpathlineto{\pgfqpoint{2.944459in}{2.330741in}}%
\pgfpathlineto{\pgfqpoint{2.948793in}{2.308712in}}%
\pgfpathlineto{\pgfqpoint{2.953126in}{2.295025in}}%
\pgfpathlineto{\pgfqpoint{2.957460in}{2.287715in}}%
\pgfpathlineto{\pgfqpoint{2.961793in}{2.284874in}}%
\pgfpathlineto{\pgfqpoint{2.966127in}{2.284928in}}%
\pgfpathlineto{\pgfqpoint{2.970461in}{2.286706in}}%
\pgfpathlineto{\pgfqpoint{2.974794in}{2.289407in}}%
\pgfpathlineto{\pgfqpoint{2.992129in}{2.302217in}}%
\pgfpathlineto{\pgfqpoint{3.005130in}{2.312154in}}%
\pgfpathlineto{\pgfqpoint{3.009464in}{2.315913in}}%
\pgfpathlineto{\pgfqpoint{3.013797in}{2.320115in}}%
\pgfpathlineto{\pgfqpoint{3.018131in}{2.324918in}}%
\pgfpathlineto{\pgfqpoint{3.022465in}{2.330501in}}%
\pgfpathlineto{\pgfqpoint{3.026798in}{2.337069in}}%
\pgfpathlineto{\pgfqpoint{3.031132in}{2.344858in}}%
\pgfpathlineto{\pgfqpoint{3.035466in}{2.354143in}}%
\pgfpathlineto{\pgfqpoint{3.039799in}{2.365243in}}%
\pgfpathlineto{\pgfqpoint{3.044133in}{2.378525in}}%
\pgfpathlineto{\pgfqpoint{3.048467in}{2.394402in}}%
\pgfpathlineto{\pgfqpoint{3.052800in}{2.413322in}}%
\pgfpathlineto{\pgfqpoint{3.057134in}{2.435738in}}%
\pgfpathlineto{\pgfqpoint{3.061467in}{2.462043in}}%
\pgfpathlineto{\pgfqpoint{3.065801in}{2.492454in}}%
\pgfpathlineto{\pgfqpoint{3.070135in}{2.526809in}}%
\pgfpathlineto{\pgfqpoint{3.083136in}{2.638978in}}%
\pgfpathlineto{\pgfqpoint{3.087469in}{2.667259in}}%
\pgfpathlineto{\pgfqpoint{3.091803in}{2.680729in}}%
\pgfpathlineto{\pgfqpoint{3.096137in}{2.672483in}}%
\pgfpathlineto{\pgfqpoint{3.100470in}{2.638489in}}%
\pgfpathlineto{\pgfqpoint{3.104804in}{2.580556in}}%
\pgfpathlineto{\pgfqpoint{3.117805in}{2.362476in}}%
\pgfpathlineto{\pgfqpoint{3.122139in}{2.309314in}}%
\pgfpathlineto{\pgfqpoint{3.126472in}{2.272601in}}%
\pgfpathlineto{\pgfqpoint{3.130806in}{2.250145in}}%
\pgfpathlineto{\pgfqpoint{3.135140in}{2.238293in}}%
\pgfpathlineto{\pgfqpoint{3.139473in}{2.233423in}}%
\pgfpathlineto{\pgfqpoint{3.143807in}{2.232610in}}%
\pgfpathlineto{\pgfqpoint{3.148141in}{2.233754in}}%
\pgfpathlineto{\pgfqpoint{3.152474in}{2.235460in}}%
\pgfpathlineto{\pgfqpoint{3.156808in}{2.236857in}}%
\pgfpathlineto{\pgfqpoint{3.161141in}{2.237426in}}%
\pgfpathlineto{\pgfqpoint{3.165475in}{2.236874in}}%
\pgfpathlineto{\pgfqpoint{3.169809in}{2.235036in}}%
\pgfpathlineto{\pgfqpoint{3.174142in}{2.231809in}}%
\pgfpathlineto{\pgfqpoint{3.178476in}{2.227110in}}%
\pgfpathlineto{\pgfqpoint{3.182810in}{2.220849in}}%
\pgfpathlineto{\pgfqpoint{3.187143in}{2.212914in}}%
\pgfpathlineto{\pgfqpoint{3.191477in}{2.203160in}}%
\pgfpathlineto{\pgfqpoint{3.195811in}{2.191415in}}%
\pgfpathlineto{\pgfqpoint{3.200144in}{2.177483in}}%
\pgfpathlineto{\pgfqpoint{3.204478in}{2.161172in}}%
\pgfpathlineto{\pgfqpoint{3.208812in}{2.142322in}}%
\pgfpathlineto{\pgfqpoint{3.213145in}{2.120872in}}%
\pgfpathlineto{\pgfqpoint{3.217479in}{2.096944in}}%
\pgfpathlineto{\pgfqpoint{3.226146in}{2.043914in}}%
\pgfpathlineto{\pgfqpoint{3.230480in}{2.017348in}}%
\pgfpathlineto{\pgfqpoint{3.234814in}{1.993713in}}%
\pgfpathlineto{\pgfqpoint{3.239147in}{1.976247in}}%
\pgfpathlineto{\pgfqpoint{3.243481in}{1.968649in}}%
\pgfpathlineto{\pgfqpoint{3.247815in}{1.974271in}}%
\pgfpathlineto{\pgfqpoint{3.252148in}{1.994931in}}%
\pgfpathlineto{\pgfqpoint{3.256482in}{2.029793in}}%
\pgfpathlineto{\pgfqpoint{3.260816in}{2.075031in}}%
\pgfpathlineto{\pgfqpoint{3.269483in}{2.172573in}}%
\pgfpathlineto{\pgfqpoint{3.273816in}{2.213834in}}%
\pgfpathlineto{\pgfqpoint{3.278150in}{2.246023in}}%
\pgfpathlineto{\pgfqpoint{3.282484in}{2.268848in}}%
\pgfpathlineto{\pgfqpoint{3.286817in}{2.283449in}}%
\pgfpathlineto{\pgfqpoint{3.291151in}{2.291589in}}%
\pgfpathlineto{\pgfqpoint{3.295485in}{2.295073in}}%
\pgfpathlineto{\pgfqpoint{3.299818in}{2.295440in}}%
\pgfpathlineto{\pgfqpoint{3.304152in}{2.293865in}}%
\pgfpathlineto{\pgfqpoint{3.308486in}{2.291167in}}%
\pgfpathlineto{\pgfqpoint{3.317153in}{2.284293in}}%
\pgfpathlineto{\pgfqpoint{3.325820in}{2.276728in}}%
\pgfpathlineto{\pgfqpoint{3.334488in}{2.268507in}}%
\pgfpathlineto{\pgfqpoint{3.338821in}{2.263909in}}%
\pgfpathlineto{\pgfqpoint{3.343155in}{2.258792in}}%
\pgfpathlineto{\pgfqpoint{3.347489in}{2.252977in}}%
\pgfpathlineto{\pgfqpoint{3.351822in}{2.246264in}}%
\pgfpathlineto{\pgfqpoint{3.356156in}{2.238421in}}%
\pgfpathlineto{\pgfqpoint{3.360490in}{2.229185in}}%
\pgfpathlineto{\pgfqpoint{3.364823in}{2.218256in}}%
\pgfpathlineto{\pgfqpoint{3.369157in}{2.205297in}}%
\pgfpathlineto{\pgfqpoint{3.373490in}{2.189934in}}%
\pgfpathlineto{\pgfqpoint{3.377824in}{2.171770in}}%
\pgfpathlineto{\pgfqpoint{3.382158in}{2.150417in}}%
\pgfpathlineto{\pgfqpoint{3.386491in}{2.125552in}}%
\pgfpathlineto{\pgfqpoint{3.390825in}{2.097030in}}%
\pgfpathlineto{\pgfqpoint{3.395159in}{2.065052in}}%
\pgfpathlineto{\pgfqpoint{3.408160in}{1.961761in}}%
\pgfpathlineto{\pgfqpoint{3.412493in}{1.935512in}}%
\pgfpathlineto{\pgfqpoint{3.416827in}{1.922146in}}%
\pgfpathlineto{\pgfqpoint{3.421161in}{1.927600in}}%
\pgfpathlineto{\pgfqpoint{3.425494in}{1.955577in}}%
\pgfpathlineto{\pgfqpoint{3.429828in}{2.005127in}}%
\pgfpathlineto{\pgfqpoint{3.434162in}{2.069730in}}%
\pgfpathlineto{\pgfqpoint{3.438495in}{2.139136in}}%
\pgfpathlineto{\pgfqpoint{3.442829in}{2.203169in}}%
\pgfpathlineto{\pgfqpoint{3.447163in}{2.255055in}}%
\pgfpathlineto{\pgfqpoint{3.451496in}{2.292483in}}%
\pgfpathlineto{\pgfqpoint{3.455830in}{2.316593in}}%
\pgfpathlineto{\pgfqpoint{3.460164in}{2.330243in}}%
\pgfpathlineto{\pgfqpoint{3.464497in}{2.336605in}}%
\pgfpathlineto{\pgfqpoint{3.468831in}{2.338419in}}%
\pgfpathlineto{\pgfqpoint{3.473164in}{2.337751in}}%
\pgfpathlineto{\pgfqpoint{3.486165in}{2.332759in}}%
\pgfpathlineto{\pgfqpoint{3.490499in}{2.332095in}}%
\pgfpathlineto{\pgfqpoint{3.494833in}{2.332365in}}%
\pgfpathlineto{\pgfqpoint{3.499166in}{2.333665in}}%
\pgfpathlineto{\pgfqpoint{3.503500in}{2.336059in}}%
\pgfpathlineto{\pgfqpoint{3.507834in}{2.339607in}}%
\pgfpathlineto{\pgfqpoint{3.512167in}{2.344388in}}%
\pgfpathlineto{\pgfqpoint{3.516501in}{2.350506in}}%
\pgfpathlineto{\pgfqpoint{3.520835in}{2.358108in}}%
\pgfpathlineto{\pgfqpoint{3.525168in}{2.367380in}}%
\pgfpathlineto{\pgfqpoint{3.529502in}{2.378554in}}%
\pgfpathlineto{\pgfqpoint{3.533836in}{2.391898in}}%
\pgfpathlineto{\pgfqpoint{3.538169in}{2.407712in}}%
\pgfpathlineto{\pgfqpoint{3.542503in}{2.426301in}}%
\pgfpathlineto{\pgfqpoint{3.546837in}{2.447929in}}%
\pgfpathlineto{\pgfqpoint{3.551170in}{2.472748in}}%
\pgfpathlineto{\pgfqpoint{3.555504in}{2.500671in}}%
\pgfpathlineto{\pgfqpoint{3.564171in}{2.563108in}}%
\pgfpathlineto{\pgfqpoint{3.568505in}{2.594284in}}%
\pgfpathlineto{\pgfqpoint{3.572839in}{2.621365in}}%
\pgfpathlineto{\pgfqpoint{3.577172in}{2.639844in}}%
\pgfpathlineto{\pgfqpoint{3.581506in}{2.644664in}}%
\pgfpathlineto{\pgfqpoint{3.585839in}{2.631583in}}%
\pgfpathlineto{\pgfqpoint{3.590173in}{2.599073in}}%
\pgfpathlineto{\pgfqpoint{3.594507in}{2.549814in}}%
\pgfpathlineto{\pgfqpoint{3.603174in}{2.430051in}}%
\pgfpathlineto{\pgfqpoint{3.607508in}{2.375981in}}%
\pgfpathlineto{\pgfqpoint{3.611841in}{2.332973in}}%
\pgfpathlineto{\pgfqpoint{3.616175in}{2.302222in}}%
\pgfpathlineto{\pgfqpoint{3.620509in}{2.282477in}}%
\pgfpathlineto{\pgfqpoint{3.624842in}{2.271338in}}%
\pgfpathlineto{\pgfqpoint{3.629176in}{2.266251in}}%
\pgfpathlineto{\pgfqpoint{3.633510in}{2.265024in}}%
\pgfpathlineto{\pgfqpoint{3.637843in}{2.265999in}}%
\pgfpathlineto{\pgfqpoint{3.646511in}{2.270333in}}%
\pgfpathlineto{\pgfqpoint{3.650844in}{2.272491in}}%
\pgfpathlineto{\pgfqpoint{3.655178in}{2.274241in}}%
\pgfpathlineto{\pgfqpoint{3.659512in}{2.275460in}}%
\pgfpathlineto{\pgfqpoint{3.663845in}{2.276099in}}%
\pgfpathlineto{\pgfqpoint{3.668179in}{2.276148in}}%
\pgfpathlineto{\pgfqpoint{3.672513in}{2.275610in}}%
\pgfpathlineto{\pgfqpoint{3.676846in}{2.274484in}}%
\pgfpathlineto{\pgfqpoint{3.681180in}{2.272758in}}%
\pgfpathlineto{\pgfqpoint{3.685513in}{2.270398in}}%
\pgfpathlineto{\pgfqpoint{3.689847in}{2.267345in}}%
\pgfpathlineto{\pgfqpoint{3.694181in}{2.263515in}}%
\pgfpathlineto{\pgfqpoint{3.698514in}{2.258787in}}%
\pgfpathlineto{\pgfqpoint{3.702848in}{2.253006in}}%
\pgfpathlineto{\pgfqpoint{3.707182in}{2.245974in}}%
\pgfpathlineto{\pgfqpoint{3.711515in}{2.237447in}}%
\pgfpathlineto{\pgfqpoint{3.715849in}{2.227127in}}%
\pgfpathlineto{\pgfqpoint{3.720183in}{2.214657in}}%
\pgfpathlineto{\pgfqpoint{3.724516in}{2.199622in}}%
\pgfpathlineto{\pgfqpoint{3.728850in}{2.181556in}}%
\pgfpathlineto{\pgfqpoint{3.733184in}{2.159957in}}%
\pgfpathlineto{\pgfqpoint{3.737517in}{2.134344in}}%
\pgfpathlineto{\pgfqpoint{3.741851in}{2.104354in}}%
\pgfpathlineto{\pgfqpoint{3.746185in}{2.069914in}}%
\pgfpathlineto{\pgfqpoint{3.754852in}{1.990775in}}%
\pgfpathlineto{\pgfqpoint{3.759186in}{1.950722in}}%
\pgfpathlineto{\pgfqpoint{3.763519in}{1.916545in}}%
\pgfpathlineto{\pgfqpoint{3.767853in}{1.895449in}}%
\pgfpathlineto{\pgfqpoint{3.772187in}{1.895492in}}%
\pgfpathlineto{\pgfqpoint{3.776520in}{1.922779in}}%
\pgfpathlineto{\pgfqpoint{3.780854in}{1.977783in}}%
\pgfpathlineto{\pgfqpoint{3.785187in}{2.053222in}}%
\pgfpathlineto{\pgfqpoint{3.789521in}{2.135961in}}%
\pgfpathlineto{\pgfqpoint{3.793855in}{2.212377in}}%
\pgfpathlineto{\pgfqpoint{3.798188in}{2.273428in}}%
\pgfpathlineto{\pgfqpoint{3.802522in}{2.316308in}}%
\pgfpathlineto{\pgfqpoint{3.806856in}{2.342889in}}%
\pgfpathlineto{\pgfqpoint{3.811189in}{2.357149in}}%
\pgfpathlineto{\pgfqpoint{3.815523in}{2.363240in}}%
\pgfpathlineto{\pgfqpoint{3.819857in}{2.364566in}}%
\pgfpathlineto{\pgfqpoint{3.824190in}{2.363581in}}%
\pgfpathlineto{\pgfqpoint{3.828524in}{2.361914in}}%
\pgfpathlineto{\pgfqpoint{3.832858in}{2.360579in}}%
\pgfpathlineto{\pgfqpoint{3.837191in}{2.360174in}}%
\pgfpathlineto{\pgfqpoint{3.841525in}{2.361040in}}%
\pgfpathlineto{\pgfqpoint{3.845859in}{2.363366in}}%
\pgfpathlineto{\pgfqpoint{3.850192in}{2.367264in}}%
\pgfpathlineto{\pgfqpoint{3.854526in}{2.372819in}}%
\pgfpathlineto{\pgfqpoint{3.858860in}{2.380114in}}%
\pgfpathlineto{\pgfqpoint{3.863193in}{2.389250in}}%
\pgfpathlineto{\pgfqpoint{3.867527in}{2.400345in}}%
\pgfpathlineto{\pgfqpoint{3.871861in}{2.413527in}}%
\pgfpathlineto{\pgfqpoint{3.876194in}{2.428914in}}%
\pgfpathlineto{\pgfqpoint{3.880528in}{2.446580in}}%
\pgfpathlineto{\pgfqpoint{3.884862in}{2.466498in}}%
\pgfpathlineto{\pgfqpoint{3.889195in}{2.488458in}}%
\pgfpathlineto{\pgfqpoint{3.902196in}{2.559292in}}%
\pgfpathlineto{\pgfqpoint{3.906530in}{2.579493in}}%
\pgfpathlineto{\pgfqpoint{3.910863in}{2.593949in}}%
\pgfpathlineto{\pgfqpoint{3.915197in}{2.599683in}}%
\pgfpathlineto{\pgfqpoint{3.919531in}{2.594115in}}%
\pgfpathlineto{\pgfqpoint{3.923864in}{2.575951in}}%
\pgfpathlineto{\pgfqpoint{3.928198in}{2.545972in}}%
\pgfpathlineto{\pgfqpoint{3.932532in}{2.507239in}}%
\pgfpathlineto{\pgfqpoint{3.941199in}{2.422602in}}%
\pgfpathlineto{\pgfqpoint{3.945533in}{2.385746in}}%
\pgfpathlineto{\pgfqpoint{3.949866in}{2.356184in}}%
\pgfpathlineto{\pgfqpoint{3.954200in}{2.334502in}}%
\pgfpathlineto{\pgfqpoint{3.958534in}{2.320050in}}%
\pgfpathlineto{\pgfqpoint{3.962867in}{2.311543in}}%
\pgfpathlineto{\pgfqpoint{3.967201in}{2.307544in}}%
\pgfpathlineto{\pgfqpoint{3.971535in}{2.306759in}}%
\pgfpathlineto{\pgfqpoint{3.975868in}{2.308154in}}%
\pgfpathlineto{\pgfqpoint{3.980202in}{2.310984in}}%
\pgfpathlineto{\pgfqpoint{3.984536in}{2.314754in}}%
\pgfpathlineto{\pgfqpoint{3.988869in}{2.319175in}}%
\pgfpathlineto{\pgfqpoint{3.993203in}{2.324114in}}%
\pgfpathlineto{\pgfqpoint{3.997536in}{2.329551in}}%
\pgfpathlineto{\pgfqpoint{4.001870in}{2.335550in}}%
\pgfpathlineto{\pgfqpoint{4.006204in}{2.342236in}}%
\pgfpathlineto{\pgfqpoint{4.010537in}{2.349777in}}%
\pgfpathlineto{\pgfqpoint{4.014871in}{2.358381in}}%
\pgfpathlineto{\pgfqpoint{4.019205in}{2.368287in}}%
\pgfpathlineto{\pgfqpoint{4.023538in}{2.379762in}}%
\pgfpathlineto{\pgfqpoint{4.027872in}{2.393093in}}%
\pgfpathlineto{\pgfqpoint{4.032206in}{2.408578in}}%
\pgfpathlineto{\pgfqpoint{4.036539in}{2.426501in}}%
\pgfpathlineto{\pgfqpoint{4.040873in}{2.447093in}}%
\pgfpathlineto{\pgfqpoint{4.045207in}{2.470464in}}%
\pgfpathlineto{\pgfqpoint{4.049540in}{2.496498in}}%
\pgfpathlineto{\pgfqpoint{4.062541in}{2.582344in}}%
\pgfpathlineto{\pgfqpoint{4.066875in}{2.606991in}}%
\pgfpathlineto{\pgfqpoint{4.071209in}{2.624052in}}%
\pgfpathlineto{\pgfqpoint{4.075542in}{2.629262in}}%
\pgfpathlineto{\pgfqpoint{4.079876in}{2.619008in}}%
\pgfpathlineto{\pgfqpoint{4.084210in}{2.591819in}}%
\pgfpathlineto{\pgfqpoint{4.088543in}{2.549583in}}%
\pgfpathlineto{\pgfqpoint{4.101544in}{2.392877in}}%
\pgfpathlineto{\pgfqpoint{4.105878in}{2.351523in}}%
\pgfpathlineto{\pgfqpoint{4.110211in}{2.320765in}}%
\pgfpathlineto{\pgfqpoint{4.114545in}{2.300075in}}%
\pgfpathlineto{\pgfqpoint{4.118879in}{2.287674in}}%
\pgfpathlineto{\pgfqpoint{4.123212in}{2.281416in}}%
\pgfpathlineto{\pgfqpoint{4.127546in}{2.279331in}}%
\pgfpathlineto{\pgfqpoint{4.131880in}{2.279846in}}%
\pgfpathlineto{\pgfqpoint{4.136213in}{2.281824in}}%
\pgfpathlineto{\pgfqpoint{4.144881in}{2.287412in}}%
\pgfpathlineto{\pgfqpoint{4.153548in}{2.293013in}}%
\pgfpathlineto{\pgfqpoint{4.162215in}{2.297950in}}%
\pgfpathlineto{\pgfqpoint{4.175216in}{2.304922in}}%
\pgfpathlineto{\pgfqpoint{4.183884in}{2.310323in}}%
\pgfpathlineto{\pgfqpoint{4.188217in}{2.313572in}}%
\pgfpathlineto{\pgfqpoint{4.192551in}{2.317356in}}%
\pgfpathlineto{\pgfqpoint{4.196884in}{2.321823in}}%
\pgfpathlineto{\pgfqpoint{4.201218in}{2.327146in}}%
\pgfpathlineto{\pgfqpoint{4.205552in}{2.333529in}}%
\pgfpathlineto{\pgfqpoint{4.209885in}{2.341214in}}%
\pgfpathlineto{\pgfqpoint{4.214219in}{2.350492in}}%
\pgfpathlineto{\pgfqpoint{4.218553in}{2.361706in}}%
\pgfpathlineto{\pgfqpoint{4.222886in}{2.375258in}}%
\pgfpathlineto{\pgfqpoint{4.227220in}{2.391615in}}%
\pgfpathlineto{\pgfqpoint{4.231554in}{2.411292in}}%
\pgfpathlineto{\pgfqpoint{4.235887in}{2.434830in}}%
\pgfpathlineto{\pgfqpoint{4.240221in}{2.462727in}}%
\pgfpathlineto{\pgfqpoint{4.244555in}{2.495310in}}%
\pgfpathlineto{\pgfqpoint{4.248888in}{2.532506in}}%
\pgfpathlineto{\pgfqpoint{4.261889in}{2.656290in}}%
\pgfpathlineto{\pgfqpoint{4.266223in}{2.687692in}}%
\pgfpathlineto{\pgfqpoint{4.270557in}{2.701960in}}%
\pgfpathlineto{\pgfqpoint{4.274890in}{2.690778in}}%
\pgfpathlineto{\pgfqpoint{4.279224in}{2.649541in}}%
\pgfpathlineto{\pgfqpoint{4.283558in}{2.581144in}}%
\pgfpathlineto{\pgfqpoint{4.292225in}{2.411397in}}%
\pgfpathlineto{\pgfqpoint{4.296559in}{2.338179in}}%
\pgfpathlineto{\pgfqpoint{4.300892in}{2.283548in}}%
\pgfpathlineto{\pgfqpoint{4.305226in}{2.247705in}}%
\pgfpathlineto{\pgfqpoint{4.309559in}{2.227139in}}%
\pgfpathlineto{\pgfqpoint{4.313893in}{2.217265in}}%
\pgfpathlineto{\pgfqpoint{4.318227in}{2.213972in}}%
\pgfpathlineto{\pgfqpoint{4.322560in}{2.214156in}}%
\pgfpathlineto{\pgfqpoint{4.331228in}{2.217278in}}%
\pgfpathlineto{\pgfqpoint{4.335561in}{2.218069in}}%
\pgfpathlineto{\pgfqpoint{4.339895in}{2.217620in}}%
\pgfpathlineto{\pgfqpoint{4.344229in}{2.215676in}}%
\pgfpathlineto{\pgfqpoint{4.348562in}{2.212100in}}%
\pgfpathlineto{\pgfqpoint{4.352896in}{2.206803in}}%
\pgfpathlineto{\pgfqpoint{4.357230in}{2.199714in}}%
\pgfpathlineto{\pgfqpoint{4.361563in}{2.190757in}}%
\pgfpathlineto{\pgfqpoint{4.365897in}{2.179848in}}%
\pgfpathlineto{\pgfqpoint{4.370231in}{2.166901in}}%
\pgfpathlineto{\pgfqpoint{4.374564in}{2.151850in}}%
\pgfpathlineto{\pgfqpoint{4.378898in}{2.134687in}}%
\pgfpathlineto{\pgfqpoint{4.383232in}{2.115514in}}%
\pgfpathlineto{\pgfqpoint{4.391899in}{2.072605in}}%
\pgfpathlineto{\pgfqpoint{4.400566in}{2.029632in}}%
\pgfpathlineto{\pgfqpoint{4.404900in}{2.012209in}}%
\pgfpathlineto{\pgfqpoint{4.409233in}{2.000630in}}%
\pgfpathlineto{\pgfqpoint{4.413567in}{1.997438in}}%
\pgfpathlineto{\pgfqpoint{4.417901in}{2.004654in}}%
\pgfpathlineto{\pgfqpoint{4.422234in}{2.023050in}}%
\pgfpathlineto{\pgfqpoint{4.426568in}{2.051581in}}%
\pgfpathlineto{\pgfqpoint{4.430902in}{2.087339in}}%
\pgfpathlineto{\pgfqpoint{4.439569in}{2.163880in}}%
\pgfpathlineto{\pgfqpoint{4.443903in}{2.197024in}}%
\pgfpathlineto{\pgfqpoint{4.448236in}{2.223698in}}%
\pgfpathlineto{\pgfqpoint{4.452570in}{2.243397in}}%
\pgfpathlineto{\pgfqpoint{4.456904in}{2.256644in}}%
\pgfpathlineto{\pgfqpoint{4.461237in}{2.264507in}}%
\pgfpathlineto{\pgfqpoint{4.465571in}{2.268196in}}%
\pgfpathlineto{\pgfqpoint{4.469905in}{2.268824in}}%
\pgfpathlineto{\pgfqpoint{4.474238in}{2.267286in}}%
\pgfpathlineto{\pgfqpoint{4.478572in}{2.264238in}}%
\pgfpathlineto{\pgfqpoint{4.482906in}{2.260114in}}%
\pgfpathlineto{\pgfqpoint{4.487239in}{2.255166in}}%
\pgfpathlineto{\pgfqpoint{4.491573in}{2.249503in}}%
\pgfpathlineto{\pgfqpoint{4.495907in}{2.243127in}}%
\pgfpathlineto{\pgfqpoint{4.500240in}{2.235960in}}%
\pgfpathlineto{\pgfqpoint{4.504574in}{2.227863in}}%
\pgfpathlineto{\pgfqpoint{4.508907in}{2.218655in}}%
\pgfpathlineto{\pgfqpoint{4.513241in}{2.208120in}}%
\pgfpathlineto{\pgfqpoint{4.517575in}{2.196019in}}%
\pgfpathlineto{\pgfqpoint{4.521908in}{2.182101in}}%
\pgfpathlineto{\pgfqpoint{4.526242in}{2.166126in}}%
\pgfpathlineto{\pgfqpoint{4.530576in}{2.147891in}}%
\pgfpathlineto{\pgfqpoint{4.534909in}{2.127285in}}%
\pgfpathlineto{\pgfqpoint{4.539243in}{2.104364in}}%
\pgfpathlineto{\pgfqpoint{4.547910in}{2.053346in}}%
\pgfpathlineto{\pgfqpoint{4.552244in}{2.027377in}}%
\pgfpathlineto{\pgfqpoint{4.556578in}{2.003663in}}%
\pgfpathlineto{\pgfqpoint{4.560911in}{1.985083in}}%
\pgfpathlineto{\pgfqpoint{4.565245in}{1.975044in}}%
\pgfpathlineto{\pgfqpoint{4.569579in}{1.976842in}}%
\pgfpathlineto{\pgfqpoint{4.573912in}{1.992642in}}%
\pgfpathlineto{\pgfqpoint{4.578246in}{2.022390in}}%
\pgfpathlineto{\pgfqpoint{4.582580in}{2.063253in}}%
\pgfpathlineto{\pgfqpoint{4.591247in}{2.157035in}}%
\pgfpathlineto{\pgfqpoint{4.595581in}{2.198976in}}%
\pgfpathlineto{\pgfqpoint{4.599914in}{2.232882in}}%
\pgfpathlineto{\pgfqpoint{4.604248in}{2.257836in}}%
\pgfpathlineto{\pgfqpoint{4.608582in}{2.274503in}}%
\pgfpathlineto{\pgfqpoint{4.612915in}{2.284382in}}%
\pgfpathlineto{\pgfqpoint{4.617249in}{2.289177in}}%
\pgfpathlineto{\pgfqpoint{4.621582in}{2.290435in}}%
\pgfpathlineto{\pgfqpoint{4.625916in}{2.289384in}}%
\pgfpathlineto{\pgfqpoint{4.630250in}{2.286912in}}%
\pgfpathlineto{\pgfqpoint{4.634583in}{2.283612in}}%
\pgfpathlineto{\pgfqpoint{4.643251in}{2.275768in}}%
\pgfpathlineto{\pgfqpoint{4.651918in}{2.266887in}}%
\pgfpathlineto{\pgfqpoint{4.656252in}{2.261953in}}%
\pgfpathlineto{\pgfqpoint{4.660585in}{2.256526in}}%
\pgfpathlineto{\pgfqpoint{4.664919in}{2.250440in}}%
\pgfpathlineto{\pgfqpoint{4.669253in}{2.243497in}}%
\pgfpathlineto{\pgfqpoint{4.673586in}{2.235472in}}%
\pgfpathlineto{\pgfqpoint{4.677920in}{2.226108in}}%
\pgfpathlineto{\pgfqpoint{4.682254in}{2.215117in}}%
\pgfpathlineto{\pgfqpoint{4.686587in}{2.202177in}}%
\pgfpathlineto{\pgfqpoint{4.690921in}{2.186941in}}%
\pgfpathlineto{\pgfqpoint{4.695255in}{2.169049in}}%
\pgfpathlineto{\pgfqpoint{4.699588in}{2.148162in}}%
\pgfpathlineto{\pgfqpoint{4.703922in}{2.124022in}}%
\pgfpathlineto{\pgfqpoint{4.708256in}{2.096553in}}%
\pgfpathlineto{\pgfqpoint{4.712589in}{2.066031in}}%
\pgfpathlineto{\pgfqpoint{4.725590in}{1.969617in}}%
\pgfpathlineto{\pgfqpoint{4.729924in}{1.945895in}}%
\pgfpathlineto{\pgfqpoint{4.734257in}{1.934336in}}%
\pgfpathlineto{\pgfqpoint{4.738591in}{1.940105in}}%
\pgfpathlineto{\pgfqpoint{4.742925in}{1.966299in}}%
\pgfpathlineto{\pgfqpoint{4.747258in}{2.011937in}}%
\pgfpathlineto{\pgfqpoint{4.751592in}{2.071258in}}%
\pgfpathlineto{\pgfqpoint{4.755926in}{2.135277in}}%
\pgfpathlineto{\pgfqpoint{4.760259in}{2.194961in}}%
\pgfpathlineto{\pgfqpoint{4.764593in}{2.244077in}}%
\pgfpathlineto{\pgfqpoint{4.768927in}{2.280221in}}%
\pgfpathlineto{\pgfqpoint{4.773260in}{2.304091in}}%
\pgfpathlineto{\pgfqpoint{4.777594in}{2.318055in}}%
\pgfpathlineto{\pgfqpoint{4.781928in}{2.324904in}}%
\pgfpathlineto{\pgfqpoint{4.786261in}{2.327140in}}%
\pgfpathlineto{\pgfqpoint{4.790595in}{2.326706in}}%
\pgfpathlineto{\pgfqpoint{4.794929in}{2.324973in}}%
\pgfpathlineto{\pgfqpoint{4.803596in}{2.320888in}}%
\pgfpathlineto{\pgfqpoint{4.807930in}{2.319414in}}%
\pgfpathlineto{\pgfqpoint{4.812263in}{2.318596in}}%
\pgfpathlineto{\pgfqpoint{4.816597in}{2.318517in}}%
\pgfpathlineto{\pgfqpoint{4.820930in}{2.319214in}}%
\pgfpathlineto{\pgfqpoint{4.825264in}{2.320711in}}%
\pgfpathlineto{\pgfqpoint{4.829598in}{2.323038in}}%
\pgfpathlineto{\pgfqpoint{4.833931in}{2.326242in}}%
\pgfpathlineto{\pgfqpoint{4.838265in}{2.330400in}}%
\pgfpathlineto{\pgfqpoint{4.842599in}{2.335620in}}%
\pgfpathlineto{\pgfqpoint{4.846932in}{2.342053in}}%
\pgfpathlineto{\pgfqpoint{4.851266in}{2.349892in}}%
\pgfpathlineto{\pgfqpoint{4.855600in}{2.359376in}}%
\pgfpathlineto{\pgfqpoint{4.859933in}{2.370798in}}%
\pgfpathlineto{\pgfqpoint{4.864267in}{2.384500in}}%
\pgfpathlineto{\pgfqpoint{4.868601in}{2.400866in}}%
\pgfpathlineto{\pgfqpoint{4.872934in}{2.420309in}}%
\pgfpathlineto{\pgfqpoint{4.877268in}{2.443227in}}%
\pgfpathlineto{\pgfqpoint{4.881602in}{2.469929in}}%
\pgfpathlineto{\pgfqpoint{4.885935in}{2.500499in}}%
\pgfpathlineto{\pgfqpoint{4.890269in}{2.534586in}}%
\pgfpathlineto{\pgfqpoint{4.898936in}{2.607696in}}%
\pgfpathlineto{\pgfqpoint{4.903270in}{2.640598in}}%
\pgfpathlineto{\pgfqpoint{4.907604in}{2.664256in}}%
\pgfpathlineto{\pgfqpoint{4.911937in}{2.672075in}}%
\pgfpathlineto{\pgfqpoint{4.916271in}{2.658146in}}%
\pgfpathlineto{\pgfqpoint{4.920605in}{2.619967in}}%
\pgfpathlineto{\pgfqpoint{4.924938in}{2.560783in}}%
\pgfpathlineto{\pgfqpoint{4.933605in}{2.417859in}}%
\pgfpathlineto{\pgfqpoint{4.937939in}{2.355486in}}%
\pgfpathlineto{\pgfqpoint{4.942273in}{2.307684in}}%
\pgfpathlineto{\pgfqpoint{4.946606in}{2.275078in}}%
\pgfpathlineto{\pgfqpoint{4.950940in}{2.255350in}}%
\pgfpathlineto{\pgfqpoint{4.955274in}{2.245096in}}%
\pgfpathlineto{\pgfqpoint{4.959607in}{2.241048in}}%
\pgfpathlineto{\pgfqpoint{4.963941in}{2.240601in}}%
\pgfpathlineto{\pgfqpoint{4.968275in}{2.241888in}}%
\pgfpathlineto{\pgfqpoint{4.976942in}{2.245171in}}%
\pgfpathlineto{\pgfqpoint{4.981276in}{2.245935in}}%
\pgfpathlineto{\pgfqpoint{4.985609in}{2.245702in}}%
\pgfpathlineto{\pgfqpoint{4.989943in}{2.244327in}}%
\pgfpathlineto{\pgfqpoint{4.994277in}{2.241721in}}%
\pgfpathlineto{\pgfqpoint{4.998610in}{2.237811in}}%
\pgfpathlineto{\pgfqpoint{5.002944in}{2.232515in}}%
\pgfpathlineto{\pgfqpoint{5.007278in}{2.225724in}}%
\pgfpathlineto{\pgfqpoint{5.011611in}{2.217293in}}%
\pgfpathlineto{\pgfqpoint{5.015945in}{2.207041in}}%
\pgfpathlineto{\pgfqpoint{5.020279in}{2.194751in}}%
\pgfpathlineto{\pgfqpoint{5.024612in}{2.180177in}}%
\pgfpathlineto{\pgfqpoint{5.028946in}{2.163068in}}%
\pgfpathlineto{\pgfqpoint{5.033279in}{2.143205in}}%
\pgfpathlineto{\pgfqpoint{5.037613in}{2.120461in}}%
\pgfpathlineto{\pgfqpoint{5.041947in}{2.094904in}}%
\pgfpathlineto{\pgfqpoint{5.050614in}{2.037549in}}%
\pgfpathlineto{\pgfqpoint{5.054948in}{2.008447in}}%
\pgfpathlineto{\pgfqpoint{5.059281in}{1.982364in}}%
\pgfpathlineto{\pgfqpoint{5.063615in}{1.963017in}}%
\pgfpathlineto{\pgfqpoint{5.067949in}{1.954737in}}%
\pgfpathlineto{\pgfqpoint{5.072282in}{1.961489in}}%
\pgfpathlineto{\pgfqpoint{5.076616in}{1.985394in}}%
\pgfpathlineto{\pgfqpoint{5.080950in}{2.025327in}}%
\pgfpathlineto{\pgfqpoint{5.085283in}{2.076541in}}%
\pgfpathlineto{\pgfqpoint{5.093951in}{2.184113in}}%
\pgfpathlineto{\pgfqpoint{5.098284in}{2.228016in}}%
\pgfpathlineto{\pgfqpoint{5.102618in}{2.261272in}}%
\pgfpathlineto{\pgfqpoint{5.106952in}{2.284055in}}%
\pgfpathlineto{\pgfqpoint{5.111285in}{2.298023in}}%
\pgfpathlineto{\pgfqpoint{5.115619in}{2.305354in}}%
\pgfpathlineto{\pgfqpoint{5.119953in}{2.308122in}}%
\pgfpathlineto{\pgfqpoint{5.124286in}{2.308013in}}%
\pgfpathlineto{\pgfqpoint{5.128620in}{2.306266in}}%
\pgfpathlineto{\pgfqpoint{5.137287in}{2.300904in}}%
\pgfpathlineto{\pgfqpoint{5.145954in}{2.295513in}}%
\pgfpathlineto{\pgfqpoint{5.154622in}{2.290962in}}%
\pgfpathlineto{\pgfqpoint{5.180624in}{2.278657in}}%
\pgfpathlineto{\pgfqpoint{5.184957in}{2.275930in}}%
\pgfpathlineto{\pgfqpoint{5.189291in}{2.272742in}}%
\pgfpathlineto{\pgfqpoint{5.193625in}{2.268964in}}%
\pgfpathlineto{\pgfqpoint{5.193625in}{2.268964in}}%
\pgfusepath{stroke}%
\end{pgfscope}%
\begin{pgfscope}%
\pgfpathrectangle{\pgfqpoint{0.647840in}{1.855124in}}{\pgfqpoint{4.762251in}{0.887161in}}%
\pgfusepath{clip}%
\pgfsetrectcap%
\pgfsetroundjoin%
\pgfsetlinewidth{1.505625pt}%
\definecolor{currentstroke}{rgb}{0.203922,0.541176,0.741176}%
\pgfsetstrokecolor{currentstroke}%
\pgfsetdash{}{0pt}%
\pgfpathmoveto{\pgfqpoint{3.031132in}{2.344867in}}%
\pgfpathlineto{\pgfqpoint{3.035466in}{2.354045in}}%
\pgfpathlineto{\pgfqpoint{3.039799in}{2.365012in}}%
\pgfpathlineto{\pgfqpoint{3.044133in}{2.378175in}}%
\pgfpathlineto{\pgfqpoint{3.048467in}{2.394029in}}%
\pgfpathlineto{\pgfqpoint{3.052800in}{2.412889in}}%
\pgfpathlineto{\pgfqpoint{3.057134in}{2.435344in}}%
\pgfpathlineto{\pgfqpoint{3.061467in}{2.461634in}}%
\pgfpathlineto{\pgfqpoint{3.065801in}{2.491948in}}%
\pgfpathlineto{\pgfqpoint{3.070135in}{2.526192in}}%
\pgfpathlineto{\pgfqpoint{3.083136in}{2.638014in}}%
\pgfpathlineto{\pgfqpoint{3.087469in}{2.667394in}}%
\pgfpathlineto{\pgfqpoint{3.091803in}{2.682820in}}%
\pgfpathlineto{\pgfqpoint{3.096137in}{2.677198in}}%
\pgfpathlineto{\pgfqpoint{3.100470in}{2.645796in}}%
\pgfpathlineto{\pgfqpoint{3.104804in}{2.590053in}}%
\pgfpathlineto{\pgfqpoint{3.109138in}{2.516779in}}%
\pgfpathlineto{\pgfqpoint{3.113471in}{2.438791in}}%
\pgfpathlineto{\pgfqpoint{3.117805in}{2.367812in}}%
\pgfpathlineto{\pgfqpoint{3.122139in}{2.311620in}}%
\pgfpathlineto{\pgfqpoint{3.126472in}{2.272592in}}%
\pgfpathlineto{\pgfqpoint{3.130806in}{2.248400in}}%
\pgfpathlineto{\pgfqpoint{3.135140in}{2.235496in}}%
\pgfpathlineto{\pgfqpoint{3.139473in}{2.229913in}}%
\pgfpathlineto{\pgfqpoint{3.143807in}{2.228638in}}%
\pgfpathlineto{\pgfqpoint{3.148141in}{2.229669in}}%
\pgfpathlineto{\pgfqpoint{3.156808in}{2.232646in}}%
\pgfpathlineto{\pgfqpoint{3.161141in}{2.233129in}}%
\pgfpathlineto{\pgfqpoint{3.165475in}{2.232502in}}%
\pgfpathlineto{\pgfqpoint{3.169809in}{2.230489in}}%
\pgfpathlineto{\pgfqpoint{3.174142in}{2.227105in}}%
\pgfpathlineto{\pgfqpoint{3.178476in}{2.222146in}}%
\pgfpathlineto{\pgfqpoint{3.182810in}{2.215602in}}%
\pgfpathlineto{\pgfqpoint{3.187143in}{2.207350in}}%
\pgfpathlineto{\pgfqpoint{3.191477in}{2.197257in}}%
\pgfpathlineto{\pgfqpoint{3.195811in}{2.184972in}}%
\pgfpathlineto{\pgfqpoint{3.200144in}{2.170508in}}%
\pgfpathlineto{\pgfqpoint{3.204478in}{2.153700in}}%
\pgfpathlineto{\pgfqpoint{3.208812in}{2.134405in}}%
\pgfpathlineto{\pgfqpoint{3.213145in}{2.112658in}}%
\pgfpathlineto{\pgfqpoint{3.217479in}{2.088718in}}%
\pgfpathlineto{\pgfqpoint{3.230480in}{2.012707in}}%
\pgfpathlineto{\pgfqpoint{3.234814in}{1.992128in}}%
\pgfpathlineto{\pgfqpoint{3.239147in}{1.978746in}}%
\pgfpathlineto{\pgfqpoint{3.243481in}{1.975977in}}%
\pgfpathlineto{\pgfqpoint{3.247815in}{1.986220in}}%
\pgfpathlineto{\pgfqpoint{3.252148in}{2.010546in}}%
\pgfpathlineto{\pgfqpoint{3.256482in}{2.047152in}}%
\pgfpathlineto{\pgfqpoint{3.269483in}{2.182484in}}%
\pgfpathlineto{\pgfqpoint{3.273816in}{2.219350in}}%
\pgfpathlineto{\pgfqpoint{3.278150in}{2.247534in}}%
\pgfpathlineto{\pgfqpoint{3.282484in}{2.267147in}}%
\pgfpathlineto{\pgfqpoint{3.286817in}{2.279410in}}%
\pgfpathlineto{\pgfqpoint{3.291151in}{2.285976in}}%
\pgfpathlineto{\pgfqpoint{3.295485in}{2.288461in}}%
\pgfpathlineto{\pgfqpoint{3.299818in}{2.288274in}}%
\pgfpathlineto{\pgfqpoint{3.304152in}{2.286325in}}%
\pgfpathlineto{\pgfqpoint{3.308486in}{2.283372in}}%
\pgfpathlineto{\pgfqpoint{3.312819in}{2.279739in}}%
\pgfpathlineto{\pgfqpoint{3.317153in}{2.275701in}}%
\pgfpathlineto{\pgfqpoint{3.325820in}{2.266588in}}%
\pgfpathlineto{\pgfqpoint{3.330154in}{2.261565in}}%
\pgfpathlineto{\pgfqpoint{3.334488in}{2.256094in}}%
\pgfpathlineto{\pgfqpoint{3.338821in}{2.249968in}}%
\pgfpathlineto{\pgfqpoint{3.343155in}{2.243042in}}%
\pgfpathlineto{\pgfqpoint{3.347489in}{2.234993in}}%
\pgfpathlineto{\pgfqpoint{3.351822in}{2.225705in}}%
\pgfpathlineto{\pgfqpoint{3.356156in}{2.214895in}}%
\pgfpathlineto{\pgfqpoint{3.360490in}{2.202199in}}%
\pgfpathlineto{\pgfqpoint{3.364823in}{2.187131in}}%
\pgfpathlineto{\pgfqpoint{3.369157in}{2.169587in}}%
\pgfpathlineto{\pgfqpoint{3.373490in}{2.149019in}}%
\pgfpathlineto{\pgfqpoint{3.377824in}{2.125298in}}%
\pgfpathlineto{\pgfqpoint{3.382158in}{2.098156in}}%
\pgfpathlineto{\pgfqpoint{3.386491in}{2.068261in}}%
\pgfpathlineto{\pgfqpoint{3.399492in}{1.973082in}}%
\pgfpathlineto{\pgfqpoint{3.403826in}{1.949090in}}%
\pgfpathlineto{\pgfqpoint{3.408160in}{1.936560in}}%
\pgfpathlineto{\pgfqpoint{3.412493in}{1.940716in}}%
\pgfpathlineto{\pgfqpoint{3.416827in}{1.964839in}}%
\pgfpathlineto{\pgfqpoint{3.421161in}{2.008349in}}%
\pgfpathlineto{\pgfqpoint{3.425494in}{2.066091in}}%
\pgfpathlineto{\pgfqpoint{3.434162in}{2.188593in}}%
\pgfpathlineto{\pgfqpoint{3.438495in}{2.238108in}}%
\pgfpathlineto{\pgfqpoint{3.442829in}{2.275142in}}%
\pgfpathlineto{\pgfqpoint{3.447163in}{2.299981in}}%
\pgfpathlineto{\pgfqpoint{3.451496in}{2.314710in}}%
\pgfpathlineto{\pgfqpoint{3.455830in}{2.322042in}}%
\pgfpathlineto{\pgfqpoint{3.460164in}{2.324701in}}%
\pgfpathlineto{\pgfqpoint{3.464497in}{2.324530in}}%
\pgfpathlineto{\pgfqpoint{3.468831in}{2.322895in}}%
\pgfpathlineto{\pgfqpoint{3.477498in}{2.318694in}}%
\pgfpathlineto{\pgfqpoint{3.481832in}{2.317047in}}%
\pgfpathlineto{\pgfqpoint{3.486165in}{2.315880in}}%
\pgfpathlineto{\pgfqpoint{3.490499in}{2.315427in}}%
\pgfpathlineto{\pgfqpoint{3.494833in}{2.315729in}}%
\pgfpathlineto{\pgfqpoint{3.499166in}{2.316725in}}%
\pgfpathlineto{\pgfqpoint{3.503500in}{2.318514in}}%
\pgfpathlineto{\pgfqpoint{3.507834in}{2.320956in}}%
\pgfpathlineto{\pgfqpoint{3.512167in}{2.324233in}}%
\pgfpathlineto{\pgfqpoint{3.516501in}{2.328509in}}%
\pgfpathlineto{\pgfqpoint{3.520835in}{2.333783in}}%
\pgfpathlineto{\pgfqpoint{3.525168in}{2.340211in}}%
\pgfpathlineto{\pgfqpoint{3.529502in}{2.348085in}}%
\pgfpathlineto{\pgfqpoint{3.533836in}{2.357628in}}%
\pgfpathlineto{\pgfqpoint{3.538169in}{2.369035in}}%
\pgfpathlineto{\pgfqpoint{3.542503in}{2.382743in}}%
\pgfpathlineto{\pgfqpoint{3.546837in}{2.399179in}}%
\pgfpathlineto{\pgfqpoint{3.551170in}{2.418680in}}%
\pgfpathlineto{\pgfqpoint{3.555504in}{2.441748in}}%
\pgfpathlineto{\pgfqpoint{3.559838in}{2.468601in}}%
\pgfpathlineto{\pgfqpoint{3.564171in}{2.499193in}}%
\pgfpathlineto{\pgfqpoint{3.572839in}{2.570616in}}%
\pgfpathlineto{\pgfqpoint{3.577172in}{2.607540in}}%
\pgfpathlineto{\pgfqpoint{3.581506in}{2.642051in}}%
\pgfpathlineto{\pgfqpoint{3.585839in}{2.669380in}}%
\pgfpathlineto{\pgfqpoint{3.590173in}{2.682163in}}%
\pgfpathlineto{\pgfqpoint{3.594507in}{2.673893in}}%
\pgfpathlineto{\pgfqpoint{3.598840in}{2.640042in}}%
\pgfpathlineto{\pgfqpoint{3.603174in}{2.582519in}}%
\pgfpathlineto{\pgfqpoint{3.616175in}{2.362070in}}%
\pgfpathlineto{\pgfqpoint{3.620509in}{2.307498in}}%
\pgfpathlineto{\pgfqpoint{3.624842in}{2.269863in}}%
\pgfpathlineto{\pgfqpoint{3.629176in}{2.246830in}}%
\pgfpathlineto{\pgfqpoint{3.633510in}{2.234786in}}%
\pgfpathlineto{\pgfqpoint{3.637843in}{2.229699in}}%
\pgfpathlineto{\pgfqpoint{3.642177in}{2.228674in}}%
\pgfpathlineto{\pgfqpoint{3.646511in}{2.229833in}}%
\pgfpathlineto{\pgfqpoint{3.650844in}{2.231598in}}%
\pgfpathlineto{\pgfqpoint{3.655178in}{2.232910in}}%
\pgfpathlineto{\pgfqpoint{3.659512in}{2.233327in}}%
\pgfpathlineto{\pgfqpoint{3.663845in}{2.232523in}}%
\pgfpathlineto{\pgfqpoint{3.668179in}{2.230370in}}%
\pgfpathlineto{\pgfqpoint{3.672513in}{2.226736in}}%
\pgfpathlineto{\pgfqpoint{3.676846in}{2.221618in}}%
\pgfpathlineto{\pgfqpoint{3.681180in}{2.214953in}}%
\pgfpathlineto{\pgfqpoint{3.685513in}{2.206555in}}%
\pgfpathlineto{\pgfqpoint{3.689847in}{2.196119in}}%
\pgfpathlineto{\pgfqpoint{3.694181in}{2.183650in}}%
\pgfpathlineto{\pgfqpoint{3.698514in}{2.168988in}}%
\pgfpathlineto{\pgfqpoint{3.702848in}{2.151847in}}%
\pgfpathlineto{\pgfqpoint{3.707182in}{2.132325in}}%
\pgfpathlineto{\pgfqpoint{3.711515in}{2.110329in}}%
\pgfpathlineto{\pgfqpoint{3.715849in}{2.086262in}}%
\pgfpathlineto{\pgfqpoint{3.728850in}{2.010010in}}%
\pgfpathlineto{\pgfqpoint{3.733184in}{1.989779in}}%
\pgfpathlineto{\pgfqpoint{3.737517in}{1.977108in}}%
\pgfpathlineto{\pgfqpoint{3.741851in}{1.975247in}}%
\pgfpathlineto{\pgfqpoint{3.746185in}{1.986745in}}%
\pgfpathlineto{\pgfqpoint{3.750518in}{2.012440in}}%
\pgfpathlineto{\pgfqpoint{3.754852in}{2.050003in}}%
\pgfpathlineto{\pgfqpoint{3.767853in}{2.185220in}}%
\pgfpathlineto{\pgfqpoint{3.772187in}{2.221489in}}%
\pgfpathlineto{\pgfqpoint{3.776520in}{2.248901in}}%
\pgfpathlineto{\pgfqpoint{3.780854in}{2.267847in}}%
\pgfpathlineto{\pgfqpoint{3.785187in}{2.279535in}}%
\pgfpathlineto{\pgfqpoint{3.789521in}{2.285655in}}%
\pgfpathlineto{\pgfqpoint{3.793855in}{2.287910in}}%
\pgfpathlineto{\pgfqpoint{3.798188in}{2.287516in}}%
\pgfpathlineto{\pgfqpoint{3.802522in}{2.285288in}}%
\pgfpathlineto{\pgfqpoint{3.806856in}{2.282194in}}%
\pgfpathlineto{\pgfqpoint{3.811189in}{2.278470in}}%
\pgfpathlineto{\pgfqpoint{3.819857in}{2.269916in}}%
\pgfpathlineto{\pgfqpoint{3.824190in}{2.265147in}}%
\pgfpathlineto{\pgfqpoint{3.828524in}{2.259906in}}%
\pgfpathlineto{\pgfqpoint{3.832858in}{2.254224in}}%
\pgfpathlineto{\pgfqpoint{3.837191in}{2.247862in}}%
\pgfpathlineto{\pgfqpoint{3.841525in}{2.240649in}}%
\pgfpathlineto{\pgfqpoint{3.845859in}{2.232283in}}%
\pgfpathlineto{\pgfqpoint{3.850192in}{2.222637in}}%
\pgfpathlineto{\pgfqpoint{3.854526in}{2.211277in}}%
\pgfpathlineto{\pgfqpoint{3.858860in}{2.198013in}}%
\pgfpathlineto{\pgfqpoint{3.863193in}{2.182316in}}%
\pgfpathlineto{\pgfqpoint{3.867527in}{2.164038in}}%
\pgfpathlineto{\pgfqpoint{3.871861in}{2.142831in}}%
\pgfpathlineto{\pgfqpoint{3.876194in}{2.118382in}}%
\pgfpathlineto{\pgfqpoint{3.880528in}{2.090671in}}%
\pgfpathlineto{\pgfqpoint{3.889195in}{2.028055in}}%
\pgfpathlineto{\pgfqpoint{3.893529in}{1.996124in}}%
\pgfpathlineto{\pgfqpoint{3.897862in}{1.967451in}}%
\pgfpathlineto{\pgfqpoint{3.902196in}{1.946556in}}%
\pgfpathlineto{\pgfqpoint{3.906530in}{1.938379in}}%
\pgfpathlineto{\pgfqpoint{3.910863in}{1.947651in}}%
\pgfpathlineto{\pgfqpoint{3.915197in}{1.976556in}}%
\pgfpathlineto{\pgfqpoint{3.919531in}{2.023367in}}%
\pgfpathlineto{\pgfqpoint{3.932532in}{2.200733in}}%
\pgfpathlineto{\pgfqpoint{3.936865in}{2.247095in}}%
\pgfpathlineto{\pgfqpoint{3.941199in}{2.281025in}}%
\pgfpathlineto{\pgfqpoint{3.945533in}{2.303225in}}%
\pgfpathlineto{\pgfqpoint{3.949866in}{2.316052in}}%
\pgfpathlineto{\pgfqpoint{3.954200in}{2.322093in}}%
\pgfpathlineto{\pgfqpoint{3.958534in}{2.323983in}}%
\pgfpathlineto{\pgfqpoint{3.962867in}{2.323426in}}%
\pgfpathlineto{\pgfqpoint{3.967201in}{2.321634in}}%
\pgfpathlineto{\pgfqpoint{3.975868in}{2.317333in}}%
\pgfpathlineto{\pgfqpoint{3.980202in}{2.315749in}}%
\pgfpathlineto{\pgfqpoint{3.984536in}{2.314586in}}%
\pgfpathlineto{\pgfqpoint{3.988869in}{2.314214in}}%
\pgfpathlineto{\pgfqpoint{3.993203in}{2.314507in}}%
\pgfpathlineto{\pgfqpoint{3.997536in}{2.315538in}}%
\pgfpathlineto{\pgfqpoint{4.001870in}{2.317277in}}%
\pgfpathlineto{\pgfqpoint{4.006204in}{2.319759in}}%
\pgfpathlineto{\pgfqpoint{4.010537in}{2.323074in}}%
\pgfpathlineto{\pgfqpoint{4.014871in}{2.327185in}}%
\pgfpathlineto{\pgfqpoint{4.019205in}{2.332237in}}%
\pgfpathlineto{\pgfqpoint{4.023538in}{2.338535in}}%
\pgfpathlineto{\pgfqpoint{4.027872in}{2.346148in}}%
\pgfpathlineto{\pgfqpoint{4.032206in}{2.355322in}}%
\pgfpathlineto{\pgfqpoint{4.036539in}{2.366395in}}%
\pgfpathlineto{\pgfqpoint{4.040873in}{2.379734in}}%
\pgfpathlineto{\pgfqpoint{4.045207in}{2.395795in}}%
\pgfpathlineto{\pgfqpoint{4.049540in}{2.415016in}}%
\pgfpathlineto{\pgfqpoint{4.053874in}{2.437810in}}%
\pgfpathlineto{\pgfqpoint{4.058208in}{2.464722in}}%
\pgfpathlineto{\pgfqpoint{4.062541in}{2.495819in}}%
\pgfpathlineto{\pgfqpoint{4.066875in}{2.530818in}}%
\pgfpathlineto{\pgfqpoint{4.079876in}{2.644387in}}%
\pgfpathlineto{\pgfqpoint{4.084210in}{2.670794in}}%
\pgfpathlineto{\pgfqpoint{4.088543in}{2.680498in}}%
\pgfpathlineto{\pgfqpoint{4.092877in}{2.667225in}}%
\pgfpathlineto{\pgfqpoint{4.097210in}{2.627579in}}%
\pgfpathlineto{\pgfqpoint{4.101544in}{2.565132in}}%
\pgfpathlineto{\pgfqpoint{4.110211in}{2.413980in}}%
\pgfpathlineto{\pgfqpoint{4.114545in}{2.348887in}}%
\pgfpathlineto{\pgfqpoint{4.118879in}{2.299452in}}%
\pgfpathlineto{\pgfqpoint{4.123212in}{2.266289in}}%
\pgfpathlineto{\pgfqpoint{4.127546in}{2.246638in}}%
\pgfpathlineto{\pgfqpoint{4.131880in}{2.236700in}}%
\pgfpathlineto{\pgfqpoint{4.136213in}{2.232899in}}%
\pgfpathlineto{\pgfqpoint{4.140547in}{2.232587in}}%
\pgfpathlineto{\pgfqpoint{4.144881in}{2.233992in}}%
\pgfpathlineto{\pgfqpoint{4.149214in}{2.235764in}}%
\pgfpathlineto{\pgfqpoint{4.153548in}{2.237127in}}%
\pgfpathlineto{\pgfqpoint{4.157882in}{2.237522in}}%
\pgfpathlineto{\pgfqpoint{4.162215in}{2.236652in}}%
\pgfpathlineto{\pgfqpoint{4.166549in}{2.234585in}}%
\pgfpathlineto{\pgfqpoint{4.170883in}{2.231096in}}%
\pgfpathlineto{\pgfqpoint{4.175216in}{2.226073in}}%
\pgfpathlineto{\pgfqpoint{4.179550in}{2.219479in}}%
\pgfpathlineto{\pgfqpoint{4.183884in}{2.211256in}}%
\pgfpathlineto{\pgfqpoint{4.188217in}{2.201129in}}%
\pgfpathlineto{\pgfqpoint{4.192551in}{2.188977in}}%
\pgfpathlineto{\pgfqpoint{4.196884in}{2.174617in}}%
\pgfpathlineto{\pgfqpoint{4.201218in}{2.157853in}}%
\pgfpathlineto{\pgfqpoint{4.205552in}{2.138480in}}%
\pgfpathlineto{\pgfqpoint{4.209885in}{2.116490in}}%
\pgfpathlineto{\pgfqpoint{4.214219in}{2.092158in}}%
\pgfpathlineto{\pgfqpoint{4.227220in}{2.012743in}}%
\pgfpathlineto{\pgfqpoint{4.231554in}{1.990261in}}%
\pgfpathlineto{\pgfqpoint{4.235887in}{1.974721in}}%
\pgfpathlineto{\pgfqpoint{4.240221in}{1.969703in}}%
\pgfpathlineto{\pgfqpoint{4.244555in}{1.978466in}}%
\pgfpathlineto{\pgfqpoint{4.248888in}{2.002062in}}%
\pgfpathlineto{\pgfqpoint{4.253222in}{2.039330in}}%
\pgfpathlineto{\pgfqpoint{4.257556in}{2.085397in}}%
\pgfpathlineto{\pgfqpoint{4.266223in}{2.180431in}}%
\pgfpathlineto{\pgfqpoint{4.270557in}{2.219313in}}%
\pgfpathlineto{\pgfqpoint{4.274890in}{2.249130in}}%
\pgfpathlineto{\pgfqpoint{4.279224in}{2.269948in}}%
\pgfpathlineto{\pgfqpoint{4.283558in}{2.283112in}}%
\pgfpathlineto{\pgfqpoint{4.287891in}{2.290271in}}%
\pgfpathlineto{\pgfqpoint{4.292225in}{2.293027in}}%
\pgfpathlineto{\pgfqpoint{4.296559in}{2.293018in}}%
\pgfpathlineto{\pgfqpoint{4.300892in}{2.291186in}}%
\pgfpathlineto{\pgfqpoint{4.305226in}{2.288223in}}%
\pgfpathlineto{\pgfqpoint{4.313893in}{2.280871in}}%
\pgfpathlineto{\pgfqpoint{4.322560in}{2.272768in}}%
\pgfpathlineto{\pgfqpoint{4.331228in}{2.263780in}}%
\pgfpathlineto{\pgfqpoint{4.335561in}{2.258617in}}%
\pgfpathlineto{\pgfqpoint{4.339895in}{2.252847in}}%
\pgfpathlineto{\pgfqpoint{4.344229in}{2.246238in}}%
\pgfpathlineto{\pgfqpoint{4.348562in}{2.238486in}}%
\pgfpathlineto{\pgfqpoint{4.352896in}{2.229512in}}%
\pgfpathlineto{\pgfqpoint{4.357230in}{2.218943in}}%
\pgfpathlineto{\pgfqpoint{4.361563in}{2.206372in}}%
\pgfpathlineto{\pgfqpoint{4.365897in}{2.191566in}}%
\pgfpathlineto{\pgfqpoint{4.370231in}{2.174080in}}%
\pgfpathlineto{\pgfqpoint{4.374564in}{2.153517in}}%
\pgfpathlineto{\pgfqpoint{4.378898in}{2.129582in}}%
\pgfpathlineto{\pgfqpoint{4.383232in}{2.101979in}}%
\pgfpathlineto{\pgfqpoint{4.387565in}{2.071108in}}%
\pgfpathlineto{\pgfqpoint{4.400566in}{1.970484in}}%
\pgfpathlineto{\pgfqpoint{4.404900in}{1.943444in}}%
\pgfpathlineto{\pgfqpoint{4.409233in}{1.927908in}}%
\pgfpathlineto{\pgfqpoint{4.413567in}{1.929317in}}%
\pgfpathlineto{\pgfqpoint{4.417901in}{1.951750in}}%
\pgfpathlineto{\pgfqpoint{4.422234in}{1.995260in}}%
\pgfpathlineto{\pgfqpoint{4.426568in}{2.054968in}}%
\pgfpathlineto{\pgfqpoint{4.435235in}{2.185510in}}%
\pgfpathlineto{\pgfqpoint{4.439569in}{2.239236in}}%
\pgfpathlineto{\pgfqpoint{4.443903in}{2.279375in}}%
\pgfpathlineto{\pgfqpoint{4.448236in}{2.306312in}}%
\pgfpathlineto{\pgfqpoint{4.452570in}{2.322191in}}%
\pgfpathlineto{\pgfqpoint{4.456904in}{2.330109in}}%
\pgfpathlineto{\pgfqpoint{4.461237in}{2.332960in}}%
\pgfpathlineto{\pgfqpoint{4.465571in}{2.332896in}}%
\pgfpathlineto{\pgfqpoint{4.469905in}{2.331248in}}%
\pgfpathlineto{\pgfqpoint{4.478572in}{2.327445in}}%
\pgfpathlineto{\pgfqpoint{4.482906in}{2.326324in}}%
\pgfpathlineto{\pgfqpoint{4.487239in}{2.325843in}}%
\pgfpathlineto{\pgfqpoint{4.491573in}{2.326286in}}%
\pgfpathlineto{\pgfqpoint{4.495907in}{2.327657in}}%
\pgfpathlineto{\pgfqpoint{4.500240in}{2.330004in}}%
\pgfpathlineto{\pgfqpoint{4.504574in}{2.333347in}}%
\pgfpathlineto{\pgfqpoint{4.508907in}{2.337808in}}%
\pgfpathlineto{\pgfqpoint{4.513241in}{2.343463in}}%
\pgfpathlineto{\pgfqpoint{4.517575in}{2.350483in}}%
\pgfpathlineto{\pgfqpoint{4.521908in}{2.359048in}}%
\pgfpathlineto{\pgfqpoint{4.526242in}{2.369386in}}%
\pgfpathlineto{\pgfqpoint{4.530576in}{2.381808in}}%
\pgfpathlineto{\pgfqpoint{4.534909in}{2.396595in}}%
\pgfpathlineto{\pgfqpoint{4.539243in}{2.414098in}}%
\pgfpathlineto{\pgfqpoint{4.543577in}{2.434863in}}%
\pgfpathlineto{\pgfqpoint{4.547910in}{2.459140in}}%
\pgfpathlineto{\pgfqpoint{4.552244in}{2.487216in}}%
\pgfpathlineto{\pgfqpoint{4.556578in}{2.518553in}}%
\pgfpathlineto{\pgfqpoint{4.569579in}{2.619024in}}%
\pgfpathlineto{\pgfqpoint{4.573912in}{2.644417in}}%
\pgfpathlineto{\pgfqpoint{4.578246in}{2.656792in}}%
\pgfpathlineto{\pgfqpoint{4.582580in}{2.650579in}}%
\pgfpathlineto{\pgfqpoint{4.586913in}{2.622735in}}%
\pgfpathlineto{\pgfqpoint{4.591247in}{2.574395in}}%
\pgfpathlineto{\pgfqpoint{4.595581in}{2.511805in}}%
\pgfpathlineto{\pgfqpoint{4.599914in}{2.444638in}}%
\pgfpathlineto{\pgfqpoint{4.604248in}{2.382736in}}%
\pgfpathlineto{\pgfqpoint{4.608582in}{2.332727in}}%
\pgfpathlineto{\pgfqpoint{4.612915in}{2.296561in}}%
\pgfpathlineto{\pgfqpoint{4.617249in}{2.273157in}}%
\pgfpathlineto{\pgfqpoint{4.621582in}{2.259831in}}%
\pgfpathlineto{\pgfqpoint{4.625916in}{2.253573in}}%
\pgfpathlineto{\pgfqpoint{4.630250in}{2.251596in}}%
\pgfpathlineto{\pgfqpoint{4.634583in}{2.252231in}}%
\pgfpathlineto{\pgfqpoint{4.647584in}{2.257412in}}%
\pgfpathlineto{\pgfqpoint{4.651918in}{2.258249in}}%
\pgfpathlineto{\pgfqpoint{4.656252in}{2.258251in}}%
\pgfpathlineto{\pgfqpoint{4.660585in}{2.257390in}}%
\pgfpathlineto{\pgfqpoint{4.664919in}{2.255581in}}%
\pgfpathlineto{\pgfqpoint{4.669253in}{2.252713in}}%
\pgfpathlineto{\pgfqpoint{4.673586in}{2.248732in}}%
\pgfpathlineto{\pgfqpoint{4.677920in}{2.243559in}}%
\pgfpathlineto{\pgfqpoint{4.682254in}{2.237029in}}%
\pgfpathlineto{\pgfqpoint{4.686587in}{2.229018in}}%
\pgfpathlineto{\pgfqpoint{4.690921in}{2.219447in}}%
\pgfpathlineto{\pgfqpoint{4.695255in}{2.207884in}}%
\pgfpathlineto{\pgfqpoint{4.699588in}{2.194055in}}%
\pgfpathlineto{\pgfqpoint{4.703922in}{2.177537in}}%
\pgfpathlineto{\pgfqpoint{4.708256in}{2.158032in}}%
\pgfpathlineto{\pgfqpoint{4.712589in}{2.135306in}}%
\pgfpathlineto{\pgfqpoint{4.716923in}{2.109262in}}%
\pgfpathlineto{\pgfqpoint{4.721256in}{2.079969in}}%
\pgfpathlineto{\pgfqpoint{4.734257in}{1.981644in}}%
\pgfpathlineto{\pgfqpoint{4.738591in}{1.953890in}}%
\pgfpathlineto{\pgfqpoint{4.742925in}{1.935812in}}%
\pgfpathlineto{\pgfqpoint{4.747258in}{1.932837in}}%
\pgfpathlineto{\pgfqpoint{4.751592in}{1.949477in}}%
\pgfpathlineto{\pgfqpoint{4.755926in}{1.986870in}}%
\pgfpathlineto{\pgfqpoint{4.760259in}{2.041458in}}%
\pgfpathlineto{\pgfqpoint{4.768927in}{2.169247in}}%
\pgfpathlineto{\pgfqpoint{4.773260in}{2.224654in}}%
\pgfpathlineto{\pgfqpoint{4.777594in}{2.267315in}}%
\pgfpathlineto{\pgfqpoint{4.781928in}{2.296897in}}%
\pgfpathlineto{\pgfqpoint{4.786261in}{2.315171in}}%
\pgfpathlineto{\pgfqpoint{4.790595in}{2.324908in}}%
\pgfpathlineto{\pgfqpoint{4.794929in}{2.329001in}}%
\pgfpathlineto{\pgfqpoint{4.799262in}{2.329539in}}%
\pgfpathlineto{\pgfqpoint{4.803596in}{2.328265in}}%
\pgfpathlineto{\pgfqpoint{4.816597in}{2.322588in}}%
\pgfpathlineto{\pgfqpoint{4.820930in}{2.321646in}}%
\pgfpathlineto{\pgfqpoint{4.825264in}{2.321405in}}%
\pgfpathlineto{\pgfqpoint{4.829598in}{2.322028in}}%
\pgfpathlineto{\pgfqpoint{4.833931in}{2.323693in}}%
\pgfpathlineto{\pgfqpoint{4.838265in}{2.326174in}}%
\pgfpathlineto{\pgfqpoint{4.842599in}{2.329607in}}%
\pgfpathlineto{\pgfqpoint{4.846932in}{2.333950in}}%
\pgfpathlineto{\pgfqpoint{4.851266in}{2.339418in}}%
\pgfpathlineto{\pgfqpoint{4.855600in}{2.346170in}}%
\pgfpathlineto{\pgfqpoint{4.859933in}{2.354472in}}%
\pgfpathlineto{\pgfqpoint{4.864267in}{2.364426in}}%
\pgfpathlineto{\pgfqpoint{4.868601in}{2.376405in}}%
\pgfpathlineto{\pgfqpoint{4.872934in}{2.390664in}}%
\pgfpathlineto{\pgfqpoint{4.877268in}{2.407624in}}%
\pgfpathlineto{\pgfqpoint{4.881602in}{2.427733in}}%
\pgfpathlineto{\pgfqpoint{4.885935in}{2.451332in}}%
\pgfpathlineto{\pgfqpoint{4.890269in}{2.478593in}}%
\pgfpathlineto{\pgfqpoint{4.894603in}{2.509515in}}%
\pgfpathlineto{\pgfqpoint{4.903270in}{2.579027in}}%
\pgfpathlineto{\pgfqpoint{4.907604in}{2.613785in}}%
\pgfpathlineto{\pgfqpoint{4.911937in}{2.643636in}}%
\pgfpathlineto{\pgfqpoint{4.916271in}{2.663044in}}%
\pgfpathlineto{\pgfqpoint{4.920605in}{2.666360in}}%
\pgfpathlineto{\pgfqpoint{4.924938in}{2.648735in}}%
\pgfpathlineto{\pgfqpoint{4.929272in}{2.608086in}}%
\pgfpathlineto{\pgfqpoint{4.933605in}{2.548572in}}%
\pgfpathlineto{\pgfqpoint{4.942273in}{2.410698in}}%
\pgfpathlineto{\pgfqpoint{4.946606in}{2.351853in}}%
\pgfpathlineto{\pgfqpoint{4.950940in}{2.307155in}}%
\pgfpathlineto{\pgfqpoint{4.955274in}{2.276937in}}%
\pgfpathlineto{\pgfqpoint{4.959607in}{2.258770in}}%
\pgfpathlineto{\pgfqpoint{4.963941in}{2.249554in}}%
\pgfpathlineto{\pgfqpoint{4.968275in}{2.245946in}}%
\pgfpathlineto{\pgfqpoint{4.972608in}{2.245847in}}%
\pgfpathlineto{\pgfqpoint{4.976942in}{2.247169in}}%
\pgfpathlineto{\pgfqpoint{4.985609in}{2.250639in}}%
\pgfpathlineto{\pgfqpoint{4.989943in}{2.251571in}}%
\pgfpathlineto{\pgfqpoint{4.994277in}{2.251482in}}%
\pgfpathlineto{\pgfqpoint{4.998610in}{2.250518in}}%
\pgfpathlineto{\pgfqpoint{5.002944in}{2.248338in}}%
\pgfpathlineto{\pgfqpoint{5.007278in}{2.244968in}}%
\pgfpathlineto{\pgfqpoint{5.011611in}{2.240333in}}%
\pgfpathlineto{\pgfqpoint{5.015945in}{2.234337in}}%
\pgfpathlineto{\pgfqpoint{5.020279in}{2.226805in}}%
\pgfpathlineto{\pgfqpoint{5.024612in}{2.217618in}}%
\pgfpathlineto{\pgfqpoint{5.028946in}{2.206491in}}%
\pgfpathlineto{\pgfqpoint{5.033279in}{2.193220in}}%
\pgfpathlineto{\pgfqpoint{5.037613in}{2.177414in}}%
\pgfpathlineto{\pgfqpoint{5.041947in}{2.158957in}}%
\pgfpathlineto{\pgfqpoint{5.046280in}{2.137562in}}%
\pgfpathlineto{\pgfqpoint{5.050614in}{2.113120in}}%
\pgfpathlineto{\pgfqpoint{5.054948in}{2.085510in}}%
\pgfpathlineto{\pgfqpoint{5.067949in}{1.993682in}}%
\pgfpathlineto{\pgfqpoint{5.072282in}{1.967468in}}%
\pgfpathlineto{\pgfqpoint{5.076616in}{1.949411in}}%
\pgfpathlineto{\pgfqpoint{5.080950in}{1.944700in}}%
\pgfpathlineto{\pgfqpoint{5.085283in}{1.957170in}}%
\pgfpathlineto{\pgfqpoint{5.089617in}{1.988281in}}%
\pgfpathlineto{\pgfqpoint{5.093951in}{2.035674in}}%
\pgfpathlineto{\pgfqpoint{5.106952in}{2.205131in}}%
\pgfpathlineto{\pgfqpoint{5.111285in}{2.247786in}}%
\pgfpathlineto{\pgfqpoint{5.115619in}{2.278532in}}%
\pgfpathlineto{\pgfqpoint{5.119953in}{2.298577in}}%
\pgfpathlineto{\pgfqpoint{5.124286in}{2.310091in}}%
\pgfpathlineto{\pgfqpoint{5.128620in}{2.315442in}}%
\pgfpathlineto{\pgfqpoint{5.132953in}{2.316997in}}%
\pgfpathlineto{\pgfqpoint{5.137287in}{2.316238in}}%
\pgfpathlineto{\pgfqpoint{5.141621in}{2.314286in}}%
\pgfpathlineto{\pgfqpoint{5.150288in}{2.309396in}}%
\pgfpathlineto{\pgfqpoint{5.154622in}{2.307338in}}%
\pgfpathlineto{\pgfqpoint{5.158955in}{2.305686in}}%
\pgfpathlineto{\pgfqpoint{5.163289in}{2.304466in}}%
\pgfpathlineto{\pgfqpoint{5.167623in}{2.303760in}}%
\pgfpathlineto{\pgfqpoint{5.171956in}{2.303474in}}%
\pgfpathlineto{\pgfqpoint{5.176290in}{2.303614in}}%
\pgfpathlineto{\pgfqpoint{5.180624in}{2.304283in}}%
\pgfpathlineto{\pgfqpoint{5.184957in}{2.305370in}}%
\pgfpathlineto{\pgfqpoint{5.189291in}{2.306924in}}%
\pgfpathlineto{\pgfqpoint{5.193625in}{2.308963in}}%
\pgfpathlineto{\pgfqpoint{5.193625in}{2.308963in}}%
\pgfusepath{stroke}%
\end{pgfscope}%
\begin{pgfscope}%
\pgfsetrectcap%
\pgfsetmiterjoin%
\pgfsetlinewidth{1.003750pt}%
\definecolor{currentstroke}{rgb}{1.000000,1.000000,1.000000}%
\pgfsetstrokecolor{currentstroke}%
\pgfsetdash{}{0pt}%
\pgfpathmoveto{\pgfqpoint{0.647840in}{1.855124in}}%
\pgfpathlineto{\pgfqpoint{0.647840in}{2.742285in}}%
\pgfusepath{stroke}%
\end{pgfscope}%
\begin{pgfscope}%
\pgfsetrectcap%
\pgfsetmiterjoin%
\pgfsetlinewidth{1.003750pt}%
\definecolor{currentstroke}{rgb}{1.000000,1.000000,1.000000}%
\pgfsetstrokecolor{currentstroke}%
\pgfsetdash{}{0pt}%
\pgfpathmoveto{\pgfqpoint{5.410091in}{1.855124in}}%
\pgfpathlineto{\pgfqpoint{5.410091in}{2.742285in}}%
\pgfusepath{stroke}%
\end{pgfscope}%
\begin{pgfscope}%
\pgfsetrectcap%
\pgfsetmiterjoin%
\pgfsetlinewidth{1.003750pt}%
\definecolor{currentstroke}{rgb}{1.000000,1.000000,1.000000}%
\pgfsetstrokecolor{currentstroke}%
\pgfsetdash{}{0pt}%
\pgfpathmoveto{\pgfqpoint{0.647840in}{1.855124in}}%
\pgfpathlineto{\pgfqpoint{5.410091in}{1.855124in}}%
\pgfusepath{stroke}%
\end{pgfscope}%
\begin{pgfscope}%
\pgfsetrectcap%
\pgfsetmiterjoin%
\pgfsetlinewidth{1.003750pt}%
\definecolor{currentstroke}{rgb}{1.000000,1.000000,1.000000}%
\pgfsetstrokecolor{currentstroke}%
\pgfsetdash{}{0pt}%
\pgfpathmoveto{\pgfqpoint{0.647840in}{2.742285in}}%
\pgfpathlineto{\pgfqpoint{5.410091in}{2.742285in}}%
\pgfusepath{stroke}%
\end{pgfscope}%
\begin{pgfscope}%
\pgfsetbuttcap%
\pgfsetmiterjoin%
\definecolor{currentfill}{rgb}{0.898039,0.898039,0.898039}%
\pgfsetfillcolor{currentfill}%
\pgfsetlinewidth{0.000000pt}%
\definecolor{currentstroke}{rgb}{0.000000,0.000000,0.000000}%
\pgfsetstrokecolor{currentstroke}%
\pgfsetstrokeopacity{0.000000}%
\pgfsetdash{}{0pt}%
\pgfpathmoveto{\pgfqpoint{0.647840in}{0.524382in}}%
\pgfpathlineto{\pgfqpoint{5.410091in}{0.524382in}}%
\pgfpathlineto{\pgfqpoint{5.410091in}{1.411543in}}%
\pgfpathlineto{\pgfqpoint{0.647840in}{1.411543in}}%
\pgfpathclose%
\pgfusepath{fill}%
\end{pgfscope}%
\begin{pgfscope}%
\pgfpathrectangle{\pgfqpoint{0.647840in}{0.524382in}}{\pgfqpoint{4.762251in}{0.887161in}}%
\pgfusepath{clip}%
\pgfsetrectcap%
\pgfsetroundjoin%
\pgfsetlinewidth{0.803000pt}%
\definecolor{currentstroke}{rgb}{1.000000,1.000000,1.000000}%
\pgfsetstrokecolor{currentstroke}%
\pgfsetdash{}{0pt}%
\pgfpathmoveto{\pgfqpoint{0.864306in}{0.524382in}}%
\pgfpathlineto{\pgfqpoint{0.864306in}{1.411543in}}%
\pgfusepath{stroke}%
\end{pgfscope}%
\begin{pgfscope}%
\pgfsetbuttcap%
\pgfsetroundjoin%
\definecolor{currentfill}{rgb}{0.333333,0.333333,0.333333}%
\pgfsetfillcolor{currentfill}%
\pgfsetlinewidth{0.803000pt}%
\definecolor{currentstroke}{rgb}{0.333333,0.333333,0.333333}%
\pgfsetstrokecolor{currentstroke}%
\pgfsetdash{}{0pt}%
\pgfsys@defobject{currentmarker}{\pgfqpoint{0.000000in}{-0.048611in}}{\pgfqpoint{0.000000in}{0.000000in}}{%
\pgfpathmoveto{\pgfqpoint{0.000000in}{0.000000in}}%
\pgfpathlineto{\pgfqpoint{0.000000in}{-0.048611in}}%
\pgfusepath{stroke,fill}%
}%
\begin{pgfscope}%
\pgfsys@transformshift{0.864306in}{0.524382in}%
\pgfsys@useobject{currentmarker}{}%
\end{pgfscope}%
\end{pgfscope}%
\begin{pgfscope}%
\definecolor{textcolor}{rgb}{0.333333,0.333333,0.333333}%
\pgfsetstrokecolor{textcolor}%
\pgfsetfillcolor{textcolor}%
\pgftext[x=0.864306in,y=0.427160in,,top]{\color{textcolor}\rmfamily\fontsize{10.000000}{12.000000}\selectfont \(\displaystyle 80.0\)}%
\end{pgfscope}%
\begin{pgfscope}%
\pgfpathrectangle{\pgfqpoint{0.647840in}{0.524382in}}{\pgfqpoint{4.762251in}{0.887161in}}%
\pgfusepath{clip}%
\pgfsetrectcap%
\pgfsetroundjoin%
\pgfsetlinewidth{0.803000pt}%
\definecolor{currentstroke}{rgb}{1.000000,1.000000,1.000000}%
\pgfsetstrokecolor{currentstroke}%
\pgfsetdash{}{0pt}%
\pgfpathmoveto{\pgfqpoint{1.406012in}{0.524382in}}%
\pgfpathlineto{\pgfqpoint{1.406012in}{1.411543in}}%
\pgfusepath{stroke}%
\end{pgfscope}%
\begin{pgfscope}%
\pgfsetbuttcap%
\pgfsetroundjoin%
\definecolor{currentfill}{rgb}{0.333333,0.333333,0.333333}%
\pgfsetfillcolor{currentfill}%
\pgfsetlinewidth{0.803000pt}%
\definecolor{currentstroke}{rgb}{0.333333,0.333333,0.333333}%
\pgfsetstrokecolor{currentstroke}%
\pgfsetdash{}{0pt}%
\pgfsys@defobject{currentmarker}{\pgfqpoint{0.000000in}{-0.048611in}}{\pgfqpoint{0.000000in}{0.000000in}}{%
\pgfpathmoveto{\pgfqpoint{0.000000in}{0.000000in}}%
\pgfpathlineto{\pgfqpoint{0.000000in}{-0.048611in}}%
\pgfusepath{stroke,fill}%
}%
\begin{pgfscope}%
\pgfsys@transformshift{1.406012in}{0.524382in}%
\pgfsys@useobject{currentmarker}{}%
\end{pgfscope}%
\end{pgfscope}%
\begin{pgfscope}%
\definecolor{textcolor}{rgb}{0.333333,0.333333,0.333333}%
\pgfsetstrokecolor{textcolor}%
\pgfsetfillcolor{textcolor}%
\pgftext[x=1.406012in,y=0.427160in,,top]{\color{textcolor}\rmfamily\fontsize{10.000000}{12.000000}\selectfont \(\displaystyle 82.5\)}%
\end{pgfscope}%
\begin{pgfscope}%
\pgfpathrectangle{\pgfqpoint{0.647840in}{0.524382in}}{\pgfqpoint{4.762251in}{0.887161in}}%
\pgfusepath{clip}%
\pgfsetrectcap%
\pgfsetroundjoin%
\pgfsetlinewidth{0.803000pt}%
\definecolor{currentstroke}{rgb}{1.000000,1.000000,1.000000}%
\pgfsetstrokecolor{currentstroke}%
\pgfsetdash{}{0pt}%
\pgfpathmoveto{\pgfqpoint{1.947719in}{0.524382in}}%
\pgfpathlineto{\pgfqpoint{1.947719in}{1.411543in}}%
\pgfusepath{stroke}%
\end{pgfscope}%
\begin{pgfscope}%
\pgfsetbuttcap%
\pgfsetroundjoin%
\definecolor{currentfill}{rgb}{0.333333,0.333333,0.333333}%
\pgfsetfillcolor{currentfill}%
\pgfsetlinewidth{0.803000pt}%
\definecolor{currentstroke}{rgb}{0.333333,0.333333,0.333333}%
\pgfsetstrokecolor{currentstroke}%
\pgfsetdash{}{0pt}%
\pgfsys@defobject{currentmarker}{\pgfqpoint{0.000000in}{-0.048611in}}{\pgfqpoint{0.000000in}{0.000000in}}{%
\pgfpathmoveto{\pgfqpoint{0.000000in}{0.000000in}}%
\pgfpathlineto{\pgfqpoint{0.000000in}{-0.048611in}}%
\pgfusepath{stroke,fill}%
}%
\begin{pgfscope}%
\pgfsys@transformshift{1.947719in}{0.524382in}%
\pgfsys@useobject{currentmarker}{}%
\end{pgfscope}%
\end{pgfscope}%
\begin{pgfscope}%
\definecolor{textcolor}{rgb}{0.333333,0.333333,0.333333}%
\pgfsetstrokecolor{textcolor}%
\pgfsetfillcolor{textcolor}%
\pgftext[x=1.947719in,y=0.427160in,,top]{\color{textcolor}\rmfamily\fontsize{10.000000}{12.000000}\selectfont \(\displaystyle 85.0\)}%
\end{pgfscope}%
\begin{pgfscope}%
\pgfpathrectangle{\pgfqpoint{0.647840in}{0.524382in}}{\pgfqpoint{4.762251in}{0.887161in}}%
\pgfusepath{clip}%
\pgfsetrectcap%
\pgfsetroundjoin%
\pgfsetlinewidth{0.803000pt}%
\definecolor{currentstroke}{rgb}{1.000000,1.000000,1.000000}%
\pgfsetstrokecolor{currentstroke}%
\pgfsetdash{}{0pt}%
\pgfpathmoveto{\pgfqpoint{2.489425in}{0.524382in}}%
\pgfpathlineto{\pgfqpoint{2.489425in}{1.411543in}}%
\pgfusepath{stroke}%
\end{pgfscope}%
\begin{pgfscope}%
\pgfsetbuttcap%
\pgfsetroundjoin%
\definecolor{currentfill}{rgb}{0.333333,0.333333,0.333333}%
\pgfsetfillcolor{currentfill}%
\pgfsetlinewidth{0.803000pt}%
\definecolor{currentstroke}{rgb}{0.333333,0.333333,0.333333}%
\pgfsetstrokecolor{currentstroke}%
\pgfsetdash{}{0pt}%
\pgfsys@defobject{currentmarker}{\pgfqpoint{0.000000in}{-0.048611in}}{\pgfqpoint{0.000000in}{0.000000in}}{%
\pgfpathmoveto{\pgfqpoint{0.000000in}{0.000000in}}%
\pgfpathlineto{\pgfqpoint{0.000000in}{-0.048611in}}%
\pgfusepath{stroke,fill}%
}%
\begin{pgfscope}%
\pgfsys@transformshift{2.489425in}{0.524382in}%
\pgfsys@useobject{currentmarker}{}%
\end{pgfscope}%
\end{pgfscope}%
\begin{pgfscope}%
\definecolor{textcolor}{rgb}{0.333333,0.333333,0.333333}%
\pgfsetstrokecolor{textcolor}%
\pgfsetfillcolor{textcolor}%
\pgftext[x=2.489425in,y=0.427160in,,top]{\color{textcolor}\rmfamily\fontsize{10.000000}{12.000000}\selectfont \(\displaystyle 87.5\)}%
\end{pgfscope}%
\begin{pgfscope}%
\pgfpathrectangle{\pgfqpoint{0.647840in}{0.524382in}}{\pgfqpoint{4.762251in}{0.887161in}}%
\pgfusepath{clip}%
\pgfsetrectcap%
\pgfsetroundjoin%
\pgfsetlinewidth{0.803000pt}%
\definecolor{currentstroke}{rgb}{1.000000,1.000000,1.000000}%
\pgfsetstrokecolor{currentstroke}%
\pgfsetdash{}{0pt}%
\pgfpathmoveto{\pgfqpoint{3.031132in}{0.524382in}}%
\pgfpathlineto{\pgfqpoint{3.031132in}{1.411543in}}%
\pgfusepath{stroke}%
\end{pgfscope}%
\begin{pgfscope}%
\pgfsetbuttcap%
\pgfsetroundjoin%
\definecolor{currentfill}{rgb}{0.333333,0.333333,0.333333}%
\pgfsetfillcolor{currentfill}%
\pgfsetlinewidth{0.803000pt}%
\definecolor{currentstroke}{rgb}{0.333333,0.333333,0.333333}%
\pgfsetstrokecolor{currentstroke}%
\pgfsetdash{}{0pt}%
\pgfsys@defobject{currentmarker}{\pgfqpoint{0.000000in}{-0.048611in}}{\pgfqpoint{0.000000in}{0.000000in}}{%
\pgfpathmoveto{\pgfqpoint{0.000000in}{0.000000in}}%
\pgfpathlineto{\pgfqpoint{0.000000in}{-0.048611in}}%
\pgfusepath{stroke,fill}%
}%
\begin{pgfscope}%
\pgfsys@transformshift{3.031132in}{0.524382in}%
\pgfsys@useobject{currentmarker}{}%
\end{pgfscope}%
\end{pgfscope}%
\begin{pgfscope}%
\definecolor{textcolor}{rgb}{0.333333,0.333333,0.333333}%
\pgfsetstrokecolor{textcolor}%
\pgfsetfillcolor{textcolor}%
\pgftext[x=3.031132in,y=0.427160in,,top]{\color{textcolor}\rmfamily\fontsize{10.000000}{12.000000}\selectfont \(\displaystyle 90.0\)}%
\end{pgfscope}%
\begin{pgfscope}%
\pgfpathrectangle{\pgfqpoint{0.647840in}{0.524382in}}{\pgfqpoint{4.762251in}{0.887161in}}%
\pgfusepath{clip}%
\pgfsetrectcap%
\pgfsetroundjoin%
\pgfsetlinewidth{0.803000pt}%
\definecolor{currentstroke}{rgb}{1.000000,1.000000,1.000000}%
\pgfsetstrokecolor{currentstroke}%
\pgfsetdash{}{0pt}%
\pgfpathmoveto{\pgfqpoint{3.572839in}{0.524382in}}%
\pgfpathlineto{\pgfqpoint{3.572839in}{1.411543in}}%
\pgfusepath{stroke}%
\end{pgfscope}%
\begin{pgfscope}%
\pgfsetbuttcap%
\pgfsetroundjoin%
\definecolor{currentfill}{rgb}{0.333333,0.333333,0.333333}%
\pgfsetfillcolor{currentfill}%
\pgfsetlinewidth{0.803000pt}%
\definecolor{currentstroke}{rgb}{0.333333,0.333333,0.333333}%
\pgfsetstrokecolor{currentstroke}%
\pgfsetdash{}{0pt}%
\pgfsys@defobject{currentmarker}{\pgfqpoint{0.000000in}{-0.048611in}}{\pgfqpoint{0.000000in}{0.000000in}}{%
\pgfpathmoveto{\pgfqpoint{0.000000in}{0.000000in}}%
\pgfpathlineto{\pgfqpoint{0.000000in}{-0.048611in}}%
\pgfusepath{stroke,fill}%
}%
\begin{pgfscope}%
\pgfsys@transformshift{3.572839in}{0.524382in}%
\pgfsys@useobject{currentmarker}{}%
\end{pgfscope}%
\end{pgfscope}%
\begin{pgfscope}%
\definecolor{textcolor}{rgb}{0.333333,0.333333,0.333333}%
\pgfsetstrokecolor{textcolor}%
\pgfsetfillcolor{textcolor}%
\pgftext[x=3.572839in,y=0.427160in,,top]{\color{textcolor}\rmfamily\fontsize{10.000000}{12.000000}\selectfont \(\displaystyle 92.5\)}%
\end{pgfscope}%
\begin{pgfscope}%
\pgfpathrectangle{\pgfqpoint{0.647840in}{0.524382in}}{\pgfqpoint{4.762251in}{0.887161in}}%
\pgfusepath{clip}%
\pgfsetrectcap%
\pgfsetroundjoin%
\pgfsetlinewidth{0.803000pt}%
\definecolor{currentstroke}{rgb}{1.000000,1.000000,1.000000}%
\pgfsetstrokecolor{currentstroke}%
\pgfsetdash{}{0pt}%
\pgfpathmoveto{\pgfqpoint{4.114545in}{0.524382in}}%
\pgfpathlineto{\pgfqpoint{4.114545in}{1.411543in}}%
\pgfusepath{stroke}%
\end{pgfscope}%
\begin{pgfscope}%
\pgfsetbuttcap%
\pgfsetroundjoin%
\definecolor{currentfill}{rgb}{0.333333,0.333333,0.333333}%
\pgfsetfillcolor{currentfill}%
\pgfsetlinewidth{0.803000pt}%
\definecolor{currentstroke}{rgb}{0.333333,0.333333,0.333333}%
\pgfsetstrokecolor{currentstroke}%
\pgfsetdash{}{0pt}%
\pgfsys@defobject{currentmarker}{\pgfqpoint{0.000000in}{-0.048611in}}{\pgfqpoint{0.000000in}{0.000000in}}{%
\pgfpathmoveto{\pgfqpoint{0.000000in}{0.000000in}}%
\pgfpathlineto{\pgfqpoint{0.000000in}{-0.048611in}}%
\pgfusepath{stroke,fill}%
}%
\begin{pgfscope}%
\pgfsys@transformshift{4.114545in}{0.524382in}%
\pgfsys@useobject{currentmarker}{}%
\end{pgfscope}%
\end{pgfscope}%
\begin{pgfscope}%
\definecolor{textcolor}{rgb}{0.333333,0.333333,0.333333}%
\pgfsetstrokecolor{textcolor}%
\pgfsetfillcolor{textcolor}%
\pgftext[x=4.114545in,y=0.427160in,,top]{\color{textcolor}\rmfamily\fontsize{10.000000}{12.000000}\selectfont \(\displaystyle 95.0\)}%
\end{pgfscope}%
\begin{pgfscope}%
\pgfpathrectangle{\pgfqpoint{0.647840in}{0.524382in}}{\pgfqpoint{4.762251in}{0.887161in}}%
\pgfusepath{clip}%
\pgfsetrectcap%
\pgfsetroundjoin%
\pgfsetlinewidth{0.803000pt}%
\definecolor{currentstroke}{rgb}{1.000000,1.000000,1.000000}%
\pgfsetstrokecolor{currentstroke}%
\pgfsetdash{}{0pt}%
\pgfpathmoveto{\pgfqpoint{4.656252in}{0.524382in}}%
\pgfpathlineto{\pgfqpoint{4.656252in}{1.411543in}}%
\pgfusepath{stroke}%
\end{pgfscope}%
\begin{pgfscope}%
\pgfsetbuttcap%
\pgfsetroundjoin%
\definecolor{currentfill}{rgb}{0.333333,0.333333,0.333333}%
\pgfsetfillcolor{currentfill}%
\pgfsetlinewidth{0.803000pt}%
\definecolor{currentstroke}{rgb}{0.333333,0.333333,0.333333}%
\pgfsetstrokecolor{currentstroke}%
\pgfsetdash{}{0pt}%
\pgfsys@defobject{currentmarker}{\pgfqpoint{0.000000in}{-0.048611in}}{\pgfqpoint{0.000000in}{0.000000in}}{%
\pgfpathmoveto{\pgfqpoint{0.000000in}{0.000000in}}%
\pgfpathlineto{\pgfqpoint{0.000000in}{-0.048611in}}%
\pgfusepath{stroke,fill}%
}%
\begin{pgfscope}%
\pgfsys@transformshift{4.656252in}{0.524382in}%
\pgfsys@useobject{currentmarker}{}%
\end{pgfscope}%
\end{pgfscope}%
\begin{pgfscope}%
\definecolor{textcolor}{rgb}{0.333333,0.333333,0.333333}%
\pgfsetstrokecolor{textcolor}%
\pgfsetfillcolor{textcolor}%
\pgftext[x=4.656252in,y=0.427160in,,top]{\color{textcolor}\rmfamily\fontsize{10.000000}{12.000000}\selectfont \(\displaystyle 97.5\)}%
\end{pgfscope}%
\begin{pgfscope}%
\pgfpathrectangle{\pgfqpoint{0.647840in}{0.524382in}}{\pgfqpoint{4.762251in}{0.887161in}}%
\pgfusepath{clip}%
\pgfsetrectcap%
\pgfsetroundjoin%
\pgfsetlinewidth{0.803000pt}%
\definecolor{currentstroke}{rgb}{1.000000,1.000000,1.000000}%
\pgfsetstrokecolor{currentstroke}%
\pgfsetdash{}{0pt}%
\pgfpathmoveto{\pgfqpoint{5.197958in}{0.524382in}}%
\pgfpathlineto{\pgfqpoint{5.197958in}{1.411543in}}%
\pgfusepath{stroke}%
\end{pgfscope}%
\begin{pgfscope}%
\pgfsetbuttcap%
\pgfsetroundjoin%
\definecolor{currentfill}{rgb}{0.333333,0.333333,0.333333}%
\pgfsetfillcolor{currentfill}%
\pgfsetlinewidth{0.803000pt}%
\definecolor{currentstroke}{rgb}{0.333333,0.333333,0.333333}%
\pgfsetstrokecolor{currentstroke}%
\pgfsetdash{}{0pt}%
\pgfsys@defobject{currentmarker}{\pgfqpoint{0.000000in}{-0.048611in}}{\pgfqpoint{0.000000in}{0.000000in}}{%
\pgfpathmoveto{\pgfqpoint{0.000000in}{0.000000in}}%
\pgfpathlineto{\pgfqpoint{0.000000in}{-0.048611in}}%
\pgfusepath{stroke,fill}%
}%
\begin{pgfscope}%
\pgfsys@transformshift{5.197958in}{0.524382in}%
\pgfsys@useobject{currentmarker}{}%
\end{pgfscope}%
\end{pgfscope}%
\begin{pgfscope}%
\definecolor{textcolor}{rgb}{0.333333,0.333333,0.333333}%
\pgfsetstrokecolor{textcolor}%
\pgfsetfillcolor{textcolor}%
\pgftext[x=5.197958in,y=0.427160in,,top]{\color{textcolor}\rmfamily\fontsize{10.000000}{12.000000}\selectfont \(\displaystyle 100.0\)}%
\end{pgfscope}%
\begin{pgfscope}%
\definecolor{textcolor}{rgb}{0.333333,0.333333,0.333333}%
\pgfsetstrokecolor{textcolor}%
\pgfsetfillcolor{textcolor}%
\pgftext[x=3.028965in,y=0.248148in,,top]{\color{textcolor}\rmfamily\fontsize{12.000000}{14.400000}\selectfont t}%
\end{pgfscope}%
\begin{pgfscope}%
\pgfpathrectangle{\pgfqpoint{0.647840in}{0.524382in}}{\pgfqpoint{4.762251in}{0.887161in}}%
\pgfusepath{clip}%
\pgfsetrectcap%
\pgfsetroundjoin%
\pgfsetlinewidth{0.803000pt}%
\definecolor{currentstroke}{rgb}{1.000000,1.000000,1.000000}%
\pgfsetstrokecolor{currentstroke}%
\pgfsetdash{}{0pt}%
\pgfpathmoveto{\pgfqpoint{0.647840in}{0.831537in}}%
\pgfpathlineto{\pgfqpoint{5.410091in}{0.831537in}}%
\pgfusepath{stroke}%
\end{pgfscope}%
\begin{pgfscope}%
\pgfsetbuttcap%
\pgfsetroundjoin%
\definecolor{currentfill}{rgb}{0.333333,0.333333,0.333333}%
\pgfsetfillcolor{currentfill}%
\pgfsetlinewidth{0.803000pt}%
\definecolor{currentstroke}{rgb}{0.333333,0.333333,0.333333}%
\pgfsetstrokecolor{currentstroke}%
\pgfsetdash{}{0pt}%
\pgfsys@defobject{currentmarker}{\pgfqpoint{-0.048611in}{0.000000in}}{\pgfqpoint{0.000000in}{0.000000in}}{%
\pgfpathmoveto{\pgfqpoint{0.000000in}{0.000000in}}%
\pgfpathlineto{\pgfqpoint{-0.048611in}{0.000000in}}%
\pgfusepath{stroke,fill}%
}%
\begin{pgfscope}%
\pgfsys@transformshift{0.647840in}{0.831537in}%
\pgfsys@useobject{currentmarker}{}%
\end{pgfscope}%
\end{pgfscope}%
\begin{pgfscope}%
\definecolor{textcolor}{rgb}{0.333333,0.333333,0.333333}%
\pgfsetstrokecolor{textcolor}%
\pgfsetfillcolor{textcolor}%
\pgftext[x=0.411728in,y=0.783312in,left,base]{\color{textcolor}\rmfamily\fontsize{10.000000}{12.000000}\selectfont \(\displaystyle 20\)}%
\end{pgfscope}%
\begin{pgfscope}%
\pgfpathrectangle{\pgfqpoint{0.647840in}{0.524382in}}{\pgfqpoint{4.762251in}{0.887161in}}%
\pgfusepath{clip}%
\pgfsetrectcap%
\pgfsetroundjoin%
\pgfsetlinewidth{0.803000pt}%
\definecolor{currentstroke}{rgb}{1.000000,1.000000,1.000000}%
\pgfsetstrokecolor{currentstroke}%
\pgfsetdash{}{0pt}%
\pgfpathmoveto{\pgfqpoint{0.647840in}{1.315335in}}%
\pgfpathlineto{\pgfqpoint{5.410091in}{1.315335in}}%
\pgfusepath{stroke}%
\end{pgfscope}%
\begin{pgfscope}%
\pgfsetbuttcap%
\pgfsetroundjoin%
\definecolor{currentfill}{rgb}{0.333333,0.333333,0.333333}%
\pgfsetfillcolor{currentfill}%
\pgfsetlinewidth{0.803000pt}%
\definecolor{currentstroke}{rgb}{0.333333,0.333333,0.333333}%
\pgfsetstrokecolor{currentstroke}%
\pgfsetdash{}{0pt}%
\pgfsys@defobject{currentmarker}{\pgfqpoint{-0.048611in}{0.000000in}}{\pgfqpoint{0.000000in}{0.000000in}}{%
\pgfpathmoveto{\pgfqpoint{0.000000in}{0.000000in}}%
\pgfpathlineto{\pgfqpoint{-0.048611in}{0.000000in}}%
\pgfusepath{stroke,fill}%
}%
\begin{pgfscope}%
\pgfsys@transformshift{0.647840in}{1.315335in}%
\pgfsys@useobject{currentmarker}{}%
\end{pgfscope}%
\end{pgfscope}%
\begin{pgfscope}%
\definecolor{textcolor}{rgb}{0.333333,0.333333,0.333333}%
\pgfsetstrokecolor{textcolor}%
\pgfsetfillcolor{textcolor}%
\pgftext[x=0.411728in,y=1.267110in,left,base]{\color{textcolor}\rmfamily\fontsize{10.000000}{12.000000}\selectfont \(\displaystyle 40\)}%
\end{pgfscope}%
\begin{pgfscope}%
\definecolor{textcolor}{rgb}{0.333333,0.333333,0.333333}%
\pgfsetstrokecolor{textcolor}%
\pgfsetfillcolor{textcolor}%
\pgftext[x=0.356173in,y=0.967963in,,bottom,rotate=90.000000]{\color{textcolor}\rmfamily\fontsize{12.000000}{14.400000}\selectfont z}%
\end{pgfscope}%
\begin{pgfscope}%
\pgfpathrectangle{\pgfqpoint{0.647840in}{0.524382in}}{\pgfqpoint{4.762251in}{0.887161in}}%
\pgfusepath{clip}%
\pgfsetrectcap%
\pgfsetroundjoin%
\pgfsetlinewidth{1.505625pt}%
\definecolor{currentstroke}{rgb}{0.886275,0.290196,0.200000}%
\pgfsetstrokecolor{currentstroke}%
\pgfsetdash{}{0pt}%
\pgfpathmoveto{\pgfqpoint{0.864306in}{0.649603in}}%
\pgfpathlineto{\pgfqpoint{0.868639in}{0.649927in}}%
\pgfpathlineto{\pgfqpoint{0.872973in}{0.655996in}}%
\pgfpathlineto{\pgfqpoint{0.877306in}{0.669513in}}%
\pgfpathlineto{\pgfqpoint{0.881640in}{0.692584in}}%
\pgfpathlineto{\pgfqpoint{0.885974in}{0.727599in}}%
\pgfpathlineto{\pgfqpoint{0.890307in}{0.776874in}}%
\pgfpathlineto{\pgfqpoint{0.894641in}{0.841911in}}%
\pgfpathlineto{\pgfqpoint{0.898975in}{0.922230in}}%
\pgfpathlineto{\pgfqpoint{0.911976in}{1.195955in}}%
\pgfpathlineto{\pgfqpoint{0.916309in}{1.262486in}}%
\pgfpathlineto{\pgfqpoint{0.920643in}{1.300320in}}%
\pgfpathlineto{\pgfqpoint{0.924977in}{1.307796in}}%
\pgfpathlineto{\pgfqpoint{0.929310in}{1.289685in}}%
\pgfpathlineto{\pgfqpoint{0.933644in}{1.254208in}}%
\pgfpathlineto{\pgfqpoint{0.937978in}{1.209610in}}%
\pgfpathlineto{\pgfqpoint{0.946645in}{1.115442in}}%
\pgfpathlineto{\pgfqpoint{0.950979in}{1.071392in}}%
\pgfpathlineto{\pgfqpoint{0.955312in}{1.030505in}}%
\pgfpathlineto{\pgfqpoint{0.959646in}{0.992716in}}%
\pgfpathlineto{\pgfqpoint{0.963980in}{0.957719in}}%
\pgfpathlineto{\pgfqpoint{0.968313in}{0.925169in}}%
\pgfpathlineto{\pgfqpoint{0.976981in}{0.866266in}}%
\pgfpathlineto{\pgfqpoint{0.985648in}{0.814314in}}%
\pgfpathlineto{\pgfqpoint{0.994315in}{0.768398in}}%
\pgfpathlineto{\pgfqpoint{0.998649in}{0.747563in}}%
\pgfpathlineto{\pgfqpoint{1.002982in}{0.728123in}}%
\pgfpathlineto{\pgfqpoint{1.007316in}{0.710102in}}%
\pgfpathlineto{\pgfqpoint{1.011650in}{0.693566in}}%
\pgfpathlineto{\pgfqpoint{1.015983in}{0.678632in}}%
\pgfpathlineto{\pgfqpoint{1.020317in}{0.665491in}}%
\pgfpathlineto{\pgfqpoint{1.024651in}{0.654430in}}%
\pgfpathlineto{\pgfqpoint{1.028984in}{0.645877in}}%
\pgfpathlineto{\pgfqpoint{1.033318in}{0.640442in}}%
\pgfpathlineto{\pgfqpoint{1.037652in}{0.638984in}}%
\pgfpathlineto{\pgfqpoint{1.041985in}{0.642686in}}%
\pgfpathlineto{\pgfqpoint{1.046319in}{0.653123in}}%
\pgfpathlineto{\pgfqpoint{1.050653in}{0.672312in}}%
\pgfpathlineto{\pgfqpoint{1.054986in}{0.702660in}}%
\pgfpathlineto{\pgfqpoint{1.059320in}{0.746726in}}%
\pgfpathlineto{\pgfqpoint{1.063654in}{0.806638in}}%
\pgfpathlineto{\pgfqpoint{1.067987in}{0.883042in}}%
\pgfpathlineto{\pgfqpoint{1.072321in}{0.973648in}}%
\pgfpathlineto{\pgfqpoint{1.080988in}{1.166906in}}%
\pgfpathlineto{\pgfqpoint{1.085322in}{1.245556in}}%
\pgfpathlineto{\pgfqpoint{1.089655in}{1.296912in}}%
\pgfpathlineto{\pgfqpoint{1.093989in}{1.316290in}}%
\pgfpathlineto{\pgfqpoint{1.098323in}{1.306452in}}%
\pgfpathlineto{\pgfqpoint{1.102656in}{1.275245in}}%
\pgfpathlineto{\pgfqpoint{1.106990in}{1.231765in}}%
\pgfpathlineto{\pgfqpoint{1.119991in}{1.089325in}}%
\pgfpathlineto{\pgfqpoint{1.124325in}{1.046788in}}%
\pgfpathlineto{\pgfqpoint{1.128658in}{1.007614in}}%
\pgfpathlineto{\pgfqpoint{1.132992in}{0.971499in}}%
\pgfpathlineto{\pgfqpoint{1.137326in}{0.938059in}}%
\pgfpathlineto{\pgfqpoint{1.141659in}{0.906946in}}%
\pgfpathlineto{\pgfqpoint{1.150327in}{0.850659in}}%
\pgfpathlineto{\pgfqpoint{1.158994in}{0.801210in}}%
\pgfpathlineto{\pgfqpoint{1.163328in}{0.778828in}}%
\pgfpathlineto{\pgfqpoint{1.167661in}{0.757965in}}%
\pgfpathlineto{\pgfqpoint{1.171995in}{0.738637in}}%
\pgfpathlineto{\pgfqpoint{1.176329in}{0.720901in}}%
\pgfpathlineto{\pgfqpoint{1.180662in}{0.704869in}}%
\pgfpathlineto{\pgfqpoint{1.184996in}{0.690724in}}%
\pgfpathlineto{\pgfqpoint{1.189329in}{0.678744in}}%
\pgfpathlineto{\pgfqpoint{1.193663in}{0.669337in}}%
\pgfpathlineto{\pgfqpoint{1.197997in}{0.663084in}}%
\pgfpathlineto{\pgfqpoint{1.202330in}{0.660794in}}%
\pgfpathlineto{\pgfqpoint{1.206664in}{0.663567in}}%
\pgfpathlineto{\pgfqpoint{1.210998in}{0.672854in}}%
\pgfpathlineto{\pgfqpoint{1.215331in}{0.690487in}}%
\pgfpathlineto{\pgfqpoint{1.219665in}{0.718622in}}%
\pgfpathlineto{\pgfqpoint{1.223999in}{0.759515in}}%
\pgfpathlineto{\pgfqpoint{1.228332in}{0.815009in}}%
\pgfpathlineto{\pgfqpoint{1.232666in}{0.885620in}}%
\pgfpathlineto{\pgfqpoint{1.237000in}{0.969319in}}%
\pgfpathlineto{\pgfqpoint{1.245667in}{1.149345in}}%
\pgfpathlineto{\pgfqpoint{1.250001in}{1.224572in}}%
\pgfpathlineto{\pgfqpoint{1.254334in}{1.275970in}}%
\pgfpathlineto{\pgfqpoint{1.258668in}{1.298511in}}%
\pgfpathlineto{\pgfqpoint{1.263002in}{1.293645in}}%
\pgfpathlineto{\pgfqpoint{1.267335in}{1.267696in}}%
\pgfpathlineto{\pgfqpoint{1.271669in}{1.228683in}}%
\pgfpathlineto{\pgfqpoint{1.289004in}{1.050680in}}%
\pgfpathlineto{\pgfqpoint{1.293337in}{1.011610in}}%
\pgfpathlineto{\pgfqpoint{1.297671in}{0.975346in}}%
\pgfpathlineto{\pgfqpoint{1.302004in}{0.941599in}}%
\pgfpathlineto{\pgfqpoint{1.310672in}{0.880499in}}%
\pgfpathlineto{\pgfqpoint{1.319339in}{0.826489in}}%
\pgfpathlineto{\pgfqpoint{1.328006in}{0.778459in}}%
\pgfpathlineto{\pgfqpoint{1.336674in}{0.735767in}}%
\pgfpathlineto{\pgfqpoint{1.345341in}{0.698090in}}%
\pgfpathlineto{\pgfqpoint{1.349675in}{0.681121in}}%
\pgfpathlineto{\pgfqpoint{1.354008in}{0.665448in}}%
\pgfpathlineto{\pgfqpoint{1.358342in}{0.651160in}}%
\pgfpathlineto{\pgfqpoint{1.362676in}{0.638409in}}%
\pgfpathlineto{\pgfqpoint{1.367009in}{0.627430in}}%
\pgfpathlineto{\pgfqpoint{1.371343in}{0.618577in}}%
\pgfpathlineto{\pgfqpoint{1.375677in}{0.612371in}}%
\pgfpathlineto{\pgfqpoint{1.380010in}{0.609564in}}%
\pgfpathlineto{\pgfqpoint{1.384344in}{0.611218in}}%
\pgfpathlineto{\pgfqpoint{1.388678in}{0.618796in}}%
\pgfpathlineto{\pgfqpoint{1.393011in}{0.634254in}}%
\pgfpathlineto{\pgfqpoint{1.397345in}{0.660078in}}%
\pgfpathlineto{\pgfqpoint{1.401678in}{0.699171in}}%
\pgfpathlineto{\pgfqpoint{1.406012in}{0.754458in}}%
\pgfpathlineto{\pgfqpoint{1.410346in}{0.827975in}}%
\pgfpathlineto{\pgfqpoint{1.414679in}{0.919356in}}%
\pgfpathlineto{\pgfqpoint{1.427680in}{1.228260in}}%
\pgfpathlineto{\pgfqpoint{1.432014in}{1.298873in}}%
\pgfpathlineto{\pgfqpoint{1.436348in}{1.334527in}}%
\pgfpathlineto{\pgfqpoint{1.440681in}{1.335094in}}%
\pgfpathlineto{\pgfqpoint{1.445015in}{1.308105in}}%
\pgfpathlineto{\pgfqpoint{1.449349in}{1.264260in}}%
\pgfpathlineto{\pgfqpoint{1.462350in}{1.112188in}}%
\pgfpathlineto{\pgfqpoint{1.466683in}{1.066997in}}%
\pgfpathlineto{\pgfqpoint{1.471017in}{1.025863in}}%
\pgfpathlineto{\pgfqpoint{1.475351in}{0.988395in}}%
\pgfpathlineto{\pgfqpoint{1.479684in}{0.954084in}}%
\pgfpathlineto{\pgfqpoint{1.484018in}{0.922475in}}%
\pgfpathlineto{\pgfqpoint{1.488352in}{0.893215in}}%
\pgfpathlineto{\pgfqpoint{1.492685in}{0.866057in}}%
\pgfpathlineto{\pgfqpoint{1.497019in}{0.840845in}}%
\pgfpathlineto{\pgfqpoint{1.501352in}{0.817495in}}%
\pgfpathlineto{\pgfqpoint{1.505686in}{0.795989in}}%
\pgfpathlineto{\pgfqpoint{1.510020in}{0.776369in}}%
\pgfpathlineto{\pgfqpoint{1.514353in}{0.758742in}}%
\pgfpathlineto{\pgfqpoint{1.518687in}{0.743295in}}%
\pgfpathlineto{\pgfqpoint{1.523021in}{0.730310in}}%
\pgfpathlineto{\pgfqpoint{1.527354in}{0.720195in}}%
\pgfpathlineto{\pgfqpoint{1.531688in}{0.713516in}}%
\pgfpathlineto{\pgfqpoint{1.536022in}{0.711039in}}%
\pgfpathlineto{\pgfqpoint{1.540355in}{0.713765in}}%
\pgfpathlineto{\pgfqpoint{1.544689in}{0.722959in}}%
\pgfpathlineto{\pgfqpoint{1.549023in}{0.740131in}}%
\pgfpathlineto{\pgfqpoint{1.553356in}{0.766938in}}%
\pgfpathlineto{\pgfqpoint{1.557690in}{0.804931in}}%
\pgfpathlineto{\pgfqpoint{1.562024in}{0.855091in}}%
\pgfpathlineto{\pgfqpoint{1.566357in}{0.917118in}}%
\pgfpathlineto{\pgfqpoint{1.575025in}{1.064483in}}%
\pgfpathlineto{\pgfqpoint{1.579358in}{1.137158in}}%
\pgfpathlineto{\pgfqpoint{1.583692in}{1.198028in}}%
\pgfpathlineto{\pgfqpoint{1.588026in}{1.239852in}}%
\pgfpathlineto{\pgfqpoint{1.592359in}{1.259027in}}%
\pgfpathlineto{\pgfqpoint{1.596693in}{1.256409in}}%
\pgfpathlineto{\pgfqpoint{1.601027in}{1.236305in}}%
\pgfpathlineto{\pgfqpoint{1.605360in}{1.204462in}}%
\pgfpathlineto{\pgfqpoint{1.614027in}{1.125654in}}%
\pgfpathlineto{\pgfqpoint{1.622695in}{1.046230in}}%
\pgfpathlineto{\pgfqpoint{1.631362in}{0.974419in}}%
\pgfpathlineto{\pgfqpoint{1.640029in}{0.910741in}}%
\pgfpathlineto{\pgfqpoint{1.648697in}{0.854090in}}%
\pgfpathlineto{\pgfqpoint{1.657364in}{0.803494in}}%
\pgfpathlineto{\pgfqpoint{1.666031in}{0.758310in}}%
\pgfpathlineto{\pgfqpoint{1.674699in}{0.718164in}}%
\pgfpathlineto{\pgfqpoint{1.679032in}{0.699932in}}%
\pgfpathlineto{\pgfqpoint{1.683366in}{0.682946in}}%
\pgfpathlineto{\pgfqpoint{1.687700in}{0.667254in}}%
\pgfpathlineto{\pgfqpoint{1.692033in}{0.652952in}}%
\pgfpathlineto{\pgfqpoint{1.696367in}{0.640193in}}%
\pgfpathlineto{\pgfqpoint{1.700701in}{0.629219in}}%
\pgfpathlineto{\pgfqpoint{1.705034in}{0.620391in}}%
\pgfpathlineto{\pgfqpoint{1.709368in}{0.614235in}}%
\pgfpathlineto{\pgfqpoint{1.713701in}{0.611513in}}%
\pgfpathlineto{\pgfqpoint{1.718035in}{0.613293in}}%
\pgfpathlineto{\pgfqpoint{1.722369in}{0.621049in}}%
\pgfpathlineto{\pgfqpoint{1.726702in}{0.636744in}}%
\pgfpathlineto{\pgfqpoint{1.731036in}{0.662860in}}%
\pgfpathlineto{\pgfqpoint{1.735370in}{0.702288in}}%
\pgfpathlineto{\pgfqpoint{1.739703in}{0.757903in}}%
\pgfpathlineto{\pgfqpoint{1.744037in}{0.831659in}}%
\pgfpathlineto{\pgfqpoint{1.748371in}{0.923062in}}%
\pgfpathlineto{\pgfqpoint{1.761372in}{1.229679in}}%
\pgfpathlineto{\pgfqpoint{1.765705in}{1.298966in}}%
\pgfpathlineto{\pgfqpoint{1.770039in}{1.333451in}}%
\pgfpathlineto{\pgfqpoint{1.774373in}{1.333238in}}%
\pgfpathlineto{\pgfqpoint{1.778706in}{1.305909in}}%
\pgfpathlineto{\pgfqpoint{1.783040in}{1.262062in}}%
\pgfpathlineto{\pgfqpoint{1.796041in}{1.110648in}}%
\pgfpathlineto{\pgfqpoint{1.800375in}{1.065640in}}%
\pgfpathlineto{\pgfqpoint{1.804708in}{1.024639in}}%
\pgfpathlineto{\pgfqpoint{1.809042in}{0.987258in}}%
\pgfpathlineto{\pgfqpoint{1.813375in}{0.953000in}}%
\pgfpathlineto{\pgfqpoint{1.817709in}{0.921415in}}%
\pgfpathlineto{\pgfqpoint{1.822043in}{0.892158in}}%
\pgfpathlineto{\pgfqpoint{1.826376in}{0.864985in}}%
\pgfpathlineto{\pgfqpoint{1.830710in}{0.839739in}}%
\pgfpathlineto{\pgfqpoint{1.835044in}{0.816336in}}%
\pgfpathlineto{\pgfqpoint{1.839377in}{0.794756in}}%
\pgfpathlineto{\pgfqpoint{1.843711in}{0.775035in}}%
\pgfpathlineto{\pgfqpoint{1.848045in}{0.757274in}}%
\pgfpathlineto{\pgfqpoint{1.852378in}{0.741652in}}%
\pgfpathlineto{\pgfqpoint{1.856712in}{0.728439in}}%
\pgfpathlineto{\pgfqpoint{1.861046in}{0.718030in}}%
\pgfpathlineto{\pgfqpoint{1.865379in}{0.710974in}}%
\pgfpathlineto{\pgfqpoint{1.869713in}{0.708017in}}%
\pgfpathlineto{\pgfqpoint{1.874047in}{0.710140in}}%
\pgfpathlineto{\pgfqpoint{1.878380in}{0.718593in}}%
\pgfpathlineto{\pgfqpoint{1.882714in}{0.734883in}}%
\pgfpathlineto{\pgfqpoint{1.887048in}{0.760691in}}%
\pgfpathlineto{\pgfqpoint{1.891381in}{0.797638in}}%
\pgfpathlineto{\pgfqpoint{1.895715in}{0.846843in}}%
\pgfpathlineto{\pgfqpoint{1.900049in}{0.908223in}}%
\pgfpathlineto{\pgfqpoint{1.904382in}{0.979652in}}%
\pgfpathlineto{\pgfqpoint{1.913050in}{1.130672in}}%
\pgfpathlineto{\pgfqpoint{1.917383in}{1.194028in}}%
\pgfpathlineto{\pgfqpoint{1.921717in}{1.238689in}}%
\pgfpathlineto{\pgfqpoint{1.926050in}{1.260513in}}%
\pgfpathlineto{\pgfqpoint{1.930384in}{1.259929in}}%
\pgfpathlineto{\pgfqpoint{1.934718in}{1.241073in}}%
\pgfpathlineto{\pgfqpoint{1.939051in}{1.209768in}}%
\pgfpathlineto{\pgfqpoint{1.943385in}{1.171577in}}%
\pgfpathlineto{\pgfqpoint{1.956386in}{1.050507in}}%
\pgfpathlineto{\pgfqpoint{1.965053in}{0.977966in}}%
\pgfpathlineto{\pgfqpoint{1.973721in}{0.913717in}}%
\pgfpathlineto{\pgfqpoint{1.982388in}{0.856600in}}%
\pgfpathlineto{\pgfqpoint{1.991055in}{0.805580in}}%
\pgfpathlineto{\pgfqpoint{1.999723in}{0.759965in}}%
\pgfpathlineto{\pgfqpoint{2.008390in}{0.719318in}}%
\pgfpathlineto{\pgfqpoint{2.017057in}{0.683434in}}%
\pgfpathlineto{\pgfqpoint{2.021391in}{0.667296in}}%
\pgfpathlineto{\pgfqpoint{2.025724in}{0.652428in}}%
\pgfpathlineto{\pgfqpoint{2.030058in}{0.638935in}}%
\pgfpathlineto{\pgfqpoint{2.034392in}{0.626990in}}%
\pgfpathlineto{\pgfqpoint{2.038725in}{0.616858in}}%
\pgfpathlineto{\pgfqpoint{2.043059in}{0.608935in}}%
\pgfpathlineto{\pgfqpoint{2.047393in}{0.603802in}}%
\pgfpathlineto{\pgfqpoint{2.051726in}{0.602293in}}%
\pgfpathlineto{\pgfqpoint{2.056060in}{0.605583in}}%
\pgfpathlineto{\pgfqpoint{2.060394in}{0.615290in}}%
\pgfpathlineto{\pgfqpoint{2.064727in}{0.633559in}}%
\pgfpathlineto{\pgfqpoint{2.069061in}{0.663079in}}%
\pgfpathlineto{\pgfqpoint{2.073395in}{0.706919in}}%
\pgfpathlineto{\pgfqpoint{2.077728in}{0.768002in}}%
\pgfpathlineto{\pgfqpoint{2.082062in}{0.848005in}}%
\pgfpathlineto{\pgfqpoint{2.086396in}{0.945624in}}%
\pgfpathlineto{\pgfqpoint{2.095063in}{1.163235in}}%
\pgfpathlineto{\pgfqpoint{2.099397in}{1.255770in}}%
\pgfpathlineto{\pgfqpoint{2.103730in}{1.318175in}}%
\pgfpathlineto{\pgfqpoint{2.108064in}{1.343573in}}%
\pgfpathlineto{\pgfqpoint{2.112398in}{1.334556in}}%
\pgfpathlineto{\pgfqpoint{2.116731in}{1.300507in}}%
\pgfpathlineto{\pgfqpoint{2.121065in}{1.252577in}}%
\pgfpathlineto{\pgfqpoint{2.129732in}{1.147887in}}%
\pgfpathlineto{\pgfqpoint{2.134066in}{1.099433in}}%
\pgfpathlineto{\pgfqpoint{2.138399in}{1.055253in}}%
\pgfpathlineto{\pgfqpoint{2.142733in}{1.015190in}}%
\pgfpathlineto{\pgfqpoint{2.147067in}{0.978747in}}%
\pgfpathlineto{\pgfqpoint{2.151400in}{0.945392in}}%
\pgfpathlineto{\pgfqpoint{2.155734in}{0.914681in}}%
\pgfpathlineto{\pgfqpoint{2.160068in}{0.886286in}}%
\pgfpathlineto{\pgfqpoint{2.164401in}{0.859990in}}%
\pgfpathlineto{\pgfqpoint{2.168735in}{0.835665in}}%
\pgfpathlineto{\pgfqpoint{2.173069in}{0.813260in}}%
\pgfpathlineto{\pgfqpoint{2.177402in}{0.792792in}}%
\pgfpathlineto{\pgfqpoint{2.181736in}{0.774345in}}%
\pgfpathlineto{\pgfqpoint{2.186070in}{0.758082in}}%
\pgfpathlineto{\pgfqpoint{2.190403in}{0.744256in}}%
\pgfpathlineto{\pgfqpoint{2.194737in}{0.733237in}}%
\pgfpathlineto{\pgfqpoint{2.199071in}{0.725542in}}%
\pgfpathlineto{\pgfqpoint{2.203404in}{0.721865in}}%
\pgfpathlineto{\pgfqpoint{2.207738in}{0.723122in}}%
\pgfpathlineto{\pgfqpoint{2.212072in}{0.730465in}}%
\pgfpathlineto{\pgfqpoint{2.216405in}{0.745280in}}%
\pgfpathlineto{\pgfqpoint{2.220739in}{0.769101in}}%
\pgfpathlineto{\pgfqpoint{2.225072in}{0.803403in}}%
\pgfpathlineto{\pgfqpoint{2.229406in}{0.849214in}}%
\pgfpathlineto{\pgfqpoint{2.233740in}{0.906498in}}%
\pgfpathlineto{\pgfqpoint{2.238073in}{0.973411in}}%
\pgfpathlineto{\pgfqpoint{2.246741in}{1.116701in}}%
\pgfpathlineto{\pgfqpoint{2.251074in}{1.178421in}}%
\pgfpathlineto{\pgfqpoint{2.255408in}{1.223572in}}%
\pgfpathlineto{\pgfqpoint{2.259742in}{1.247785in}}%
\pgfpathlineto{\pgfqpoint{2.264075in}{1.250752in}}%
\pgfpathlineto{\pgfqpoint{2.268409in}{1.235730in}}%
\pgfpathlineto{\pgfqpoint{2.272743in}{1.207856in}}%
\pgfpathlineto{\pgfqpoint{2.277076in}{1.172357in}}%
\pgfpathlineto{\pgfqpoint{2.294411in}{1.018348in}}%
\pgfpathlineto{\pgfqpoint{2.303078in}{0.950485in}}%
\pgfpathlineto{\pgfqpoint{2.311746in}{0.890107in}}%
\pgfpathlineto{\pgfqpoint{2.320413in}{0.836289in}}%
\pgfpathlineto{\pgfqpoint{2.329080in}{0.788332in}}%
\pgfpathlineto{\pgfqpoint{2.337747in}{0.745859in}}%
\pgfpathlineto{\pgfqpoint{2.342081in}{0.726653in}}%
\pgfpathlineto{\pgfqpoint{2.346415in}{0.708838in}}%
\pgfpathlineto{\pgfqpoint{2.350748in}{0.692490in}}%
\pgfpathlineto{\pgfqpoint{2.355082in}{0.677735in}}%
\pgfpathlineto{\pgfqpoint{2.359416in}{0.664774in}}%
\pgfpathlineto{\pgfqpoint{2.363749in}{0.653907in}}%
\pgfpathlineto{\pgfqpoint{2.368083in}{0.645573in}}%
\pgfpathlineto{\pgfqpoint{2.372417in}{0.640401in}}%
\pgfpathlineto{\pgfqpoint{2.376750in}{0.639270in}}%
\pgfpathlineto{\pgfqpoint{2.381084in}{0.643387in}}%
\pgfpathlineto{\pgfqpoint{2.385418in}{0.654358in}}%
\pgfpathlineto{\pgfqpoint{2.389751in}{0.674227in}}%
\pgfpathlineto{\pgfqpoint{2.394085in}{0.705419in}}%
\pgfpathlineto{\pgfqpoint{2.398419in}{0.750484in}}%
\pgfpathlineto{\pgfqpoint{2.402752in}{0.811483in}}%
\pgfpathlineto{\pgfqpoint{2.407086in}{0.888906in}}%
\pgfpathlineto{\pgfqpoint{2.411420in}{0.980213in}}%
\pgfpathlineto{\pgfqpoint{2.420087in}{1.172758in}}%
\pgfpathlineto{\pgfqpoint{2.424421in}{1.249781in}}%
\pgfpathlineto{\pgfqpoint{2.428754in}{1.298999in}}%
\pgfpathlineto{\pgfqpoint{2.433088in}{1.316242in}}%
\pgfpathlineto{\pgfqpoint{2.437421in}{1.304711in}}%
\pgfpathlineto{\pgfqpoint{2.441755in}{1.272443in}}%
\pgfpathlineto{\pgfqpoint{2.446089in}{1.228478in}}%
\pgfpathlineto{\pgfqpoint{2.459090in}{1.086321in}}%
\pgfpathlineto{\pgfqpoint{2.463423in}{1.044025in}}%
\pgfpathlineto{\pgfqpoint{2.467757in}{1.005073in}}%
\pgfpathlineto{\pgfqpoint{2.472091in}{0.969152in}}%
\pgfpathlineto{\pgfqpoint{2.476424in}{0.935878in}}%
\pgfpathlineto{\pgfqpoint{2.480758in}{0.904906in}}%
\pgfpathlineto{\pgfqpoint{2.489425in}{0.848855in}}%
\pgfpathlineto{\pgfqpoint{2.498093in}{0.799602in}}%
\pgfpathlineto{\pgfqpoint{2.502426in}{0.777309in}}%
\pgfpathlineto{\pgfqpoint{2.506760in}{0.756531in}}%
\pgfpathlineto{\pgfqpoint{2.511094in}{0.737287in}}%
\pgfpathlineto{\pgfqpoint{2.515427in}{0.719633in}}%
\pgfpathlineto{\pgfqpoint{2.519761in}{0.703686in}}%
\pgfpathlineto{\pgfqpoint{2.524095in}{0.689630in}}%
\pgfpathlineto{\pgfqpoint{2.528428in}{0.677747in}}%
\pgfpathlineto{\pgfqpoint{2.532762in}{0.668450in}}%
\pgfpathlineto{\pgfqpoint{2.537095in}{0.662327in}}%
\pgfpathlineto{\pgfqpoint{2.541429in}{0.660198in}}%
\pgfpathlineto{\pgfqpoint{2.545763in}{0.663173in}}%
\pgfpathlineto{\pgfqpoint{2.550096in}{0.672720in}}%
\pgfpathlineto{\pgfqpoint{2.554430in}{0.690686in}}%
\pgfpathlineto{\pgfqpoint{2.558764in}{0.719242in}}%
\pgfpathlineto{\pgfqpoint{2.563097in}{0.760649in}}%
\pgfpathlineto{\pgfqpoint{2.567431in}{0.816731in}}%
\pgfpathlineto{\pgfqpoint{2.571765in}{0.887951in}}%
\pgfpathlineto{\pgfqpoint{2.576098in}{0.972170in}}%
\pgfpathlineto{\pgfqpoint{2.584766in}{1.152403in}}%
\pgfpathlineto{\pgfqpoint{2.589099in}{1.227126in}}%
\pgfpathlineto{\pgfqpoint{2.593433in}{1.277680in}}%
\pgfpathlineto{\pgfqpoint{2.597767in}{1.299242in}}%
\pgfpathlineto{\pgfqpoint{2.602100in}{1.293484in}}%
\pgfpathlineto{\pgfqpoint{2.606434in}{1.266869in}}%
\pgfpathlineto{\pgfqpoint{2.610768in}{1.227451in}}%
\pgfpathlineto{\pgfqpoint{2.628102in}{1.049322in}}%
\pgfpathlineto{\pgfqpoint{2.632436in}{1.010336in}}%
\pgfpathlineto{\pgfqpoint{2.636770in}{0.974155in}}%
\pgfpathlineto{\pgfqpoint{2.641103in}{0.940485in}}%
\pgfpathlineto{\pgfqpoint{2.649770in}{0.879526in}}%
\pgfpathlineto{\pgfqpoint{2.658438in}{0.825641in}}%
\pgfpathlineto{\pgfqpoint{2.667105in}{0.777733in}}%
\pgfpathlineto{\pgfqpoint{2.675772in}{0.735170in}}%
\pgfpathlineto{\pgfqpoint{2.684440in}{0.697649in}}%
\pgfpathlineto{\pgfqpoint{2.688773in}{0.680775in}}%
\pgfpathlineto{\pgfqpoint{2.693107in}{0.665217in}}%
\pgfpathlineto{\pgfqpoint{2.697441in}{0.651072in}}%
\pgfpathlineto{\pgfqpoint{2.701774in}{0.638503in}}%
\pgfpathlineto{\pgfqpoint{2.706108in}{0.627762in}}%
\pgfpathlineto{\pgfqpoint{2.710442in}{0.619224in}}%
\pgfpathlineto{\pgfqpoint{2.714775in}{0.613444in}}%
\pgfpathlineto{\pgfqpoint{2.719109in}{0.611213in}}%
\pgfpathlineto{\pgfqpoint{2.723443in}{0.613646in}}%
\pgfpathlineto{\pgfqpoint{2.727776in}{0.622275in}}%
\pgfpathlineto{\pgfqpoint{2.732110in}{0.639128in}}%
\pgfpathlineto{\pgfqpoint{2.736444in}{0.666752in}}%
\pgfpathlineto{\pgfqpoint{2.740777in}{0.708070in}}%
\pgfpathlineto{\pgfqpoint{2.745111in}{0.765910in}}%
\pgfpathlineto{\pgfqpoint{2.749444in}{0.842019in}}%
\pgfpathlineto{\pgfqpoint{2.753778in}{0.935471in}}%
\pgfpathlineto{\pgfqpoint{2.762445in}{1.147329in}}%
\pgfpathlineto{\pgfqpoint{2.766779in}{1.240208in}}%
\pgfpathlineto{\pgfqpoint{2.771113in}{1.305500in}}%
\pgfpathlineto{\pgfqpoint{2.775446in}{1.335457in}}%
\pgfpathlineto{\pgfqpoint{2.779780in}{1.331295in}}%
\pgfpathlineto{\pgfqpoint{2.784114in}{1.301261in}}%
\pgfpathlineto{\pgfqpoint{2.788447in}{1.256018in}}%
\pgfpathlineto{\pgfqpoint{2.801448in}{1.104810in}}%
\pgfpathlineto{\pgfqpoint{2.805782in}{1.060305in}}%
\pgfpathlineto{\pgfqpoint{2.810116in}{1.019781in}}%
\pgfpathlineto{\pgfqpoint{2.814449in}{0.982821in}}%
\pgfpathlineto{\pgfqpoint{2.818783in}{0.948925in}}%
\pgfpathlineto{\pgfqpoint{2.823117in}{0.917656in}}%
\pgfpathlineto{\pgfqpoint{2.827450in}{0.888680in}}%
\pgfpathlineto{\pgfqpoint{2.831784in}{0.861766in}}%
\pgfpathlineto{\pgfqpoint{2.836118in}{0.836768in}}%
\pgfpathlineto{\pgfqpoint{2.840451in}{0.813610in}}%
\pgfpathlineto{\pgfqpoint{2.844785in}{0.792278in}}%
\pgfpathlineto{\pgfqpoint{2.849118in}{0.772819in}}%
\pgfpathlineto{\pgfqpoint{2.853452in}{0.755344in}}%
\pgfpathlineto{\pgfqpoint{2.857786in}{0.740042in}}%
\pgfpathlineto{\pgfqpoint{2.862119in}{0.727202in}}%
\pgfpathlineto{\pgfqpoint{2.866453in}{0.717238in}}%
\pgfpathlineto{\pgfqpoint{2.870787in}{0.710723in}}%
\pgfpathlineto{\pgfqpoint{2.875120in}{0.708436in}}%
\pgfpathlineto{\pgfqpoint{2.879454in}{0.711394in}}%
\pgfpathlineto{\pgfqpoint{2.883788in}{0.720885in}}%
\pgfpathlineto{\pgfqpoint{2.888121in}{0.738446in}}%
\pgfpathlineto{\pgfqpoint{2.892455in}{0.765761in}}%
\pgfpathlineto{\pgfqpoint{2.896789in}{0.804409in}}%
\pgfpathlineto{\pgfqpoint{2.901122in}{0.855372in}}%
\pgfpathlineto{\pgfqpoint{2.905456in}{0.918319in}}%
\pgfpathlineto{\pgfqpoint{2.914123in}{1.067449in}}%
\pgfpathlineto{\pgfqpoint{2.918457in}{1.140635in}}%
\pgfpathlineto{\pgfqpoint{2.922791in}{1.201567in}}%
\pgfpathlineto{\pgfqpoint{2.927124in}{1.242983in}}%
\pgfpathlineto{\pgfqpoint{2.931458in}{1.261394in}}%
\pgfpathlineto{\pgfqpoint{2.935792in}{1.257855in}}%
\pgfpathlineto{\pgfqpoint{2.940125in}{1.236870in}}%
\pgfpathlineto{\pgfqpoint{2.944459in}{1.204308in}}%
\pgfpathlineto{\pgfqpoint{2.953126in}{1.124643in}}%
\pgfpathlineto{\pgfqpoint{2.961793in}{1.044913in}}%
\pgfpathlineto{\pgfqpoint{2.970461in}{0.973038in}}%
\pgfpathlineto{\pgfqpoint{2.979128in}{0.909372in}}%
\pgfpathlineto{\pgfqpoint{2.987795in}{0.852745in}}%
\pgfpathlineto{\pgfqpoint{2.996463in}{0.802155in}}%
\pgfpathlineto{\pgfqpoint{3.005130in}{0.756936in}}%
\pgfpathlineto{\pgfqpoint{3.013797in}{0.716674in}}%
\pgfpathlineto{\pgfqpoint{3.022465in}{0.681194in}}%
\pgfpathlineto{\pgfqpoint{3.026798in}{0.665281in}}%
\pgfpathlineto{\pgfqpoint{3.031132in}{0.650665in}}%
\pgfpathlineto{\pgfqpoint{3.035466in}{0.637468in}}%
\pgfpathlineto{\pgfqpoint{3.039799in}{0.625880in}}%
\pgfpathlineto{\pgfqpoint{3.044133in}{0.616197in}}%
\pgfpathlineto{\pgfqpoint{3.048467in}{0.608854in}}%
\pgfpathlineto{\pgfqpoint{3.052800in}{0.604487in}}%
\pgfpathlineto{\pgfqpoint{3.057134in}{0.604004in}}%
\pgfpathlineto{\pgfqpoint{3.061467in}{0.608677in}}%
\pgfpathlineto{\pgfqpoint{3.065801in}{0.620243in}}%
\pgfpathlineto{\pgfqpoint{3.070135in}{0.640973in}}%
\pgfpathlineto{\pgfqpoint{3.074468in}{0.673662in}}%
\pgfpathlineto{\pgfqpoint{3.078802in}{0.721389in}}%
\pgfpathlineto{\pgfqpoint{3.083136in}{0.786871in}}%
\pgfpathlineto{\pgfqpoint{3.087469in}{0.871208in}}%
\pgfpathlineto{\pgfqpoint{3.091803in}{0.972048in}}%
\pgfpathlineto{\pgfqpoint{3.100470in}{1.187479in}}%
\pgfpathlineto{\pgfqpoint{3.104804in}{1.273370in}}%
\pgfpathlineto{\pgfqpoint{3.109138in}{1.326795in}}%
\pgfpathlineto{\pgfqpoint{3.113471in}{1.343206in}}%
\pgfpathlineto{\pgfqpoint{3.117805in}{1.327207in}}%
\pgfpathlineto{\pgfqpoint{3.122139in}{1.288992in}}%
\pgfpathlineto{\pgfqpoint{3.130806in}{1.186613in}}%
\pgfpathlineto{\pgfqpoint{3.135140in}{1.135481in}}%
\pgfpathlineto{\pgfqpoint{3.139473in}{1.088111in}}%
\pgfpathlineto{\pgfqpoint{3.143807in}{1.044998in}}%
\pgfpathlineto{\pgfqpoint{3.148141in}{1.005870in}}%
\pgfpathlineto{\pgfqpoint{3.152474in}{0.970211in}}%
\pgfpathlineto{\pgfqpoint{3.156808in}{0.937508in}}%
\pgfpathlineto{\pgfqpoint{3.161141in}{0.907347in}}%
\pgfpathlineto{\pgfqpoint{3.165475in}{0.879431in}}%
\pgfpathlineto{\pgfqpoint{3.169809in}{0.853564in}}%
\pgfpathlineto{\pgfqpoint{3.174142in}{0.829638in}}%
\pgfpathlineto{\pgfqpoint{3.178476in}{0.807615in}}%
\pgfpathlineto{\pgfqpoint{3.182810in}{0.787524in}}%
\pgfpathlineto{\pgfqpoint{3.187143in}{0.769463in}}%
\pgfpathlineto{\pgfqpoint{3.191477in}{0.753608in}}%
\pgfpathlineto{\pgfqpoint{3.195811in}{0.740229in}}%
\pgfpathlineto{\pgfqpoint{3.200144in}{0.729718in}}%
\pgfpathlineto{\pgfqpoint{3.204478in}{0.722619in}}%
\pgfpathlineto{\pgfqpoint{3.208812in}{0.719663in}}%
\pgfpathlineto{\pgfqpoint{3.213145in}{0.721806in}}%
\pgfpathlineto{\pgfqpoint{3.217479in}{0.730249in}}%
\pgfpathlineto{\pgfqpoint{3.221813in}{0.746423in}}%
\pgfpathlineto{\pgfqpoint{3.226146in}{0.771892in}}%
\pgfpathlineto{\pgfqpoint{3.230480in}{0.808119in}}%
\pgfpathlineto{\pgfqpoint{3.234814in}{0.856039in}}%
\pgfpathlineto{\pgfqpoint{3.239147in}{0.915405in}}%
\pgfpathlineto{\pgfqpoint{3.243481in}{0.984027in}}%
\pgfpathlineto{\pgfqpoint{3.252148in}{1.127919in}}%
\pgfpathlineto{\pgfqpoint{3.256482in}{1.187993in}}%
\pgfpathlineto{\pgfqpoint{3.260816in}{1.230380in}}%
\pgfpathlineto{\pgfqpoint{3.265149in}{1.251268in}}%
\pgfpathlineto{\pgfqpoint{3.269483in}{1.251009in}}%
\pgfpathlineto{\pgfqpoint{3.273816in}{1.233354in}}%
\pgfpathlineto{\pgfqpoint{3.278150in}{1.203647in}}%
\pgfpathlineto{\pgfqpoint{3.282484in}{1.167068in}}%
\pgfpathlineto{\pgfqpoint{3.295485in}{1.049401in}}%
\pgfpathlineto{\pgfqpoint{3.304152in}{0.978002in}}%
\pgfpathlineto{\pgfqpoint{3.312819in}{0.914455in}}%
\pgfpathlineto{\pgfqpoint{3.321487in}{0.857876in}}%
\pgfpathlineto{\pgfqpoint{3.330154in}{0.807405in}}%
\pgfpathlineto{\pgfqpoint{3.338821in}{0.762502in}}%
\pgfpathlineto{\pgfqpoint{3.347489in}{0.722936in}}%
\pgfpathlineto{\pgfqpoint{3.351822in}{0.705182in}}%
\pgfpathlineto{\pgfqpoint{3.356156in}{0.688858in}}%
\pgfpathlineto{\pgfqpoint{3.360490in}{0.674084in}}%
\pgfpathlineto{\pgfqpoint{3.364823in}{0.661047in}}%
\pgfpathlineto{\pgfqpoint{3.369157in}{0.650030in}}%
\pgfpathlineto{\pgfqpoint{3.373490in}{0.641452in}}%
\pgfpathlineto{\pgfqpoint{3.377824in}{0.635912in}}%
\pgfpathlineto{\pgfqpoint{3.382158in}{0.634256in}}%
\pgfpathlineto{\pgfqpoint{3.386491in}{0.637652in}}%
\pgfpathlineto{\pgfqpoint{3.390825in}{0.647665in}}%
\pgfpathlineto{\pgfqpoint{3.395159in}{0.666310in}}%
\pgfpathlineto{\pgfqpoint{3.399492in}{0.696020in}}%
\pgfpathlineto{\pgfqpoint{3.403826in}{0.739424in}}%
\pgfpathlineto{\pgfqpoint{3.408160in}{0.798787in}}%
\pgfpathlineto{\pgfqpoint{3.412493in}{0.874976in}}%
\pgfpathlineto{\pgfqpoint{3.416827in}{0.965977in}}%
\pgfpathlineto{\pgfqpoint{3.425494in}{1.162591in}}%
\pgfpathlineto{\pgfqpoint{3.429828in}{1.243945in}}%
\pgfpathlineto{\pgfqpoint{3.434162in}{1.298022in}}%
\pgfpathlineto{\pgfqpoint{3.438495in}{1.319549in}}%
\pgfpathlineto{\pgfqpoint{3.442829in}{1.310970in}}%
\pgfpathlineto{\pgfqpoint{3.447163in}{1.280169in}}%
\pgfpathlineto{\pgfqpoint{3.451496in}{1.236507in}}%
\pgfpathlineto{\pgfqpoint{3.464497in}{1.092579in}}%
\pgfpathlineto{\pgfqpoint{3.468831in}{1.049658in}}%
\pgfpathlineto{\pgfqpoint{3.473164in}{1.010201in}}%
\pgfpathlineto{\pgfqpoint{3.477498in}{0.973888in}}%
\pgfpathlineto{\pgfqpoint{3.481832in}{0.940318in}}%
\pgfpathlineto{\pgfqpoint{3.486165in}{0.909126in}}%
\pgfpathlineto{\pgfqpoint{3.494833in}{0.852804in}}%
\pgfpathlineto{\pgfqpoint{3.503500in}{0.803464in}}%
\pgfpathlineto{\pgfqpoint{3.507834in}{0.781198in}}%
\pgfpathlineto{\pgfqpoint{3.512167in}{0.760506in}}%
\pgfpathlineto{\pgfqpoint{3.516501in}{0.741416in}}%
\pgfpathlineto{\pgfqpoint{3.520835in}{0.724006in}}%
\pgfpathlineto{\pgfqpoint{3.525168in}{0.708415in}}%
\pgfpathlineto{\pgfqpoint{3.529502in}{0.694863in}}%
\pgfpathlineto{\pgfqpoint{3.533836in}{0.683678in}}%
\pgfpathlineto{\pgfqpoint{3.538169in}{0.675334in}}%
\pgfpathlineto{\pgfqpoint{3.542503in}{0.670493in}}%
\pgfpathlineto{\pgfqpoint{3.546837in}{0.670067in}}%
\pgfpathlineto{\pgfqpoint{3.551170in}{0.675275in}}%
\pgfpathlineto{\pgfqpoint{3.555504in}{0.687683in}}%
\pgfpathlineto{\pgfqpoint{3.559838in}{0.709204in}}%
\pgfpathlineto{\pgfqpoint{3.564171in}{0.741976in}}%
\pgfpathlineto{\pgfqpoint{3.568505in}{0.788030in}}%
\pgfpathlineto{\pgfqpoint{3.572839in}{0.848635in}}%
\pgfpathlineto{\pgfqpoint{3.577172in}{0.923288in}}%
\pgfpathlineto{\pgfqpoint{3.590173in}{1.179148in}}%
\pgfpathlineto{\pgfqpoint{3.594507in}{1.243380in}}%
\pgfpathlineto{\pgfqpoint{3.598840in}{1.282001in}}%
\pgfpathlineto{\pgfqpoint{3.603174in}{1.292745in}}%
\pgfpathlineto{\pgfqpoint{3.607508in}{1.279124in}}%
\pgfpathlineto{\pgfqpoint{3.611841in}{1.248112in}}%
\pgfpathlineto{\pgfqpoint{3.616175in}{1.207183in}}%
\pgfpathlineto{\pgfqpoint{3.629176in}{1.074210in}}%
\pgfpathlineto{\pgfqpoint{3.633510in}{1.033771in}}%
\pgfpathlineto{\pgfqpoint{3.637843in}{0.996108in}}%
\pgfpathlineto{\pgfqpoint{3.642177in}{0.961035in}}%
\pgfpathlineto{\pgfqpoint{3.650844in}{0.897579in}}%
\pgfpathlineto{\pgfqpoint{3.659512in}{0.841504in}}%
\pgfpathlineto{\pgfqpoint{3.668179in}{0.791542in}}%
\pgfpathlineto{\pgfqpoint{3.676846in}{0.746898in}}%
\pgfpathlineto{\pgfqpoint{3.685513in}{0.707050in}}%
\pgfpathlineto{\pgfqpoint{3.694181in}{0.671675in}}%
\pgfpathlineto{\pgfqpoint{3.698514in}{0.655626in}}%
\pgfpathlineto{\pgfqpoint{3.702848in}{0.640687in}}%
\pgfpathlineto{\pgfqpoint{3.707182in}{0.626913in}}%
\pgfpathlineto{\pgfqpoint{3.711515in}{0.614400in}}%
\pgfpathlineto{\pgfqpoint{3.715849in}{0.603310in}}%
\pgfpathlineto{\pgfqpoint{3.720183in}{0.593894in}}%
\pgfpathlineto{\pgfqpoint{3.724516in}{0.586535in}}%
\pgfpathlineto{\pgfqpoint{3.728850in}{0.581799in}}%
\pgfpathlineto{\pgfqpoint{3.733184in}{0.580507in}}%
\pgfpathlineto{\pgfqpoint{3.737517in}{0.583835in}}%
\pgfpathlineto{\pgfqpoint{3.741851in}{0.593425in}}%
\pgfpathlineto{\pgfqpoint{3.746185in}{0.611494in}}%
\pgfpathlineto{\pgfqpoint{3.750518in}{0.640885in}}%
\pgfpathlineto{\pgfqpoint{3.754852in}{0.684950in}}%
\pgfpathlineto{\pgfqpoint{3.759186in}{0.747050in}}%
\pgfpathlineto{\pgfqpoint{3.763519in}{0.829432in}}%
\pgfpathlineto{\pgfqpoint{3.767853in}{0.931340in}}%
\pgfpathlineto{\pgfqpoint{3.776520in}{1.163169in}}%
\pgfpathlineto{\pgfqpoint{3.780854in}{1.263484in}}%
\pgfpathlineto{\pgfqpoint{3.785187in}{1.331713in}}%
\pgfpathlineto{\pgfqpoint{3.789521in}{1.359761in}}%
\pgfpathlineto{\pgfqpoint{3.793855in}{1.350317in}}%
\pgfpathlineto{\pgfqpoint{3.798188in}{1.313889in}}%
\pgfpathlineto{\pgfqpoint{3.802522in}{1.262953in}}%
\pgfpathlineto{\pgfqpoint{3.811189in}{1.153542in}}%
\pgfpathlineto{\pgfqpoint{3.815523in}{1.103819in}}%
\pgfpathlineto{\pgfqpoint{3.819857in}{1.058988in}}%
\pgfpathlineto{\pgfqpoint{3.824190in}{1.018739in}}%
\pgfpathlineto{\pgfqpoint{3.828524in}{0.982444in}}%
\pgfpathlineto{\pgfqpoint{3.832858in}{0.949480in}}%
\pgfpathlineto{\pgfqpoint{3.837191in}{0.919349in}}%
\pgfpathlineto{\pgfqpoint{3.841525in}{0.891701in}}%
\pgfpathlineto{\pgfqpoint{3.845859in}{0.866320in}}%
\pgfpathlineto{\pgfqpoint{3.850192in}{0.843098in}}%
\pgfpathlineto{\pgfqpoint{3.854526in}{0.822020in}}%
\pgfpathlineto{\pgfqpoint{3.858860in}{0.803156in}}%
\pgfpathlineto{\pgfqpoint{3.863193in}{0.786664in}}%
\pgfpathlineto{\pgfqpoint{3.867527in}{0.772798in}}%
\pgfpathlineto{\pgfqpoint{3.871861in}{0.761927in}}%
\pgfpathlineto{\pgfqpoint{3.876194in}{0.754560in}}%
\pgfpathlineto{\pgfqpoint{3.880528in}{0.751367in}}%
\pgfpathlineto{\pgfqpoint{3.884862in}{0.753203in}}%
\pgfpathlineto{\pgfqpoint{3.889195in}{0.761113in}}%
\pgfpathlineto{\pgfqpoint{3.893529in}{0.776295in}}%
\pgfpathlineto{\pgfqpoint{3.897862in}{0.799992in}}%
\pgfpathlineto{\pgfqpoint{3.902196in}{0.833279in}}%
\pgfpathlineto{\pgfqpoint{3.906530in}{0.876683in}}%
\pgfpathlineto{\pgfqpoint{3.910863in}{0.929683in}}%
\pgfpathlineto{\pgfqpoint{3.919531in}{1.053987in}}%
\pgfpathlineto{\pgfqpoint{3.923864in}{1.115380in}}%
\pgfpathlineto{\pgfqpoint{3.928198in}{1.167756in}}%
\pgfpathlineto{\pgfqpoint{3.932532in}{1.205426in}}%
\pgfpathlineto{\pgfqpoint{3.936865in}{1.225161in}}%
\pgfpathlineto{\pgfqpoint{3.941199in}{1.226905in}}%
\pgfpathlineto{\pgfqpoint{3.945533in}{1.213313in}}%
\pgfpathlineto{\pgfqpoint{3.949866in}{1.188491in}}%
\pgfpathlineto{\pgfqpoint{3.954200in}{1.156669in}}%
\pgfpathlineto{\pgfqpoint{3.975868in}{0.981071in}}%
\pgfpathlineto{\pgfqpoint{3.984536in}{0.919857in}}%
\pgfpathlineto{\pgfqpoint{3.993203in}{0.865270in}}%
\pgfpathlineto{\pgfqpoint{4.001870in}{0.817018in}}%
\pgfpathlineto{\pgfqpoint{4.006204in}{0.795259in}}%
\pgfpathlineto{\pgfqpoint{4.010537in}{0.775134in}}%
\pgfpathlineto{\pgfqpoint{4.014871in}{0.756734in}}%
\pgfpathlineto{\pgfqpoint{4.019205in}{0.740211in}}%
\pgfpathlineto{\pgfqpoint{4.023538in}{0.725785in}}%
\pgfpathlineto{\pgfqpoint{4.027872in}{0.713781in}}%
\pgfpathlineto{\pgfqpoint{4.032206in}{0.704650in}}%
\pgfpathlineto{\pgfqpoint{4.036539in}{0.699016in}}%
\pgfpathlineto{\pgfqpoint{4.040873in}{0.697718in}}%
\pgfpathlineto{\pgfqpoint{4.045207in}{0.701857in}}%
\pgfpathlineto{\pgfqpoint{4.049540in}{0.712821in}}%
\pgfpathlineto{\pgfqpoint{4.053874in}{0.732269in}}%
\pgfpathlineto{\pgfqpoint{4.058208in}{0.762011in}}%
\pgfpathlineto{\pgfqpoint{4.062541in}{0.803713in}}%
\pgfpathlineto{\pgfqpoint{4.066875in}{0.858350in}}%
\pgfpathlineto{\pgfqpoint{4.071209in}{0.925390in}}%
\pgfpathlineto{\pgfqpoint{4.084210in}{1.156686in}}%
\pgfpathlineto{\pgfqpoint{4.088543in}{1.217184in}}%
\pgfpathlineto{\pgfqpoint{4.092877in}{1.256102in}}%
\pgfpathlineto{\pgfqpoint{4.097210in}{1.270617in}}%
\pgfpathlineto{\pgfqpoint{4.101544in}{1.262756in}}%
\pgfpathlineto{\pgfqpoint{4.105878in}{1.237864in}}%
\pgfpathlineto{\pgfqpoint{4.110211in}{1.202274in}}%
\pgfpathlineto{\pgfqpoint{4.127546in}{1.038587in}}%
\pgfpathlineto{\pgfqpoint{4.136213in}{0.966678in}}%
\pgfpathlineto{\pgfqpoint{4.144881in}{0.903241in}}%
\pgfpathlineto{\pgfqpoint{4.153548in}{0.846870in}}%
\pgfpathlineto{\pgfqpoint{4.162215in}{0.796469in}}%
\pgfpathlineto{\pgfqpoint{4.170883in}{0.751295in}}%
\pgfpathlineto{\pgfqpoint{4.179550in}{0.710819in}}%
\pgfpathlineto{\pgfqpoint{4.188217in}{0.674657in}}%
\pgfpathlineto{\pgfqpoint{4.196884in}{0.642577in}}%
\pgfpathlineto{\pgfqpoint{4.201218in}{0.628054in}}%
\pgfpathlineto{\pgfqpoint{4.205552in}{0.614578in}}%
\pgfpathlineto{\pgfqpoint{4.209885in}{0.602219in}}%
\pgfpathlineto{\pgfqpoint{4.214219in}{0.591096in}}%
\pgfpathlineto{\pgfqpoint{4.218553in}{0.581403in}}%
\pgfpathlineto{\pgfqpoint{4.222886in}{0.573438in}}%
\pgfpathlineto{\pgfqpoint{4.227220in}{0.567651in}}%
\pgfpathlineto{\pgfqpoint{4.231554in}{0.564707in}}%
\pgfpathlineto{\pgfqpoint{4.235887in}{0.565577in}}%
\pgfpathlineto{\pgfqpoint{4.240221in}{0.571643in}}%
\pgfpathlineto{\pgfqpoint{4.244555in}{0.584831in}}%
\pgfpathlineto{\pgfqpoint{4.248888in}{0.607725in}}%
\pgfpathlineto{\pgfqpoint{4.253222in}{0.643580in}}%
\pgfpathlineto{\pgfqpoint{4.257556in}{0.696094in}}%
\pgfpathlineto{\pgfqpoint{4.261889in}{0.768662in}}%
\pgfpathlineto{\pgfqpoint{4.266223in}{0.862841in}}%
\pgfpathlineto{\pgfqpoint{4.270557in}{0.976007in}}%
\pgfpathlineto{\pgfqpoint{4.279224in}{1.216004in}}%
\pgfpathlineto{\pgfqpoint{4.283558in}{1.308089in}}%
\pgfpathlineto{\pgfqpoint{4.287891in}{1.360992in}}%
\pgfpathlineto{\pgfqpoint{4.292225in}{1.371218in}}%
\pgfpathlineto{\pgfqpoint{4.296559in}{1.346314in}}%
\pgfpathlineto{\pgfqpoint{4.300892in}{1.299459in}}%
\pgfpathlineto{\pgfqpoint{4.313893in}{1.132968in}}%
\pgfpathlineto{\pgfqpoint{4.318227in}{1.084947in}}%
\pgfpathlineto{\pgfqpoint{4.322560in}{1.042153in}}%
\pgfpathlineto{\pgfqpoint{4.326894in}{1.003952in}}%
\pgfpathlineto{\pgfqpoint{4.331228in}{0.969611in}}%
\pgfpathlineto{\pgfqpoint{4.335561in}{0.938505in}}%
\pgfpathlineto{\pgfqpoint{4.339895in}{0.910181in}}%
\pgfpathlineto{\pgfqpoint{4.344229in}{0.884347in}}%
\pgfpathlineto{\pgfqpoint{4.348562in}{0.860848in}}%
\pgfpathlineto{\pgfqpoint{4.352896in}{0.839643in}}%
\pgfpathlineto{\pgfqpoint{4.357230in}{0.820789in}}%
\pgfpathlineto{\pgfqpoint{4.361563in}{0.804440in}}%
\pgfpathlineto{\pgfqpoint{4.365897in}{0.790851in}}%
\pgfpathlineto{\pgfqpoint{4.370231in}{0.780390in}}%
\pgfpathlineto{\pgfqpoint{4.374564in}{0.773559in}}%
\pgfpathlineto{\pgfqpoint{4.378898in}{0.771010in}}%
\pgfpathlineto{\pgfqpoint{4.383232in}{0.773554in}}%
\pgfpathlineto{\pgfqpoint{4.387565in}{0.782153in}}%
\pgfpathlineto{\pgfqpoint{4.391899in}{0.797866in}}%
\pgfpathlineto{\pgfqpoint{4.396233in}{0.821729in}}%
\pgfpathlineto{\pgfqpoint{4.400566in}{0.854525in}}%
\pgfpathlineto{\pgfqpoint{4.404900in}{0.896444in}}%
\pgfpathlineto{\pgfqpoint{4.409233in}{0.946632in}}%
\pgfpathlineto{\pgfqpoint{4.422234in}{1.115773in}}%
\pgfpathlineto{\pgfqpoint{4.426568in}{1.161685in}}%
\pgfpathlineto{\pgfqpoint{4.430902in}{1.193988in}}%
\pgfpathlineto{\pgfqpoint{4.435235in}{1.210206in}}%
\pgfpathlineto{\pgfqpoint{4.439569in}{1.210517in}}%
\pgfpathlineto{\pgfqpoint{4.443903in}{1.197290in}}%
\pgfpathlineto{\pgfqpoint{4.448236in}{1.174048in}}%
\pgfpathlineto{\pgfqpoint{4.452570in}{1.144409in}}%
\pgfpathlineto{\pgfqpoint{4.461237in}{1.077181in}}%
\pgfpathlineto{\pgfqpoint{4.469905in}{1.010081in}}%
\pgfpathlineto{\pgfqpoint{4.478572in}{0.948215in}}%
\pgfpathlineto{\pgfqpoint{4.487239in}{0.892696in}}%
\pgfpathlineto{\pgfqpoint{4.495907in}{0.843670in}}%
\pgfpathlineto{\pgfqpoint{4.500240in}{0.821659in}}%
\pgfpathlineto{\pgfqpoint{4.504574in}{0.801410in}}%
\pgfpathlineto{\pgfqpoint{4.508907in}{0.783049in}}%
\pgfpathlineto{\pgfqpoint{4.513241in}{0.766762in}}%
\pgfpathlineto{\pgfqpoint{4.517575in}{0.752817in}}%
\pgfpathlineto{\pgfqpoint{4.521908in}{0.741584in}}%
\pgfpathlineto{\pgfqpoint{4.526242in}{0.733567in}}%
\pgfpathlineto{\pgfqpoint{4.530576in}{0.729441in}}%
\pgfpathlineto{\pgfqpoint{4.534909in}{0.730077in}}%
\pgfpathlineto{\pgfqpoint{4.539243in}{0.736572in}}%
\pgfpathlineto{\pgfqpoint{4.543577in}{0.750239in}}%
\pgfpathlineto{\pgfqpoint{4.547910in}{0.772533in}}%
\pgfpathlineto{\pgfqpoint{4.552244in}{0.804861in}}%
\pgfpathlineto{\pgfqpoint{4.556578in}{0.848233in}}%
\pgfpathlineto{\pgfqpoint{4.560911in}{0.902699in}}%
\pgfpathlineto{\pgfqpoint{4.565245in}{0.966668in}}%
\pgfpathlineto{\pgfqpoint{4.573912in}{1.105517in}}%
\pgfpathlineto{\pgfqpoint{4.578246in}{1.166753in}}%
\pgfpathlineto{\pgfqpoint{4.582580in}{1.212904in}}%
\pgfpathlineto{\pgfqpoint{4.586913in}{1.239354in}}%
\pgfpathlineto{\pgfqpoint{4.591247in}{1.245235in}}%
\pgfpathlineto{\pgfqpoint{4.595581in}{1.233172in}}%
\pgfpathlineto{\pgfqpoint{4.599914in}{1.207844in}}%
\pgfpathlineto{\pgfqpoint{4.604248in}{1.174275in}}%
\pgfpathlineto{\pgfqpoint{4.625916in}{0.988349in}}%
\pgfpathlineto{\pgfqpoint{4.634583in}{0.924398in}}%
\pgfpathlineto{\pgfqpoint{4.643251in}{0.867413in}}%
\pgfpathlineto{\pgfqpoint{4.651918in}{0.816666in}}%
\pgfpathlineto{\pgfqpoint{4.660585in}{0.771722in}}%
\pgfpathlineto{\pgfqpoint{4.664919in}{0.751391in}}%
\pgfpathlineto{\pgfqpoint{4.669253in}{0.732521in}}%
\pgfpathlineto{\pgfqpoint{4.673586in}{0.715184in}}%
\pgfpathlineto{\pgfqpoint{4.677920in}{0.699504in}}%
\pgfpathlineto{\pgfqpoint{4.682254in}{0.685673in}}%
\pgfpathlineto{\pgfqpoint{4.686587in}{0.673981in}}%
\pgfpathlineto{\pgfqpoint{4.690921in}{0.664845in}}%
\pgfpathlineto{\pgfqpoint{4.695255in}{0.658859in}}%
\pgfpathlineto{\pgfqpoint{4.699588in}{0.656851in}}%
\pgfpathlineto{\pgfqpoint{4.703922in}{0.659945in}}%
\pgfpathlineto{\pgfqpoint{4.708256in}{0.669624in}}%
\pgfpathlineto{\pgfqpoint{4.712589in}{0.687759in}}%
\pgfpathlineto{\pgfqpoint{4.716923in}{0.716559in}}%
\pgfpathlineto{\pgfqpoint{4.721256in}{0.758329in}}%
\pgfpathlineto{\pgfqpoint{4.725590in}{0.814941in}}%
\pgfpathlineto{\pgfqpoint{4.729924in}{0.886892in}}%
\pgfpathlineto{\pgfqpoint{4.734257in}{0.972034in}}%
\pgfpathlineto{\pgfqpoint{4.742925in}{1.154252in}}%
\pgfpathlineto{\pgfqpoint{4.747258in}{1.229654in}}%
\pgfpathlineto{\pgfqpoint{4.751592in}{1.280455in}}%
\pgfpathlineto{\pgfqpoint{4.755926in}{1.301822in}}%
\pgfpathlineto{\pgfqpoint{4.760259in}{1.295558in}}%
\pgfpathlineto{\pgfqpoint{4.764593in}{1.268317in}}%
\pgfpathlineto{\pgfqpoint{4.768927in}{1.228311in}}%
\pgfpathlineto{\pgfqpoint{4.781928in}{1.091137in}}%
\pgfpathlineto{\pgfqpoint{4.786261in}{1.049068in}}%
\pgfpathlineto{\pgfqpoint{4.790595in}{1.010027in}}%
\pgfpathlineto{\pgfqpoint{4.794929in}{0.973832in}}%
\pgfpathlineto{\pgfqpoint{4.799262in}{0.940174in}}%
\pgfpathlineto{\pgfqpoint{4.807930in}{0.879289in}}%
\pgfpathlineto{\pgfqpoint{4.816597in}{0.825527in}}%
\pgfpathlineto{\pgfqpoint{4.825264in}{0.777799in}}%
\pgfpathlineto{\pgfqpoint{4.833931in}{0.735505in}}%
\pgfpathlineto{\pgfqpoint{4.838265in}{0.716308in}}%
\pgfpathlineto{\pgfqpoint{4.842599in}{0.698411in}}%
\pgfpathlineto{\pgfqpoint{4.846932in}{0.681849in}}%
\pgfpathlineto{\pgfqpoint{4.851266in}{0.666701in}}%
\pgfpathlineto{\pgfqpoint{4.855600in}{0.653104in}}%
\pgfpathlineto{\pgfqpoint{4.859933in}{0.641272in}}%
\pgfpathlineto{\pgfqpoint{4.864267in}{0.631530in}}%
\pgfpathlineto{\pgfqpoint{4.868601in}{0.624355in}}%
\pgfpathlineto{\pgfqpoint{4.872934in}{0.620436in}}%
\pgfpathlineto{\pgfqpoint{4.877268in}{0.620742in}}%
\pgfpathlineto{\pgfqpoint{4.881602in}{0.626614in}}%
\pgfpathlineto{\pgfqpoint{4.885935in}{0.639844in}}%
\pgfpathlineto{\pgfqpoint{4.890269in}{0.662718in}}%
\pgfpathlineto{\pgfqpoint{4.894603in}{0.697947in}}%
\pgfpathlineto{\pgfqpoint{4.898936in}{0.748341in}}%
\pgfpathlineto{\pgfqpoint{4.903270in}{0.816055in}}%
\pgfpathlineto{\pgfqpoint{4.907604in}{0.901279in}}%
\pgfpathlineto{\pgfqpoint{4.920605in}{1.202531in}}%
\pgfpathlineto{\pgfqpoint{4.924938in}{1.277943in}}%
\pgfpathlineto{\pgfqpoint{4.929272in}{1.321242in}}%
\pgfpathlineto{\pgfqpoint{4.933605in}{1.330120in}}%
\pgfpathlineto{\pgfqpoint{4.937939in}{1.310141in}}%
\pgfpathlineto{\pgfqpoint{4.942273in}{1.271058in}}%
\pgfpathlineto{\pgfqpoint{4.959607in}{1.076705in}}%
\pgfpathlineto{\pgfqpoint{4.963941in}{1.034808in}}%
\pgfpathlineto{\pgfqpoint{4.968275in}{0.996515in}}%
\pgfpathlineto{\pgfqpoint{4.972608in}{0.961379in}}%
\pgfpathlineto{\pgfqpoint{4.976942in}{0.928957in}}%
\pgfpathlineto{\pgfqpoint{4.981276in}{0.898887in}}%
\pgfpathlineto{\pgfqpoint{4.985609in}{0.870903in}}%
\pgfpathlineto{\pgfqpoint{4.989943in}{0.844824in}}%
\pgfpathlineto{\pgfqpoint{4.994277in}{0.820537in}}%
\pgfpathlineto{\pgfqpoint{4.998610in}{0.797988in}}%
\pgfpathlineto{\pgfqpoint{5.002944in}{0.777175in}}%
\pgfpathlineto{\pgfqpoint{5.007278in}{0.758151in}}%
\pgfpathlineto{\pgfqpoint{5.011611in}{0.741026in}}%
\pgfpathlineto{\pgfqpoint{5.015945in}{0.725991in}}%
\pgfpathlineto{\pgfqpoint{5.020279in}{0.713330in}}%
\pgfpathlineto{\pgfqpoint{5.024612in}{0.703457in}}%
\pgfpathlineto{\pgfqpoint{5.028946in}{0.696948in}}%
\pgfpathlineto{\pgfqpoint{5.033279in}{0.694593in}}%
\pgfpathlineto{\pgfqpoint{5.037613in}{0.697435in}}%
\pgfpathlineto{\pgfqpoint{5.041947in}{0.706814in}}%
\pgfpathlineto{\pgfqpoint{5.046280in}{0.724357in}}%
\pgfpathlineto{\pgfqpoint{5.050614in}{0.751890in}}%
\pgfpathlineto{\pgfqpoint{5.054948in}{0.791190in}}%
\pgfpathlineto{\pgfqpoint{5.059281in}{0.843488in}}%
\pgfpathlineto{\pgfqpoint{5.063615in}{0.908686in}}%
\pgfpathlineto{\pgfqpoint{5.067949in}{0.984411in}}%
\pgfpathlineto{\pgfqpoint{5.076616in}{1.143015in}}%
\pgfpathlineto{\pgfqpoint{5.080950in}{1.208100in}}%
\pgfpathlineto{\pgfqpoint{5.085283in}{1.252475in}}%
\pgfpathlineto{\pgfqpoint{5.089617in}{1.272216in}}%
\pgfpathlineto{\pgfqpoint{5.093951in}{1.268457in}}%
\pgfpathlineto{\pgfqpoint{5.098284in}{1.246177in}}%
\pgfpathlineto{\pgfqpoint{5.102618in}{1.211827in}}%
\pgfpathlineto{\pgfqpoint{5.111285in}{1.128766in}}%
\pgfpathlineto{\pgfqpoint{5.119953in}{1.046818in}}%
\pgfpathlineto{\pgfqpoint{5.128620in}{0.973708in}}%
\pgfpathlineto{\pgfqpoint{5.137287in}{0.909326in}}%
\pgfpathlineto{\pgfqpoint{5.145954in}{0.852197in}}%
\pgfpathlineto{\pgfqpoint{5.154622in}{0.801147in}}%
\pgfpathlineto{\pgfqpoint{5.163289in}{0.755378in}}%
\pgfpathlineto{\pgfqpoint{5.171956in}{0.714317in}}%
\pgfpathlineto{\pgfqpoint{5.180624in}{0.677520in}}%
\pgfpathlineto{\pgfqpoint{5.189291in}{0.644655in}}%
\pgfpathlineto{\pgfqpoint{5.193625in}{0.629629in}}%
\pgfpathlineto{\pgfqpoint{5.193625in}{0.629629in}}%
\pgfusepath{stroke}%
\end{pgfscope}%
\begin{pgfscope}%
\pgfpathrectangle{\pgfqpoint{0.647840in}{0.524382in}}{\pgfqpoint{4.762251in}{0.887161in}}%
\pgfusepath{clip}%
\pgfsetrectcap%
\pgfsetroundjoin%
\pgfsetlinewidth{1.505625pt}%
\definecolor{currentstroke}{rgb}{0.203922,0.541176,0.741176}%
\pgfsetstrokecolor{currentstroke}%
\pgfsetdash{}{0pt}%
\pgfpathmoveto{\pgfqpoint{3.031132in}{0.650832in}}%
\pgfpathlineto{\pgfqpoint{3.035466in}{0.637775in}}%
\pgfpathlineto{\pgfqpoint{3.039799in}{0.626113in}}%
\pgfpathlineto{\pgfqpoint{3.044133in}{0.616239in}}%
\pgfpathlineto{\pgfqpoint{3.048467in}{0.609127in}}%
\pgfpathlineto{\pgfqpoint{3.052800in}{0.604734in}}%
\pgfpathlineto{\pgfqpoint{3.057134in}{0.603864in}}%
\pgfpathlineto{\pgfqpoint{3.061467in}{0.608087in}}%
\pgfpathlineto{\pgfqpoint{3.065801in}{0.618852in}}%
\pgfpathlineto{\pgfqpoint{3.070135in}{0.638880in}}%
\pgfpathlineto{\pgfqpoint{3.074468in}{0.670210in}}%
\pgfpathlineto{\pgfqpoint{3.078802in}{0.715379in}}%
\pgfpathlineto{\pgfqpoint{3.083136in}{0.778160in}}%
\pgfpathlineto{\pgfqpoint{3.087469in}{0.860015in}}%
\pgfpathlineto{\pgfqpoint{3.091803in}{0.959327in}}%
\pgfpathlineto{\pgfqpoint{3.100470in}{1.176124in}}%
\pgfpathlineto{\pgfqpoint{3.104804in}{1.266096in}}%
\pgfpathlineto{\pgfqpoint{3.109138in}{1.324805in}}%
\pgfpathlineto{\pgfqpoint{3.113471in}{1.347877in}}%
\pgfpathlineto{\pgfqpoint{3.117805in}{1.334284in}}%
\pgfpathlineto{\pgfqpoint{3.122139in}{1.298720in}}%
\pgfpathlineto{\pgfqpoint{3.130806in}{1.194789in}}%
\pgfpathlineto{\pgfqpoint{3.135140in}{1.142532in}}%
\pgfpathlineto{\pgfqpoint{3.139473in}{1.094495in}}%
\pgfpathlineto{\pgfqpoint{3.143807in}{1.051701in}}%
\pgfpathlineto{\pgfqpoint{3.148141in}{1.011965in}}%
\pgfpathlineto{\pgfqpoint{3.152474in}{0.975963in}}%
\pgfpathlineto{\pgfqpoint{3.156808in}{0.943044in}}%
\pgfpathlineto{\pgfqpoint{3.161141in}{0.912686in}}%
\pgfpathlineto{\pgfqpoint{3.165475in}{0.884948in}}%
\pgfpathlineto{\pgfqpoint{3.169809in}{0.859598in}}%
\pgfpathlineto{\pgfqpoint{3.174142in}{0.836001in}}%
\pgfpathlineto{\pgfqpoint{3.178476in}{0.813949in}}%
\pgfpathlineto{\pgfqpoint{3.182810in}{0.794573in}}%
\pgfpathlineto{\pgfqpoint{3.187143in}{0.776615in}}%
\pgfpathlineto{\pgfqpoint{3.191477in}{0.761259in}}%
\pgfpathlineto{\pgfqpoint{3.195811in}{0.748367in}}%
\pgfpathlineto{\pgfqpoint{3.200144in}{0.738776in}}%
\pgfpathlineto{\pgfqpoint{3.204478in}{0.733122in}}%
\pgfpathlineto{\pgfqpoint{3.208812in}{0.731952in}}%
\pgfpathlineto{\pgfqpoint{3.213145in}{0.735710in}}%
\pgfpathlineto{\pgfqpoint{3.217479in}{0.746217in}}%
\pgfpathlineto{\pgfqpoint{3.221813in}{0.765172in}}%
\pgfpathlineto{\pgfqpoint{3.226146in}{0.793912in}}%
\pgfpathlineto{\pgfqpoint{3.230480in}{0.832856in}}%
\pgfpathlineto{\pgfqpoint{3.234814in}{0.882907in}}%
\pgfpathlineto{\pgfqpoint{3.239147in}{0.943096in}}%
\pgfpathlineto{\pgfqpoint{3.252148in}{1.143147in}}%
\pgfpathlineto{\pgfqpoint{3.256482in}{1.195281in}}%
\pgfpathlineto{\pgfqpoint{3.260816in}{1.229235in}}%
\pgfpathlineto{\pgfqpoint{3.265149in}{1.242369in}}%
\pgfpathlineto{\pgfqpoint{3.269483in}{1.237658in}}%
\pgfpathlineto{\pgfqpoint{3.273816in}{1.216781in}}%
\pgfpathlineto{\pgfqpoint{3.278150in}{1.186595in}}%
\pgfpathlineto{\pgfqpoint{3.286817in}{1.112330in}}%
\pgfpathlineto{\pgfqpoint{3.295485in}{1.037209in}}%
\pgfpathlineto{\pgfqpoint{3.304152in}{0.967532in}}%
\pgfpathlineto{\pgfqpoint{3.312819in}{0.906744in}}%
\pgfpathlineto{\pgfqpoint{3.321487in}{0.852265in}}%
\pgfpathlineto{\pgfqpoint{3.330154in}{0.803525in}}%
\pgfpathlineto{\pgfqpoint{3.338821in}{0.760513in}}%
\pgfpathlineto{\pgfqpoint{3.343155in}{0.741253in}}%
\pgfpathlineto{\pgfqpoint{3.347489in}{0.723675in}}%
\pgfpathlineto{\pgfqpoint{3.351822in}{0.707828in}}%
\pgfpathlineto{\pgfqpoint{3.356156in}{0.693124in}}%
\pgfpathlineto{\pgfqpoint{3.360490in}{0.680939in}}%
\pgfpathlineto{\pgfqpoint{3.364823in}{0.671983in}}%
\pgfpathlineto{\pgfqpoint{3.369157in}{0.665323in}}%
\pgfpathlineto{\pgfqpoint{3.373490in}{0.662623in}}%
\pgfpathlineto{\pgfqpoint{3.377824in}{0.664643in}}%
\pgfpathlineto{\pgfqpoint{3.382158in}{0.673031in}}%
\pgfpathlineto{\pgfqpoint{3.386491in}{0.689857in}}%
\pgfpathlineto{\pgfqpoint{3.390825in}{0.717187in}}%
\pgfpathlineto{\pgfqpoint{3.395159in}{0.757409in}}%
\pgfpathlineto{\pgfqpoint{3.399492in}{0.811988in}}%
\pgfpathlineto{\pgfqpoint{3.403826in}{0.881693in}}%
\pgfpathlineto{\pgfqpoint{3.408160in}{0.964980in}}%
\pgfpathlineto{\pgfqpoint{3.416827in}{1.145484in}}%
\pgfpathlineto{\pgfqpoint{3.421161in}{1.221798in}}%
\pgfpathlineto{\pgfqpoint{3.425494in}{1.274406in}}%
\pgfpathlineto{\pgfqpoint{3.429828in}{1.297759in}}%
\pgfpathlineto{\pgfqpoint{3.434162in}{1.294268in}}%
\pgfpathlineto{\pgfqpoint{3.438495in}{1.268456in}}%
\pgfpathlineto{\pgfqpoint{3.442829in}{1.230468in}}%
\pgfpathlineto{\pgfqpoint{3.455830in}{1.094273in}}%
\pgfpathlineto{\pgfqpoint{3.464497in}{1.013495in}}%
\pgfpathlineto{\pgfqpoint{3.468831in}{0.976828in}}%
\pgfpathlineto{\pgfqpoint{3.473164in}{0.943037in}}%
\pgfpathlineto{\pgfqpoint{3.481832in}{0.881930in}}%
\pgfpathlineto{\pgfqpoint{3.490499in}{0.827766in}}%
\pgfpathlineto{\pgfqpoint{3.499166in}{0.779527in}}%
\pgfpathlineto{\pgfqpoint{3.507834in}{0.736553in}}%
\pgfpathlineto{\pgfqpoint{3.516501in}{0.698742in}}%
\pgfpathlineto{\pgfqpoint{3.520835in}{0.681466in}}%
\pgfpathlineto{\pgfqpoint{3.525168in}{0.665784in}}%
\pgfpathlineto{\pgfqpoint{3.529502in}{0.651352in}}%
\pgfpathlineto{\pgfqpoint{3.533836in}{0.638770in}}%
\pgfpathlineto{\pgfqpoint{3.538169in}{0.627742in}}%
\pgfpathlineto{\pgfqpoint{3.542503in}{0.618615in}}%
\pgfpathlineto{\pgfqpoint{3.546837in}{0.611618in}}%
\pgfpathlineto{\pgfqpoint{3.551170in}{0.607814in}}%
\pgfpathlineto{\pgfqpoint{3.555504in}{0.608380in}}%
\pgfpathlineto{\pgfqpoint{3.559838in}{0.614614in}}%
\pgfpathlineto{\pgfqpoint{3.564171in}{0.626974in}}%
\pgfpathlineto{\pgfqpoint{3.568505in}{0.648819in}}%
\pgfpathlineto{\pgfqpoint{3.572839in}{0.681342in}}%
\pgfpathlineto{\pgfqpoint{3.577172in}{0.727225in}}%
\pgfpathlineto{\pgfqpoint{3.581506in}{0.790315in}}%
\pgfpathlineto{\pgfqpoint{3.585839in}{0.872642in}}%
\pgfpathlineto{\pgfqpoint{3.590173in}{0.971567in}}%
\pgfpathlineto{\pgfqpoint{3.598840in}{1.185019in}}%
\pgfpathlineto{\pgfqpoint{3.603174in}{1.272336in}}%
\pgfpathlineto{\pgfqpoint{3.607508in}{1.326732in}}%
\pgfpathlineto{\pgfqpoint{3.611841in}{1.345767in}}%
\pgfpathlineto{\pgfqpoint{3.616175in}{1.329968in}}%
\pgfpathlineto{\pgfqpoint{3.620509in}{1.294279in}}%
\pgfpathlineto{\pgfqpoint{3.629176in}{1.190562in}}%
\pgfpathlineto{\pgfqpoint{3.633510in}{1.138735in}}%
\pgfpathlineto{\pgfqpoint{3.637843in}{1.090875in}}%
\pgfpathlineto{\pgfqpoint{3.642177in}{1.047992in}}%
\pgfpathlineto{\pgfqpoint{3.646511in}{1.008842in}}%
\pgfpathlineto{\pgfqpoint{3.650844in}{0.972524in}}%
\pgfpathlineto{\pgfqpoint{3.655178in}{0.939799in}}%
\pgfpathlineto{\pgfqpoint{3.659512in}{0.909351in}}%
\pgfpathlineto{\pgfqpoint{3.663845in}{0.881856in}}%
\pgfpathlineto{\pgfqpoint{3.668179in}{0.856371in}}%
\pgfpathlineto{\pgfqpoint{3.672513in}{0.832955in}}%
\pgfpathlineto{\pgfqpoint{3.676846in}{0.811133in}}%
\pgfpathlineto{\pgfqpoint{3.681180in}{0.791714in}}%
\pgfpathlineto{\pgfqpoint{3.685513in}{0.774172in}}%
\pgfpathlineto{\pgfqpoint{3.689847in}{0.758781in}}%
\pgfpathlineto{\pgfqpoint{3.694181in}{0.746642in}}%
\pgfpathlineto{\pgfqpoint{3.698514in}{0.737671in}}%
\pgfpathlineto{\pgfqpoint{3.702848in}{0.732661in}}%
\pgfpathlineto{\pgfqpoint{3.707182in}{0.731528in}}%
\pgfpathlineto{\pgfqpoint{3.711515in}{0.736069in}}%
\pgfpathlineto{\pgfqpoint{3.715849in}{0.747294in}}%
\pgfpathlineto{\pgfqpoint{3.720183in}{0.766862in}}%
\pgfpathlineto{\pgfqpoint{3.724516in}{0.795797in}}%
\pgfpathlineto{\pgfqpoint{3.728850in}{0.835594in}}%
\pgfpathlineto{\pgfqpoint{3.733184in}{0.886571in}}%
\pgfpathlineto{\pgfqpoint{3.737517in}{0.947560in}}%
\pgfpathlineto{\pgfqpoint{3.750518in}{1.148901in}}%
\pgfpathlineto{\pgfqpoint{3.754852in}{1.198954in}}%
\pgfpathlineto{\pgfqpoint{3.759186in}{1.231236in}}%
\pgfpathlineto{\pgfqpoint{3.763519in}{1.242822in}}%
\pgfpathlineto{\pgfqpoint{3.767853in}{1.236290in}}%
\pgfpathlineto{\pgfqpoint{3.772187in}{1.215305in}}%
\pgfpathlineto{\pgfqpoint{3.776520in}{1.183444in}}%
\pgfpathlineto{\pgfqpoint{3.785187in}{1.108624in}}%
\pgfpathlineto{\pgfqpoint{3.793855in}{1.034068in}}%
\pgfpathlineto{\pgfqpoint{3.802522in}{0.965496in}}%
\pgfpathlineto{\pgfqpoint{3.811189in}{0.904069in}}%
\pgfpathlineto{\pgfqpoint{3.819857in}{0.849867in}}%
\pgfpathlineto{\pgfqpoint{3.828524in}{0.801508in}}%
\pgfpathlineto{\pgfqpoint{3.832858in}{0.779360in}}%
\pgfpathlineto{\pgfqpoint{3.837191in}{0.758867in}}%
\pgfpathlineto{\pgfqpoint{3.841525in}{0.739813in}}%
\pgfpathlineto{\pgfqpoint{3.845859in}{0.722467in}}%
\pgfpathlineto{\pgfqpoint{3.850192in}{0.706691in}}%
\pgfpathlineto{\pgfqpoint{3.854526in}{0.692918in}}%
\pgfpathlineto{\pgfqpoint{3.858860in}{0.681928in}}%
\pgfpathlineto{\pgfqpoint{3.863193in}{0.673547in}}%
\pgfpathlineto{\pgfqpoint{3.867527in}{0.667969in}}%
\pgfpathlineto{\pgfqpoint{3.871861in}{0.666410in}}%
\pgfpathlineto{\pgfqpoint{3.876194in}{0.670293in}}%
\pgfpathlineto{\pgfqpoint{3.880528in}{0.681012in}}%
\pgfpathlineto{\pgfqpoint{3.884862in}{0.700612in}}%
\pgfpathlineto{\pgfqpoint{3.889195in}{0.730794in}}%
\pgfpathlineto{\pgfqpoint{3.893529in}{0.774440in}}%
\pgfpathlineto{\pgfqpoint{3.897862in}{0.832543in}}%
\pgfpathlineto{\pgfqpoint{3.902196in}{0.905216in}}%
\pgfpathlineto{\pgfqpoint{3.910863in}{1.079620in}}%
\pgfpathlineto{\pgfqpoint{3.915197in}{1.164726in}}%
\pgfpathlineto{\pgfqpoint{3.919531in}{1.234317in}}%
\pgfpathlineto{\pgfqpoint{3.923864in}{1.278720in}}%
\pgfpathlineto{\pgfqpoint{3.928198in}{1.295603in}}%
\pgfpathlineto{\pgfqpoint{3.932532in}{1.287146in}}%
\pgfpathlineto{\pgfqpoint{3.936865in}{1.259142in}}%
\pgfpathlineto{\pgfqpoint{3.941199in}{1.219134in}}%
\pgfpathlineto{\pgfqpoint{3.958534in}{1.043303in}}%
\pgfpathlineto{\pgfqpoint{3.962867in}{1.004868in}}%
\pgfpathlineto{\pgfqpoint{3.971535in}{0.935928in}}%
\pgfpathlineto{\pgfqpoint{3.975868in}{0.904817in}}%
\pgfpathlineto{\pgfqpoint{3.984536in}{0.848577in}}%
\pgfpathlineto{\pgfqpoint{3.993203in}{0.797714in}}%
\pgfpathlineto{\pgfqpoint{4.001870in}{0.752808in}}%
\pgfpathlineto{\pgfqpoint{4.006204in}{0.731729in}}%
\pgfpathlineto{\pgfqpoint{4.010537in}{0.712452in}}%
\pgfpathlineto{\pgfqpoint{4.014871in}{0.694600in}}%
\pgfpathlineto{\pgfqpoint{4.019205in}{0.677966in}}%
\pgfpathlineto{\pgfqpoint{4.023538in}{0.662446in}}%
\pgfpathlineto{\pgfqpoint{4.027872in}{0.648049in}}%
\pgfpathlineto{\pgfqpoint{4.032206in}{0.635218in}}%
\pgfpathlineto{\pgfqpoint{4.036539in}{0.623518in}}%
\pgfpathlineto{\pgfqpoint{4.040873in}{0.613857in}}%
\pgfpathlineto{\pgfqpoint{4.045207in}{0.607039in}}%
\pgfpathlineto{\pgfqpoint{4.049540in}{0.603352in}}%
\pgfpathlineto{\pgfqpoint{4.053874in}{0.602961in}}%
\pgfpathlineto{\pgfqpoint{4.058208in}{0.607917in}}%
\pgfpathlineto{\pgfqpoint{4.062541in}{0.620134in}}%
\pgfpathlineto{\pgfqpoint{4.066875in}{0.642744in}}%
\pgfpathlineto{\pgfqpoint{4.071209in}{0.678199in}}%
\pgfpathlineto{\pgfqpoint{4.075542in}{0.730431in}}%
\pgfpathlineto{\pgfqpoint{4.079876in}{0.800966in}}%
\pgfpathlineto{\pgfqpoint{4.084210in}{0.890179in}}%
\pgfpathlineto{\pgfqpoint{4.097210in}{1.209055in}}%
\pgfpathlineto{\pgfqpoint{4.101544in}{1.289289in}}%
\pgfpathlineto{\pgfqpoint{4.105878in}{1.334272in}}%
\pgfpathlineto{\pgfqpoint{4.110211in}{1.341965in}}%
\pgfpathlineto{\pgfqpoint{4.114545in}{1.320466in}}%
\pgfpathlineto{\pgfqpoint{4.118879in}{1.278280in}}%
\pgfpathlineto{\pgfqpoint{4.131880in}{1.124060in}}%
\pgfpathlineto{\pgfqpoint{4.136213in}{1.078300in}}%
\pgfpathlineto{\pgfqpoint{4.140547in}{1.036528in}}%
\pgfpathlineto{\pgfqpoint{4.144881in}{0.997774in}}%
\pgfpathlineto{\pgfqpoint{4.149214in}{0.962966in}}%
\pgfpathlineto{\pgfqpoint{4.157882in}{0.901117in}}%
\pgfpathlineto{\pgfqpoint{4.162215in}{0.873644in}}%
\pgfpathlineto{\pgfqpoint{4.166549in}{0.847972in}}%
\pgfpathlineto{\pgfqpoint{4.170883in}{0.824257in}}%
\pgfpathlineto{\pgfqpoint{4.175216in}{0.803064in}}%
\pgfpathlineto{\pgfqpoint{4.179550in}{0.783608in}}%
\pgfpathlineto{\pgfqpoint{4.183884in}{0.765700in}}%
\pgfpathlineto{\pgfqpoint{4.188217in}{0.750320in}}%
\pgfpathlineto{\pgfqpoint{4.192551in}{0.737668in}}%
\pgfpathlineto{\pgfqpoint{4.196884in}{0.727245in}}%
\pgfpathlineto{\pgfqpoint{4.201218in}{0.720760in}}%
\pgfpathlineto{\pgfqpoint{4.205552in}{0.718872in}}%
\pgfpathlineto{\pgfqpoint{4.209885in}{0.722576in}}%
\pgfpathlineto{\pgfqpoint{4.214219in}{0.733134in}}%
\pgfpathlineto{\pgfqpoint{4.218553in}{0.751477in}}%
\pgfpathlineto{\pgfqpoint{4.222886in}{0.779285in}}%
\pgfpathlineto{\pgfqpoint{4.227220in}{0.818116in}}%
\pgfpathlineto{\pgfqpoint{4.231554in}{0.868364in}}%
\pgfpathlineto{\pgfqpoint{4.235887in}{0.929972in}}%
\pgfpathlineto{\pgfqpoint{4.248888in}{1.141326in}}%
\pgfpathlineto{\pgfqpoint{4.253222in}{1.197526in}}%
\pgfpathlineto{\pgfqpoint{4.257556in}{1.234161in}}%
\pgfpathlineto{\pgfqpoint{4.261889in}{1.249620in}}%
\pgfpathlineto{\pgfqpoint{4.266223in}{1.245576in}}%
\pgfpathlineto{\pgfqpoint{4.270557in}{1.224729in}}%
\pgfpathlineto{\pgfqpoint{4.274890in}{1.194019in}}%
\pgfpathlineto{\pgfqpoint{4.279224in}{1.157731in}}%
\pgfpathlineto{\pgfqpoint{4.292225in}{1.041613in}}%
\pgfpathlineto{\pgfqpoint{4.300892in}{0.971287in}}%
\pgfpathlineto{\pgfqpoint{4.309559in}{0.909103in}}%
\pgfpathlineto{\pgfqpoint{4.318227in}{0.853012in}}%
\pgfpathlineto{\pgfqpoint{4.326894in}{0.802734in}}%
\pgfpathlineto{\pgfqpoint{4.335561in}{0.758169in}}%
\pgfpathlineto{\pgfqpoint{4.344229in}{0.719994in}}%
\pgfpathlineto{\pgfqpoint{4.348562in}{0.703008in}}%
\pgfpathlineto{\pgfqpoint{4.352896in}{0.687750in}}%
\pgfpathlineto{\pgfqpoint{4.357230in}{0.673646in}}%
\pgfpathlineto{\pgfqpoint{4.361563in}{0.661697in}}%
\pgfpathlineto{\pgfqpoint{4.365897in}{0.652743in}}%
\pgfpathlineto{\pgfqpoint{4.370231in}{0.646505in}}%
\pgfpathlineto{\pgfqpoint{4.374564in}{0.643987in}}%
\pgfpathlineto{\pgfqpoint{4.378898in}{0.645839in}}%
\pgfpathlineto{\pgfqpoint{4.383232in}{0.654367in}}%
\pgfpathlineto{\pgfqpoint{4.387565in}{0.670659in}}%
\pgfpathlineto{\pgfqpoint{4.391899in}{0.697672in}}%
\pgfpathlineto{\pgfqpoint{4.396233in}{0.737049in}}%
\pgfpathlineto{\pgfqpoint{4.400566in}{0.791679in}}%
\pgfpathlineto{\pgfqpoint{4.404900in}{0.862797in}}%
\pgfpathlineto{\pgfqpoint{4.409233in}{0.948497in}}%
\pgfpathlineto{\pgfqpoint{4.417901in}{1.138930in}}%
\pgfpathlineto{\pgfqpoint{4.422234in}{1.221974in}}%
\pgfpathlineto{\pgfqpoint{4.426568in}{1.280650in}}%
\pgfpathlineto{\pgfqpoint{4.430902in}{1.308598in}}%
\pgfpathlineto{\pgfqpoint{4.435235in}{1.308366in}}%
\pgfpathlineto{\pgfqpoint{4.439569in}{1.283813in}}%
\pgfpathlineto{\pgfqpoint{4.443903in}{1.244075in}}%
\pgfpathlineto{\pgfqpoint{4.461237in}{1.058751in}}%
\pgfpathlineto{\pgfqpoint{4.469905in}{0.981976in}}%
\pgfpathlineto{\pgfqpoint{4.478572in}{0.916155in}}%
\pgfpathlineto{\pgfqpoint{4.487239in}{0.858893in}}%
\pgfpathlineto{\pgfqpoint{4.491573in}{0.832294in}}%
\pgfpathlineto{\pgfqpoint{4.495907in}{0.807765in}}%
\pgfpathlineto{\pgfqpoint{4.504574in}{0.763428in}}%
\pgfpathlineto{\pgfqpoint{4.508907in}{0.743159in}}%
\pgfpathlineto{\pgfqpoint{4.513241in}{0.724348in}}%
\pgfpathlineto{\pgfqpoint{4.517575in}{0.707718in}}%
\pgfpathlineto{\pgfqpoint{4.521908in}{0.692421in}}%
\pgfpathlineto{\pgfqpoint{4.526242in}{0.678637in}}%
\pgfpathlineto{\pgfqpoint{4.530576in}{0.666524in}}%
\pgfpathlineto{\pgfqpoint{4.534909in}{0.656936in}}%
\pgfpathlineto{\pgfqpoint{4.539243in}{0.650644in}}%
\pgfpathlineto{\pgfqpoint{4.543577in}{0.648054in}}%
\pgfpathlineto{\pgfqpoint{4.547910in}{0.649934in}}%
\pgfpathlineto{\pgfqpoint{4.552244in}{0.658562in}}%
\pgfpathlineto{\pgfqpoint{4.556578in}{0.676719in}}%
\pgfpathlineto{\pgfqpoint{4.560911in}{0.706031in}}%
\pgfpathlineto{\pgfqpoint{4.565245in}{0.749809in}}%
\pgfpathlineto{\pgfqpoint{4.569579in}{0.808489in}}%
\pgfpathlineto{\pgfqpoint{4.573912in}{0.883325in}}%
\pgfpathlineto{\pgfqpoint{4.578246in}{0.972322in}}%
\pgfpathlineto{\pgfqpoint{4.586913in}{1.161425in}}%
\pgfpathlineto{\pgfqpoint{4.591247in}{1.239571in}}%
\pgfpathlineto{\pgfqpoint{4.595581in}{1.291858in}}%
\pgfpathlineto{\pgfqpoint{4.599914in}{1.313822in}}%
\pgfpathlineto{\pgfqpoint{4.604248in}{1.304315in}}%
\pgfpathlineto{\pgfqpoint{4.608582in}{1.273224in}}%
\pgfpathlineto{\pgfqpoint{4.612915in}{1.230694in}}%
\pgfpathlineto{\pgfqpoint{4.625916in}{1.089534in}}%
\pgfpathlineto{\pgfqpoint{4.630250in}{1.047231in}}%
\pgfpathlineto{\pgfqpoint{4.638917in}{0.971817in}}%
\pgfpathlineto{\pgfqpoint{4.643251in}{0.938353in}}%
\pgfpathlineto{\pgfqpoint{4.651918in}{0.878165in}}%
\pgfpathlineto{\pgfqpoint{4.660585in}{0.826043in}}%
\pgfpathlineto{\pgfqpoint{4.669253in}{0.779010in}}%
\pgfpathlineto{\pgfqpoint{4.677920in}{0.738408in}}%
\pgfpathlineto{\pgfqpoint{4.682254in}{0.720177in}}%
\pgfpathlineto{\pgfqpoint{4.686587in}{0.703998in}}%
\pgfpathlineto{\pgfqpoint{4.690921in}{0.689439in}}%
\pgfpathlineto{\pgfqpoint{4.695255in}{0.676423in}}%
\pgfpathlineto{\pgfqpoint{4.699588in}{0.665760in}}%
\pgfpathlineto{\pgfqpoint{4.703922in}{0.657925in}}%
\pgfpathlineto{\pgfqpoint{4.708256in}{0.653945in}}%
\pgfpathlineto{\pgfqpoint{4.712589in}{0.654606in}}%
\pgfpathlineto{\pgfqpoint{4.716923in}{0.660911in}}%
\pgfpathlineto{\pgfqpoint{4.721256in}{0.674794in}}%
\pgfpathlineto{\pgfqpoint{4.725590in}{0.698419in}}%
\pgfpathlineto{\pgfqpoint{4.729924in}{0.734505in}}%
\pgfpathlineto{\pgfqpoint{4.734257in}{0.784302in}}%
\pgfpathlineto{\pgfqpoint{4.738591in}{0.849613in}}%
\pgfpathlineto{\pgfqpoint{4.742925in}{0.929620in}}%
\pgfpathlineto{\pgfqpoint{4.755926in}{1.198643in}}%
\pgfpathlineto{\pgfqpoint{4.760259in}{1.263109in}}%
\pgfpathlineto{\pgfqpoint{4.764593in}{1.299148in}}%
\pgfpathlineto{\pgfqpoint{4.768927in}{1.305250in}}%
\pgfpathlineto{\pgfqpoint{4.773260in}{1.286581in}}%
\pgfpathlineto{\pgfqpoint{4.777594in}{1.249893in}}%
\pgfpathlineto{\pgfqpoint{4.786261in}{1.158372in}}%
\pgfpathlineto{\pgfqpoint{4.794929in}{1.068367in}}%
\pgfpathlineto{\pgfqpoint{4.799262in}{1.028145in}}%
\pgfpathlineto{\pgfqpoint{4.803596in}{0.990753in}}%
\pgfpathlineto{\pgfqpoint{4.812263in}{0.923870in}}%
\pgfpathlineto{\pgfqpoint{4.816597in}{0.893571in}}%
\pgfpathlineto{\pgfqpoint{4.825264in}{0.837825in}}%
\pgfpathlineto{\pgfqpoint{4.829598in}{0.812528in}}%
\pgfpathlineto{\pgfqpoint{4.838265in}{0.767091in}}%
\pgfpathlineto{\pgfqpoint{4.842599in}{0.746157in}}%
\pgfpathlineto{\pgfqpoint{4.846932in}{0.726770in}}%
\pgfpathlineto{\pgfqpoint{4.855600in}{0.692143in}}%
\pgfpathlineto{\pgfqpoint{4.859933in}{0.677239in}}%
\pgfpathlineto{\pgfqpoint{4.864267in}{0.664000in}}%
\pgfpathlineto{\pgfqpoint{4.868601in}{0.652273in}}%
\pgfpathlineto{\pgfqpoint{4.872934in}{0.642903in}}%
\pgfpathlineto{\pgfqpoint{4.877268in}{0.636372in}}%
\pgfpathlineto{\pgfqpoint{4.881602in}{0.633185in}}%
\pgfpathlineto{\pgfqpoint{4.885935in}{0.634229in}}%
\pgfpathlineto{\pgfqpoint{4.890269in}{0.641106in}}%
\pgfpathlineto{\pgfqpoint{4.894603in}{0.655882in}}%
\pgfpathlineto{\pgfqpoint{4.898936in}{0.680498in}}%
\pgfpathlineto{\pgfqpoint{4.903270in}{0.717882in}}%
\pgfpathlineto{\pgfqpoint{4.907604in}{0.769692in}}%
\pgfpathlineto{\pgfqpoint{4.911937in}{0.838074in}}%
\pgfpathlineto{\pgfqpoint{4.916271in}{0.922400in}}%
\pgfpathlineto{\pgfqpoint{4.929272in}{1.210959in}}%
\pgfpathlineto{\pgfqpoint{4.933605in}{1.278908in}}%
\pgfpathlineto{\pgfqpoint{4.937939in}{1.316855in}}%
\pgfpathlineto{\pgfqpoint{4.942273in}{1.320917in}}%
\pgfpathlineto{\pgfqpoint{4.946606in}{1.299510in}}%
\pgfpathlineto{\pgfqpoint{4.950940in}{1.259727in}}%
\pgfpathlineto{\pgfqpoint{4.959607in}{1.162828in}}%
\pgfpathlineto{\pgfqpoint{4.963941in}{1.114209in}}%
\pgfpathlineto{\pgfqpoint{4.968275in}{1.069598in}}%
\pgfpathlineto{\pgfqpoint{4.972608in}{1.028260in}}%
\pgfpathlineto{\pgfqpoint{4.976942in}{0.990255in}}%
\pgfpathlineto{\pgfqpoint{4.981276in}{0.955234in}}%
\pgfpathlineto{\pgfqpoint{4.989943in}{0.893195in}}%
\pgfpathlineto{\pgfqpoint{4.998610in}{0.838968in}}%
\pgfpathlineto{\pgfqpoint{5.002944in}{0.814018in}}%
\pgfpathlineto{\pgfqpoint{5.007278in}{0.790961in}}%
\pgfpathlineto{\pgfqpoint{5.011611in}{0.769594in}}%
\pgfpathlineto{\pgfqpoint{5.015945in}{0.750045in}}%
\pgfpathlineto{\pgfqpoint{5.020279in}{0.732335in}}%
\pgfpathlineto{\pgfqpoint{5.024612in}{0.716564in}}%
\pgfpathlineto{\pgfqpoint{5.028946in}{0.702861in}}%
\pgfpathlineto{\pgfqpoint{5.033279in}{0.691687in}}%
\pgfpathlineto{\pgfqpoint{5.037613in}{0.683457in}}%
\pgfpathlineto{\pgfqpoint{5.041947in}{0.678052in}}%
\pgfpathlineto{\pgfqpoint{5.046280in}{0.677412in}}%
\pgfpathlineto{\pgfqpoint{5.050614in}{0.682339in}}%
\pgfpathlineto{\pgfqpoint{5.054948in}{0.694184in}}%
\pgfpathlineto{\pgfqpoint{5.059281in}{0.715402in}}%
\pgfpathlineto{\pgfqpoint{5.063615in}{0.747325in}}%
\pgfpathlineto{\pgfqpoint{5.067949in}{0.792443in}}%
\pgfpathlineto{\pgfqpoint{5.072282in}{0.851601in}}%
\pgfpathlineto{\pgfqpoint{5.076616in}{0.924464in}}%
\pgfpathlineto{\pgfqpoint{5.089617in}{1.173172in}}%
\pgfpathlineto{\pgfqpoint{5.093951in}{1.236416in}}%
\pgfpathlineto{\pgfqpoint{5.098284in}{1.275204in}}%
\pgfpathlineto{\pgfqpoint{5.102618in}{1.287579in}}%
\pgfpathlineto{\pgfqpoint{5.106952in}{1.274784in}}%
\pgfpathlineto{\pgfqpoint{5.111285in}{1.245707in}}%
\pgfpathlineto{\pgfqpoint{5.119953in}{1.161937in}}%
\pgfpathlineto{\pgfqpoint{5.128620in}{1.074935in}}%
\pgfpathlineto{\pgfqpoint{5.132953in}{1.034943in}}%
\pgfpathlineto{\pgfqpoint{5.141621in}{0.962729in}}%
\pgfpathlineto{\pgfqpoint{5.150288in}{0.899301in}}%
\pgfpathlineto{\pgfqpoint{5.158955in}{0.842800in}}%
\pgfpathlineto{\pgfqpoint{5.167623in}{0.792630in}}%
\pgfpathlineto{\pgfqpoint{5.176290in}{0.747839in}}%
\pgfpathlineto{\pgfqpoint{5.184957in}{0.707931in}}%
\pgfpathlineto{\pgfqpoint{5.193625in}{0.671794in}}%
\pgfpathlineto{\pgfqpoint{5.193625in}{0.671794in}}%
\pgfusepath{stroke}%
\end{pgfscope}%
\begin{pgfscope}%
\pgfsetrectcap%
\pgfsetmiterjoin%
\pgfsetlinewidth{1.003750pt}%
\definecolor{currentstroke}{rgb}{1.000000,1.000000,1.000000}%
\pgfsetstrokecolor{currentstroke}%
\pgfsetdash{}{0pt}%
\pgfpathmoveto{\pgfqpoint{0.647840in}{0.524382in}}%
\pgfpathlineto{\pgfqpoint{0.647840in}{1.411543in}}%
\pgfusepath{stroke}%
\end{pgfscope}%
\begin{pgfscope}%
\pgfsetrectcap%
\pgfsetmiterjoin%
\pgfsetlinewidth{1.003750pt}%
\definecolor{currentstroke}{rgb}{1.000000,1.000000,1.000000}%
\pgfsetstrokecolor{currentstroke}%
\pgfsetdash{}{0pt}%
\pgfpathmoveto{\pgfqpoint{5.410091in}{0.524382in}}%
\pgfpathlineto{\pgfqpoint{5.410091in}{1.411543in}}%
\pgfusepath{stroke}%
\end{pgfscope}%
\begin{pgfscope}%
\pgfsetrectcap%
\pgfsetmiterjoin%
\pgfsetlinewidth{1.003750pt}%
\definecolor{currentstroke}{rgb}{1.000000,1.000000,1.000000}%
\pgfsetstrokecolor{currentstroke}%
\pgfsetdash{}{0pt}%
\pgfpathmoveto{\pgfqpoint{0.647840in}{0.524382in}}%
\pgfpathlineto{\pgfqpoint{5.410091in}{0.524382in}}%
\pgfusepath{stroke}%
\end{pgfscope}%
\begin{pgfscope}%
\pgfsetrectcap%
\pgfsetmiterjoin%
\pgfsetlinewidth{1.003750pt}%
\definecolor{currentstroke}{rgb}{1.000000,1.000000,1.000000}%
\pgfsetstrokecolor{currentstroke}%
\pgfsetdash{}{0pt}%
\pgfpathmoveto{\pgfqpoint{0.647840in}{1.411543in}}%
\pgfpathlineto{\pgfqpoint{5.410091in}{1.411543in}}%
\pgfusepath{stroke}%
\end{pgfscope}%
\begin{pgfscope}%
\pgfsetbuttcap%
\pgfsetmiterjoin%
\definecolor{currentfill}{rgb}{0.269412,0.269412,0.269412}%
\pgfsetfillcolor{currentfill}%
\pgfsetfillopacity{0.500000}%
\pgfsetlinewidth{0.501875pt}%
\definecolor{currentstroke}{rgb}{0.269412,0.269412,0.269412}%
\pgfsetstrokecolor{currentstroke}%
\pgfsetstrokeopacity{0.500000}%
\pgfsetdash{}{0pt}%
\pgfpathmoveto{\pgfqpoint{5.560891in}{1.960420in}}%
\pgfpathlineto{\pgfqpoint{6.867413in}{1.960420in}}%
\pgfpathquadraticcurveto{\pgfqpoint{6.895191in}{1.960420in}}{\pgfqpoint{6.895191in}{1.988198in}}%
\pgfpathlineto{\pgfqpoint{6.895191in}{2.361655in}}%
\pgfpathquadraticcurveto{\pgfqpoint{6.895191in}{2.389432in}}{\pgfqpoint{6.867413in}{2.389432in}}%
\pgfpathlineto{\pgfqpoint{5.560891in}{2.389432in}}%
\pgfpathquadraticcurveto{\pgfqpoint{5.533113in}{2.389432in}}{\pgfqpoint{5.533113in}{2.361655in}}%
\pgfpathlineto{\pgfqpoint{5.533113in}{1.988198in}}%
\pgfpathquadraticcurveto{\pgfqpoint{5.533113in}{1.960420in}}{\pgfqpoint{5.560891in}{1.960420in}}%
\pgfpathclose%
\pgfusepath{stroke,fill}%
\end{pgfscope}%
\begin{pgfscope}%
\pgfsetbuttcap%
\pgfsetmiterjoin%
\definecolor{currentfill}{rgb}{0.898039,0.898039,0.898039}%
\pgfsetfillcolor{currentfill}%
\pgfsetlinewidth{0.501875pt}%
\definecolor{currentstroke}{rgb}{0.800000,0.800000,0.800000}%
\pgfsetstrokecolor{currentstroke}%
\pgfsetdash{}{0pt}%
\pgfpathmoveto{\pgfqpoint{5.533113in}{1.988198in}}%
\pgfpathlineto{\pgfqpoint{6.839635in}{1.988198in}}%
\pgfpathquadraticcurveto{\pgfqpoint{6.867413in}{1.988198in}}{\pgfqpoint{6.867413in}{2.015976in}}%
\pgfpathlineto{\pgfqpoint{6.867413in}{2.389432in}}%
\pgfpathquadraticcurveto{\pgfqpoint{6.867413in}{2.417210in}}{\pgfqpoint{6.839635in}{2.417210in}}%
\pgfpathlineto{\pgfqpoint{5.533113in}{2.417210in}}%
\pgfpathquadraticcurveto{\pgfqpoint{5.505336in}{2.417210in}}{\pgfqpoint{5.505336in}{2.389432in}}%
\pgfpathlineto{\pgfqpoint{5.505336in}{2.015976in}}%
\pgfpathquadraticcurveto{\pgfqpoint{5.505336in}{1.988198in}}{\pgfqpoint{5.533113in}{1.988198in}}%
\pgfpathclose%
\pgfusepath{stroke,fill}%
\end{pgfscope}%
\begin{pgfscope}%
\pgfsetrectcap%
\pgfsetroundjoin%
\pgfsetlinewidth{1.505625pt}%
\definecolor{currentstroke}{rgb}{0.886275,0.290196,0.200000}%
\pgfsetstrokecolor{currentstroke}%
\pgfsetdash{}{0pt}%
\pgfpathmoveto{\pgfqpoint{5.560891in}{2.313044in}}%
\pgfpathlineto{\pgfqpoint{5.838669in}{2.313044in}}%
\pgfusepath{stroke}%
\end{pgfscope}%
\begin{pgfscope}%
\definecolor{textcolor}{rgb}{0.000000,0.000000,0.000000}%
\pgfsetstrokecolor{textcolor}%
\pgfsetfillcolor{textcolor}%
\pgftext[x=5.949780in,y=2.264432in,left,base]{\color{textcolor}\rmfamily\fontsize{10.000000}{12.000000}\selectfont Ground Truth}%
\end{pgfscope}%
\begin{pgfscope}%
\pgfsetrectcap%
\pgfsetroundjoin%
\pgfsetlinewidth{1.505625pt}%
\definecolor{currentstroke}{rgb}{0.203922,0.541176,0.741176}%
\pgfsetstrokecolor{currentstroke}%
\pgfsetdash{}{0pt}%
\pgfpathmoveto{\pgfqpoint{5.560891in}{2.119371in}}%
\pgfpathlineto{\pgfqpoint{5.838669in}{2.119371in}}%
\pgfusepath{stroke}%
\end{pgfscope}%
\begin{pgfscope}%
\definecolor{textcolor}{rgb}{0.000000,0.000000,0.000000}%
\pgfsetstrokecolor{textcolor}%
\pgfsetfillcolor{textcolor}%
\pgftext[x=5.949780in,y=2.070760in,left,base]{\color{textcolor}\rmfamily\fontsize{10.000000}{12.000000}\selectfont Prediction}%
\end{pgfscope}%
\end{pgfpicture}%
\makeatother%
\endgroup%

  \caption{Using an \gls{esn} to replicate the climate of the Lorenz Attractor.}
  \label{fig:lorenz63}
\end{figure}
