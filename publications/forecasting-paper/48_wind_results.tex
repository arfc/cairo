Finally, we performed the same previous prediction tasks for wind energy.
Figure \ref{fig:wind48} and Figure \ref{fig:wind04} show the best forecasts for
48 and 4-hours ahead, respectively. All features except air pressure improved
the forecast. Including solar elevation angle improved the 48-hour ahead
forecast the most, while adding windspeed improved the 4-hour ahead the most.
Those results are shown in Table \ref{tab:wind48} and \ref{tab:wind04}
respectively.
These results are somewhat worse than the results from other work.


\begin{figure*}[h]
  \centering
  % \includegraphics[width=0.8\textwidth]{48_wind_elevation_prediction.png}
  \input{./images/48_wind_elevation_prediction.pgf}
  \caption{The optimized 48-hour ahead wind energy prediction. The inputs for this forecast were wind energy and solar elevation angle. \textit{Hyperparameters}: Reservoir Size:1000, Sparsity: 0.1, Spectral Radius: 0.9, Noise: 0.0001, Training Length: 19100, Prediction Window: 48, Random state: 85}
  \label{fig:wind48}
\end{figure*}
% \begin{center}
  \begin{table*}[h]
    \centering
    \caption{Tabulated error for 48-hour ahead wind forecasts with various coupled quantities. Improvement indicates the percentage improvement over the base case of forecasting wind energy alone.}
    \label{tab:wind48}
    \begin{tabular}{l|r|r|r|r|r}
      & & & & Improvement & Improvement \\
      Scenario &NRMSE & MAE & RMSE & MAE (\%) & RMSE (\%)\\
      \hline
      Wind Energy & 0.93167 & 0.1035 & 0.1308 & [-] & [-] \\
      Wind + Sun Elevation & 0.81220 & 0.0857 & 0.1141 & -17.1981 & -12.7676 \\
      Wind + Humidity & 0.84950 & 0.0952 & 0.1193 & -8.0193 & -8.7620 \\
      Wind + Pressure & 0.98345 & 0.1076 & 0.1381 & +3.9614 & +5.5810 \\
      Wind + Wet Bulb Temp. & 0.84323 & 0.0886 & 0.1184 & -14.3961 & -9.4801 \\
      Wind + Dry Bulb Temp. & 0.86365 & 0.0815 & 0.1213 & -21.2560 & -7.2630 \\
      Wind + Wind Speed & 0.84180 & 0.0763 & 0.1182 & -26.2802 & -9.6330 \\
    \end{tabular}
  \end{table*}
% \end{center}
