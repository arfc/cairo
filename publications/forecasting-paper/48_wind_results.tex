Finally, we performed the same prediction tasks as before for wind energy.
Figure \ref{fig:wind48} and Figure \ref{fig:wind04} show the best forecasts
that minimized the RMSE for
48- and 4-hours ahead, respectively. All features except air pressure improved
the forecast. Including solar elevation angle improved the 48-hour ahead
forecast the most, while adding windspeed improved the 4-hour ahead forecast
the most. Those results are shown in Table \ref{tab:wind48} and \ref{tab:wind04}
respectively. Lop\'ez et al. 2018 developed an advanced algorithm based on
\gls{esn} by replacing the neurons in the reservoir with \gls{lstm} blocks \cite{lopez_wind_2018}. Our conventional \gls{esn} achieved
results comparable to the algorithm from Lop\'ez et al. 2018 by simply adding a
single meteorological predictor.
Chitsazan et al. 2017 \cite{chitsazan_wind_2017} compared wind speed
forecasting with a conventional \gls{esn} to an \gls{esn} with nonlinear
readouts and achieved better results with the base model than we did
\cite{chitsazan_wind_2017}. However, this could be attributed to the fact that
they used data with 10-minute resolution and 10-minute prediction steps.
Unfortunately they did not include information about the hyper-parameters used
for each prediction task.

Still, the state-of-the-art Weather Research and Forecasting model
\cite{powers_weather_2017}, a numerical weather prediction model, far
outperforms our best results. Additionally, the 48-hour ahead forecast accuracy
needs to be an order of magnitude lower than our model for meaningful economic
benefit. Thus, conventional \glspl{esn} are currently insufficient for
applications in grid planning \cite{wang_quantifying_2016}.

\begin{figure}[H]
  \centering
  \resizebox{\columnwidth}{!}{\input{./images/48_wind_elevation_prediction.pgf}}
  \caption{The optimized 48-hour ahead wind energy prediction that minimized
  the RMSE. The training inputs for this forecast were historical wind energy and solar elevation
  angle. \textit{Hyperparameters}: Reservoir Size:1000, Sparsity: 0.1, Spectral
  Radius: 0.9, Noise: 0.0001, Training Length: 19100, Prediction Window: 48,
  Random state: 85}
  \label{fig:wind48}
\end{figure}
% \FloatBarrier

\begin{table*}[h]
  \centering
  \caption{Tabulated error for 48-hour ahead wind forecasts with various coupled quantities.}
  \label{tab:wind48}
  \resizebox{\textwidth}{!}{
  \begin{tabular}{lrrrrr}
    \hline
    & & & & Improvement & Improvement \\
    Scenario &NRMSE & MAE & RMSE & MAE (\%) & RMSE (\%)\\
    \hline
    Wind Energy & 0.93167 & 0.1035 & 0.1308 & [-] & [-] \\
    Wind + Sun Elevation & 0.81220 & 0.0857 & 0.1141 & -17.1981 & -12.7676 \\
    Wind + Humidity & 0.84950 & 0.0952 & 0.1193 & -8.0193 & -8.7620 \\
    Wind + Pressure & 0.98345 & 0.1076 & 0.1381 & +3.9614 & +5.5810 \\
    Wind + Wet Bulb Temp. & 0.84323 & 0.0886 & 0.1184 & -14.3961 & -9.4801 \\
    Wind + Dry Bulb Temp. & 0.86365 & 0.0815 & 0.1213 & -21.2560 & -7.2630 \\
    Wind + Wind Speed & 0.84180 & 0.0763 & 0.1182 & -26.2802 & -9.6330 \\
    \hline
  \end{tabular}
  } % end resizebox
\end{table*}
% \FloatBarrier
